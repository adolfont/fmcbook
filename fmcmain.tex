%{{{ [vim] 
% vim:foldmarker=%{{{,%}}}
% vim:foldmethod=marker
% vim:foldcolumn=7
%}}}
%% fmcmain.tex
%% author: Thanos Tsouanas <thanos@tsouanas.org>
%% Copyright (c) 2016, 2017 Thanos Tsouanas <thanos@tsouanas.org>
%% All rights reserved.

%%{{{ chapter: Introduction 
\chapter Introdução.

%%{{{ Afirmações vs.~objetos 
\section Afirmações vs.~objetos.

\endsection
%%}}}

%%{{{ Axioms, definitions, theorems, proofs 
\section Axiomas, definições, teoremas, provas.

\endsection
%%}}}

%%{{{ Bound and free variables 
\section Variáveis ligades e livres.
\label{Bound_and_free_variables}%

\endsection
%%}}}

%%{{{ Notation 
\section Notação.

\note.
\label{defeq_vs_defiff}%
$\defeq$ vs $\defiff$.

\endsection
%%}}}

%%{{{ Sorts of numbers 
\section Tipos de números.
% cyclic definitions
% from reals to nats
% from nats to reals
\endsection
%%}}}

%%{{{ Sets, functions, relations 
\section Conjuntos, funções, relações.

\TODO Aridade.

\endsection
%%}}}

%%{{{ Problems 
\problems.

\endproblems
%%}}}

%%{{{ Further reading 
\further.

Matemática elementar:
\cite{simmonsprecalculus} (para uma revisão rápida);
\cite{langbasicmath} (para um estudo mais profundo).

Sobre geometria euclideana:
\cite{elements},
\cite{coxeterrevisited};
\cite{hartshorneeuclidbeyond}.

Sobre calculus:
\cite{spivakcalculus},
\cite{hardypuremath};
\cite{apostol1}~\&~\cite{apostol2},
\cite{loomissternberg}.

Sobre álgebra linear:
\cite{janichlinalg},
\cite{halmoslapb},
\cite{halmosfdvs}.

\endfurther
%%}}}

\endchapter
%%}}}

%%{{{ chapter: Rational and irrational numbers 
\chapter Números racionais e irracionais.

%%{{{ Rationals and the pythagoreans 
\section Os racionais e os pitagoreanos.

\endsection
%%}}}

%%{{{ sqrt(2) is not rational 
\section O $\sqrt 2$ não é racional.

\endsection
%%}}}

%%{{{ But sqrt(2) surely is a number 
\section Mas o $\sqrt 2$ com certeza é um número.

\endsection
%%}}}

%%{{{ Irrational numbers 
\section Números irracionais.

\endsection
%%}}}

%%{{{ sqrt(3) is irrational 
\section O $\sqrt 3$ é irracional.

\endsection
%%}}}

%%{{{ A lemma 
\section Um lemma.

\endsection
%%}}}

%%{{{ What happens with sqrt(4) and sqrt(5) 
\section O que acontece com $\sqrt 4$ e $\sqrt 5$.

\endsection
%%}}}

%%{{{ A generalization theorem 
\section Um teorema de generalização.

\endsection
%%}}}

%%{{{ More irrational numbers 
\section Mais números irracionais.

\endsection
%%}}}

%%{{{ Algebraic and transcendental numbers 
\section Números algébricos e transcendentais.

\endsection
%%}}}

%%{{{ Problems 
\problems.

\endproblems
%%}}}

%%{{{ Further reading 
\further.

\cite{nivenirrational}.

\endfurther
%%}}}

\endchapter
%%}}}

%%{{{ chapter: Languages 
\chapter Linguagens.

%%{{{ Numbers, numerals, digits 
\section Números, numerais, dígitos.
% dígito aka "algarismo"

\note.
\tdefined{número}%
\tdefined{numeral}%
\tdefined{dígito}%
\iisee{algarismo}{dígito}%
Aceitamos por enquanto como dado o conceito dos números que usamos
para contar:
$$
0, 1, 2, 3, \dots, 247, 248, 249, \dots
$$
Usando então apenas um \emph{alfabeto} composto de 10 símbolos
$$
\digit 0\ 
\digit 1\ 
\digit 2\ 
\digit 3\ 
\digit 4\ 
\digit 5\ 
\digit 6\ 
\digit 7\ 
\digit 8\ 
\digit 9
$$
e seguindo as regras bem-conhecidas do sistema decimal conseguimos
denotar qualquer um dos números, mesmo que tem uma infinidade deles!

Chamamos esses símbolos \dterm{dígitos} (ou \dterm{algarismos}),
e as palavras (ou ``strings'') que formamos justapondo esses dígitos que
representam os números, \dterm{numerais}.
Sem contexto, lendo o ``10'' existe já uma ambigüidade:
é o numeral $\numeral {10}$ ou o número dês?
Para apreciar essa diferença ainda mais, note que o numeral $\numeral {10}$,
pode representar outro número em outro contexto.
Por exemplo, no sistema binário, o numeral $\numeral {10}$
representa o número dois.
E a ambigüidade pode ser ainda maior lendo ``1'':
é o numeral $\numeral {1}$; o número um; ou o dígito~$\digit 1$?
Quando o contexto é suficiente para entender, não precisamos mudar a fonte
como acabei de fazer aqui, nem escrever explicitamente o que é.
Note que existem numerais bem diferentes para denotar esses números:
o numeral (romano) {XII}
e o numeral (grego) $\iota \beta$ denotam o mesmo número: $12$.

\blah.
Temos então umas pequenas linguágens que nos permitem descrever \emph{números}.
Não fatos sobre números.
Não operações que involvem números.
Números.
Quais números?
Todos os números \emph{naturais}, cuja totalidade simbolizamos com $\nats$.

\endsection
%%}}}

%%{{{ Arithmetic expressions: syntax vs.~semantics 
\section Expressões aritméticas: sintaxe vs.~semântica.

%%{{{ A first example expression 
\note.
\tdefined{precedência}%
Considere agora a expressão
$$
1 + 5 \ntimes 2
$$
que involve os numerais $1$, $5$, e $2$,
e os símbolos de funções $+$ (adição) e $\ntimes$ (multiplicação).
O que ela representa?
A multiplicação de $1+5$ com $2$, ou a adição de $1$ com $5\ntimes 2$?
A segunda opção, graças a uma convenção que temos%
---e que você provavelmente já encontrou na vida.
Digamos que a $\ntimes$ ``pega mais forte'' do que a $+$,
então precisamos ``a~aplicar'' primeiro.
Mais formalmente, a $\ntimes$ tem uma \dterm{precedência} mais alta que a da $+$.
Quando não temos convenções como essa, usamos parenteses para tirar a ambigüidade
e deixar claro como parsear uma expressão.
Então temos
$$
(1 + 5) \ntimes 2 \neq 1 + 5 \ntimes 2 = 1 + (5 \ntimes 2).
$$
%%}}}

%%{{{ One more example; syntax-vs-semantics 
\note.
\tdefined{associatividade}[esquerda e direita]%
\tdefined{igualdade}[sintáctica]%
\tdefined{igualdade}[semântica]%
\sdefined {\holed A = \holed B}      {igualdade semântica}%
\sdefined {\holed A \syneq \holed B} {igualdade sintáctica}%
E a expressão
$$
1+5+2
$$
representa o quê?
Não seja tentado dizer ``tanto faz'', pois mesmo que as duas
possíveis interpretações
$$
(1+5) + 2
\qqqquad \text{e} \qqqquad
1 + (5+2)
$$
\emph{denotam valores} iguais,
elas expressam algo diferente:
$$
\align
(1 + 5) + 2&: \quad\text{adicione o $1+5$ com o $2$};\\
1 + (5 + 2)&: \quad\text{adicione o $1$ com o $5+2$}.
\endalign
$$
Então\dots
$$
(1 + 5) + 2 \askeq 1 + (5 + 2)
$$
Como \emph{expressões} (a \emph{sintaxe}) são diferentes;
como \emph{valores} (a \emph{semântica}) são iguais,
pois denotam o mesmo objeto: o número $6$.
Normalmente em matemática ligamos sobre as denotações das expressões,
e logo escrevemos igualdades como
$$
(1 + 5) + 2 = 6 + 2 = 8 = 1 + 7 = 1 + (5 + 2).
$$
Ou seja, o símbolo `$=$' em geral denota \dterm{igualdade semântica}:
$A=B$ significa que os dois lados, $A$ e $B$, denotam o mesmo objeto.
Querendo representar \dterm{igualdade sintáctica}, as vezes usamos
outros símbolos.  Vamos usar o `$\syneq$' agora, de modo que:
$$
\align
1 + 2 = 3
&\qqquad\text{mas}\qqquad
1 + 2 \synneq 3;\\
(1 + 5) + 2 = 1 + (5 + 2)
&\qqquad\text{mas}\qqquad
(1 + 5) + 2 \synneq 1 + (5 + 2);
\quad\text{etc.}
\endalign
$$

Voltando à expressão $1 + 5 + 2$, precisamos mais uma convenção
para atribuir uma associatividade esquerda ou direita.
Vamos concordar que $a + b + c$ representa a expressão
$((a + b) + c)$, ou seja, atribuimos em $+$ uma
\emph{associatividade esquerda}.

Mas $+$ não é uma operação associativa?
Sim, e isso quis dizer que como valores,
$$
((a + b) + c) = (a + (b + c)).
$$
Mas como expressões, são diferentes, e logo sem essa convenção,
$a + b + c$ não representaria nenhuma expressão de aritmética!

\exercise.
Sejam $a,b,c$ números naturais.
Usando $=$ para igualdade semântica e $\syneq$ para igualdade
sintáctica, decida para cada uma das afirmações seguintes
se é verdadeira ou falsa:
\doublecolumns
\beginol
\li $a + b + c                \syneq  a + (b + c)$
\li $a + b + c                \syneq  (a + b) + c$
\li $a + b + c                =       a + (b + c)$
\li $a + b + c                =       (a + b) + c$
\li $2 \ntimes 0 + 3          =       0 + 3$
\li $2 \ntimes 0 + 3          \syneq  0 + 3$
\li $(2 \ntimes 0) + 3 + 0    =       1 + 1 + 1$
\li $2 \ntimes 0 + 3          \syneq  1 + 1 + 1$
\li $2 \ntimes 0 + 3          \syneq  2 \ntimes (0 + 3)$
\li $2 \ntimes 0 + 3          =       2 \ntimes (0 + 3)$
\li $2 \ntimes 0 + 3          \syneq  (2 \ntimes 0) + 3$
\li $1 + 2                    \syneq  2 + 1$
\endol
\singlecolumn

\solution
Temos:
\doublecolumns
\beginol
\li $a + b + c                \synneq a + (b + c)$
\li $a + b + c                \syneq  (a + b) + c$
\li $a + b + c                =       a + (b + c)$
\li $a + b + c                =       (a + b) + c$
\li $2 \ntimes 0 + 3          =       0 + 3$
\li $2 \ntimes 0 + 3          \synneq 0 + 3$
\li $(2 \ntimes 0) + 3 + 0    =       1 + 1 + 1$
\li $2 \ntimes 0 + 3          \synneq 1 + 1 + 1$
\li $2 \ntimes 0 + 3          \synneq 2 \ntimes (0 + 3)$
\li $2 \ntimes 0 + 3          \neq    2 \ntimes (0 + 3)$
\li $2 \ntimes 0 + 3          \syneq  (2 \ntimes 0) + 3$
\li $1 + 2                    \synneq 2 + 1$
\endol
\singlecolumn

\endexercise

\exercise.
Verdade ou falso?:
$$
A \syneq B \implies A = B
$$

\endexercise

%%}}}

\endsection
%%}}}

%%{{{ Derivation trees 
\section Arvores de derivação.

\note Parsing.
\tdefined{parsing}%
\tdefined{árvore}[sintáctica]%
\tdefined{árvore}[de derivação]%
Lendo uma expressão ``linear'' como a
``$1 + 5 \ntimes 2$''
nos a \dterm{parseamos} para revelar sua estrutura,
freqüentemente representada numa forma bidimensional, como uma
\dterm{árvore sintáctica}.
Temos então as árvores:
$$
1 + (5 \ntimes 2)
\quad\leadsto\quad
\gathered
\tikzpicture[scale=0.8]
\node [circle,draw] (z) {$+$}
  child {node [circle,draw] (a) {$1$}}
  child {node [circle,draw] (b) {$\vphantom+\ntimes$}
    child {node [circle,draw] (b1) {$5$}}
    child {node [circle,draw] (b2) {$2$}}
  };
\endtikzpicture
\endgathered
\qqqquad
(1 + 5) \ntimes 2
\quad\leadsto\quad
\gathered
\tikzpicture[scale=0.8]
\node [circle,draw] (z) {$\vphantom+\ntimes$}
  child {node [circle,draw] (a) {$+$}
    child {node [circle,draw] (a1) {$1$}}
    child {node [circle,draw] (a2) {$5$}}
  }
  child {node [circle,draw] (b) {$2$}};
\endtikzpicture
\endgathered
$$
Não vamos usar mais esse tipo de árvore sintáctica nessas notas.

\note Arvores de derivação.
\tdefined{árvore de derivação}%
Em vez, vamos ficar usando \dterm{árvores de derivação} como essas:
$$
\def\fCenter{}
\xalignat2
&\centerAlignProof
\AxiomC{1}
\AxiomC{5}
\AxiomC{2}
\RightLabel{$\ntimes$}
\BinaryInf$\fCenter (5 \ntimes 2)$
\RightLabel{$+$}
\BinaryInf$\fCenter 1 + (5 \ntimes 2)$
\DisplayProof{}
&
&\centerAlignProof
\AxiomC{1}
\AxiomC{5}
\RightLabel{$+$}
\BinaryInf$\fCenter (1 + 5)$
\AxiomC{2}
\RightLabel{$\ntimes$}
\BinaryInf$\fCenter (1 + 5) \ntimes 2$
\DisplayProof{}
\endxalignat
$$

Sendo esse nosso primeiro contato com árvores sintácticas, vamos explicar
em detalhe como as escrevemos.  Começamos então com a expressão (linear)
que queremos parsear:
$$
(1 + 5) \ntimes 2.
$$
Graças à sua parêntese, o ``operador principal'' (o mais ``externo'')
é o $\ntimes$.  Isso quis dizer que, no final das contas, essa expressão
representa uma multiplicação de duas coisas.
Exatamente por isso, reduzimos essa expressão em duas novas, escrevendo
uma linha em cima dela, onde temos agora dois lugares para botar essas
duas coisas.  No lado da linha, escrevemos sua ``justificativa''
$$
\def\fCenter{}
\centerAlignProof
\AxiomC{$\hole$}
\AxiomC{$\hole$}
\RightLabel{$\holed \ntimes$}
\BinaryInf$\fCenter (1 + 5) \ntimes 2$
\DisplayProof{}
$$
e nos dois buracos que aparecem botamos as expressões que estão
nos lados desse $\ntimes$:
$$
\def\fCenter{}
\centerAlignProof
\AxiomC{$\holed {(1 + 5)}$}
\AxiomC{$\holed {\vphantom(2}$}
\RightLabel{$\ntimes$}
\BinaryInf$\fCenter (1 + 5) \ntimes 2$
\DisplayProof{}
$$
Agora $2$ já é uma expressão \dterm{atómica} (ou seja, inquebrável),
mas a $(1 + 5)$ não é então repetimos o mesmo processo nela:
$$
\def\fCenter{}
\centerAlignProof
\AxiomC{$\holed 1$}
\AxiomC{$\holed 5$}
\RightLabel{$\holed +$}
\BinaryInf$\fCenter (1 + 5)$
\AxiomC{${\vphantom(2}$}
\RightLabel{$\ntimes$}
\BinaryInf$\fCenter (1 + 5) \ntimes 2$
\DisplayProof{}
$$
Chegamos finalmente na árvore
$$
\def\fCenter{}
\centerAlignProof
\AxiomC{$1$}
\AxiomC{$5$}
\RightLabel{$+$}
\BinaryInf$\fCenter (1 + 5)$
\AxiomC{${\vphantom(2}$}
\RightLabel{$\ntimes$}
\BinaryInf$\fCenter (1 + 5) \ntimes 2$
\DisplayProof{}
$$
que mostra como a expressão aritmética $(1 + 5) \ntimes 2$ na sua \dterm{raiz}
(ou \dterm{root}) é composta por os numeráis $1$, $5$, e $2$
(que são as \dterm{folhas} (ou \dterm{leaves}) dessa árvore.

\endsection
%%}}}

%%{{{ Grammars and BNF notation 
\section Gramáticas e a notação BNF.
\label{BNF_notation}%
\tdefined{BNF}%
\tdefined{gramática}%
\iisee{Backus--Naur form}{BNF}%

%%{{{ A first try 
\note Uma primeira tentativa.
\label{BNF_a_first_try}%
Vamos começar diretamente com um exemplo de uso da
notação~\dterm{BNF}
(Backus\Backus{}--Naur\Naur{} form),
para descrever uma linguagem de expressões aritméticas, usando
a \dterm{gramática} seguinte:
%%}}}

%%{{{ g: ArExp_grammar_1 
\grammar ArExp (1).
\label{ArExp_grammar_1}%
$$
\align
\bnf{ArExp} &\bnfeq 0 \bnfor 1 \bnfor 2 \bnfor 3 \bnfor \dotsb \tag{1}\\
\bnf{ArExp} &\bnfeq (\bnf{ArExp} + \bnf{ArExp})                \tag{2}
\endalign
$$
%%}}}

%%{{{ Explanation 
\blah.
O que tudo isso significa?
A primeira linha, é uma regra dizendo:
uma expressão aritmética pode ser um dos
$0$, $1$, $2$, $3$, \dots.
A segunda linha é mais interessante: uma expressão aritmética pode começar
com o símbolo `(',
depois ter uma expressão aritmética,
depois o símbolo `+',
depois mais uma expressão aritmética,
e finalmente o símbolo `)'.
A idéia é que o que aparece com ângulos é algo que precisa ser substituido,
com uma das opções que aparecem no lado direito de alguma regra que começa com ele.

Começando com o $\bnf{ArExp}$ ficamos substituindo até não aparece mais nada em ângulos.
Et voilà: neste momento temos criado uma expressão aritmética.
%%}}}

%%{{{ eg: BNF_first_example 
\example.
Use as regras (1)--(2) da~\ref{ArExp_grammar_1} em cima para criar uma expressão
aritmética.
Comece usando a regra (2).
$$
\align
\bnf{ArExp}
&\leadstoby {(2)} (\bnf{ArExp} + \bnf{ArExp})\\
&\leadstoby {(1)} (\bnf{ArExp} + 3)\\
&\leadstoby {(2)} ((\bnf{ArExp} + \bnf{ArExp}) + 3)\\
&\leadstoby {(1)} ((128 + \bnf{ArExp}) + 3)\\
&\leadstoby {(1)} ((128 + 0) + 3)
\endalign
$$
\endexample
%%}}}

%%{{{ eg: BNF_direct_example 
\example.
Use a mesma gramática (agora sem restrição) para criar uma expressão aritmética.
$$
\align
\bnf{ArExp}
&\leadstoby {(1)} 17
\endalign
$$
\endexample
%%}}}

%%{{{ x: BNF_with_goal 
\exercise.
Mostre como usar a~\ref{ArExp_grammar_1} para gerar a expressão aritmética
$((1 + (2 + 2)) + 3)$.

\solution
Uma solução é a seguinte:
$$
\align
\bnf{ArExp}
&\leadstoby {(2)} (\bnf{ArExp} + \bnf{ArExp})\\
&\leadstoby {(1)} (\bnf{ArExp} + 3)\\
&\leadstoby {(2)} ((\bnf{ArExp} + \bnf{ArExp}) + 3)\\
&\leadstoby {(2)} ((\bnf{ArExp} + (\bnf{ArExp} + \bnf{ArExp})) + 3)\\
&\leadstoby {(1)} ((1 + (\bnf{ArExp} + \bnf{ArExp})) + 3)\\
&\leadstoby {(1)} ((1 + (2 + \bnf{ArExp})) + 3)\\
&\leadstoby {(1)} ((1 + (2 + 2)) + 3)
\endalign
$$

\endexercise
%%}}}

% doesn't have to be the same expression for the same <name> 

%%{{{ Q: problems_of_first_BNF
\question.
\label{problems_of_first_BNF}%
Quais são uns defeitos dessa primeira tentativa?
O que podemos fazer para a melhorar?
\spoiler.
%%}}}

%%{{{ A: problems_of_first_BNF_answer 
\note Resposta.
\label{problems_of_first_BNF_answer}%
Umas deficiências são:

\beginol
\li A linguagem gerada por essa gramática não é suficiente para representar expressões que involvem outras operações, como $-$, $\ntimes$, $\div$, etc.
\li A regra (1) tem uma infinitade de casos (graças aos `$\dotsb$').
\li As regras e os nomes escolhidos não refletam bem nossa idéia.
\endol
%%}}}

%%{{{ x: solve_first_problem_of_ArExp 
\exercise.
\label{solve_first_problem_of_ArExp}%
Apenas alterando a segunda regra da~\ref{ArExp_grammar_1}, resolva a primeira deficiência.

\hint
Falta só adicionar 3 mais casos na segunda regra, imitando para os outros operadores o caso do $+$.

\solution
Temos:
$$
\align
\bnf{ArExp} &\bnfeq 0 \bnfor 1 \bnfor 2 \bnfor 3 \bnfor \dotsb \tag{1}\\
\bnf{ArExp}
&\bnfeq (\bnf{ArExp} + \bnf{ArExp})\tag{2}\\
&\bnfOR (\bnf{ArExp} - \bnf{ArExp})\\
&\bnfOR (\bnf{ArExp} \ntimes \bnf{ArExp})\\
&\bnfOR (\bnf{ArExp} \div \bnf{ArExp})
\endalign
$$

\endexercise
%%}}}

%%{{{ A second try 
\note Uma segunda tentativa.
A solução que encontramos no~\ref{solve_first_problem_of_ArExp}
não é a coisa mais elegante do mundo.
Tem muita repetição que podemos evitar, definindo uma nova regra em nossa gramática:
%%}}}

%%{{{ g: ArExp_grammar_2 
\grammar ArExp (2).
\label{ArExp_grammar_2}%
$$
\align
\bnf{ArExp} &\bnfeq 0 \bnfor 1 \bnfor 2 \bnfor 3 \bnfor \dotsb \tag{1}\\
\bnf{ArExp} &\bnfeq (\bnf{ArExp} \bnf{BinOp} \bnf{ArExp})\tag{2}\\
\bnf{BinOp} &\bnfeq + \bnfor - \bnfor \ntimes \bnfor \div\tag{3}
\endalign
$$
%%}}}

\blah.
Ainda a gramática não refleta bem nossa idéia.
Podemos a melhorar, com mais regras e com nomes melhores
que deixam mais claras nossas intenções:

%%{{{ g: ArExp_grammar_3 
\grammar ArExp (3).
\label{ArExp_grammar_3}%
$$
\align
\bnf{ArExp} &\bnfeq \bnf{Num} \bnfor \bnf{OpExp} \tag{0}\\
\bnf{Num}   &\bnfeq 0 \bnfor 1 \bnfor 2 \bnfor 3 \bnfor \dotsb \tag{1}\\
\bnf{OpExp} &\bnfeq (\bnf{ArExp} \bnf{BinOp} \bnf{ArExp})\tag{2}\\
\bnf{BinOp} &\bnfeq + \bnfor - \bnfor \ntimes \bnfor \div\tag{3}
\endalign
$$
%%}}}

\blah.
Falta achar um jeito para remover esses ``$\dotsb$'' ainda,
mas vamos deixar isso para depois (\ref{remove_dots_from_ArExp_grammar}).

%%{{{ Nat_grammar 
\grammar Nat.
\label{Nat_grammar}%
$$
\align
\bnf{Nat} &\bnfeq 0 \bnfor S \bnf{Nat}
\endalign
$$
%%}}}

%%{{{ x: generate_some_nats_from_grammar 
\exercise.
\label{generate_some_nats_from_grammar}%
Quais são umas das palavras que podes gerar com a~\ref{Nat_grammar}?
Pode usar sua linguagem como numerais para os naturais?

\endexercise
%%}}}

\endsection
%%}}}

%%{{{ Metalanguages 
\section Metalinguagens.

\note.
\tdefined{metalinguagem}%
\tdefined{linguagem-objeto}%
Já encontramos o conceito de linguagem como um objeto de estudo.
Logo vamos estudar bem mais linguagens, de lógica matemática,
estudar linguágens de programação, etc.
É preciso entender que enquanto estudando uma linguagem,
esse próprio estudo acontece também usando uma (outra) linguagem.
Aqui usamos por exemplo português, provando propriedades, dando definições,
afirmando relações, etc., de outras linguagens que estudamos,
como da aritmética, de lógica matemática, de linguagens de programação, etc.
Para enfatisar essa diferença e para tirar certas ambigüidades,
chamamos \dterm{linguagem-objeto} a linguagem que estudamos,
e \dterm{metalinguagem} a linguagem que usamos para falar
sobre a linguagem-objeto.

\note.
\tdefined{metavariável}%
Imagine que você trabalha como programador e teu chefe lhe pediu
fazer uma mudança no código de um programa escrito na linguagem de
programação C.
Ele disse:
<<Cada variável $\alpha$ de tipo $\tau$ que aparece no código fonte tem que ser renomeada
para $\alpha\code{\_of\_}\tau$.>>
<<Por exemplo,>> ele continuou abrindo o programa no seu editor,
<<essa variável aqui $\code{i}$ que é de tipo $\code{int}$,
precisa ser renomeada para $\code{i\_of\_int}$;
e essa $\code{count}$ também para $\code{count\_of\_int}$;
e essa $\code{mean}$ de $\code{float}$, para $\code{mean\_of\_float}$,
etc.>>
\endgraf
Nesse pedido---obviamente sem noção, algo muito comum em pedidos de chefes
de programadores---aparecem duas ``espécies'' de variáveis diferentes:
as $\alpha$ e $\tau$ são variáveis de uma espécie;
as $\code{i}$, $\code{i\_of\_int}$,
$\code{count}$, $\code{count\_of\_int}$,
$\code{mean}$, e $\code{mean\_of\_float}$
de outra.
Chamamos a $\alpha$ e $\tau$ \dterm{metavariável},
pois elas pertencem na metalinguagem,
e não na linguagem-objeto,
que nesse exemplo é a linguagem de programação C.

\endsection
%%}}}

%%{{{ Abbreviations and syntactic sugar 
\section Abreviações e açúcar sintáctico.

%%{{{ x: ArExp_outer_parentheses_missing 
\exercise.
\label{ArExp_outer_parentheses_missing}%
Tente gerar a expressão
$$
(1 + 5) \ntimes 2
$$
usando a~\ref{ArExp_grammar_2}.

\solution
Não tem como!

\endexercise
%%}}}

%%{{{ Abbreviations 
\note Abreviações.
\tdefined{abreviação}%
Seguindo nossa~\ref{ArExp_grammar_2}, cada vez que escrevemos um operador binário
começamos e terminamos com `$($' e `$)$' respectivamente.
Logo, ``$1 + 2$'' nem é uma expressão gerada por essa gramática!
Mas como é tedioso botar as parenteses mais externas, temos a convenção
de as omitir.
Logo, consideramos a ``$1 + 2$'' como uma \dterm{abreviação} da expressão aritmética ``$(1 + 2)$''.
Então qual é o primeiro caráter da $1 + 2$?  É sim o `$($',
pois consideramos o $1+2$ apenas como um nome que usamos na
metalinguagem para denotar a expressão aritmética $(1+2)$,
que pertence na linguagem-objeto.
%%}}}

%%{{{ beware: abbr_not_always_shorter 
\beware.
\label{abbr_not_always_shorter}%
Não se iluda com a palavra ``abreviação'' que usamos aqui:
uma expressão pode ser mais curta do que uma das suas abreviações!
Nosso motivo não é preguiça de escrever expressões mais longas,
mas sim ajudar nossos olhos humanos a parsear.
%%}}}

%%{{{ Syntactic sugar 
\note Açúcar sintáctico.
\tdefined{açúcar sintáctico}%
Querendo enriquecer uma linguagem com um novo conceito, uma nova operação,
etc., parece que precisamos aumentar sua sintaxe para adicionar certos
símbolos e formas para corresponder nessas novas idéias.
Mas isso não é sempre necessário.
Por exemplo, suponha que trabalhamos com a linguagem da~\ref{ArExp_grammar_3},
e queremos a usar com sua interpretação canônica, onde suas expressões aritméticas
geradas denotam realmente as operações que conhecemos desde pequenos.
Agora, queremos adicionar uma operação unária $S$, escrita na forma prefixa,
onde a idéia é que $Sn$ denota o sucessor de $n$ (o próximo inteiro).
Nesse caso, em vez de realmente alterar a sintaxe da nossa linguagem,
definimos como \dterm{açúcar sintáctico} o uso de $S$ tal que, para qualquer
expressão aritmética $\alpha$, o $S\alpha$ denota a expressão $(\alpha + 1)$.
Por exemplo, $S4$ é apenas uma abreviação para o $(4 + 1)$,
e $SS4$ só pode denotar o $((4 + 1) + 1)$.%
\footnote{Percebeu que esse $\alpha$ aqui é uma metavariável?}
Açúcar sintáctico é muito usado em linguagens de programação,
para adular os programadores (que ganham assim um mecanismo
``doce'' para usar nos seus programas) sem mexer e complicar
a linguagem de verdade.
%%}}}

%%{{{ x: for_while_sugar 
\exercise.
\label{for_while_sugar}%
Mostre como um $\code{while}$ loop pode ser implementado como
açúcar sintáctico numa linguagem que tem $\code{for}$ loops,
e vice versa.

\endexercise
%%}}}

\endsection
%%}}}

%%{{{ Problems 
\problems.

%%{{{ decimal_numerals_BNF_problem
\problem.
\label{decimal_numerals_BNF_problem}%
Usando BNF, defina uma gramática para a linguagem de todos os
numerais que representam os naturais no sistema decimal.
Embuta-la na gramática das expressões aritméticas
para eliminar os ``$\dotsb$''.

\endproblem
%%}}}

%%{{{ remove_dots_from_ArExp_grammar 
\problem.
\label{remove_dots_from_ArExp_grammar}%
Com base a~\ref{ArExp_grammar_3} defina uma gramática que gera a mesma linguagem,
sem usar ``$\dotsb$''.

\hint
Já resolveu o~\ref{decimal_numerals_BNF_problem}?

\endproblem
%%}}}

%%{{{ ArExp_fact_grammar 
\problem.
\label{ArExp_with_factorial}%
Aumenta tua gramática do~\ref{remove_dots_from_ArExp_grammar} para
gerar expressões aritméticas que usam o operador unitário (e postfixo)
do factorial, que denotamos com $!$, escrevendo por exemplo
$8!$ para o factorial de $8$.
Note que não usamos parenteses para aplicar o factorial:
$$
((2 + 3!)! \ntimes 0!!)
$$

\endproblem
%%}}}

%%{{{ ArExp_fact_vars_grammar 
\problem.
\label{ArExp_fact_vars_grammar}%
Aumenta tua gramática do~\ref{ArExp_with_factorial} para
gerar expressões aritméticas que usam as variáveis
$$
x,y,z,
x',y',z',
x'',y'',z'',
x''',y''',z''',
\dotsc
$$

\endproblem
%%}}}

%%{{{ g: NatList_grammar 
\grammar.
\label{NatList_grammar}%
$$
\align
\bnf{L} &\bnfeq [\,] \bnfor (\bnf{Nat} :: \bnf{L})\\
\intertext{onde $\bnf{Nat}$ é o}
\bnf{Nat} &\bnfeq 0 \bnfor S\bnf{Nat}
\endalign
$$
da~\ref{Nat_grammar}.
%%}}}

%%{{{ NatList_grammar_problem 
\problem.
\label{NatList_grammar_problem}%
Escreva umas expressões geradas por a~\ref{NatList_grammar} e ache um possível
uso dessa gramática: o que podemos representar com essa linguagem?

\endproblem
%%}}}

\endproblems
%%}}}

%%{{{ Further reading 
\further.

\cite[Cap.~4]{alicebook}.
\cite[Cap.~2]{curryfoundations}.

\endfurther
%%}}}

\endchapter
%%}}}

%%{{{ chapter: The language of propositional logic 
\chapter A linguagem de lógica proposicional.

%%{{{ Propositions 
\section Proposições.

\note.
Considere as seguinte proposições:
\beginol
\li Se $x < y$ e $y < z$ então $z < x$.
\li O $\sqrt 2$ é irracional.
\li O $3$ é um múltiplo de $10$.
\li O $5$ é um divisor dos $25$ e $26$.
\li O $n$ é menor de $8$ ou maior de $4$.
\endol

\endsection
%%}}}

%%{{{ Its syntax 
\section Sua sintaxe.

\note Dados.
Um conjunto de símbolos
$$
\zolvars = \set{ P_0, P_1, P_2, \dotsc }
$$
chamados \dterm{variáveis proposicionais}.

%%{{{ df: propositional_formula 
\definition Fórmula.
\label{propositional_formula}%
\tdefined{fórmula}[proposicional]
\beginul
\li Se $P$ é uma variável proposicional, então $P$ é uma fórmula.
\li Se $A$ é uma fórmula, então $\lnot A$ é uma fórmula.
\li Se $A,B$ são fórmulas, então:
\item{-} $(A \limplies B)$ é uma fórmula;
\item{-} $(A \lor B)$ é uma fórmula;
\item{-} $(A \land B)$ é uma fórmula.
\endul
Nada mais é uma fórmula.
Uma fórmula que consiste em apenas uma variável proposicional é chamada
\dterm{fórmula atómica}.
%%}}}

%%{{{ g: zolang_grammar 
\grammar.
\label{zolang_grammar}%
Escrevemos
$$
\align
F &\bnfeq A \bnfor \lnot F \bnfor (F \limplies F) \bnfor (F \land F) \bnfor (F \lor F)\\
\intertext{onde $A$ denota as fórmulas atómicas da $\zolang$, ou seja as variáveis proposicionais:}
A &\bnfeq P_0 \bnfor P_1 \bnfor P_2 \bnfor \dotsb
\endalign
$$
%%}}}

\endsection
%%}}}

%%{{{ Its semantics 
\section Sua semântica.

\endsection
%%}}}

%%{{{ Syntactic sugar 
\section Açúcar sintáctico.

%%{{{ Notation: limplied_liff_abbr 
\note Notação.
\label{limplied_liff_abbr}%
Se $A,B$ são fórmulas, usamos as seguintes abreviações:
$$
\align
(A \liff B)     &\abbreq ((A \limplies B) \land (B \limplies A))\\
(A \limplied B) &\abbreq (B \limplies A)
\endalign
$$
%%}}}

%%{{{ x: second_character_of_limplied_abbr 
\exercise.
\label{second_character_of_limplied_abbr}%
Qual é o segundo caráter da expressão $(A \limplied B)$?

\hint
Lembre-se que aqui $A,B$ são apenas \emph{metavariáveis}
que denotam algumas fórmulas, sobre quais não sabemos nada mais
fora do fato que são fórmulas (bem formadas).

\solution
O primeiro caráter da fórmula $B$.

\endexercise
%%}}}

\endsection
%%}}}

%%{{{ Its limitations 
\section Suas limitações.

\endsection
%%}}}

%%{{{ Problems 
\problems.

\endproblems
%%}}}

%%{{{ Further reading 
\further.

\cite[Cap.~1]{velleman}.

\endfurther
%%}}}

\endchapter
%%}}}

%%{{{ chapter: The language of predicate logic 
\chapter A linguagem de lógica de predicados.

%%{{{ Its syntax 
\section Sua sintaxe.

%%{{{ givens_for_a_FOL 
\note Dados.
Um conjunto infinito de símbolos de variáveis
$$
\folvars  = \set{x_0, x_1, x_2, x_3, \dotsc}.
$$
Conjuntos de símbolos de constantes, de funções, e de predicados:
$$
\folcons, \folfuns, \folpreds.
$$
Uma função $\folarity : \folfuns \union \folpreds \to \nats$ de aridade,
que atribua para cada símbolo nesses conjuntos sua aridade.
\endgraf
Alternativamente, podemos grupear os símbolos de funções por aridade, e similarmente o símbolos de predicados, escrevendo assim
$\folfuns_n$, para o conjunto de símbolos de funções de aridade $n$, e similarmente $\folpreds_n$ para os símbolos de predicados.
Assim as duas afirmações são equivalentes:
$$
f \in \folfuns \mland \folarity(f) = n
\iff
f \in \folfuns_n.
$$
Note também que podemos considerar funções de aridade $0$ como constantes,
e assim nem precisamos um conjunto especial para esses símbolos.
%%}}}

%%{{{ eg: full_FOL_example 
\example.
Tomando
$$
\align
\folcons  &= \set{c_0, c_1, c_2, c_3, \dotsc}\\
\folfuns  &= \set{f_0, f_1, f_2, f_3, \dotsc}\\
\folpreds &= \set{P_0, P_1, P_2, P_3, \dotsc}
\endalign
$$
e especificando uma função de aridade $\folarity$ criamos uma FOL.
Alternativamente tomamos
$$
\xalignat2
\folcons =
\folfuns_0   &= \set{f^0_0, f^0_1, f^0_2, f^0_3, \dotsc} & \folpreds_0  &= \set{P^0_0, P^0_1, P^0_2, P^0_3, \dotsc}\\
\folfuns_1   &= \set{f^1_0, f^1_1, f^1_2, f^1_3, \dotsc} & \folpreds_1  &= \set{P^1_0, P^1_1, P^1_2, P^1_3, \dotsc}\\
\folfuns_2   &= \set{f^2_0, f^2_1, f^2_2, f^2_3, \dotsc} & \folpreds_2  &= \set{P^2_0, P^2_1, P^2_2, P^2_3, \dotsc}\\
             &\eqvdots                                   &              &\eqvdots                                    
\endxalignat
$$
onde temos denotado a aridade de cada símbolo como um ``exponente''.
\endexample
%%}}}

%%{{{ df: FOL_term 
\definition Termo.
\label{FOL_term}%
\tdefined{FOL}[termo]%
\beginul
\li Se $x \in \folvars$ então $x$ é um termo.
\li Se $c \in \folcons$ então $c$ é um termo.
\li Se $f \in \folfuns_n$ e $t_1,\dotsc,t_n$ são termos, então $f(t_1,\dotsc,t_n)$ é um termo.
\endul
\noindent Nada mais é um termo.
Denotamos o conjunto de termos com $\folterms$.
Resumindo esquematicamente a definição, temos:
$$
\align
x \in \folvars &\implies x \in \folterms\\
c \in \folcons &\implies c \in \folterms\\
\left.
\aligned
f &\in \folfuns_n\\
t_1,\dotsc,t_n &\in \folterms
\endaligned
\right\}
&\implies f(t_1,\dotsc,t_n) \in \folterms\\
\endalign
$$
%%}}}

%%{{{ df: FOL_atomic_formula 
\definition Fórmula atômica.
\label{FOL_atomic_formula}%
\tdefined{FOL}[fórmula atômica]%
\beginul
\li Se $P \in \folpreds_n$ e $t_1,\dotsc,t_n$ são termos, então $P(t_1,\dotsc,t_n)$ é uma \dterm{fórmula atómica}.
\endul
%%}}}

%%{{{ df: FOL_formula 
\definition Fórmula.
\label{FOL_formula}%
\tdefined{FOL}[fôrmula]%
\beginul
\li Se $A$ é uma fórmula atómica, então $A$ é uma fórmula.
\li Se $A$ é uma fórmula, então $\lnot A$ é uma fórmula.
\li Se $A,B$ são fórmulas, então:
\item{-} $(A \limplies B)$ é uma fórmula;
\item{-} $(A \lor B)$ é uma fórmula;
\item{-} $(A \land B)$ é uma fórmula.
\li Se $A$ é uma fórmula e $x$ é uma variável, então:
\item{-} $\forall x A$ é uma fórmula;
\item{-} $\exists x A$ é uma fórmula.
\endul
%%}}}

%%{{{ Practical abuse of BNF for FOL 
\grammar FOL.
\label{FOL_grammar}%
Seguindo uma prática comum, escrevemos apenas
$$
\align
F &\bnfeq
\mathit{A}
\bnfor (\lnot F)
\bnfor (F \limplies F)
\bnfor (F \land F)
\bnfor (F \lor F)
\bnfor \forall v F
\bnfor \exists v F
\endalign
$$
onde $A$ denota fórmulas atômicas e $v$ denota qualquer variável da nossa FOL.
%%}}}

%%{{{ x: BNF_for_FOL 
\exercise.
\label{BNF_for_FOL}%
Supondo que $\bnf{At}$ pode ser substituito por qualquer fórmula atômica duma FOL (e por nada mais),
defina umas gramáticas em BNF para gerar a linguagem das suas fórmulas bem formadas.

\solution
Aqui uma solução:
$$
\align
\bnf{F}     &\bnfeq \bnf{At} \bnfor (\lnot \bnf{F}) \bnfor (\bnf{F} \bnf{Bin} \bnf{F}) \bnfor \bnf{Q}\bnf{Var}\bnf{F}\\
\bnf{Bin}   &\bnfeq \lor \bnfor \land \bnfor \limplies\\
\bnf{Q}     &\bnfeq \forall \bnfor \exists\\
\bnf{Var}   &\bnfeq x_0 \bnfor x_1 \bnfor x_2 \bnfor \dotsb\\
\intertext{Como já encontramos, um fácil jeito para evitar os ``$\dotsb$'' seria usar:}
\bnf{Var}   &\bnfeq x \bnfor \bnf{Var}'
\endalign
$$

\endexercise
%%}}}

\endsection
%%}}}

%%{{{ Its semantics 
\section Sua semântica.

\endsection
%%}}}

%%{{{ Universes and structures 
\section Universos e mundos.

\endsection
%%}}}

%%{{{ Translations to and from FOL 
\section Traduzindo de e para FOL.
\label{FOL_translations}%

\endsection
%%}}}

%%{{{ Syntactic sugar 
\section Açúcar sintáctico.

%%{{{ x: exists_x_in_A_Px_sugar 
\exercise.
Ache uma fórmula de FOL equivalente da $\lexists {x \in A} {P(x)}$.

\solution
$$
\lexists {x \in A} {P(x)}
\iff
\lexists x \paren{ x\in A \land P(x) }
$$

\endexercise
%%}}}

%%{{{ x: forall_x_in_A_Px_sugar 
\exercise.
Ache uma fórmula de FOL equivalente da $\lforall {x \in A} {P(x)}$.

\solution
$$
\lforall {x \in A} {P(x)}
\iff
\forall x \paren{ x\in A \limplies P(x) }
$$

\endexercise
%%}}}

\endsection
%%}}}

%%{{{ Its limitations 
\section Suas limitações.

\endsection
%%}}}

%%{{{ Problems 
\problems.

\endproblems
%%}}}

%%{{{ Further reading 
\further.

\cite[Cap.~2]{velleman}.

\endfurther
%%}}}

\endchapter
%%}}}

%%{{{ chapter: Proof strategies 
\chapter Estratégias de provas.

%%{{{ Proofs as games 
\section Provas como jogos.

\endsection
%%}}}

%%{{{ Attacking the logical structure of a proposition 
\section Atacando a estrutura logical duma proposição.

\endsection
%%}}}

%%{{{ Reductio ad absurdum 
\section Reductio ad absurdum.

\endsection
%%}}}

%%{{{ Proofs by cases 
\section Provas por casos.

\endsection
%%}}}

%%{{{ Existence and uniqueness proofs 
\section Provas de existência e unicidade.

\endsection
%%}}}

%%{{{ Problems 
\problems.

\endproblems
%%}}}

%%{{{ Further reading 
\further.

\cite[Cap.~3]{velleman}

\endfurther
%%}}}

\endchapter
%%}}}

%%{{{ chapter: Recursion 
\chapter Recursão.

%%{{{ Defining functions 
\section Definindo funções.

\definition Os números Fibonacci.
\label{fibonacci}
\tdefined{Fibonacci}[seqüência]
Definimos os \emph{números Fibonacci}\Fibonacci[seqüência]\tdefined{Fibonacci}[números]{}
recursivamente assim:
$$
\align
    F_0     &= 0 \\
    F_1     &= 1 \\
    F_{n+2} &= F_{n+1} + F_n.
\endalign
$$

\note Computando seus valores.

\exercise Os números Lucas.
\label{lucas}%
\tdefined{Lucas}[números]%
Os números Lucas\Lucas[números]\tdefined{Lucas}[números]{} são definidos similarmente:
$$
\align
L_0     &= 2\\
L_1     &= 1\\
L_{n+2} &= L_{n+1} + L_n.
\endalign
$$
Calcule o valor $L_{12}$.

\endexercise

\endsection
%%}}}

%%{{{ The set of natural numbers formally 
\section O conjunto dos números naturais formalmente.
\label{Nats_formally}%

%%{{{ df: nats_plus_recursive_def 
\definition Adição.
Definimos a operação $+$ no $\nats$ pelas:
$$
\align
n + 0   &= n      \tag{a1}\\
n + S m &= S(n+m) \tag{a2}
\endalign
$$
%%}}}

%%{{{ eg: three_plus_two_formally 
\example.
\label{three_plus_two_formally}%
Calcule a soma $SSS0 + SS0$.

\solution
Temos a expressão
$$
SSS0 + SS0.
$$
Qual equação aplica?
Com certeza não podemos aplicar a primeira (na direção ``$\Rightarrow$''),
pois nossa expressão não tem a forma $n + 0$.
Por que não?  O primeiro termo na nossa expressão, o $SSS0$, não é um problema
pois ele pode ``casar'' com o $n$ do lado esquerdo da (a1).
Mas nosso segundo termo, o $SS0$,
não pode casar com o $0$, então a (a1) não é aplicável.
A segunda equação é sim, pois nossos termos podem casar assim
com as variáveis da (a2):
$$
\alignat2
{\munderbrace{SSS0}n} + {S\munderbrace{S0}m}\\
\intertext{Tomando $n\asseq SSS0$ e $m\asseq S0$ substituimos nossa expressão por seu
igual seguindo a (a2):}
{\munderbrace{SSS0}n} + {S\munderbrace{S0}m}
&= S\bigparen{ {\munderbrace{SSS0}n} + {\munderbrace{S0}m} }  \qqby{por (a2)}\\
\intertext{Depois um passo de cálculo então chegamos na expressão
$S(SSS0 + S0)$.
Como nenhuma equação tem a forma $S(\text{\thole}) = \text{\lthole}$,
olhamos ``dentro'' da nossa expressão para achar nas suas subexpressões
possíveis ``casamentos'' com nossas equações.
Focamos então na subexpressão sublinhada
$S(\underline{SSS0 + S0})$:
vamos tentar substituí-la por algo igual.
Novamente a primeira equação não é aplicavel
por causa do novo segundo termo ($S0$), mas a (a2) é:}
S\bigparen{{\munderbrace{SSS0}n} + {S\munderbrace{0}m}}\\
\intertext{Tomando agora $n\asseq SSS0$ e $m\asseq 0$ substituimos de novo
seguindo a (a2):}
S\toverbrace{\bigparen{{\munderbrace{SSS0}n} + {S\munderbrace{0}m}}}{isso}
&= S\toverbrace{S\bigparen{ {\munderbrace{SSS0}n} + {\munderbrace{0}m} }}{por isso}  \qqby{por (a2)}\\
\intertext{Agora focamos na subexpressão $SS(\underline{SSS0 + 0})$ e podemos finalmente
aplicar a primeira equação:}
SS\bigparen{{\munderbrace{SSS0}n} + 0}\\
\intertext{então tomando $n\asseq SSS0$ substituimos}
SS\toverbrace{\bigparen{{\munderbrace{SSS0}n} + 0}}{isso}
&= SS\toverbrace{ {\munderbrace{SSS0}n} }{por isso}  \qqby{por (a1)}
\endalignat
$$
Finalmente chegamos no resultado: no termo $SSSSS0$.
Nunca mais vamos escrever tudo isso com tanto detalhe!
Esse cálculo que acabamos de fazer, escrevemos curtamente nessa forma:
$$
\alignat2
SSS0 + SS0
&= S(SSS0 + S0) \qqby {por (a2)}\\
&= SS(SSS0 + 0) \qqby {por (a2)}\\
&= SSSSS0       \qqby {por (a1)}
\endalignat
$$
escrevendo apenas em cada linha o que foi usado.

\endexample
%%}}}

%%{{{ x: zero_plus_four_formally 
\exercise.
\label{zero_plus_four_formally}%
Calcule a soma dos $0 + SSSS0$.

\endexercise
%%}}}

%%{{{ Evaluation strategy 
\note Estratégias de evaluação.
\tdefined{estratégia}[de evaluação]%
Vamos dizer que queremos calcular começando com uma expressão mais complexa,
como por exemplo a
$$
0 + \bigparen{0 + S\bigparen{(SS0 + 0) + 0}}.
$$
Como procedimos?
A expressão enteira não pode ser substituida pois nenhuma das (a1)--(a2) tem
essa forma, mas aparecem várias subexpressões em quais podemos \emph{focar}
para nosso próximo passo de cálculo:
$$
\align
0 + \underline{\bigparen{0 + S\bigparen{(SS0 + 0) + 0}}},&\quad\text{casando com (a2)}\\
0 + \bigparen{0 + S\underline{\bigparen{(SS0 + 0) + 0}}},&\quad\text{casando com (a1)}\\
0 + \bigparen{0 + S\bigparen{\underline{(SS0 + 0)} + 0}},&\quad\text{casando com (a1)}.
\endalign
$$
Podemos seguir uma \dterm{estratégia de evaluação} específica, por exemplo,
focando sempre na expressão que aparece primeira à esquerda; ou podemos
escolher cada vez onde focar aleatoriamente; etc.
No~\ref{three_plus_two_plus_one} tu vai ter que escolher onde focar
várias vezes.
%%}}}

%%{{{ x: three_plus_two_plus_one 
\exercise.
\label{three_plus_two_plus_one}%
Calcule os valor das expressões $SSS0 + (SS0 + S0)$ e $(SSS0 + SS0) + S0$.

\solution
Um caminho para calcular a primeira é o seguinte:
$$
\alignat2
SSS0 + \underline{(SS0 + S0)}
&= SSS0 + S\underline{(SS0 + 0)} \qqby{por (a2)}\\
&= \underline{SSS0 + SSS0}       \qqby{por (a1)}\\
&= S\underline{(SSS0 + SS0)}     \qqby{por (a2)}\\
&= SS\underline{(SSS0 + S0)}     \qqby{por (a2)}\\
&= SSS\underline{(SSS0 + 0)}     \qqby{por (a2)}\\
&= SSSSSS0                       \qqby{por (a1)}\\
\intertext{e um caminho para calcular a segunda é o:}
\underline{(SSS0 + SS0) + S0}
&= S(\underline{(SSS0 + SS0)} + 0)  \qqby{por (a2)}\\
&= S(S\underline{(SSS0 + S0)} + 0)  \qqby{por (a2)}\\
&= S\underline{(SS(SSS0 + 0) + 0)}  \qqby{por (a2)}\\
&= SSS\underline{(SSS0 + 0)}        \qqby{por (a1)}\\
&= SSSSSS0                          \qqby{por (a1)}
\endalignat
$$

\endexercise
%%}}}

%%{{{ x: nats_exp_recursive_def 
\exercise.
\label{nats_double_def}%
Define (recursivamente) a função $d : \Nat \to \Nat$ que dobra sua entrada.
Verifique que o dobro de $3$ é $6$.

\hint
Pensando fora do $\Nat$:
como podemos calcular o valor de $2(n+1)$, se sabemos como dobrar
qualquer número menor de $n+1$?

\hint
$2(n+1) = 2n + 1$.

\solution
Definimos:
$$
\align
d( 0 )  &= 0        \tag{D1}\\
d( Sn ) &= SSd(n).  \tag{D2}
\endalign
$$
Calculamos:
$$
\alignat2
d( SSS0 )
&= SSd(SS0)    \qqby{por (D2)}\\
&= SSSSd(S0)   \qqby{por (D2)}\\
&= SSSSSSd(0)  \qqby{por (D2)}\\
&= SSSSSS0.    \qqby{por (D1)}
\endalignat
$$

\endexercise
%%}}}

%%{{{ x: nats_ntimes_recursive_def 
\exercise.
\label{nats_ntimes_recursive_def}%
Define a multiplicação no $\nats$.

\hint
Precisa de novo duas equações.

\hint
$$
\align
n \ntimes 0  &= \text{\lthole}\\
n \ntimes Sm &= \text{\lthole}
\endalign
$$

\hint
Na segunda equação, no seu lado direito, tu tem acesso no valor
$n \ntimes m$, pois é ``mais simples'' do que o $n \ntimes Sm$.
Isso é o poder da recursão: podes considerar o problema que tu
tá tentando resolver (definir a multplicação), como resolvido
para as ``entradas mais simples''.

\endexercise
%%}}}

%%{{{ x: nats_exp_recursive_def 
\exercise.
\label{nats_exp_recursive_def}%
Define a exponencação no $\nats$.

\endexercise
%%}}}

%%{{{ nats_leq_recursive_def 
\definition Ordem.
Com que temos definido podemos já definir recursivamente
uma \emph{relação} no $\nats$.  A relação de ordem $\leq$:
$$
\align
0  \leq m  &\iff \True      \tag{LE1}\\
Sn \leq 0  &\iff \False     \tag{LE2}\\
Sn \leq Sm &\iff n \leq m   \tag{LE3}
\endalign
$$
%%}}}

%%{{{ eg: two_leq_four_but_four_notleq_two 
\example.
\label{two_leq_four_but_four_notleq_two}%
Ache se $SS0 \leq SSSS0$ e se $SSSS0 \leq SS0$.

\solution
Calculamos:
$$
\alignat2
SS0 \leq SSSS0
&\iff S0 \leq SSS0  \qqby{por (LE3)}\\
&\iff 0  \leq SS0   \qqby{por (LE3)}\\
&\iff \True         \qqby{por (LE1)}
\endalignat
$$
ou seja, realmente $SS0 \leq SSSS0$.
No outro lado, calculamos
$$
\alignat2
SSSS0 \leq SS0
&\iff SSS0 \leq S0  \qqby{por (LE3)}\\
&\iff SS0  \leq 0   \qqby{por (LE3)}\\
&\iff \False        \qqby{por (LE2)}
\endalignat
$$
ou seja, $SSSS0 \not\leq SS0$.

\endexample
%%}}}

\endsection
%%}}}

%%{{{ Sums and products, formally 
\section Somatórios e produtórios formalmente.

\endsection
%%}}}

%%{{{ Recursing over formulas 
\section Recursando nas fórmulas.

%%{{{ eg: bincon_count 
\example.
\label{bincon_count}%
Defina uma função $f : \zolang \to \nats$ que calcula o número
de conectivos binários que aparecem na sua entrada.
Use-lá para calcular os conectivos binários da expressão
$$
\lnot(P_4 \limplies (P_9 \land \lnot P_9)).
$$

\solution
Seguindo a definição de $\zolang$, cada um dos seus elementos
é formado por uma de certas regras.  Basta escrever então como
calcular o número desejado para cada um desses casos:
$$
\alignat2
f(p)                &= 0,          &\quad&\text{($p\in \mathrm{Pvar}$)} \tag{BC$_{P}$}\\
f(\lnot A)          &= f(A),            &&\text{($A\in \zolang$)}       \tag{BC$_{\lnot}$}\\
f((A \limplies B))  &= f(A) + 1 + f(B), &&\text{($A,B\in \zolang$)}     \tag{BC$_{\limplies}$}\\
f((A \land B))      &= f(A) + 1 + f(B), &&\text{($A,B\in \zolang$)}     \tag{BC$_{\land}$}\\
f((A \lor B))       &= f(A) + 1 + f(B), &&\text{($A,B\in \zolang$)}     \tag{BC$_{\lor}$}\\
\intertext{Preguiçosamente, podemos condensar as três últimas equações em úma só, assim:}
f((A \heartsuit B)) &= f(A) + 1 + f(B), &&\text{($A,B\in \zolang$, e $\heartsuit \in \set{\limplies,\land,\lor}$)}
\endalignat
$$
Aplicando nossa função na fórmula dada calculamos:
$$
\alignat2
f( \lnot(P_4 \limplies (P_9 \land \lnot P_9)) )
&= f( (P_4 \limplies (P_9 \land \lnot P_9)) )   \qqby{por BC$_{\lnot}$}\\
&= f( P_4 ) + 1 + f( (P_9 \land \lnot P_9) )    \qqby{por BC$_{\limplies}$}\\
&= 0 + 1 + f( (P_9 \land \lnot P_9) )           \qqby{por BC$_{P}$}\\
&= 0 + 1 + ( f( P_9 ) + 1 + f ( \lnot P_9 ) )   \qqby{por BC$_{\land}$}\\
&= 0 + 1 + ( 0 + 1 + f ( \lnot P_9 ) )          \qqby{por BC$_{P}$}\\
&= 0 + 1 + ( 0 + 1 + f ( P_9 ) )                \qqby{por BC$_{\lnot}$}\\
&= 0 + 1 + ( 0 + 1 + 0 )                        \qqby{por BC$_{P}$}\\
&= 2
\endalignat
$$

\endexample
%%}}}

\endsection
%%}}}

%%{{{ Problems 
\problems.

%%{{{ prob: quot_rem_eq_prob 
\problem.
\label{quot_rem_eq_prob}
Considere as funções $q$, $r$, e $t$ definidas recursivamente:
$$
\xalignat3
q       &\eqtype \nats \to \nats   & r      &\eqtype \nats \to \nats   &  t          &\eqtype \nats^2 \to \bools     \\
q(0)    &=       0\phantom{666}    & r(0)   &=       0\phantom{666}    &  t(0,   0)  &=       \True          \\
q(1)    &=       0\phantom{666}    & r(1)   &=       1\phantom{666}    &  t(0,  Sn)  &=       \False         \\
q(2)    &=       0\phantom{666}    & r(2)   &=       2\phantom{666}    &  t(Sm,  0)  &=       \False         \\
q(n+3)  &=       q(n) + 1          & r(n+3) &=       r(n)              &  t(Sm, Sn)  &=       t(m,n)             
\endxalignat
$$
O que cada função calcula?

\hint
Calcule os valores $q(11)$, $q(12)$, $r(11)$, e $r(12)$.

\solution
\item{($q$:)}
    $q(n) = \bigg\lfloor{\dfrac n 3}\bigg\rfloor$.
\item{($r$:)}
    O $r(n)$ é o resto da divisão do $n$ por $3$.
\item{($t$:)}
    A $t$ decida se suas entradas são iguais ou não.

\endproblem
%%}}}

\problem.
\label{products_in_disguise}
Seja $h : \nats\to\nats$.
Defina recursivamente as funções $t : \nats \to \nats$ e $T : \nats^2 \to \nats$ que satisfazem:
$$
\align
t(n)   &= h(0)h(1)\dotsb h(n-1)     = \prod\limits_{i=0}^{n-1} h(i);\\
T(m,n) &= h(m)h(m+1)\dotsb h(m+n-1) = \prod\limits_{i=m}^{m+n-1} h(i).
\endalign
$$

\hint
Depois de definir confira tuas definições seguindo elas para calcular uns valores.
Por exemplo, $t(2)$ e $T(5,2)$ devem dar os resultados
$$
\xalignat2
t(2) &= h(0)h(1)
&
T(5,2) &= h(5)h(6).
\endxalignat
$$

\hint
Mesmo que a função $T$ tem aridade 2, escolhendo bem,
tu não precisarás escrever 4 equações, mas apenas 2.

\solution
Definimos
$$
\xalignat2
t      &\eqtype \nats\to\nats   &  T        &\eqtype \nats^2\to\reals\\
t(0)   &= 1                     &  T(m,0)   &= 1                     \\
t(n+1) &= h(n) \ntimes t(n)     &  T(m,k+1) &= h(m+k) \ntimes T(m,k).  
\endxalignat
$$
Tendo definido primeiro a $T$, podemos definir a $t$ assim:
$$
t(n) = T(0,n).
$$

\endproblem

\problem.
Tentando resolver o~\ref{products_in_disguise},
% STUDENT: Victor
um aluno definiu corretamente a $t$ e depois a usou
na sua definição de $T$, assim:
$$
T(m,n) = {t(m+n)}/{t(m)}.
$$
Qual o problema com essa definição?
(Assuma como conhecida uma definição recursiva da
operação $/$ de divisão inteira.)

\hint
Qual é o contradomínio da $h$?

\hint
$0\in\nats$.

\hint
O que acontece se $h(i) = 0$ para algum $i \in \nats$?

\solution
Se $h(i) = 0$ para algum $i \in \nats$,
o $t(j)=0$ para todo $j>i$.
Assim, a expressão
${t(m+n)}/{t(m)}$ não é definida para qualquer $m>i$.
Observe que a solução seria certa se
o contradomínio da $h$ fosse o $\nats\setminus\set0$.

\endproblem

\endproblems
%%}}}

%%{{{ Further reading 
\further.

\endfurther
%%}}}

\endchapter
%%}}}

%%{{{ chapter: Induction 
\chapter Indução.

%%{{{ The principle of finite induction 
\section O princípio da indução finita (PIF).

\axiom Princípio da indução finita (PIF).
\tdefined{PIF}
\iiseealso{PIF}{indução}
\iisee{princípio}[da indução finita]{indução}
\tdefined{indução}
\label{FIP}
Seja $P(n)$ uma propriedade de naturais.
Se $P(0)$ e, para todo $k\in\nats$,
$P(k)$ implica $P(k+1)$,
então $P(n)$ é verdade para todo $n\in\nats$.

\note.
Esquematicamente o PIF:
$$
\left.
\aligned
\text{\casestyle{Base:}}\quad            & P(0)\\
\text{\casestyle{Passo Indutivo:}}\quad  & \lforall {k\in\nats} {P(k)\limplies P(k+1)}\quad
\endaligned
\right\}
\implies
\lforall {n\in\nats} {P(n)}.
$$

\example.
\label{gauss_child}
Prove que
$$
    1 + 2 + \cdots + n = \frac { n (n+1) } 2
$$
para todo $n\in\nats$.

\solution
Por indução, verificamos que a propriedade
$$
P(n)
\defiff
1 + 2 + \cdots + n = \frac { n (n+1) } 2
$$
é válida para todo $n$.
Primeiramente provamos a base
$$
P(0)
\iff
\underbrace{1 + 2 + \cdots + 0}_{\text{somatório vazio}}
= \frac { 0 (0+1) } { 2 }.
$$
Realmente, o lado esquerdo é um somatório vazio,
logo $0$, e o lado direito é o número $0(0+1)/2 = 0$ também.
Concluimos que $P(0)$.
\endgraf
Seja $k\in\nats$ tal que $P(k)$, ou seja,
$$
1 + 2 + \cdots + k = \frac { k (k+1) } 2.\tag{H.I.}
$$
Calculamos:
$$
\alignat 3
P(k+1)
&\iff& 1 + 2 + \cdots + (k+1)        &= \frac { (k+1) (k+2) } 2 \\
&\iff& 1 + 2 + \cdots + k + (k+1)    &= \frac { (k+1) (k+2) } 2 \\
&\iff& \frac { k (k+1) } 2  + (k+1)  &= \frac { (k+1) (k+2) } 2 \qqby{pela H.I.} \\
&\iff& k \cancel{(k+1)} + 2 \cancel{(k+1)} &= \cancel{(k+1)} (k+2) \\
&\iff& k + 2                         &= k + 2.
\endalignat
$$

\endexample

\exercise.
Prove que
$$
    1 + 8 + 27 + \cdots + n^3 = \paren{\frac { n (n+1) } { 2 } }^2
$$
para todo $n\in\nats$.

\hint
Lembre que o somatório vazio é $0$.

\endexercise

\exercise.
Observando os valores de:
$$
\align
    1
    +
    \frac 1 2
    &= 2 - \frac 1 2\\
    1
    +
    \frac 1 2
    +
    \frac 1 4
    &= 2 - \frac 1 4\\
    1
    +
    \frac 1 2
    +
    \frac 1 4
    +
    \frac 1 8
    &= 2 - \frac 1 8,
\intertext{adivinhe uma fórmula geral para o somatório}
    1 + \frac 1 2 + \frac 1 4 + \cdots + \frac 1 {2^n} &= \text{?}
\endalign
$$
e prove que ela é válida para todo $n\in\nats$.

\endexercise

\exercise.
Calculando os valores de:
$$
    \paren{1 - \frac 1 2},
   \qquad 
    \paren{1 - \frac 1 2}
    \paren{1 - \frac 1 3},
   \qquad
    \paren{1 - \frac 1 2}
    \paren{1 - \frac 1 3}
    \paren{1 - \frac 1 4},
$$
adivinhe uma fórmula geral para o produtório
$$
\prod_{i=2}^n\paren{ 1 - \frac 1 i}
=
    \paren{1 - \frac 1 2}
    \paren{1 - \frac 1 3}
    \paren{1 - \frac 1 4}
\dotsb
    \paren{1 - \frac 1 n}
$$
e prove que ela é válida para todo inteiro $n \geq 2$.

\endexercise

\exercise.
Prove que para todo $n \in\nats$,
$$
\sum_{i=1}^n i^2
= \frac {n^3} 3 + \frac {n^2} 2 + \frac n 6.
$$

\endexercise

\exercise.
Prove que para todo $n \in\nats$,
$$
\align
\sum_{i=1}^n i^3 &= \paren{\sum_{i=1}^n i}^2\\
\text{ou seja,}\qquad
1^3 + 2^3 + \dotsb + n^3 &= \paren{ 1 + 2 + 3 + \dotsb + n }^2.
\endalign
$$

\endexercise

\exercise.
\label{sum_of_threes_and_fives}
Qualquer número inteiro positivo $n \geq 8$ pode ser escrito
como somatório de $3$'s e $5$'s.

\endexercise

\exercise.
Seja
$$
P(n) \defiff 1 + 2 + \dotsb + n = \frac 1 8 \paren{2n + 1}^2.
$$
\item{(1)} Prove que para todo $k\in\nats$, se $P(k)$ então $P(k+1)$.
\item{(2)} Critique a sentença: ``\emph{Logo, por indução temos que $P(n)$ é válido para todo $n\in\nats$.}''
\item{(3)} Mudando o ``$=$'' para ``$>$'' ou ``$<$'', defina um novo predicado $P'(n)$, válido para todo $n\in\nats$, e prove a validade dele por indução.

\endexercise

\exercise.
Prove que para todo $n\in\nats$,
$$
\sum_{i=0}^n F_i = F_{n+2} - 1,
$$
onde $F_n$ o $n$-ésimo número Fibonacci.

\endexercise

\exercise.
Prove que para todo $n\in\nats$,
$$
\sum_{i=1}^n F_i^2 = F_n F_{n+1},
$$
onde $F_n$ o $n$-ésimo número Fibonacci.

\endexercise

\exercise.
Prove que para todo inteiro $n \geq 1$.
$$
    \frac 1 {1 \cdot 2} +
    \frac 1 {2 \cdot 3} +
    \frac 1 {3 \cdot 4} + \cdots + 
    \frac 1 {n (n+1)}
    =
    \frac n {n+1}
$$

\endexercise

\exercise.
Prove que para todo inteiro $n \geq 5$,
$$
    n^2 < 2^n.
$$

\endexercise

\endsection

%%}}}

%%{{{ When one base is not enough 
\section Quando uma base não é suficiente.

\proposition.
\label{lucas_altdef_first_attempt}
Para todo inteiro $n \geq 1$,
seja $\ell : \nats\setminus\set0\to\nats$ a função definida pela equação
$$
    \ell(n) = F_{n-1} + F_{n+1},
$$
onde $F_n$ é o $n$-ésimo número \Fibonacci{}Fibonacci (veja~\ref{fibonacci}).
Queremos mostrar que para todo $n \geq 1$, $L_n = \ell(n)$,
onde $L_n$ é o $n$-ésimo número \Lucas{}Lucas (veja~\ref{lucas}).
\wrongproof.
Nos vamos provar por indução que \emph{para todo $n \geq 1$, $L_n = \ell(n)$}.
Vamos primeiramente verificar que para $n=1$, realmente temos $L_n = \ell(n)$:
$$
\alignat 2
\ell(1) &= F_0 + F_2      \qqby{def.~de $\ell(n)$} \\
        &= 0 + F_1 + F_0  \qqby{def.~de $F_n$}     \\
        &= 0 + 1 + 0      \qqby{def.~de $F_n$}     \\
        &= 1              \\
        &= L_1.           \qqby{def.~de $L_n$}
\intertext{Seja $k\in\nats$ com $k\geq 2$, tal que $L_{k-1} = \ell(k-1)$.
Realmente temos}
L_k
&= L_{k-1} + L_{k-2}                        \qqby{def.~de $L_n$}\\
&= \ell(k-1) + \ell(k-2)                    \qqby{H.I.}\\
&= (F_{k-2} + F_k) + (F_{k-3} + F_{k-1})    \qqby{def.~de $\ell(n)$}\\
&= (F_{k-2} + F_{k-3}) + (F_k + F_{k-1})    \qqby{ass.~e com.~de $+$}\\
&= F_{k-1} + F_{k+1}                        \qqby{def.~de $F_n$}\\
&= \ell(k)                                  \qqby{def.~de $\ell(n)$}.
\endalignat
$$
que termina nossa prova.
\mistaqed

\exercise.
\label{lucas_altdef_find_error}
Na prova em cima roubamos.
Ache onde e explique como, e pense numa solução.

\endexercise

\proposition.
\label{lucas_altdef_final}
Com a notação da~\ref{lucas_altdef_first_attempt},
para todo $n\geq 1$, $\ell(n) = L_n$.
\proof.
Nos vamos provar por indução que \emph{para todo $n \geq 1$, $L_n = \ell(n)$}.
Vamos primeiramente verificar que para $n=1$ e $n=2$, realmente temos $L_n = \ell(n)$.
Para $n=1$:
$$
\alignat 2
\ell(1) &= F_0 + F_2      \qqby{def.~de $\ell(n)$} \\
        &= 0 + F_1 + F_0  \qqby{def.~de $F_n$}     \\
        &= 0 + 1 + 0      \qqby{def.~de $F_n$}     \\
        &= 1              \\
        &= L_1.           \qqby{def.~de $L_n$}
\endalignat
$$
E para $n=2$:
$$
\xalignat2
\alignedat 2
      L_2 &= L_1 + L_0  \qqby{def.~de $L_n$} \\
          &= 1 + 2      \qqby{def.~de $L_n$} \\
          &= 3          
\endalignedat
&&
\alignedat 2
\ell(2) &= F_1 + F_3          \qqby{def.~de $\ell(n)$}  \\
        &= 1 + F_2 + F_1      \qqby{def.~de $F_n$}      \\
        &= 1 + 1 + 1          \\
        &= 3.
\endalignedat
&
\endxalignat
$$
Seja $k\in\nats$ com $k\geq 3$ tal que
$$
L_{k-1} = \ell(k-1)
\qquad
\text{e}
\qquad
L_{k-2} = \ell(k-2)
$$
(nossas \emph{duas} hipoteses indutivas).
Vamos provar que $L_k = \ell(k)$.
Realmente temos
$$
\alignat 2
L_k
&= L_{k-1} + L_{k-2}                        \qqby{def.~de $L_n$}\\
&= \ell(k-1) + \ell(k-2)                    \qqby{H.I.}\\
&= (F_{k-2} + F_k) + (F_{k-3} + F_{k-1})    \qqby{def.~de $\ell(n)$, $k\geq3$}\\
&= (F_{k-2} + F_{k-3}) + (F_k + F_{k-1})    \qqby{ass.~e com.~de $+$}\\
&= F_{k-1} + F_{k+1}                        \qqby{def.~de $F_n$, $k\geq3$}\\
&= \ell(k)                                  \qqby{def.~de $\ell(n)$},
\endalignat
$$
que termina nossa prova.
\qed

\exercise.
Ache uma nova prova do~\ref{sum_of_threes_and_fives} por indução
com três bases.

\hint
$k = (k-3) + 3$.

\solution
Seja $k\geq 8 + 3 = 11$ tal que $k-1$, $k-2$, e $k-3$
podem ser escritos como somatórios de $3$'s e $5$'s (H.I.).
Temos
$$
\alignat 2
k &= (k-3) + 3      \\
  &= (3x + 5y) + 3  \quad\text{para alguns $x,y\in\nats$} \qqby{pela H.I.} \\
  &= 3(x+1) + 5y.
\endalignat
$$
Como precisamos a vericidade da proposição para o valor $k-3$,
devemos mostrar as $3$ bases, para os inteiros $8$, $9$, e $10$:
$$
\alignat 2
8  &= 3 + 5     &&= 3\ntimes 1 + 5\ntimes 1\\
9  &= 3 + 3 + 3 &&= 3\ntimes 3 + 5\ntimes 0\\
10 &= 5 + 5     &&= 3\ntimes 0 + 5\ntimes 2.
\endalignat
$$

\endexercise

\endsection
%%}}}

%%{{{ Many variables 
\section Muitas variáveis.

%TODO explain that it's just one more tool to attack theorems
%TODO XXX Maybe refer to a weaker result from "irrationality"?
\lemma.
\label{odd_to_any_power_is_odd}
Para todo $n\in\nats$, e todo impar $k\in\ints$, $k^n$ é impar.
\proof.
Seja $k\in\ints$ impar, então $k=2a+1$ para um $a\in\ints$.%
\footnote{%
Aqui consideramos a seguinte definição de ``impar'':
\emph{um inteiro $n$ é \dterm{impar} sse existe inteiro $k$ tal que $n = 2k+1$.}
}
Vamos provar por indução que para todo $n\in\nats$, $k^n$ é impar.
Se $n=0$, imediatamente $k^0 = 1$ e é impar ($1 = 2\ntimes 0 + 1$).
Suponha que para algum $t\in\nats$, $k^t$ é impar, ou seja $k^t = 2b+1$ para um $b\in\ints$.
Falta provar que $k^{t+1}$ também é impar.
Calculando,
$$
\alignat2
k^{t+1}
&= k k^t               \qqby{definição de exponenciação} \\
&= (2a + 1) (2b + 1)   \qqby{hipoteses}    \\
&= 4ab + 2a + 2b + 1   \\
&= 2(2ab + a + b) + 1.
\endalignat
$$
Logo, como $2ab + a + b\in\ints$, $k^{t+1}$ é impar.
\qed

\endsection
%%}}}

%%{{{ Strong induction 
\section O princípio da indução finita forte (PIFF).
\label{Strong_induction}

\note.
O seguinte princípio parece mais forte que o PIF.
Na verdade, nos naturais, os dois princípios são equivalentes:

\theorem Princípio da indução finita forte (PIFF).
\tdefined{PIFF}
\iiseealso{PIFF}{indução}
\tdefined{indução}[forte]
\iisee{princípio}[da indução finita forte]{indução forte}
\label{FSIP}
Seja $P(n)$ uma propriedade de naturais.
Se para todo $k\in\nats$,
a hipótese que $P(i)$ é verdade para todo $i<k$
implica que $P(k)$ também é,
então $P(n)$ é verdade para todo $n\in\nats$.

\note.
Esquematicamente o PIFF:
$$
\lforall
    {k\in\nats}
    {\vphantom{\bigg(}\Big(\lforall {i < k} {P(i)}\Big) \limplies P(k)}
\implies
\lforall
    {n\in\nats}
    {P(n)}.
$$

\endsection
%%}}}

%%{{{ The well-ordering principle 
\section O princípio da boa ordem (PBO).
\label{WOP_principle}%

\definition Elemento mínimo.
\tdefined{mínimo}%
\sdefined {\min {\holed A}} {mínimo do \holed A}%
Seja $A$ um conjunto ordenado.
Um elemento $m\in A$ é chamado
\dterm{(elemento) mínimo do $A$}
sse
$m$ é menor de todos os outros elementos do $A$.
Quando o conjunto $A$ possui elemento mínimo,
escrevemos $\min A$ para o denotar:
$$
m = \dsym{\min A} \defiff m\in A \;\land\; \lforall {a\in A} {a \neq m \limplies m < a}.
$$

Note que necessariamente o mínimo dum conjunto $A$ pertence no $A$.
Para justificar a definição do símbolo $\min A$, \emph{devemos} provar o seguinte:

\exercise Unicidade de mínimo.
Um conjunto não pode ter mais que um elemento mínimo.

\hint
Suponha que tem dois mínima diferentes
e aplicando a definição de ``mínimo''
e propriedades de $<$,
chega numa contradição.

\hint
Pela transitividade, $a<b$ e $b<a$ implicam uma contradição.

\endexercise

\exercise.
Seja $U$ um conjunto unitário ordenado.
Prove que ele tem elemento mínimo.

\endexercise

\exercise.
Seja $A$ um conjunto ordenado, e um subconjunto finito $A_0\finsubseteq A$,
com $A_0\neq\emptyset$.
O $A_0$ possui elemento mínimo.

\hint
Indução no tamanho do $A_0$.

\endexercise

\theorem Princípio da boa ordem (PBO).
\tdefined{PBO}%
\tdefined{princípio}[da boa ordem]%
\label{WOP}%
\iiseealso{PBO}{princípio da boa ordem}%
Cada subconjunto não vazio do\/ $\nats$ possui mínimo.
\sketch.
Seja $P(n)$ a propriedade que cada subconjunto $A\subseteq\nats$
que possui elementos menores ou iguais a $n$ tem mínimo.
Vamos mostrar por indução que para todo $n$, $P(n)$.
\qes

\exercise.
Quais dos $\ints$, $\rats$, $\rats_{>0}$, $\rats_{\geq0}$, $\reals$,
satisfázem a propriedade de boa ordem?

\endexercise

\exercise.
Seja $a\in\reals$.
Quais dos $\rats_{\geq a}$, $\reals_{\geq a}$,
e $\set{ 2^{-n} \st n \in\nats, n \leq a}$
satisfázem a propriedade de boa ordem?

\endexercise

\theorem.
\label{no_int_between_0_and_1}%
Não existe inteiro $k$ tal que\/ $0 < k < 1$.
\sketch.
Suponha que existe um tal inteiro $k$.
Então o conjunto $C=\set{c\in\ints\st 0 < c < 1}$ de todos os ``contraexemplos''
não é vazio.
Aplique o PBO para tomar seu menor elemento, e ache um outro contraexemplo,
ainda menor, chegando assim numa contradição.
\qes

\exercise.
Prove o \ref{no_int_between_0_and_1} usando indução.

\hint
Nenhum inteiro negativo satisfaz $0 < m < 1$.
Então para provar que nenhum inteiro fica estritamente entre $0$ e $1$, basta provar
que todos os não-negativos $n$ satisfazem $n=0$ ou $n\geq 1$

\hint
Seja
$
P(n) \defiff \text{$n=0$ ou $n\geq1$}
$.

\hint
A base é trivial.

\hint
Para o passo indutivo, tomando um $k\in\nats$ tal que $P(k)$, separa tua prova
em dois casos, dependendo da razão que o $P(k)$ seja verdade.

\endexercise

\exercise.
Sejam $a,b\in\ints$ com $ab=1$.
Prove que $a=\pm1$ e $b=\pm1$.

\endexercise

\theorem Indução, forma com conjuntos.
\tdefined{indução}[forma com conjuntos]
\label{PIF_set_form}
Seja $A\subseteq\nats$ tal que $0\in A$ e $n+1\in A$ para todo $n\in A$.
$A=\nats$.
\sketch.
Caso contrario, existeriam ``contraexemplos'', ou seja, naturais que não pertencem no $A$.
Aplique a PBO para escolher o menor tal contraexemplo, e chega num absurdo.
\qes

\exercise.
Prove que se um subconjunto $A$ de inteiros tem elementos maiores que um número fixo $n_0$,
então existe o $\min\set{a\in A\st n_0 < a}$.

\endexercise

\endsection
%%}}}

%%{{{ Problems 
\problems.

%%{{{ prob: every_finite_set_of_reals_has_min_and_max 
\problem.
\label{every_finite_set_of_reals_has_min_and_max}%
Cada conjunto finito e não vazio $A\subseteq\reals$ possui
elemento mínimo $\min A$ e máximo $\max A$.

\hint
Indução no número de elementos do $A$.

\endproblem
%%}}}

%%{{{ prob: triminos 
\problem Triminôs.
\label{triminos}%

\endproblem
%%}}}

%%{{{ prob: induction_iff_strong_induction 
\problem.
\label{induction_iff_strong_induction}%
\ii{PIF}%
\ii{PIFF}%
$\text{PIF} \iff \text{PIFF}$.

\endproblem
%%}}}

%%{{{ prob: where_is_the_base_of_strong_induction 
\problem Cadê a base da indução forte?.
\label{where_is_the_base_of_strong_induction}%
Seguindo o teorema em cima, parece que não precisamos provar uma ``base''
na indução forte.
Critique a seguinte afirmação:
``\emph{Quando quero provar um teorema da forma
$\forall n P(n)$
usando indução, eu preciso mostrar uma(s) base(s)
$P(0), P(1), \dotsc, P(b)$
e depois provar o $P(k+1)$,
dado apenas umas $b+1$ hipóteses:
$P(k), P(k-1), \dotsc, P(k-b)$.
No outro lado, usando indução forte eu preciso mostrar menos
coisas---não tenho que mostrar nenhuma base---,
e, alem disso, no meu esforço para provar o $P(k+1)$,
eu não vou ter apenas umas poucas hipóteses, mas
todos os $P(i)$ para $i<k+1$!
Como os dois princípios são equivalentes no $\nats$,
eu vou sempre usar indução forte.}''

\hint Quantos $i\in\nats$ satisfazem $i < 0$?

\endproblem
%%}}}

%%{{{ prob: ackermann_function 
\problem.
\label{ackermann_function}%
Considere a função recursiva $\alpha : \nats^2 \to \nats$ definida pelas equações:
$$
\align
\alpha(0,x)     &= x+1                      \tag{K1}\\
\alpha(n+1,0)   &= \alpha(n,1)              \tag{K2}\\
\alpha(n+1,x+1) &= \alpha(n,\alpha(n+1, x)))\tag{K3}  
\endalign
$$
\beginil
\item{(i)} Calcule o valor $\alpha(3,2)$, indicando para cada passo qual equação foi usada.
\item{(ii)} Prove que para todo $x\in\nats$, $\alpha(1,x) = x + 2$.
\item{(iii)} Prove que para todo $x\in\nats$, $\alpha(2,x) = 2x + 3$.
\endil
A função $\alpha$ é conhecida como função de \Ackermann{}Ackermann,
e vamos a encontrar novamente bem depois, no~\ref{Recursive_functions}.

\endproblem
%%}}}

\endproblems
%%}}}

%%{{{ Further reading 
\further.

Mais informações sobre indução e indução forte,
veja o~\cite[Cap.~6]{velleman}.
Sobre o princípio da boa ordem e indução,
veja o~\cite[\S\S1.4--1.5]{babybm}

\endfurther
%%}}}

\endchapter
%%}}}

%%{{{ chapter: Functional programming 
\chapter Programação funcional.

%%{{{ Problems 
\problems.

%%{{{ prob: drunk_countdown_take 
{%
\def\drunk{\namedfun{drunk}}
\def\countdown{\namedfun{countdown}}
\def\take{\namedfun{take}}
\problem.
Defina as funções com equações (recursivas):
$$
\xalignat3
\drunk      &\eqtype [\nats]  \to [\nats] &
\countdown  &\eqtype \nats    \to [\nats] &
\take       &\eqtype \nats    \to [\nats] \to [\nats].\\
\endxalignat
$$
Exemplos de uso:
$$
\align
\drunk\ [2,8]           &= [2,2,8,8]        \\
\drunk\ [1,9,8,3]       &= [1,1,9,9,8,8,3,3]\\
\countdown\ 3           &= [3,2,1,0]        \\
\countdown\ 1           &= [1,0]            \\
\take\ 3\ [2,3,5,7,11]  &= [2,3,5]          \\
\take\ 8\ [0,1,2,4]     &= [0,1,2,4]
\endalign
$$

\endproblem
}
%%}}}

\endproblems
%%}}}

%%{{{ Further reading 
\further.

\cite{huttonhaskell},
\cite{lyah};
\cite{birdthinking},
\cite{thompsonhaskell};
\cite{birdfphaskell}.

\endfurther
%%}}}

\endchapter
%%}}}

%%{{{ chapter: Enumerative combinatorics 
\chapter Combinatória enumerativa.

%%{{{ Counting principles 
\section Princípios de contagem.

\note Informalmente.
Queremos contar todas as maneiras possíveis para algo acontecer,
certas configurações, certos objetos ser escolhidos, ou ordenados, etc.
Baseamos nossas idéia em dois princípios de contagem:
da \emph{adição} e da \emph{multiplicação}.
\endgraf
{O princípio da adição, informalmente:}
Se podemos grupar todos esses objetos em grupos \emph{distintos},
tais que cada objeto pertence em \emph{exatamente um} grupo,
o número total dos objetos é igual o somatório dos tamanhos dos grupos.
\endgraf
{O princípio da multiplicação, informalmente:}
Se cada configuração pode ser descrita completamente em $n$ passos,
onde para o primeiro passo temos $a_1$ opções,
para o segundo passo temos $a_2$ opções, etc., e
\emph{em cada passo a quantidade das opções disponíveis
não depende nas escolhas anteriores},
então existem em total $a_1a_2\dotsb a_n$ configurações possíveis.

\principle da adição, formalmente.
\tdefined{princípio}[da adição]
Seja $A$ conjunto finito, e $A_1,\dotsc,A_n$ subconjuntos dele tais que cada elemento $a\in A$, pertence em \emph{exatamente} um dos $A_i$.
Logo,
$$
|A| = \sum_{i=1}^n |A_i|.
$$

\principle da multiplicação.
\tdefined{princípio}[da multiplicação]
Sejam $A_1,\dotsc,A_n$ conjuntos finitos.
Logo
$$
|A_1\times\dotsb\times A_n| = |A_1|\dotsb|A_n|.
$$

\example.
De quantas maneiras podemos escrever um string de tamanho $3$\dots
\item{(i)} Usando o alfabeto
$$
\set{\txt A, \txt B, \txt C, \dotsc, \txt X, \txt Y, \txt Z}?
$$
\item{(ii)} Usando o mesmo alfabeto, mas proibindo o mesmo caractere se repitir no string?

\solution
Consideramos a formação de cada string em passos, caractere a caractere.
Temos 3 posições para colocar os caracteres:
$
\underline{\phantom{\txt Z}}
\;
\underline{\phantom{\txt Z}}
\;
\underline{\phantom{\txt Z}}
\;
$.
\endgraf
Para a questão (i), temos:
$26$ maneiras para escolher o primeiro caractere,
$26$ para o segundo, e
$26$ para o último:
$$
\underbrace{
\underline{\phantom{\txt Z}}
}_{26}
\;
\underbrace{
\underline{\phantom{\txt Z}}
}_{26}
\;
\underbrace{
\underline{\phantom{\txt Z}}
}_{26}
\;.
$$
A escolha em cada passo não é afeitada por as escolhas dos passos anteriores.
Logo, pelo princípio da multiplicação tem
$$
26\ntimes 26\ntimes 26 = 26^3
$$
strings possíveis.
\endgraf
Para a questão (ii), temos:
$26$ maneiras para escolher o primeiro caractere,
$25$ para o segundo (todos menos aquele que escolhemos no passo anterior), e
$24$ para o último (similarmente):
$$
\underbrace{
\underline{\phantom{\txt Z}}
}_{26}
\;
\underbrace{
\underline{\phantom{\txt Z}}
}_{25}
\;
\underbrace{
\underline{\phantom{\txt Z}}
}_{24}
\;.
$$
Agora a escolha em cada passo realmente \emph{é afeitada} por as escolhas
dos passos anteriores!
Por exemplo, se no primeiro passo escolher o caractere $\txt C$, para
o segundo passo as opções incluem o caractere $\txt A$; 
mas se no primeiro passo escolher o caractere $\txt A$, as opções para o segundo
mudam: não temos essa opção mais.
\emph{Mesmo assim, podemos usar o princípio da multiplicação!}
Por quê?
As escolhas dos passos anteriores afeitam \emph{quais} são as escolhas do passo atual,
mas não afeitam \emph{quantas} elas são!
Por isso, chegamos no resultado aplicando mais uma vez o princípio da multiplicação:
temos
$$
26\ntimes 25\ntimes 24
$$
maneiras possíveis.
\endexample

\exercise.
Temos $4$ presentes e queremos dar para $3$ crianças tal que cada criança vai
receber apenas um presente.
\item{(i)} De quantas maneiras podemos disribuir os presentes para as crianças?
\item{(ii)} O que muda se as crianças são $4$?  Explique.

\endexercise

\exercise.
De quantas maneiras podemos escrever strings de tamanho $2$ usando o alfabeto
$$
\set{\txt A, \txt B, \txt C, \txt D},
$$
tais que as letras aparecem em ordem que concorda com a do alfabeto?
Por exemplo os string $\txt A \txt C$, $\txt B \txt B$, e $\txt C \txt D$ são aceitáveis,
mas os $\txt D \txt C$ e $\txt B \txt A$, não.

\hint
Cuidado: aqui a \emph{quantidade} das opções da segunda escolha, depende sim na escolha
anterior!

\hint
Separa os strings possíveis em colecções e conta os strings de cada colecção separadamente,
somando no final (princípio de adição) para achar o resultado.

\endexercise

\endsection
%%}}}

%%{{{ Permutations and combinations 
\section Permutações e combinações.
\label{Permutations_and_combinations}%

\question. De quantas maneiras podemos escolher $r$ objetos de $n$?

Essa questão é bastante ambígua;
por exemplo:
Os $n$ objetos são todos distintos?
Podemos repitir o mesmo objeto na nossa escolha?

\definition.
\label{comb_perm_totperm}%
\sdefined {\totperm {\holed n}} {o número de permutações (totais) de \holed n objetos}%
\sdefined {\perm {\holed n} {\holed r}} {o número de $\holed r$-permutações de \holed n objetos}%
\sdefined {\comb {\holed n} {\holed r}} {o número de $\holed r$-combinações de \holed n objetos}%
Usamos os símbolos:
$$
\align
\totperm n &:\quad\text{o número de permutações totais de $n$ objetos}\\
\perm n r  &:\quad\text{o número de $r$-permutações de $n$ objetos}\\
\comb n r  &:\quad\text{o número de $r$-combinações de $n$ objetos}
\endalign
$$
Onde entendemos que:
\beginol
\li os $n$ objetos são distíntos;
\li não podemos os repetir.
\endol
Observe que as permutações totais são apenas casos especiais de $r$-permutações.
Na literatura encontramos $r$-permutações também com o nome \dterm{arranjos},
mas nos vamos evitar esse termo aqui para evitar confusão.

\proposition Permutações totais.
\label{total_permutations}
\tdefined{permutação}[total]
\iisee{arranjo}{permutação}
$$
\totperm n = {n!}.
$$

\proposition Permutações.
\tdefined{permutação}
$$
\perm n r = \frac {n!} {(n-r)!}.
$$

\proposition Combinações.
\tdefined{combinação}
$$
\comb n r = \frac {n!} {(n-r)!\stimes r!}.
$$

\example.
10 amigos têm vontade viajar com um carro de 5 vagas.
De quantas maneiras diferentes 5 deles podem entrar no carro?
Considere que o que diferencia mesmo a configuração são apenas a posição
do motorista e do copiloto.

\solution
Vamos ver dois jeitos diferentes para contar essas configurações:
\endgraf
\proofstyle{\indent Jeito 1:}
Escrevendo
$$
\comb {10} 1
\ntimes
\comb 9 1
\ntimes
\comb 8 3
$$
já é meio auto-explicativo: formamos cada configuração em passos:
{(i)}      escolher o motorista;
{(ii)}     escolher o copiloto;
{(iii)}    escolher os passangeiros de trás.
\endgraf
\proofstyle{\indent Jeito 2:}
Outra método para formar cada configuração seria:
{(i)}      escolher os 5 que vão entrar no carro;
{(ii)}     escolher qual vai ser o motorista;
{(iii)}    escolher qual vai ser o copiloto.
pensando assim chegamos no cálculo
$$
\underbrace{\comb {10} 5}_{\text{(i)}}
\ntimes
\underbrace{\comb 5 1}_{\text{(ii)}}
\ntimes
\underbrace{\comb 4 1}_{\text{(iii)}}.
$$
\endgraf
Olhando para os dois cálculos
$$
\comb {10} 1 \ntimes \comb 9 1 \ntimes \comb 8 3
\askeq
\comb {10} 5 \ntimes \comb 5 1 \ntimes \comb 4 1
$$
não é óbvio que seus valores são iguais.  Calculamos
$$
\alignat2
\comb {10} 1 \ntimes \comb 9 1 \ntimes \comb 8 3
&= 10 \ntimes 9 \ntimes \frac {8!} {5!\stimes3!}
&&= \frac {10!} {5!\stimes3!}
\\
\comb {10} 5 \ntimes \comb 5 1 \ntimes \comb 4 1
&= \frac {10!} {5!\stimes 5!} \ntimes 5 \ntimes 4
&&= \frac {10!} {5!\stimes 3!}
\endalignat
$$
e respondemos (felizmente) que em total temos
$$
\frac {10!} {5!\stimes3!}
= \frac{10 \ntimes 9 \ntimes 8\ntimes7\ntimes6} { 3! }
= 10 \ntimes 9 \ntimes 8\ntimes7
= 5040
$$
configurações diferentes.
\endexample

\endsection
%%}}}

%%{{{ Permutations in a circle 
\section Permutações cíclicas.

\example.
\label{circle_dance_of_8}
$8$ pessoas querem dançar uma dança em qual
todos precisam formar um ciclo pegando as mãos
(e olhando para o interior do ciclo).
Em quantas configurações diferentes essa dança pode começar?

\solution
Vamos resolver esse problema seguindo duas idéias bem diferentes:
\endgraf
\proofstyle{\indent Idéia~1.}
Consideramos primeiro a questão:
``\emph{de quantas maneiras podemos permutar as $8$ pessoas numa ordem?}''
Respondemos $8!$, o número das permutações totais de $8$ objetos
(sabendo que hipercontamos para o problema original).
Mas podemos calcular que cada resposta do problema original,
corresponde em exatamente $8$ respostas do problema novo
(uma para cada ``circular shift'').
Então basta só dividir a ``hiperconta'' por $8$, e chegamos
no resultado final: $8! / 8$, ou seja, $7!$.
\endgraf
\proofstyle{\indent Idéia~2.}
\emph{Fixamos uma pessoa como ``determinante'' da configuração;}
a idéia sendo que para comparar duas configurações nos vamos
começar com o determinante, e depois comparar em ordem fixa
o resto da configuração
(por exemplo indo cada vez de uma pessoa
para quem tá no lado direito dela).
Assim, para cada permutação total das 7 outras pessoas,
temos uma permutação circular das 8 e vice-versa,
ou seja, a resposta final é $7!$.
\endexample

\midinsert
\hfil
\tikzpicture[scale=0.8]%%{{{
\draw (0,0) circle (2cm);
\foreach \t in {0,45,90,135,180,225,270,315}
    \node[circle,fill=white] (a\t) at (\t:2cm) {$\phantom{p_0}$};
\node[circle,fill=white] at (a0)   {$p_0$};
\node[circle,fill=white] at (a45)  {$p_1$};
\node[circle,fill=white] at (a90)  {$p_2$};
\node[circle,fill=white] at (a135) {$p_3$};
\node[circle,fill=white] at (a180) {$p_4$};
\node[circle,fill=white] at (a225) {$p_5$};
\node[circle,fill=white] at (a270) {$p_6$};
\node[circle,fill=white] at (a315) {$p_7$};
\endtikzpicture
%%}}}
\hfil
\tikzpicture[scale=0.8]%%{{{
\draw (0,0) circle (2cm);
\foreach \t in {0,45,90,135,180,225,270,315}
    \node[circle,fill=white] (a\t) at (\t:2cm) {$\phantom{a_0}$};
\node[circle,fill=white] at (a315)  {$p_0$};
\node[circle,fill=white] at (a0)    {$p_1$};
\node[circle,fill=white] at (a45)   {$p_2$};
\node[circle,fill=white] at (a90)   {$p_3$};
\node[circle,fill=white] at (a135)  {$p_4$};
\node[circle,fill=white] at (a180)  {$p_5$};
\node[circle,fill=white] at (a225)  {$p_6$};
\node[circle,fill=white] at (a270)  {$p_7$};
\endtikzpicture
%%}}}
\hfil
\endgraf\centerline{Uma configuração do~\refn{circle_dance_of_8} representada em dois jeitos diferentes no papel.}
\endinsert

Generalizando concluimos que:

\proposition.
As configurações circulares diferentes de $n$ objetos, $n>0$ são
$$
(n-1)!.
$$

\exercise.
O que mudará na contagem do~\ref{circle_dance_of_8},
se cada dançador pode olhar ou para o interior ou para o exterior do círculo?

\hint
Começando com a mesma resposta ($7!$) do~\ref{circle_dance_of_8},
claramente nos hipocontamos---mas quanto?

\hint
Considere uma configuração do~\ref{circle_dance_of_8}.
Com quantas configurações do problema atual ela corresponde?

\endexercise

\exercise.
O que mudará na contagem do~\ref{circle_dance_of_8},
se temos $4$ mulheres e $4$ homens e as regras da dança
mandam alternar os sexos na configuração?

\hint
Construa cada configuração em passos.

\hint
Primeiramente, esqueça os homens (ou as mulheres)
e coloca as $4$ mulheres (ou os $4$ homens)
num ciclo.
De quantas maneiras pode escolher o resto para
entrar no círculo?

\endexercise

\exercise.
Temos $8$
miçangas
diferentes, e queremos pôr todas
numa
corrente
para criar uma pulseira.
Quantas maneiras diferentes temos para o criar?

\hint
Se tu achar o problema igual com o~\ref{circle_dance_of_8},
tu hipercontarás\dots Quanto?

\hint
Veja a figura.
Pode explicar por que as três representações
correspondem na mesma configuração?
\topinsert
\centerline{
\tikzpicture[scale=0.8]%%{{{
\draw (0,0) circle (2cm);
\foreach \t in {0,45,90,135,180,225,270,315}
    \node[circle,fill=white] (a\t) at (\t:2cm) {$\phantom{a_0}$};
\node[circle,fill=red!33]       at (a0)   {$a_0$};
\node[circle,fill=green!33]     at (a45)  {$a_1$};
\node[circle,fill=blue!33]      at (a90)  {$a_2$};
\node[circle,fill=cyan!33]      at (a135) {$a_3$};
\node[circle,fill=magenta!33]   at (a180) {$a_4$};
\node[circle,fill=yellow!33]    at (a225) {$a_5$};
\node[circle,fill=orange!33]    at (a270) {$a_6$};
\node[circle,fill=brown!33]     at (a315) {$a_7$};
\endtikzpicture
%%}}}
\hfil
\tikzpicture[scale=0.8]%%{{{
\draw (0,0) circle (2cm);
\foreach \t in {0,45,90,135,180,225,270,315}
    \node[circle,fill=white] (a\t) at (\t:2cm) {$\phantom{a_0}$};
\node[circle,fill=red!33]       at (a315)  {$a_0$};
\node[circle,fill=green!33]     at (a0)    {$a_1$};
\node[circle,fill=blue!33]      at (a45)   {$a_2$};
\node[circle,fill=cyan!33]      at (a90)   {$a_3$};
\node[circle,fill=magenta!33]   at (a135)  {$a_4$};
\node[circle,fill=yellow!33]    at (a180)  {$a_5$};
\node[circle,fill=orange!33]    at (a225)  {$a_6$};
\node[circle,fill=brown!33]     at (a270)  {$a_7$};
\endtikzpicture
%%}}}
\hfil
\tikzpicture[scale=0.8]%%{{{
\draw (0,0) circle (2cm);
\foreach \t in {0,45,90,135,180,225,270,315}
    \node[circle,fill=white] (a\t) at (\t:2cm) {$\phantom{a_0}$};
\node[circle,fill=red!33]       at (a315)  {$a_0$};
\node[circle,fill=green!33]     at (a270)  {$a_1$};
\node[circle,fill=blue!33]      at (a225)  {$a_2$};
\node[circle,fill=cyan!33]      at (a180)  {$a_3$};
\node[circle,fill=magenta!33]   at (a135)  {$a_4$};
\node[circle,fill=yellow!33]    at (a90)   {$a_5$};
\node[circle,fill=orange!33]    at (a45)   {$a_6$};
\node[circle,fill=brown!33]     at (a0)    {$a_7$};
\endtikzpicture
%%}}}
}
\botcaption{}
\emph{Apenas uma} das configurações da pulseira, representada em três desenhos.
\endcaption
\endinsert

\endexercise

\endsection
%%}}}

%%{{{ Together or separated 
\section Juntos ou separados.

\example.
\label{dinner_of_8_with_couple}
Suponha que $8$ pessoas $A,B,C,D,E,F,G,H$ querem sentar num bar
mas $C$ e $D$ querem sentar juntos.
De quantas maneiras isso pode acontecer?

\solution
\emph{Vamos imaginar que $C$ e $D$ são uma pessoa, chamada $CD$.}
Nos perguntamos de quantas maneiras as 7 pessoas $A,B,CD,E,F,G,H$ podem sentar
numa mesa de bar com $7$ banquinhos.
A resposta é os permutações totais de tamanho 7, ou seja, $7!$.
Mas para cada configuração desse problema,
correspondem \emph{duas} configurações do problema original, porque os
$C$ e $D$ podem sentar em duas ordens diferentes juntos.
A resposta final:
$7! \ntimes 2$.
\endexample

\exercise.
Suponha que 8 pessoas $A,B,C,D,E,F,G,H$ querem jantar numa mesa de bar.
Em quantas configurações diferentes eles podem sentar se\dots:
\item{(1)} os $C$ e $D$ e $E$ querem sentar juntos;
\item{(2)} os $F$ e $G$ não podem sentar juntos;
\item{(3)} as duas restrições (1) e (2).

\hint
\item{(1)} Como no \ref{dinner_of_8_with_couple},
           considere o problema onde $C$, $D$, e $E$ são uma pessoa só.
\item{(2)} Conte o complementar e subtraia do total sem restrição;
\item{(3)} Conte as configurações em quais $C$, $D$, e $E$ sentam juntos e subtraia
           as configurações onde, alem disso, $F$ e $G$ também sentam juntos.

\solution
(1) Como no \ref{dinner_of_8_with_couple}, traduzimos o problema para um onde
$C$, $D$, e $E$, são uma pessoa---vamos a chamar de $CDE$---e as
$6$ pessoas $A,B,CDE,F,G,H$ querem jantar numa mesa de bar com 6 banquinhos.
Cada solução desse problema corresponde em tantas configurações quantas as $3$ pessoas
$C$, $D$, $E$ podem sentar numa ordem, ou seja $\totperm 3$ configurações.
A resposta final:
$$
\totperm 6 \ntimes \totperm 3 = 6!\stimes3! = 6!\stimes 6
$$
\endgraf
(2) Contamos o complementar:
todas as maneiras onde $F$ e $G$ sentam juntos ($7!\ntimes 2$),
e o tiramos de todas as maneiras sem restrição ($8!$):
$$
\totperm 8 - \totperm 7 \ntimes \totperm 2
= 8! - 7!\ntimes 2! = 7!(8 - 2)
= 7!\stimes 6
$$
(3) Já achamos quantas maneiras tem onde $C$, $D$, e $E$ sentam juntos: $6!\stimes6$.
Disso, precisamos subtrair as configurações onde $F$ e $G$ também sentam juntos:
para satisfazer as duas restrições, consideramos as $5$ ``pessoas''
$A,B,CDE,FG,H$, quais podem sentar numa mesa de bar de tamanho $5$ de
$$
\totperm 5 \stimes \totperm 3 \stimes \totperm 2
= 5! \stimes 3! \stimes 2!
= 5! \stimes 6 \stimes 2
= 6! \stimes 2
$$
maneiras.
A resposta final então é
$$
6!\stimes6 - 6!\stimes2 = 6!(6-2) = 6! \stimes 4.
$$

\endexercise

Generalizando:

\proposition.
O número das permutações totais de $m$ objetos distintos com a restrição que
certos $c$ deles tem que estar consecutivos é
$$
(m-c+1)! \stimes c!.
$$

\endsection
%%}}}

%%{{{ Permutations of things not all distinct 
\section Permutações de objetos não todos distintos.

\question.
De quantas maneiras podemos permutar $n$ objetos se
eles não são todos distíntos?

\example.
\label{pessimissimo}
Conte todas as palavras feitas por permutações das 12 letras da palavra
$$
\txt{PESSIMISSIMO}.
$$

\solution
Vamos contar em dois jeitos diferentes:
\endgraf
\proofstyle{Idéia 1:}
Construimos cada palavra possível ``em passos'',
usando o princípio da multplicação para achar o número total.
$$
\alignat 2
\text{Começamos com 12 espaços:}
\qquad&
\underline{\phantom {\txt M}} \ 
\underline{\phantom {\txt I}} \ 
\underline{\phantom {\txt S}} \ 
\underline{\phantom {\txt S}} \ 
\underline{\phantom {\txt I}} \ 
\underline{\phantom {\txt S}} \ 
\underline{\phantom {\txt S}} \ 
\underline{\phantom {\txt I}} \ 
\underline{\phantom {\txt P}} \ 
\underline{\phantom {\txt O}} \ 
\underline{\phantom {\txt E}} \ 
\underline{\phantom {\txt M}}
\\
\text{escolhemos onde colocar o ${\txt P}$:}
\qquad&
\underline{\phantom {\txt M}} \ 
\underline{\phantom {\txt I}} \ 
\underline{\phantom {\txt S}} \ 
\underline{\phantom {\txt S}} \ 
\underline{\phantom {\txt I}} \ 
\underline{\phantom {\txt S}} \ 
\underline{\phantom {\txt S}} \ 
\underline{\phantom {\txt I}} \ 
          {         {\txt P}} \ 
\underline{\phantom {\txt O}} \ 
\underline{\phantom {\txt E}} \ 
\underline{\phantom {\txt M}}
&\qquad&
\text{tivemos $\comb {12} 1$ opções;}
\\
\text{depois o ${\txt E}$:}
\qquad&
\underline{\phantom {\txt M}} \ 
\underline{\phantom {\txt I}} \ 
\underline{\phantom {\txt S}} \ 
\underline{\phantom {\txt S}} \ 
\underline{\phantom {\txt I}} \ 
\underline{\phantom {\txt S}} \ 
\underline{\phantom {\txt S}} \ 
\underline{\phantom {\txt I}} \ 
          {         {\txt P}} \ 
\underline{\phantom {\txt O}} \ 
          {         {\txt E}} \ 
\underline{\phantom {\txt M}}
&\qquad&
\text{tivemos $\comb {11} 1$ opções;}
\\
\text{depois os 4 ${\txt S}$:}
\qquad&
\underline{\phantom {\txt M}} \ 
\underline{\phantom {\txt I}} \ 
          {         {\txt S}} \ 
          {         {\txt S}} \ 
\underline{\phantom {\txt I}} \ 
          {         {\txt S}} \ 
          {         {\txt S}} \ 
\underline{\phantom {\txt I}} \ 
          {         {\txt P}} \ 
\underline{\phantom {\txt O}} \ 
          {         {\txt E}} \ 
\underline{\phantom {\txt M}}
&\qquad&
\text{tivemos $\comb {10} 4$ opções;}
\\
\text{depois os 3 ${\txt I}$:}
\qquad&
\underline{\phantom {\txt M}} \ 
          {         {\txt I}} \ 
          {         {\txt S}} \ 
          {         {\txt S}} \ 
          {         {\txt I}} \ 
          {         {\txt S}} \ 
          {         {\txt S}} \ 
          {         {\txt I}} \ 
          {         {\txt P}} \ 
\underline{\phantom {\txt O}} \ 
          {         {\txt E}} \ 
\underline{\phantom {\txt M}}
&\qquad&
\text{tivemos $\comb 6 3$ opções;}
\\
\text{depois os 2 ${\txt M}$:}
\qquad&
          {         {\txt M}} \ 
          {         {\txt I}} \ 
          {         {\txt S}} \ 
          {         {\txt S}} \ 
          {         {\txt I}} \ 
          {         {\txt S}} \ 
          {         {\txt S}} \ 
          {         {\txt I}} \ 
          {         {\txt P}} \ 
\underline{\phantom {\txt O}} \ 
          {         {\txt E}} \ 
          {         {\txt M}}
&\qquad&
\text{tivemos $\comb 3 2$ opções;}
\\
\text{e finalmente o ${\txt O}$:}
\qquad&
          {         {\txt M}} \ 
          {         {\txt I}} \ 
          {         {\txt S}} \ 
          {         {\txt S}} \ 
          {         {\txt I}} \ 
          {         {\txt S}} \ 
          {         {\txt S}} \ 
          {         {\txt I}} \ 
          {         {\txt P}} \ 
          {         {\txt O}} \ 
          {         {\txt E}} \ 
          {         {\txt M}}
&\qquad&
\text{tivemos $\comb 1 1$ opção.}
\endalignat
$$
Pelo princípio da multiplicação, a resposta é o produto
$$
\align
\underbrace{\comb {12} 1}_{\dsize {\txt P}}
\underbrace{\comb {11} 1}_{\dsize {\txt E}}
\underbrace{\comb {10} 4}_{\dsize4{\txt S}}
\underbrace{\comb {6}  3}_{\dsize3{\txt I}}
\underbrace{\comb {3}  2}_{\dsize2{\txt M}}
\underbrace{\comb {1}  1}_{\dsize {\txt O}}
&=
\frac
{12!}
{\cancel{11!}\stimes 1!}
\frac
{\cancel{11!}}
{\cancel{10!}\stimes 1!}
\frac
{\cancel{10!}}
{\cancel{6!}\stimes 4!}
\frac
{\cancel{6!}}
{\cancel{3!}\stimes 3!}
\frac
{\cancel{3!}}
{\cancel{1!}\stimes 2!}
\frac
{\cancel{1!}}
{0!\stimes 1!}\\
&=
\frac
{12!}
{1!
\stimes 1!
\stimes 4!
\stimes 3!
\stimes 2!
\stimes 1!
}
=
\frac
{12!}
{
4!
\stimes 3!
\stimes 2!
}.
\endalign
$$
\endgraf
\proofstyle{Idéia 2:}
Contamos as maneiras se todas as letras fossem distintas,
por exemplo marcando cada letra com índices:
$$
{\txt P}_1\ 
{\txt E}_1\ 
{\txt S}_1\ 
{\txt S}_2\ 
{\txt I}_1\ 
{\txt M}_1\ 
{\txt I}_2\ 
{\txt S}_3\ 
{\txt S}_4\ 
{\txt I}_3\ 
{\txt M}_2\ 
{\txt O}_1\,.
$$
Sabemos que são $12!$ e que
assim temos \emph{hipercontado} para nosso problema.
Por exemplo, a palavra 
$
{\txt M}
{\txt I}
{\txt S}
{\txt S}
{\txt I}
{\txt S}
{\txt S}
{\txt I}
{\txt P}
{\txt O}
{\txt E}
{\txt M}
$
corresponde em várias palavras do problema novo;
escrevemos três delas aqui como exemplos:
$$
{\txt M}\ 
{\txt I}\ 
{\txt S}\ 
{\txt S}\ 
{\txt I}\ 
{\txt S}\ 
{\txt S}\ 
{\txt I}\ 
{\txt P}\ 
{\txt O}\ 
{\txt E}\ 
{\txt M}
\ 
\transto
\ 
\left\{
\gathered
{\txt M}_1\ 
{\txt I}_1\ 
{\txt S}_1\ 
{\txt S}_2\ 
{\txt I}_2\ 
{\txt S}_3\ 
{\txt S}_4\ 
{\txt I}_3\ 
{\txt P}_1\ 
{\txt O}_1\ 
{\txt E}_1\ 
{\txt M}_2
\\
{\txt M}_2\ 
{\txt I}_1\ 
{\txt S}_1\ 
{\txt S}_2\ 
{\txt I}_2\ 
{\txt S}_3\ 
{\txt S}_4\ 
{\txt I}_3\ 
{\txt P}_1\ 
{\txt O}_1\ 
{\txt E}_1\ 
{\txt M}_1
\\
{\txt M}_2\ 
{\txt I}_1\ 
{\txt S}_4\ 
{\txt S}_3\ 
{\txt I}_2\ 
{\txt S}_2\ 
{\txt S}_1\ 
{\txt I}_3\ 
{\txt P}_1\ 
{\txt O}_1\ 
{\txt E}_1\ 
{\txt M}_1
\\
\vdots
\endgathered
\quad\qquad
\right\}
\ \text{\dots quantas?}
$$
Mas é fácil calcular quanto hipercontamos:
\emph{cada} palavra do problema original corresponde em exatamente tantas palavras
quantas as maneiras de permutar cada grupo de letras ``subindicadas'' entre si,
ou seja:
$$
\underbrace{1!}_{{\txt P}} \ntimes
\underbrace{1!}_{{\txt E}} \ntimes
\underbrace{4!}_{{\txt S}} \ntimes
\underbrace{3!}_{{\txt I}} \ntimes
\underbrace{2!}_{{\txt M}} \ntimes
\underbrace{1!}_{{\txt O}}
$$
maneiras.
Para responder então, basta dividir o número da ``hipercontagem'' por esse:
$$
\frac
{12!}
{4!\stimes 3!\stimes 2!}.
$$
\moveqedup
\endexample

\exercise.
Escolhendo outra ordem de colocar as letras
na \proofstyle{Idéia 1} do~\ref{pessimissimo}
nossas opções em cada passo seriam diferentes.
Explique porque podemos usar o princípio da multiplicação mesmo assim.

\hint
Presta atenção na frase
``\emph{em cada passo a quantidade das opções disponíveis
não depende nas escolhas anteriores}''.

\solution
O imporante é que em cada passo, qual das nossas disponíveis opções
será escolhida, não vai afeitar a quantidade das nossas opções no passo seguinte.
Isso é realmente valido nesse caso.
Se escolher colocar as letras em outra ordem, por exemplo a
$
{\txt S},
{\txt E},
{\txt I},
{\txt P},
{\txt O},
{\txt M}
$, teriamos quantidades diferentes para cada passo sim,
\emph{mas}\/:
cada uma das nossas escolhas, não afeitaria a quantidade das escolhas próximas.
Com essa ordem, chegamos no mesmo resultado (com um cálculo que de longe aparece diferente):
$$
\align
\underbrace{\comb {12} 4}_{\dsize4{\txt S}}
\underbrace{\comb {8}  1}_{\dsize {\txt E}}
\underbrace{\comb {7}  3}_{\dsize3{\txt I}}
\underbrace{\comb {4}  1}_{\dsize {\txt P}}
\underbrace{\comb {3}  1}_{\dsize {\txt O}}
\underbrace{\comb {2}  2}_{\dsize2{\txt M}}
&=
\frac
{12!}
{\cancel{8!}\stimes 4!}
\frac
{\cancel{8!}}
{\cancel{7!}\stimes 1!}
\frac
{\cancel{7!}}
{\cancel{4!}\stimes 3!}
\frac
{\cancel{4!}}
{\cancel{3!}\stimes 1!}
\frac
{\cancel{3!}}
{\cancel{2!}\stimes 1!}
\frac
{\cancel{2!}}
{0!\stimes 2!}\\
\vphantom{
\underbrace{\comb {12} 4}_{\dsize4{\txt S}}
\underbrace{\comb {8}  1}_{\dsize {\txt E}}
\underbrace{\comb {7}  3}_{\dsize3{\txt I}}
\underbrace{\comb {4}  1}_{\dsize {\txt P}}
\underbrace{\comb {3}  1}_{\dsize {\txt O}}
\underbrace{\comb {2}  2}_{\dsize2{\txt M}}
}
&=
\frac
{12!}
{4!
\stimes 1!
\stimes 3!
\stimes 1!
\stimes 1!
\stimes 2!
}\\
&=
\frac
{12!}
{
4!
\stimes 3!
\stimes 2!
}.
\endalign
$$

\endexercise

\endsection
%%}}}

%%{{{ Binomial coefficients 
\section Binomial e seus coeficientes.

\theorem Binomial.
\ii{teorema}[binomial]%
\iisee{binomial}[teorema]{teorema binomial}%
\label{binomial_theorem}%
Sejam $x,y\in\ints$ e $n\in\nats$.
$$
\align
(x+y)^n
&= \binom n 0 x^n + \binom n 1 x^{n-1}y + \dotsb + \binom n {n-1} xy^{n-1} + \binom n n y^n\\
&= \sum_{i=0}^n \binom n i x^{n-i}y^i.
\endalign
$$
\sketch.
Queremos achar quantas vezes o termo $x^{n-r}y^r$ aparece na expansão do binomial.
Escrevendo
$$
(x+y)^n = \underbrace{(x+y)(x+y)\dotsb(x+y)}_{\text{$n$ vezes}}
$$
observamos que para cada das $\comb n r$ maneiras de escolher $r$ dos termos em cima,
corresponde um termo $x^{n-r}y^r$:
``escolhe quais dos termos do produto vão oferecer seu $y$
(o resto dos termos oferecerá seus $x$'s)''.
Isso justifica os coeficientes $\binom n r$.
Por exemplo, para $n=7$ e $r=4$, a escolha
$$
(x+y)^7
=
(x+y)
\underbrace{(x+y)}_{\text{``$y$''}}\,
(x+y)
\underbrace{(x+y)}_{\text{``$y$''}}
\underbrace{(x+y)}_{\text{``$y$''}}\,
(x+y)
\underbrace{(x+y)}_{\text{``$y$''}}
$$
corresponde no produto
$xyxyyxy = x^3y^4$, e cada diferente escolha das $\binom 7 4$
correspondará com uma maneira diferente para formar o $x^3y^4$.
\qes

\endsection
%%}}}

%%{{{ Number of subsets 
\section Número de subconjuntos. 
\label{Number_of_subsets}%

\question.
Quantos subconjuntos dum conjunto finito existem?

\note Idéia 1.
\label{subset_count_using_strings}
Começamos com um exemplo de um conjunto $A$ de tamanho 6:
$$
A = \set{a, b, c, d, e, f}.
$$
Uns subconjuntos de $A$ são os:
$$
\set{a,d,e},
\qquad
\emptyset,
\qquad
\set{a},
\qquad
\set{b,c,e,f},
\qquad
A,
\qquad
\set{f},
\qquad
\dotsc
$$
Queremos contar todos os subconjuntos de $A$.
Vamos \emph{traduzir o problema} de contar os subconjuntos do $A$
para um problema que involve $n$-tuplas de dois símbolos ``0'' e ``1'', ``sim'' e ``não'', ``$\in$'' e ``$\notin$'', etc.
(Obviamente \emph{quais} são esses símbolos não afeita nada; o que importa é
que são dois símbolos distintos.)
Podemos associar agora cada dessas tuplas (ou strings) de tamanho $n$ para um
subconjunto de $A$, e vice-versa, chegando numa correspondência entre as duas
colecções de objetos.
Naturalmente associamos, por exemplo,
$$
\def\Y{1}
\def\N{0}
\underbrace{
\matrix
a  & b  & c  & d  & e  & f \\
\Y & \N & \N & \Y & \Y & \N\\
\N & \N & \N & \N & \N & \N\\
\Y & \N & \N & \N & \N & \N\\
\N & \Y & \Y & \N & \Y & \Y\\
\Y & \Y & \Y & \Y & \Y & \Y\\
\N & \N & \N & \N & \N & \Y\\
\vdots&
\vdots&
\vdots&
\vdots&
\vdots&
\vdots
\endmatrix
}_{\text{Strings de tamanho 6 do alfabeto $\set{\N,\Y}$}}
\quad\bitrans\quad
\underbrace{
\matrix
\format
\c\\
\\
\set{a,d,e}\\
\emptyset\\
\set{a}\\
\set{b,c,e,f}\\
A\\
\set{f}\\
\vdots
\endmatrix
}_{\text{Subconjuntos de $A$}}
$$
e verificamos que realmente cada configuração do problema original de subconjuntos
corresponde exatamente numa configuração do problema novo dos strings e vice-versa.
\endgraf
\emph{O que ganhamos?}
Sabemos como contar todos esses strings: são $2^6$.
Concluimos que os subconjuntos do $A$ são $2^6$ também.

Generalizando essa idéia chegamos no resultado:

\proposition.
\label{number_of_subsets_of_finite_set}
Seja $A$ conjunto finito.
$$
\card{\pset A} = 2^{\card A}.
$$

\note Idéia 2.
\label{subset_count_by_grouping}
Um outro jeito para contar todos os subconjuntos dum dado conjunto $A$, seria
os separar em grupos baseados no seu tamanho.
Assim, percebos que esses subconjuntos são naturalmente divididos em $n+1$ colecções:
subconjuntos com $0$ elementos, com $1$ elemento, \dots, com $n$ elementos.
\endgraf
\emph{O que ganhamos?}
Sabemos como contar os elementos de cada uma dessas colecções:
para formar um subconjunto de tamanho $r$, precisamos escolher $r$ dos $n$ elementos,
ou seja, existem $\comb n r$ subconjuntos de tamanho $r$.
Agora, pelo princípio da adição basta apenas somar:
são
$\sum_{i=0}^n \comb n i$.
\endgraf
\emph{Qual o problema?}
Comparando essa solução com a do item~\refn{subset_count_using_strings},
aqui temos a dificuldade de realmente calcular todos os $n$ números $\comb n i$
para os somar.
O~\ref{sum_of_all_binomial_coefficients} mostre que na verdade,
não é nada dificil calcular o somatório diretamente sem nem
calcular nenhum dos seus termos separadamente!

\proposition.
Seja $A$ conjunto finito.
$$
\card{\pset A} = \sum_{i=0}^n \comb n i,\qquad\text{onde $n = \card A$}.
$$


Combinando as duas proposições chegamos num resultado interessante:

\corollary.
Para todo $n\in\nats$,
$$
\sum_{i=0}^n \binom n i
= \binom n 0 + \binom n 1 + \dotsb + \binom n {n-1} + \binom n n
= 2^n
$$

\exercise.
\label{sum_of_all_binomial_coefficients}
Esqueça o corolário e prove que:
$$
\alignat 2
\sum_{i=0}^n \binom n i
&= \binom n 0 + \binom n 1 + \binom n 2 + \dotsb + \binom n n
&&= 2^n\\
\sum_{i=0}^n (-1)^i \binom n i
&=\binom n 0 - \binom n 1 + \binom n 2 - \dotsb + (-1)^n \binom n n
&&= 0
\endalignat
$$

\hint
Teorema binomial~\refn{binomial_theorem}.

\hint
Cada somatório é apenas um caso especial do teorema binomial.

\hint
Toma $x = y = 1$ no teorema para resolver o primeiro.

\hint
Toma $x = 1$ e $y = -1$ para resolver o segundo.

\endexercise

\endsection
%%}}}

%%{{{ Pascal's triangle 
\section O triângulo de Pascal.

\note As primeiras potências do binomial.
Calculamos:
$$
\align
(x+y)^0 &= 1\\
(x+y)^1 &= x   + y\\
(x+y)^2 &= x^2 + 2 xy    + y^2\\
(x+y)^3 &= x^3 + 3 x^2y  + 3 xy^2     + y^3\\
(x+y)^4 &= x^4 + 4 x^3y  + 6 x^2y^2   + 4xy^3      + y^4\\
(x+y)^5 &= x^5 + 5 x^4y  + 10 x^3y^2  + 10 x^2y^3  + 5 xy^4     + y^5\\
(x+y)^6 &= x^6 + 6 x^5y  + 15 x^4y^2  + 20 x^3y^3  + 15 x^2y^4  + 6 xy^5    + y^6\\
(x+y)^7 &= x^7 + 7 x^6y  + 21 x^5y^2  + 35 x^4y^3  + 35 x^3y^4  + 21 x^2y^5 + 7 xy^6    + y^7\\
(x+y)^8 &= x^8 + 8 x^7y  + 28 x^6y^2  + 56 x^5y^3  + 70 x^4y^4  + 56 x^3y^5 + 28 x^2y^6 + 8 xy^7 + y^8
\endalign
$$

\note O triângulo de Pascal.
\tdefined{triângulo}[de Pascal]
Tomando os coeficientes em cima criamos o triângulo seguinte,
conhecido como \emph{triângulo de \Pascal[triângulo]{}Pascal}:%
$$
\matrix
\format
\;\c\; &
\;\c\; &
\;\c\; &
\;\c\; &
\;\c\; &
\;\c\; &
\;\c\; &
\;\c\; &
\;\c\; &
\;\c\; &
\;\c\; \\
1\\
1&1\\
1&2&1\\
1&3&3&1\\
1&4&6&4&1\\
1&5&10&10&5&1\\
1&6&15&20&15&6&1\\
1&7&21&35&35&21&7&1\\
1&8&28&56&70&56&28&8&1\\
\;\vdots\;&
\phantom{00}&
\phantom{00}&
\phantom{00}&
\phantom{00}&
\phantom{00}&
\phantom{00}&
\phantom{00}&
\phantom{00}&
\ddots
\endmatrix
$$

Observando o triângulo, percebemos que com umas exceções---quais?---cada
número é igual à soma de dois números \emph{na linha em cima}:
aquele que fica na mesma posição, e aquele que fica na posição anterior.
(Consideramos os ``espaços'' no triangulo como se fossem $0$'s.)

\exercise.
Escreva essa relação formalmente.

\hint
O $\binom a b$ está na linha $a$, na posição $b$
(começando contar com 0).

\solution
Temos as equações:
$$
\left.
\aligned
\binom 0 0          &= 1\\
\binom 0 r          &= 0\\
\binom n r          &= \binom {n-1} r + \binom {n-1} {r-1}
\endaligned
\right\}
\quad
\text{equivalentemente}
\quad
\left\{
\aligned
\binom 0     0      &= 1\\
\binom 0     {r+1}  &= 0\\
\binom {n+1} {r+1}  &= \binom n {r+1} + \binom n r.
\endaligned
\right.
$$

\endexercise

\theorem.
\label{combinations_recursive_equation}
Para todos inteiros positivos $n$ e $r$ positivos temos:
$$
\comb n r           = \comb {n-1} r + \comb {n-1} {r-1}.
$$
\sketch.
Lembramos que $\comb n r$ é o número das maneiras que podemos
escolher $r$ de $n$ objetos.
Fixe um dos $n$ objetos e denota-lo por $s$,
para agir como ``separador'':
separamos as maneiras de escolher em dois grupos:
aquelas que escolhem (entre outros) o $s$ e aquelas que não o escolhem.
Contamos cada colecção separadamente e somamos (princípio da adição)
para achar o resultado:
$$
\comb n r
= \underbrace{\comb {n-1} r}_{\text{escolhas sem $s$}}
+ \underbrace{\comb {n-1} {r-1}}_{\text{escolhas com $s$}}.
$$
\qes

\exercise.
Prove o~\ref{combinations_recursive_equation} para todos os $n,r\in\nats$
com $0<r<n$, usando como definição do símbolo $\comb n r$ a
$$
\comb n r = \frac {n!} {(n-r)!\stimes r!}.
$$

\hint
Prove diretamente usando apenas a definição de fatorial.

\solution
Sejam $r,n\in\nats$ com $0<r<n$.
Calculamos:
$$
\alignat 4
\intertext{$\comb n r = \comb {n-1} r + \comb {n-1} {r-1}$}
&\iff\quad& \frac {n!} {(n-r)!\stimes r!} &= \frac {(n-1)!} {(n-1-r)!\stimes r!}              &\!\!{}+{}\ & \frac {(n-1)!} {(n-1-(r-1))!\stimes (r-1)!} &\qquad&\qquad\\
&\iff\quad& n! &= \frac {(n-1)!\stimes(n-r)!\stimes r!} {(n-r-1)!\stimes r!}                  &\!\!{}+{}\ & \frac {(n-1)!\stimes(n-r)!\stimes r!} {(n-r)!\stimes (r-1)!}\\
&\iff\quad& n! &= \frac {(n-1)!\stimes(n-r)!\stimes\cancel{r!}} {(n-r-1)!\stimes \cancel{r!}} &\!\!{}+{}\ & \frac {(n-1)!\stimes\cancel{(n-r)!}\stimes r!} {\cancel{(n-r)!}\stimes (r-1)!}\\
&\iff\quad& n! &= \frac {(n-1)!\stimes(n-r)\!\cancel{\stimes!\stimes}} {\cancel{(n-r-1)!}}    &\!\!{}+{}\ & \frac {(n-1)!\stimes r\cancel{\stimes!\stimes}} {\cancel{(r-1)!}}\\
&\iff\quad& n! &= {(n-1)!\stimes(n-r)} &\!\!{}+{}\ & {(n-1)!\stimes r} \\
&\iff\quad& n! &= {(n-1)!\stimes((n-r) + r)}\\
&\iff\quad& n! &= {(n-1)!\stimes n}\\
&\iff\quad& n! &= n!
\endalignat
$$

\endexercise

\exercise.
\emph{Redefina} o símbolo $\comb n r$ para todo $n,r\in\nats$
recursivamente com
$$
\align
\comb 0 0          &= 1\\
\comb 0 r          &= 0\\
\comb n r          &= \comb {n-1} r + \comb {n-1} {r-1},
\endalign
$$
e prove que para todo $n,r\in\nats$,
$\comb n r = \dfrac {n!} {(n-r)!\stimes r!}$\,.

\hint
Indução.

\endexercise

\endsection
%%}}}

%%{{{ Counting recursively 
\section Contando recursivamente.

\exercise.
\label{sequences_of_twos_and_threes}
Defina uma função $f : \nats\to\nats$ que conta as seqüências feitas por os números 2 e 3 com soma sua entrada.
Quantas seqüências de $2$'s e $3$'s existem cujos termos somam em $17$?

\hint
Recursão.

\hint
Separe as seqüências em dois grupos: aquelas que começam com 2, e aquelas que começam com 3.

\hint
Cuidado com a ``base'' $f(0)$.  Quantas seqüências de 2 e 3, somam em $0$?

\hint
Para calcular o valor de $f(17)$, \emph{não} use a definição recursiva ``top-down'', mas ``bottom-up'': calcule os valores em seqüência linear
$f(0), f(1), f(2), \dotsc$ até o valor desejado.

\endexercise

\exercise.
\label{sequences_of_twos_and_threes_restricted}
Defina uma função $g : \nats\to\nats$ que conta as seqüências feitas por os números 2 e 3 com soma sua entrada, em quais aparecem os dois números (2 e 3).
Quantas seqüências de $2$'s e $3$'s existem cujos termos somam em $18$?

\hint
Use a $f$ do~\ref{sequences_of_twos_and_threes}.

\hint
Quando $g(n) \neq f(n)$?

\hint
Considere os casos:
(1) $n$ não pode ser escrito nem como $n = 2 + 2 + \dotsb + 2$, nem como $n = 3 + 3 + \dotsb + 3$;
(2) $n$ pode ser escrito como $n = 2 + 2 + \dotsb + 2$, e como $n = 3 + 3 + \dotsb + 3$ também;
(3) nenhum dos casos (1)--(2).

\hint
$
g(n) =
\knuthcases{
\cdots\vphantom{f(n)}\cr
\cdots\vphantom{f(n)}\cr
\cdots\vphantom{f(n)}
}
$

\endexercise

\exercise Dirigindo na cidade infinita (sem destino).
\label{infinite_city_1}
No ``meio'' duma ``cidade infinita'', tem um motorista no seu carro.
Seu carro tá parado numa intersecção onde tem 3 opções:
virar esquerda; dirigir reto; virar direita.
No seu depósito tem $a$ unidades de combustível,
e sempre gasta $1$ para dirigir até a próxima intersecção.
De quantas maneiras diferentes ele pode dirigir até seu combustível acabar?
(Veja na figura, dois caminhos possíveis com $a=12$.)
\noindent
\midinsert
\noindent
\centerline{
\hfill
\tikzpicture[scale=0.666]%%{{{
%
\foreach \i in {-4,-3,-2,-1,0,1,2,3,4}
  \draw [-] (\i,-2.5) -- (\i,4.5);
\foreach \j in {-2,-1,0,1,2,3,4}
  \draw [-] (-4.5,\j) -- (4.5,\j);
\draw[rounded corners,line width=2mm,color=blue!40] (0,0) -- (0,1) -- (0,2) -- (-1,2) -- (-1,1) -- (-1,0) -- (-1,-1) -- (-2,-1) -- (-3,-1) -- (-3,0) -- (-3,1) -- (-2,1) -- (-2,2);
\draw[rounded corners,line width=2mm,color=green!40] (0,0) -- (1,0) -- (2,0) -- (3,0) -- (3,1) -- (3,2) -- (3,3) -- (3,4) -- (4,4) -- (4,3) -- (4,2) -- (3,2) -- (2,2);
\node[circle,fill=gray!20] (CR)  at (0,0) {$C$};
%
\endtikzpicture
%%}}}
\hfill
\tikzpicture[scale=0.666]%%{{{
%
\foreach \i in {-4,-3,-2,-1,0,1,2,3,4}
  \draw [-] (\i,-2.5) -- (\i,4.5);
\foreach \j in {-2,-1,0,1,2,3,4}
  \draw [-] (-4.5,\j) -- (4.5,\j);
\draw[rounded corners,line width=2mm,color=blue!40] (0,0) -- (-1.8,0) -- (-1.8,2);
\draw[rounded corners,line width=2mm,color=cyan!60] (0,0) -- (4,0) -- (4,2) -- (-3, 2) -- (-3,3) -- (-2,3) -- (-2,2);
\draw[rounded corners,line width=2mm,color=green!40] (0,0) -- (0,-1) -- (-2.1,-1) -- (-2.1,2);
\node[circle,fill=gray!20] (CR)  at (0,0) {$C$};
\node[circle,fill=gray!20] (DN)  at (-2,2) {$D$};
%
\endtikzpicture
%%}}}
\hfill
}
%\caption{Fig.~1}
\endgraf\centerline{Caminhos possíveis para os exercísios~\refn{infinite_city_1} e~\refn{infinite_city_2} respeitivamente.}
%\endcaption
\endinsert

\hint
Sem recursão!

\hint
Em cada intersecção tem $3$ opções: $\mathtt L$, $\mathtt F$, $\mathtt R$.

\solution
Em cada das $a$ intersecções que ele encontra ele tem $3$ opções.
Logo, ele pode seguir $3^a$ caminhos diferentes dirigindo até seu combustível acabar.

\endexercise

\exercise Dirigindo na montanha infinita.
\label{infinite_mountain}
No ``meio'' duma montanha de altura infinita, tem um motorista no seu carro.
Seu carro tá parado numa intersecção onde tem 4 opções:
dirigir subindo (gasta $4$ unidades de combutível);
dirigir descendo (gasta $1$);
dirigir na mesma altura clockwise (gasta $2$);
dirigir na mesma altura counter-clockwise (gasta $2$).
No seu depósito tem $a$ unidades de combistível.
De quantas maneiras diferentes ele pode dirigir até seu combustível acabar?
\ignore{
\midinsert
\tikzpicture[scale=2]
\axis[
hide axis,
domain=0:1,
y domain=0:-2*pi,
xmin=-1.5, xmax=1.5,
ymin=-1.5, ymax=1.5, zmin=-1.2, zmax=-1.2,
samples=10,
samples y=40,
z buffer=sort,
]
\addplot3[mesh,gray]
({1.1*x*cos(deg(y))},{1.1*x*sin(deg(y))},{-x});
\endaxis
\endtikzpicture
\endinsert
}

\hint
Recursão.

\hint
Seja $f(a)$ o número de caminhos diferentes que o motorista pode seguir com $a$ unidades de combustível.

\hint
Separa todos os caminhos possíveis em 3 grupos, dependendo na primeira escolha do motorista.
Conta o número de caminhos em cada grupo separadamente (recursivamente!),
e e use o princípio da adição para contar quantos são todos.

\endexercise

\exercise Dirigindo na cidade infinita (com destino).
\label{infinite_city_2}
No ``meio'' duma ``cidade infinita'', tem um motorista no seu carro.
Seu carro tá parado numa intersecção onde tem 4 opções:
dirigir na direção do norte; do leste; do sul; do oeste.
No seu depósito tem $c$ unidades de combustível
e sempre gasta $1$ para dirigir até a próxima intersecção.
De quantas maneiras diferentes ele pode dirigir até chegar no teu destino,
que fica numa distância $y$ unidades para norte e $x$ para leste?
Considere que números negativos representam descolamento para a direção oposta.
(Veja na figura onde o carro $C$ tem destino $D$, ou seja, seu descolamento
desejado é de $x=-2$, $y=2$.)

\hint
Recursão.

\hint
Seja $f(a,x,y)$ o número de caminhos diferentes que acabam com descolamento total de $y$ unidades para norte e $x$ para leste, para um motorista que tem $a$ unidades de combustível no seu carro.

\hint
Separa todos os caminhos possíveis em 4 grupos, dependendo na primeira escolha do motorista.
Conta o número de caminhos em cada grupo separadamente (recursivamente!),
e e use o princípio da adição para contar quantos são todos.

\endexercise

\endsection
%%}}}

%%{{{ Solutions of equations in integers 
\section Soluções de equações em inteiros.

\endsection
%%}}}

%%{{{ Combinations with repetitions 
\section Combinações com repetições. 

\endsection
%%}}}

%%{{{ The inclusion--exclusion principle 
\section O princípio da inclusão--exclusão.
\label{Inclusion_exclusion_principle}%

\exercise.
42 passangeiros estão viajando num avião.
\beginul
\li 11 deles não comem beef.
\li 10 deles não comem peixe.
\li 12 deles não comem frango.
\li Os passangeiros que não comem nem beef nem frango são 6.
\li O número de passangeiros que não comem nem beef nem peixe, é o mesmo com o número de passangeiros que não comem nem peixe nem frango.
\li Os passangeiros que não comem nada disso são 3.
\li Os passangeiros que comem tudo são 22.
\endul
\endgraf
\noindent
Quantos são os passangeiros que não comem nem beef nem peixe?

\endexercise

\endsection
%%}}}

%%{{{ Elementary probability 
\section Probabilidade elementar.

\endsection
%%}}}

%%{{{ Derrangements 
\section Desarranjos.

\endsection
%%}}}

%%{{{ The pigeonhole principle 
\section O princípio da casa dos pombos.

\endsection
%%}}}

%%{{{ Generating functions and recurrence relations 
\section Funções geradoras e relações de recorrência.

\endsection
%%}}}

%%{{{ Problems 
\problems.

%%{{{ prob 
\problem.
Uma turma de 28 alunos tem 12 mulheres e 16 homens.
\beginol
\li De quantas maneiras podemos escolher 5 desses alunos, para formar um time de basquete?
(Considere que as posições de basquete não importam).
\li De quantas maneiras podemos escolher 6 desses alunos, para formar um time de volei, tal que o time tem pelo menos 4 homens?
(Considere que as posições de volei não importam).
\li De quantas maneiras podemos escolher 11 desses alunos, para formar um time de futebol, tal que o time tem exatamente 3 mulheres, e um homem para goleiro?
(Considere que a única posição de futebol que importa é do goleiro.)
\li De quantas maneiras podemos escolher 3 times, um para cada esporte, sem restrição de sexo?
\endol

\endproblem
%%}}}

%%{{{ prob 
\problem.
Uma noite, depois do treino 3 times (uma de basquete, uma de vólei, e uma de
futebol), foram beber num bar que foi reservado para eles.
Como os jogadores de cada time querem sentar juntos,
o dono arrumou duas mesas cíclicas, uma com 5 e outra com 6 cadeiras, e 11 cadeiras no bar.
\endgraf
De quantas maneiras diferentes eles podem sentar?
(Considere que nas mesas cíclicas o que importa é apenas quem tá no lado de quem, mas no bar o que importa é a posição da cadeira mesmo.)

\endproblem
%%}}}

%%{{{ prob 
\problem.
Considere os inteiros $1,2,\dotsc, 30$.
Quantas das suas $30!$ permutações totais têm a propriedade que
não aparecem múltiplos de $3$ consecutivamente?

\hint
Construa cada configuração em passos e use o princípio da multiplicação.

\hint
Coloque os não-múltiplos de 3 primeiramente numa ordem, deixando espaços entre-si
para os múltiplos de 3.

\hint
Escolhe 10 dos 21 lugares possíveis para colocar os múltiplos de 3.

\solution
Primeiramente vamos esqueçer os múltiplos de 3.
O resto dos (20) números pode ser permutado de
$20!$ maneiras.
Para qualquer dessa maneira, temos $\comb {21} {10}$
opções para escolher em quais $10$ das $20+1$ possíveis posições vamos colocar os múltiplos de 3,
e para cada escolha, correspondem $10!$ diferentes permutações dos múltiplos de 3 nessas $10$ posições.
Finalmente,
$$
\underbrace{\phantom(20!\phantom)}_{\text{ordena os não-múltiplos}}\ntimes \underbrace{\comb {21} {10}}_{\text{escolhe as posições dos múltiplos}}\ntimes \underbrace{\phantom(10!\phantom)}_{\text{escolhe a ordem dos múltiplos}}
$$
das $30!$ permutações têm a propriedade desejada.

\endproblem
%%}}}

%%{{{ prob 
\problem.
Numa turma de $28$ alunos
precisamos formar duas comissões de $5$ e $6$ membros.
Cada comição tem seu presidente, seu vice-presidente, e seus membros normais.
De quantas maneiras podemos formar essas comissões\dots
\beginol
\li\dots sem restrições (cada um aluno pode participar nas duas comissões simultaneamente)?
\li\dots se nenhum aluno pode participar simultaneamente nas duas comissões?
\li\dots se os dois (únicos) irmãos entre os alunos não podem participar na mesma comissão,
e cada aluno pode participar simultaneamente nas duas?
\endol

\endproblem
%%}}}

%%{{{ prob: nikos 
\problem.
\label{nikos}
Na figura abaixo temos um mapa (as linhas correspondem em ruas).
Nikos quer caminhar do ponto $A$ para o ponto $B$, \emph{o mais rápido possível}.
\beginol
\li De quantas maneiras ele pode chegar?
\li Se ele precisa passar pelo ponto $S$?
\li Se ele precisa passar pelo ponto $S$ mas quer evitar o ponto $N$?
\endol
\noindent
\midinsert
\noindent
\tikzpicture[scale=0.666]%%{{{
%
\foreach \i in {0,1,2,3,4,5,6,7,8,9}
  \foreach \j in {0,1,2,3,4,5,6,7,8}
    \node (a\i) at (\i,\j) {};
\foreach \i in {0,1,2,3,4,5,6,7,8,9}
  \draw [-] (\i,0) -- (\i,8);
\foreach \j in {0,1,2,3,4,5,6,7,8}
  \draw [-] (0,\j) -- (9,\j);
\draw [rounded corners,line width=2mm,color=green!50] (0,0) -- (0,2) -- (3,2) -- (3,4) -- (7,4) -- (7,5) -- (8,5) -- (8,7) -- (9,7) -- (9,8);
\node[circle,fill=gray!20]  (SW)    at (-0.3,-0.3) {$A$};
\node[circle,fill=gray!20]  (NE)    at (9.3,8.3)   {$B$};
\node[circle,             inner sep=2pt,fill=green!30] (nice)  at (3,2.5)     {{\niness S}};
\node[star,star points=17,inner sep=2pt,fill=red!30]   (boom)  at (6.5,5)     {{\niness N}};
%
\endtikzpicture
%%}}}
\tikzpicture[scale=0.666]%%{{{
%
\foreach \i in {0,1,2,3,4,5,6,7,8,9}
  \foreach \j in {0,1,2,3,4,5,6,7,8}
    \node (a\i) at (\i,\j) {};
\foreach \i in {0,1,2,3,4,5,6,7,8,9}
  \draw [-] (\i,0) -- (\i,8);
\foreach \j in {0,1,2,3,4,5,6,7,8}
  \draw [-] (0,\j) -- (9,\j);
\draw [rounded corners,line width=2mm,color=red!50] (0,0) -- (0,2) -- (3,2) -- (3,4) -- (5,4) -- (5,5) -- (7,5) -- (7,7) -- (8,7) -- (8,8) -- (9,8);
\node[circle,fill=gray!20] (SW)  at (-0.3,-0.3) {$A$};
\node[circle,fill=gray!20] (NE)  at (9.3,8.3) {$B$};
\node[circle,             inner sep=2pt,fill=green!30] (nice)  at (3,2.5)     {{\niness S}};
\node[star,star points=17,inner sep=2pt,fill=red!30]   (boom)  at (6.5,5)     {{\niness N}};
%
\endtikzpicture
%%}}}
%\caption{Fig.~1}
\endgraf\centerline{Um caminho aceitável e um inaceitável no caso (3) do \ref{nikos}.}
%\endcaption
\endinsert

\endproblem
%%}}}

%%{{{ prob 
\problem.
Num jogo de lotéria, tem os números de $1$ até $60$:
$$
\matrix
    01&02& 03 &04 &05 &06 &07 &08 &09 &10\\
    11&12& 13 &14 &15 &16 &17 &18 &19 &20\\
    21&22& 23 &24 &25 &26 &27 &28 &29 &30\\
    31&32& 33 &34 &35 &36 &37 &38 &39 &40\\
    41&42& 43 &44 &45 &46 &47 &48 &49 &50\\
    51&52& 53 &54 &55 &56 &57 &58 &59 &60
\endmatrix
$$
Os organizadores do jogo, escolhem aleatoriamente 6 números
deles (sem repetições).
Esses 6 números são chamados ``a megasena''.
Um jogador marca pelo menos 6 números na sua
lotéria e se conseguir ter marcados todos os 6 da megasena, ganha.

(Marcando mais que 6 números,
as chances do jogador aumentam, mas o preço da lotéria aumenta também.)

(1)
Um jogador marcou $6$ números.
Qual a probabilidade que ele ganhe?

(2)
Uma jogadora marcou $9$ números.
Qual a probabilidade que ela ganhe?

(3)
Generalize para um jogo com $N$ números, onde $w$ deles são escolhidos,
e com um jogador que marcou $m$ números, sendo $w\leq m \leq N$.

\solution
(1)
Apenas uma escolha é a certa, então a probabilidade de ganhar é:
    $$
    \dfrac 1 {\comb {60} 6}
    = \dfrac{6!\stimes 54!} {60!}
    = \dfrac {6!} {55 \ntimes 56 \ntimes 57 \ntimes 58 \ntimes 59 \ntimes 60}
    = \dfrac {1} {11\ntimes 14\ntimes 19\ntimes 29\ntimes 59\ntimes 10}
    = \dfrac 1 {50063860}\,.
    $$
\endgraf
\noindent
(2)
Para ganhar, com certeza acertamos nos 6 números da megasena,
então temos que contar todas as maneiras de escolher os 3 outros números dos 9 que escolhemos:
    $$
    \dfrac {\comb {60-6} {9-6}} {\comb {60} 9}
    = \dfrac {\comb {54} 3} {\comb {60} 9}
    = \dfrac {54!\stimes \cancel{51!}\stimes 9!} {\cancel{51!} \stimes 3!\stimes 60!}
    = \dfrac {4\ntimes 5 \ntimes 6 \ntimes 7 \ntimes 8 \ntimes 9} {55\ntimes 56\ntimes 57\ntimes 58\ntimes 59 \ntimes 60}
    = \dfrac 3 {1787995}\,.
    $$
\endgraf
\noindent
(3)
Generalizando a solução do (2), a probabilidade é
$$
\align
\frac
{\comb {N-w} {m-w}}
{\comb N m}
&=
\frac
{(N-w)! \stimes (N-m)! \stimes m!}
{((N-w)-(m-w))! \stimes (m-w)! \stimes N!}\\
&=
\frac
{(N-w)! \stimes (N-m)! \stimes m!}
{((N\cancel{{}-{}w}-m\cancel{{}+w}))! \stimes (m-w)! \stimes N!}\\
&=
\frac
{(N-w)! \stimes \cancel{(N-m)!} \stimes m!}
{\cancel{(N-m)!} \stimes (m-w)! \stimes N!}\\
&=
\frac
{(N-w)! \stimes m!}
{(m-w)! \stimes N!}\\
&=
\prod_{i=0}^{w-1}
\frac
{m-i}
{N-i}\,.
\endalign
$$

\endproblem
%%}}}

%%{{{ prob: frogs 
\problem.
\label{frogs}
Aleco e Bego são dois sapos.
Eles estão na frente de uma escada com 11 degraus.
No 6o degrau, tem Cátia, uma cobra, com fome.
Aleco pula 1 ou 2 degraus para cima.
Bego, 1, 2 ou 3.  E ele é tóxico: se Cátia o comer, ela morre na hora.
\midinsert
\tikzpicture[scale=0.666]%%{{{
%
\node[star,star points=17,fill=red!30,inner sep=3pt]    (boom) at (6.75,6.25) {\phantom{\niness C}};
\node[                                             ]    (catia) at (6.666,6.333) {{\niness C}};
\node[circle,fill=green!40]  (aleco) at (-0.333,0.5) {{\niness A}};
\node[circle,fill=blue!30]   (bego)  at (-1.75,0.5) {{\niness B}};
\draw [->,color=green!50,line width=1mm] (aleco) to [bend left=70] (2.333,2);
\draw [->,color=blue!50,line width=1mm] (bego)  to [bend left=66] (1.666,1);
\draw [->,color=green!50,line width=1mm] (aleco) to [bend left=60] (1.333,1);
\draw [->,color=blue!50,line width=1mm] (bego)  to [bend left=62] (2.666,2);
\draw [->,color=blue!50,line width=1mm] (bego)  to [bend left=60] (3.5,3);
\foreach \i in {1,2,3,4,5,6,7,8,9,10,11} {
  \path[fill=gray!10]
    (\i,\i) -- (\i,\i-1) -- (15,\i-1) -- (15,\i) -- (\i,\i);
  \node[circle,fill=black,inner sep=0pt] (b\i) at (\i,\i)   {};
  \node[circle,fill=black,inner sep=0pt] (e\i) at (\i+1,\i) {};
  \node[circle,fill=black,inner sep=0pt] (d\i) at (\i,\i-1) {};
  \node[                  inner sep=0pt] (s\i) at (\i+0.5,\i) {};
  \draw [-,line width=0.2mm] (b\i) -- (e\i);
  \draw [-,line width=0.2mm] (b\i) -- (d\i);
  \node[                  inner sep=0pt] (t\i) at (\i+0.5,\i-0.3) {\i};
}
\draw [-] (-3,0) -- (1,0);
\draw [-] (12,11) -- (15,11);
%\foreach \j in {0,1,2,3,4,5,6,7,8}
%  \draw [-] (0,\j) -- (9,\j);
%  \draw [line width=2mm,color=green!50] (0,0) -- (0,2) -- (3,2) -- (3,4) -- (7,4) -- (7,5) -- (8,5) -- (8,7) -- (9,7) -- (9,8);
%\node[circle,fill=gray!20]  (SW)    at (-0.3,-0.3) {$A$};
%\node[circle,fill=gray!20]  (NE)    at (9.3,8.3)   {$B$};
%\node[circle,fill=green!30] (nice)  at (3,2.5)     {{\niness S}};
%\node[circle,fill=red!30]   (boom)  at (6.5,5)     {{\niness N}};
%
\endtikzpicture
%%}}}
%\caption{Fig.~2}
\endgraf\centerline{Os dois sapos do~\ref{frogs} e suas possibilidades para começar.}
%\endcaption
\endinsert
\item{(1)} Por enquanto, Cátia está dormindo profundamente.
\itemitem{a.} De quantas maneiras Aleco pode subir a escada toda?
\itemitem{b.} De quantas maneiras Bego pode subir a escada toda?
\item{(2)} Cátia acordou!
\itemitem{a.} De quantas maneiras Aleco pode subir a escada toda?
\itemitem{b.} De quantas maneiras Bego pode subir a escada toda?
\item{(3)} Bego começou subir a escada\dots{}
Qual é a probabilidade que Cátia morra?
(Considere que antes de começar,
ele já decidiu seus saltos e não tem percebido a existência da cobra.)
\ignore{
\item{(4)} O que muda na questão (3) se ao invés de decidir seu caminho desde o
início, Bega decida cada vez aleatoriamente (com probabilidades iguais) qual
dos 3 possíveis saltos ele vai fazer?
\item{(5)}
Generalize o problema (4)~para o caso onde a escada
tem uma infinidade de degraus e Cátia fica no degrau $k$.
}

\hint Recursão.

\hint
Sejam $a(n)$ e $b(n)$ o número de maneiras que Aleco e Bego
podem subir uma escada de $n$ degraus, respeitivamente.

\hint
Grupe as maneiras em colecções (para aplicar o princípio da adição),
olhando para o primeiro salto.

\solution
Sejam $a(n)$ e $b(n)$ o número de maneiras que Aleco e Bego
podem subir uma escada de $n$ degraus, respeitivamente.
Cada maneira do Aleco pode começar com 2 jeitos diferentes:
salto de 1 degrau, ou salto de 2 degraus.
Cada maneira do Bego pode começar com 3 jeitos diferentes:
salto de 1, de 2, ou de 3 degraus.
Observe que, por exemplo, se Bego começar com um pulo de 2
degraus, falta subir uma escada de $n-2$ degraus.
Pelo princípio da adição então, temos as equações recursivas:
$$
\xalignat 2
a(n) &= a(n-1) + a(n-2)                & b(n) &= b(n-1) +  b(n-2) +  b(n-3)
\intertext{validas para $n\geq 2$ e $n \geq 3$ respeitivamente.
Devemos definir os casos básicos de cada função recursiva:
$n=0,1$ para a $a(n)$, e $n=0,1,2$ para a $b(n)$:}
     &                                 & b(0) &= 1\qquad\explanation{fica}\\
a(0) &= 1 \qquad\explanation{fica}     & b(1) &= 1\qquad\explanation{pula $1$}\\
a(1) &= 1 \qquad\explanation{pula $1$} & b(2) &= 2\qquad\explanation{pula $1+1$; ou pula $2$}\\
a(n) &= a(n-1) +  a(n-2)               & b(n) &= b(n-1) +  b(n-2) +  b(n-3)
\endxalignat
$$
Calculamos os 11 primeiros valores:
$$
\matrix
a:\quad& \overbrace {1}^{a(0)}, & 1, & 2, & 3, & 5, & \phantom08,  & 13, & 21, & 34, & \phantom055,  & \phantom089,  & \overbrace {144}^{a(11)}, &\dotsc\\
b:\quad& \underbrace{1}_{b(0)}, & 1, & 2, & 4, & 7, & 13,          & 24, & 44, & 81, & 149, & 274, & \underbrace{504}_{b(11)}, &\dotsc
\endmatrix
$$
Agora temos tudo que precisamos para responder facilmente nas questões do problema.
\endgraf
\item{(1)} Precisamos apenas os valores $a(11)$ e $b(11)$:
\itemitem{a.} De $a(11) = 144$ maneiras.
\itemitem{b.} De $b(11) = 504$ maneiras.
\item{(2)} Usamos ``$n\to m$'' para ``pula diretamente do degrau $n$ para o degrau $m$'' e ``$n\transto m$'' para ``vai do degrau $n$ para o degrau $m$ pulando num jeito''.
\itemitem{a.} Para conseguir subir, Aleco necessariamente precisa chegar no degrau 5, saltar até o degrau 7, e depois continuar até o degrau 11.  Formamos cada maneira então em passos, e usando o princípio da multiplicação achamos que Aleco tem
$$
\underbrace{a(5)}_{0 \transto 5} \ntimes
\underbrace{\phantom(1\phantom)}_{5 \to 7} \ntimes
\underbrace{a(4)}_{7 \transto 11}
=
8 \ntimes 1 \ntimes 5
= 40
$$
maneiras de subir a escada toda.
\itemitem{b.} 
Para o Bego a situação não é tão simples, porque ele pode evitar a cobra de vários jeitos.
Vamos grupar eles assim:
(i)   aqueles onde ele pulou a cobra com salto de tamanho 2;
(ii)  aqueles onde ele pulou a cobra com salto de tamanho 3 desde o degrau 5;
(iii) aqueles onde ele pulou a cobra com salto de tamanho 3 desde o degrau 4.
Contamos as maneiras em cada grupo como na questão anterior, e no final as somamos (princípio da adição) para achar a resposta final: Bego tem
$$
\overbrace{
\underbrace{b(5)}_{0 \transto 5} \ntimes
\underbrace{\phantom(1\phantom)}_{5 \to 7} \ntimes
\underbrace{b(4)}_{7 \transto 11}
}^{\text{grupo (i)}}
+
\overbrace{
\underbrace{b(5)}_{0 \transto 5} \ntimes
\underbrace{\phantom(1\phantom)}_{5 \to 8} \ntimes
\underbrace{b(3)}_{8 \transto 11}
}^{\text{grupo (ii)}}
+
\overbrace{
\underbrace{b(4)}_{0 \transto 4} \ntimes
\underbrace{\phantom(1\phantom)}_{4 \to 7} \ntimes
\underbrace{b(4)}_{7 \transto 11}
}^{\text{grupo (iii)}}
=
13 \ntimes 7 + 
13 \ntimes 4 + 
7 \ntimes 7 
=
192
$$
maneiras de subir a escada toda.
\item{(3)}
Pela definição, a probabilidade que Cátia morra é a fracção
$$
\frac
{\text{todas as maneiras em quais Bego pisou no degrau 6}}
{\text{todas as maneiras possíveis}}\,,
$$
ou seja,
$$
\frac
{504-192}
{504}
=
\frac
{312}
{504}
=
\frac
{156}
{252}
=
\frac
{78}
{126}
=
\frac
{39}
{63}
=
\frac
{13}
{21}
\,.
$$

\endproblem
%%}}}

%%{{{ prob: band_maker 
\problem.
\label{band_maker}%
Temos $6$ músicos disponíveis, onde cada um toca:
\endgraf
\halign{
\hfil# & #\hfil & \hfil# & #\hfil\cr
Alex:       &violão, guitarra, baixo& Daniel:     &guitarra\cr
Bill:       &bateria                & Eduardo:    &piano, teclado, violão, fláuto\cr
Claudia:    &saxofone, clarineto    & Fagner:     &guitarra, baixo, teclado\cr
}

\noindent
(Considere que uma banda precisa \emph{pelo menos um membro},
todos os membros duma banda \emph{precisam tocar pelo menos algo na banda},
e que cada banda é diferenciada pelos músicos e
suas funções.
Por exemplo: uma bande onde Alex toca o violão (apenas) e Bill a bateria,
é diferente duma banda onde
Alex toca o violão \emph{e} a guitarra, e Bill a bateria,
mesmo que seus membros podem ser os mesmos.

\item{(1)}
Quantas bandas diferentes podemos formar?
\item{(2)}
Quantas bandas diferentes podemos formar com a restrição que nenhum
músico tocará mais que um instrumento na banda (mesmo se em geral sabe tocar mais)?
\item{(3)}
Quantas bandas diferentes podemos formar onde todos os
músicos fazem parte da banda?

\solution
(1):
$2^{14}-1$: para cada músico e cada instrumento, temos 2 opções: ``sim'' ou ``não''.
Tiramos $1$ porque hipercontamos (a ``banda vazia'').
\endgraf
\medskip
\noindent
(2):
Cada músico que toca $i$ instrumentos tem $i+1$ opções (a extra $+1$ corresponde no ``não participar na banda''):
podemos formar $4 \ntimes 2 \ntimes 3 \ntimes 2 \ntimes 5 \ntimes 4 - 1$ bandas, onde de novo tiramos 1 para excluir a ``banda vazia''.
\endgraf
\medskip
\noindent
(3):
Cada músico que toca $i$ instrumentos tem $2^i - 1$ opções (tirando a opção de ``não tocar nada'').  Então podemos formar $7\ntimes 1 \ntimes 3 \ntimes 1 \ntimes 15 \ntimes 7$ bandas.

\endproblem
%%}}}

%%{{{ prob 
\problem.
De quantas maneiras podemos escrever um string ternário
(usando o alfabeto $\set{0, 1, 2}$)
de tamanho 7,
tais que \emph{não aparece neles o substring $00$}.
\endgraf
Por exemplo:
$$
\align
0112220                                           &\qquad\text{é um string aceitável;}\\
2\underline{00}1\underline{0\overline0}\overline0 &\qquad\text{não é.}
\endalign
$$

\hint
Recursão.

\hint
Seja $a(n)$ o número dos strings ternários de tamanho $n$ tais que não aparece
neles o substring ${00}$.

\hint
Defina a $a(n)$ e depois calcule o $a(7)$, calculando em ordem os
$a(0),a(1),\dotsc,a(7)$.

\solution
Seja $a(n)$ o número dos strings ternários de tamanho $n$ tais que não aparece
neles o substring ${00}$.
Queremos achar o $a(7)$.
\endgraf
Observe que:
$$
\align
    a(0) &= 1 \qqqquad\explanation{o string vazio: ``$\,$''}\\
    a(1) &= 3 \qqqquad\explanation{os strings: ``$0$'', ``$1$'', e ``$2$''}\\
    a(n) &=
      \underbrace{a(n-1)}_{1\ldots}
    + \underbrace{a(n-1)}_{2\ldots}
    + \underbrace{a(n-2)}_{{01}\ldots}
    + \underbrace{a(n-2)}_{{02}\ldots}\\
         &= 2a(n-1) + 2a(n-2)\\
         &= 2(a(n-1) + a(n-2))
\endalign
$$
Então calculamos os primeiros $8$ termos da seqüência:
$$
1,\quad 3,\quad 8,\quad 22,\quad 60,\quad 164,\quad 448,\quad \underbrace{1224}_{a(7)}.
$$

\endproblem
%%}}}

%%{{{ prob: pessimissimo 
\problem.
\label{pessimissimo}%
Contar todas as palavras feitas por permutações das 12 letras da palavra
$$
\txt{PESSIMISSIMO}
$$
onde\dots
\item{(1)}
A palavra começa com $\txt P$.
\item{(2)}
Todos os $\txt I$ aparecem \emph{juntos}.
\item{(3)}
Os $\txt M$ aparecem \emph{separados}.
\item{(4)}
Nenhum dos $\txt S$ aparece ao lado de outro $\txt S$.

\hint
Para o (4), seria diferente se a restrição fosse
``os $\txt S$ não aparecem todos juntos''.

\solution
(1):
Como somos obrigados começar a palavra com ${\txt P}$,
precisamos apenas contar as permutações das letras da palavra
$
\txt{ESSIMISSIMO}
$,
que sabemos que são
$$
\frac
{11!}
{4!\stimes3!\stimes 2!}.
$$
\endgraf
\medskip
\noindent
(2):
Podemos considerar que temos apenas um $I$:
$
\dfrac
 {10!}
 {4!\stimes  2!}
$
\endgraf
\medskip
\noindent
(3):
Contamos em quantas palavras eles aparecem juntos,
e usando princípio da adição, os subtraimos das permutações sem restrição.
$$
\underbrace{
\,
\dfrac
 {12!}
 {4!\stimes  3!\stimes  2!}
\,
}_{\text{todas}}
 -
\underbrace{
\,
\dfrac
 {11!}
 {4!\stimes  3!}
\,
}_{\text{$\txt M$ juntos}}
$$
\endgraf
\noindent(4):
Construimos cada dessas palavras em passos.
Primeiramente escolhemos uma das permutações da palavra sem os ${\txt M}$'s:
$$
\phantom{\underline{\phantom{{\txt M}}}}*
\phantom{\underline{\phantom{{\txt M}}}}*
\phantom{\underline{\phantom{{\txt M}}}}*
\phantom{\underline{\phantom{{\txt M}}}}*
\phantom{\underline{\phantom{{\txt M}}}}*
\phantom{\underline{\phantom{{\txt M}}}}*
\phantom{\underline{\phantom{{\txt M}}}}*
\phantom{\underline{\phantom{{\txt M}}}}*
\phantom{\underline{\phantom{{\txt M}}}}
$$
(temos 
$
\dfrac
{8!}
{3!\stimes 2!}
$
opções).
\endgraf
\medskip
No próximo passo escolemos em qual das $9$ posições possíves colocamos os $M$:
$$
\underline{\phantom{{\txt M}}}*
\underline{\phantom{{\txt M}}}*
\underline{\phantom{{\txt M}}}*
\underline{\phantom{{\txt M}}}*
\underline{\phantom{{\txt M}}}*
\underline{\phantom{{\txt M}}}*
\underline{\phantom{{\txt M}}}*
\underline{\phantom{{\txt M}}}*
\underline{\phantom{{\txt M}}}
$$
Pelo princípio da multiplicação então, temos
$
\dfrac
{8!}
{3!\stimes 2!}
\cdot
\comb 9 4
$
palavras que satisfazem essa restrição.

\endproblem
%%}}}

%%{{{ prob 
\problem.
De quantas maneiras podemos escrever um string usando o alfabeto
de 26 letras
$$
\txt A, \txt B, \txt C, \dotsc, \txt X, \txt Y, \txt Z,
$$
tais que as vogais aparecem na ordem estrita alfabética, e as consoantes na órdem oposta?
(As vogais sendo as letras $\txt A$, $\txt E$, $\txt I$, $\txt O$, $\txt U$, $\txt Y$.)
{Por exemplo:}
$$
\align
\txt{TEDUCY}    &\qquad\text{é um string aceitável};\\
\txt{DETUCY}    &\qquad\text{não é ($\txt D \not> \txt T$)};\\
\txt{TEDUCA}    &\qquad\text{não é ($\txt U \not< \txt A$)}.
\endalign
$$
\item{(1)} \dots se os strings são de tamanho 26 e os vogais aparecem todos juntos;
\item{(2)} \dots se os strings são de tamanho 12 e aparecem todos os vogais;
\item{(3)} \dots se os strings são de tamanho 3;
\item{(4)} \dots se os strings são de tamanho $\ell$, com $0\leq\ell\leq 26$.

\solution
(1)
Como a ordem das consoantes e das vogais é predeterminada e as vogais devem aparecer juntas,
a única escolha que precisamos fazer é onde colocar as vogais, e temos $21$ possíveis posições.
Então existem $21$ tais strings.
\endgraf
\noindent
(2)
$$
\underbrace{\comb {20} 6}_{\text{\eightrm consoantes}},
\underbrace{\comb {12} 6}_{\text{\eightrm suas posições}}.
$$
\endgraf
\noindent
(3)
Separamos todos os strings que queremos contar em quatro grupos e contamos cada um separadamente:
$$
{
\overbrace{
\underbrace{\comb {20} 3}_{\text{as c.}}
}^{\text{3 c., 0 v.}}
}
+
{
\overbrace{
\underbrace{\comb {20} 2}_{\text{as c.}}
\underbrace{\comb 6 1}_{\text{a v.}}
\underbrace{\comb 3 2}_{\text{pos.~c.}}
}^{\text{2 c., 1 v.}}
}
+
\overbrace{
\underbrace{\comb {20} 1}_{\text{a c.}}
\underbrace{\comb 6 2}_{\text{as v.}}
\underbrace{\comb 3 1}_{\text{pos.~c.}}
}^{\text{1 c., 2 v.}}
+
\overbrace{
\underbrace{\comb 6 3}_{\text{as v.}}
}^{\text{0 c., 3 v.}}.
$$
Podemos descrever o resultado numa forma mais uniforme e mais fácil para generalizar:
$$
\sum_{i=0}^3
\underbrace{\comb {20} {3-i}}_{\text{as $3-i$ c.}}
\underbrace{\comb 6 i}_{\text{as $i$ v.}}
\underbrace{\comb 3 i}_{\text{pos.~v.}}
=
\sum_{\Sb c+v=3\\ c,v\in\nats\endSb}
\underbrace{\comb {20} c}_{\text{as c.}}
\underbrace{\comb 6 v}_{\text{as v.}}
\underbrace{\comb 3 v}_{\text{pos.~v.}}
$$
\endgraf
\noindent
(4)
Seguindo a última forma do (3), temos
$$
\sum_{i=0}^{\ell}
\underbrace{\comb {20} {\ell-i}}_{\text{as $\ell-i$ c.}}
\underbrace{\comb 6 i}_{\text{as $i$ v.}}
\underbrace{\comb {\ell} i}_{\text{pos.~v.}}
$$
maneiras.  O somatório pode ser escrito também assim:
$$
\sum
\bigg\{
\underbrace{\comb {20} c}_{\text{as c.}}
\underbrace{\comb 6 v}_{\text{as v.}}
\underbrace{\comb {\ell} v}_{\text{pos.~v.}}
\ \Big|\ 
c+v=\ell, \ 0\leq c \leq 20, \ 0\leq v \leq 6, \ c,v\in\nats
\bigg\}.
$$
Note que como o conjunto em cima é finito,
a adição é comutativa e associativa; logo, nosso somatório é bem definido.

\endproblem
%%}}}

%%{{{ prob: roulette_multiple_balls 
\problem.
\label{roulette_multiple_balls}
Numa roleta dum cassino tem ``pockets'' (ou ``casas'') numerados com:
$$
00, 0, 1, 2, \dotsc, 36
$$
e cada um deles é suficientemente profundo para caber até 8 bolinhas.
O crupiê joga 8 bolinas na roleta no mesmo tempo.
De quantas maneiras elas podem cair nos pockets se\dots
\item{(1)}\dots as bolinhas são distintas e não importa sua ordem dentro um pocket.
\item{(2)}\dots as bolinhas são todas iguais.
\endgraf

\solution
\noindent
(1) São permutações com repetições: $38 ^ 8$.
\endgraf
\noindent
(2) São combinações com repetições: $\comb {38 + 8 - 1} 8 = \comb {45} 8$.

\endproblem
%%}}}

%%{{{ prob 
\problem.
De quantas maneiras podemos escrever um string usando
letras do alfabeto $\set{\txt A, \txt B, \txt C, \txt D}$,
tais que \emph{cada letra é usada exatamente duas vezes
mas não aparece consecutivamente no string}?
Por exemplo:
$$
\align
\txt{ABADCDBC}                      &\qquad\text{é um string aceitável;}\\
\txt{ABAC$\underline{\txt{DD}}$BC}  &\qquad\text{não é.}
\endalign
$$

\hint
Inclusão--exclusão.

\hint
Considere as propriedades:
$$
\xalignat 4
 \alpha  &: \text{aparece o $\txt{AA}$}
&\beta   &: \text{aparece o $\txt{BB}$}
&\gamma  &: \text{aparece o $\txt{CC}$}
&\delta  &: \text{aparece o $\txt{DD}$}.
\endxalignat
$$

\solution
Seja $N$ o número de permutações totais das létras
e defina as 4 propriedades
$$
\xalignat 4
 \alpha  &: \text{aparece o $\txt{AA}$}
&\beta   &: \text{aparece o $\txt{BB}$}
&\gamma  &: \text{aparece o $\txt{CC}$}
&\delta  &: \text{aparece o $\txt{DD}$}.
\endxalignat
$$
Procuramos o número dos strings de tamanho 8 que não tenham nenhuma dessas 4 propriedades.
Assim que calcular os $N(\alpha),\dotsc,N(\alpha,\beta,\gamma,\delta)$
o princípio da inclusão--exclusão, vai nos dar o número que procuramos.
\endgraf
Observamos que
$$
\gather
N(\alpha) =N(\beta) =N(\gamma) =N(\delta)\\
N(\alpha,\beta) =N(\alpha,\gamma) = \dotsb = N(\gamma,\delta)\\
N(\alpha,\beta,\gamma) = \dotsb = N(\beta,\gamma,\delta).
\endgather
$$
\endgraf
Calculamos os
$$
\align
N
&= \frac {8!} {2!\stimes 2!\stimes 2!\stimes 2!} = 2520\\
N(\alpha)
&= \frac {7!} {2!\stimes 2!\stimes 2!} = \frac {7!} 8 = 7\ntimes 6 \ntimes 5 \ntimes 3 = 630\\
N(\alpha,\beta)
&= \frac {6!} {2!\stimes 2!} = \frac {6!} 4 = 180 \\
N(\alpha,\beta,\gamma)
&= \frac {5!} {2!} = \frac {5!} 2 = 60 \\
N(\alpha,\beta,\gamma,\delta)
&= {4!} = 24.
\endalign
$$
\endgraf
Para responder, temos
$$
\gather
    N
    - \comb 4 1 N(\alpha)
    + \comb 4 2 N(\alpha,\beta)
    - \comb 4 3 N(\alpha,\beta,\gamma)
    + N(\alpha,\beta,\gamma,\delta)\\
    =
    2520 - 4\ntimes 630 + 6\ntimes 180 - 4\ntimes 60 + 24
    =
    864
\endgather
$$
tais permutações.

\endproblem
%%}}}

%%{{{ prob: parity_respecting_strings_mutual_recursion 
\problem.
\label{parity_respecting_strings_mutual_recursion}
De quantas maneiras podemos escrever um string binário
(usando o alfabeto $\set{0, 1}$) de tamanho 12,
tais que: 
\beginol
\li os $0$'s aparecem apenas em grupos maximais de tamanho par;
\li os $1$'s aparecem apenas em grupos maximais de tamanho impar.
\endol
Por exemplo:
$$
\align
{000000111001}                         &\qquad\text{é um string aceitável;}\\
{100\underline{11}001\underline{000}1} &\qquad\text{não é.}
\endalign
$$

\hint
Separe os strings em dois grupos, aqueles que terminam em $0$ e aqueles que terminam em $1$,
e conta os strings de cada grupo usando recursão.

\hint
Sejam $a(n)$ e $b(n)$ o número de strings binários de tamánho $n$ que terminam em $0$ e em $1$ respectivamente.

\endproblem
%%}}}

%%{{{ prob: xyzzy_lemmings 
\problem.
\label{xyzzy_lemmings}%
\def\MM{\ensuremath{\mathtt{M}}}
\def\FB{\ensuremath{\mathtt{F}}}
\def\ST{\ensuremath{\mathtt{B}}}
Xÿźźÿ o Mago Bravo decidiu matar todos os lemmings que ele guarda no seu quintal.
Seus feitiços são os:
\beginul
\li ``magic missile'', que mata 2 lemmings simultaneamente, e gasta 1 ponto ``mana'';
\li ``fireball'', que mata 3 lemmings simultaneamente, e gasta 2 pontos ``mana''.
\endul
\noindent
Alem dos feitiços, Xÿźźÿ pode usar seu bastão para matar
os lemmings (que não custa nada, e mata 1 lemming com cada batida).
\endgraf
Suponha que o mago \emph{nunca} lançará um feitiço que mataria mais lemmings do
que tem (ou seu quintal vai se queimar).
Ele tem $m$ pontos ``mana'' e existem $n$ lemmings no seu quintal.
Em quantas maneiras diferentes ele pode destruir todos os lemmings se\dots
\beginol
\li os lemmings são indistinguíveis?
\li os lemmings são distinguíveis?
\li os lemmings são distinguíveis e cada vez que Xÿźźÿ mata um usando seu bastão,
ele \emph{ganha} um ponto de ``mana''?
\endol
\noindent
(Para os casos que os lemmings são distinguíveis,
o mago escolhe também \emph{quais} dos lemmings ele matará cada vez.)

\hint
Recursão.

\hint
Seja $f(m,n)$ o número de maneiras que Xÿźźÿ pode matar todos os $n$ lemmings,
começando com $m$ pontos de mana.

\endproblem
%%}}}

\endproblems
%%}}}

%%{{{ History 
\history.

Pascal não foi o primeiro de estudar o ``triângulo aritmético'',
cuja existência e sua relação com a teorema binomial já eram
conhecidas desde uns séculos antes do seu nascimento.
Mesmo assim, seu estudo
\emph{``Traité du triangle arithmétique, avec quelques autres petits traitez
sur la mesme matière''},
publicado no ano \oldstyle{1654} (depois da sua morte)
popularizou o triângulo e suas diversas aplicações e
propriedades~(\cite{pascaltriangle}).

\endhistory
%%}}}

%%{{{ Further reading 
\further.

Veja o~\cite{nivencount}.

\endfurther
%%}}}

\endchapter
%%}}}

%%{{{ chapter: Number theory I: divisibility 
\chapter Teoria de números I: divisibilidade.

%%{{{ Factorization 
\section Fatorização.

\question.
\emph{Dado um inteiro positivo $n$, como podemos ``o quebrar'' em blocos de construção?}

\note.
Vamos primeiramente responder nessa questão com outra:
\emph{Qual seria nosso ``cimento''?}
Tome como exemplo o número $n=28$.
Usando $+$ para o construir com blocos, podemos o quebrar:
$$
\align
28 &= 16 + 12.
\intertext{E agora quebramos esses dois blocos:}
   &= \overbrace{10 + 6}^{16} + \overbrace{11 + 1}^{12};
\intertext{e esses:}
   &= \overbrace{3 + 7}^{10} + \overbrace{3 + 3}^{6} + \overbrace{4 + 7}^{11} + 1
\intertext{e o $1$ não é mais ``quebrável''.
Nesse quesito, ele é um bloco atómico, um ``tijolo''.
Repetimos esse processo até chegar num somatório cujos termos são todos tijolos:}
 &= \underbrace{1 + 1 + 1 + 1 + \dotsb + 1}_{\text{$28$ termos}}.
\endalign
$$
O leitor é convidado pensar sobre as próximas observações:
\beginol
\li
Começando com qualquer inteiro positivo $n$,
depois um \emph{finito} número de passos, o processo termina:
nenhum dos termos que ficam pode ser quebrado.
\li
Existe apenas um tipo de bloco atómico: o $1$.
Podemos então formar qualquer número $n$ começando com $n$ tijolos ($n$ $1$'s)
e usando a operação $+$ para os juntar.
\li
Não faz sentido considerar o $0$ como tijolo, pois escrevendo o $1$ como
$$
1 = 1 + 0 \qqqquad\text{ou}\qqqquad
1 = 0 + 1
$$
não conseguimos o ``quebrar'' em peças menores.  Pelo contrario,
ele aparece novamente, da mesma forma, no lado direito.
\endol

\exercise.
\label{partitioning_restricted_best_strategy}
Qual é a melhor estratégia para desconstruir o $n$ em $1$'s conseguindo
o menor número de passos possível?  Quantos passos precisa?
Suponha que em cada passo tu tens que escolher apenas \emph{um} termo (não
atómico) e decidir em quais duas partes tu o quebrarás.

\hint
Olha na forma final do somatório.
O que acontece em cada passo?

\solution
Todas as estratégias são iguais: em cada passo, um $+$ é adicionado,
e todas terminam no mesmo somatório com $n$ $1$'s e $n-1$ $+$'s.
Então começando com qualquer número $n$, depois de $n-1$ passos chegamos
na sua forma $n = 1+1+\dotsb +1$.

\endexercise

\exercise.
Prove formalmente tua resposta no~\ref{partitioning_restricted_best_strategy}.

\hint
Como nos livramos dos pontos informais ``$\dotsb$''?

\hint
Indução.

\hint
Seja $steps(x)$ o número de passos necessários para quebrar o $x$ em $1$'s.

\endexercise

\note ``$\,\ntimes\,$'' ao invés de ``$\,+\,$''.
Vamos agora usar como cimento a operação $\ntimes$,
ilustrando o processo com o número $2016$:
\goodbreak
$$
\align
2016
&= 12 \ntimes 168
\intertext{e repitimos\dots}
&= \overbrace{4\ntimes 3}^{12} \ntimes \overbrace{28 \ntimes 6}^{12}
\intertext{e agora vamos ver:
o $4$ pode se quebrado sim ($2\ntimes2$), mas o $3$?
Escrever $3 = 3\ntimes 1$ com certeza não é um jeito aceitável para quebrar o $3$
em blocos de construção ``mais principais'': o lado direto é mais complexo!
Quebrando com $\ntimes$ então, o $3$ é um bloco atómico, um tijolo!
Continuando:}
&= \overbrace{2\ntimes 2}^{4}{} \ntimes 3 \ntimes (7\ntimes 4) \ntimes (2\ntimes3)\\
&= 2\ntimes 2 \ntimes 3 \ntimes 7 \ntimes \overbrace{2\ntimes2}^4{} \ntimes 2\ntimes3
\intertext{onde todos os fatores são atómicos.
Podemos construir o número $2016$ então assim:}
2016 &= 2\ntimes 2 \ntimes 3 \ntimes 7 \ntimes 2\ntimes2 \ntimes 2\ntimes3,
\endalign
$$
usando os tijolos 2, 3, e 7, e a operação de multiplicação.
Nos vamos definir formalmente esses tijolos (são os \ii{primo}[informalmente]\dterm{primos},~\ref{prime}),
e estudar suas propriedades.

Ilustrando com o mesmo número $2016$, um outro caminho para processar seria o seguinte:
$$
\align
2016
&= 48 \ntimes 42 \\
&= \overbrace{8 \ntimes 6}^{48} \ntimes \overbrace{6\ntimes 7}^{42}\\
&= \overbrace{2\ntimes 4}^{8} \ntimes \overbrace{2\ntimes 3}^{6} \ntimes \overbrace{2\ntimes 3}^{6}{}\ntimes 7\\
&= 2 \ntimes \overbrace{2\ntimes 2}^{4}{} \ntimes 2\ntimes 3 \ntimes 2\ntimes 3\ntimes 7
\intertext{então no final temos:}
2016&= 2 \ntimes 2 \ntimes 2 \ntimes 2 \ntimes 3 \ntimes 2 \ntimes 3 \ntimes 7.
\endalign
$$

\exercise.
\label{fundamental_theorem_of_arithmetic_omen}
O que tu percebes sobre as duas desconstruções?:
$$
\align
2016 &= 2\ntimes 2 \ntimes 3 \ntimes 7 \ntimes 2\ntimes2 \ntimes 2\ntimes3;\\
2016 &= 2 \ntimes 2 \ntimes 2 \ntimes 2 \ntimes 3 \ntimes 2 \ntimes 3 \ntimes 7.
\endalign
$$

\solution
Esquecendo a ordem que os fatores parecem, são iguais:
cada construção precisa os mesmos blocos atómicos (o $2$, o $3$, e o $7$),
e cada um deles foi usado o mesmo número de vezes:
$2016 = 2^5 3^2 7$.

\endexercise

\exercise.
\label{factorize_some_integers}%
Fatorize os inteiros $15$, $16$, $17$, $81$, $100$, $280$, $2015$, e $2017$
em fatores primos.

\solution
Calculamos:
$$
\alignat 4
15   &= 3 \ntimes 5 \qquad&  17   &= 17          \qquad& 100  &= 2^2\ntimes 5^2           \qquad& 2015 &= 5\ntimes 13\ntimes 31  \\
16   &= 2^4         \qquad&  81   &= 3^4         \qquad& 280  &= 2^3 \ntimes 5 \ntimes 7  \qquad& 2017 &= 2017.
\endalignat
$$

\endexercise

\codeit FactorNaive.
\label{program_factor_naive}
Escreva um programa que mostra para cada entrada, sua fatorizações em primos.
Executa teu programa para verificar tuas respostas no~\ref{factorize_some_integers}.
\endcodeit

\question.
Usando adição, nos precisamos apenas um tipo de tijolo para construir qualquer
inteiro positivo: o $1$.
Usando a multiplicação nos já percebemos que vários tipos são necessários; mas quantos?

\note.
A pergunta e sua resposta não são triviais!
Recomendo para ti, tentar responder e \emph{provar} tua afirmação.
Logo vamos encontrar a resposta (de Euclides)
que é um dos teoremas mais famosos e importantes na história de matemática
(\ref{primes_is_infinite}).

No fim do capítulo chegamos no resultado principal
que esclareçará o~\ref{fundamental_theorem_of_arithmetic_omen}:
o Teorema Fundamental de Aritmética (\refn{fundamental_theorem_of_arithmetic}),
ilustrado (parcialmente) já por \Euclid{}Euclides
(circa~\oldstyle{300}~a.C.)~nos seus \Euclid[Elementos]\emph{Elementos}
de Euclides\Euclid~\cite{elements}
e provado completamente e formalmente por \Gauss{}Gauss
(no ano \oldstyle{1798}) no seu
\emph{Disquisitiones Arithmeticæ}\Gauss[Disquisitiones Arithmiticæ]~\cite{disquisitiones}.

\endsection
%%}}}

%{{{ Divisibility 
\section Divisibilidade.

\definition Divisibilidade.
\label{divides}%
\tdefined{divide}%
\tdefined{divisor}%
\tdefined{multiplo}%
\tdefined{divisível}%
\sdefined {\holed a\divides \holed b} {\holed a divide \holed b}%
Sejam $a,b\in\ints$.
Digamos que \dterm{o $a$ divide o $b$} (ou \dterm{o $b$ é divisível por $a$}), sse existe $b = ak$ para algum $k\in\ints$.
Nesse caso, escrevemos $a\divides b$.
Em símbolos:
$$
a\divides b \defiff \lexists {k\in\ints} {b = ak}.
$$
Os \dterm{divisores} do $a$ são todos os inteiros $d$ tais que $d \divides a$.
Naturalmente, usamos a notação $a\ndivides b$ quando $a$ não divide $b$.

\example. $3 \divides 12$, porque $12 = 3 \ntimes 4$ e $4\in\ints$, mas
$8 \ndivides 12$, porque nenhum inteiro $u$ satisfaz $12 = 8u$.
\endexample

\exercise Propriedades da divisibilidade.
\label{divides_properties}
Sejam $a,b,x,y,m\in\ints$.
Prove que:
\doublecolumns
\beginol
\li[1_is_the_bottom_of_divides] $1 \divides a$
\li[0_is_the_top_of_divides] $a \divides 0$
\li $a \divides b \implies a \divides bx$
\li $a \divides b \implies a \divides -b \mland -a \divides b$
\li $a \divides b \mland a \divides c \implies a \divides b + c$
\li $a \divides b \mland a \divides c \implies a \divides bx + cy$
\li $a \divides b \mland b \neq 0 \implies \abs a \leq \abs b$
\li se $m\neq0$ então: $a \divides b \!\iff\! ma \divides mb$.
\endol
\singlecolumn

\hint Aplique a definição do $\divides$.

\endexercise

\exercise Mais propriedades da divisibilidade.
\label{divides_is_almost_antisymmetric}%
\label{divides_is_a_partial_order_on_nats}%
Para todos $a,b,c\in\ints$,
$$
\alignat 2
&a \divides a                                           &&\text{(reflexividade)}\\
&a \divides b \mland b \divides c \implies a \divides c &&\text{(transitividade)}\\
&a \divides b \mland b \divides a \implies \abs a = \abs b.   &&
\intertext{
Se $a,b\in\nats$, a terceira propriedade fica mais forte:
}
&a \divides b \mland b \divides a \implies a = b        &\quad&\text{(antisimetria)}.
\endalignat
$$

\endexercise

\proposition.
\label{wrong_property_of_product_dividing_common_multiple}
Sejam $a,b,m\in\ints$.  Se $a \divides m$ e $b \divides m$, então $ab \divides m$.
\wrongproof.
Como $a\divides m$, pela definição de $\divides$, existe $u\in\ints$ tal que
$a = mu$.  Similarmente, como $b\divides m$, existe $v\in\ints$ tal que
$b = mv$.  Multiplicando as duas equações por partes, temos
$$
ab = (mu)(mv) = m(umv),
$$
e como $umv\in\ints$, $ab \divides m$.
\mistaqed

\exercise.
Ache o erro na prova em cima e \emph{prove} que a proposição é falsa!

\hint Presta atenção na definição do $\divides$.

\hint Procure um contraexemplo onde $a$ e $b$ tem um fator em comun.

\solution
O erro fica na aplicação da definição de $a \divides b$\thinspace:
ao invés de $\lexists {k\in\ints} {b = ak}$,
a prova usou $\lexists {k\in\ints} {a = bk}$.
\endgraf
Para ver que a proposição realmente é falsa, considere o contraexemplo seguinte:
$$
a = 6,\qquad
b = 15,\qquad
m = 30.
$$
Realmente temos
$6  \divides 30$ e 
$15 \divides 30$,
mas
$6\ntimes 15 = 90 \ndivides 30$.

\endexercise

\note.
Nos encontramos a ``versão correta'' da proposição falsa em cima no item~\refn{wrong_property_of_product_dividing_common_multiple} depois (\ref{product_of_coprimes_divides_common_multiple}).

\exercise.
\label{implications_with_divisibility_of_linear_combinations}
Sejam $a,b,c\in\ints$.
Para cada afirmação abaixo, prove-la ou desprove-la:
$$
\alignat2
\text{(i)}   &\qquad& a \divides \phantom1b + c\phantom1                          &\implies a \divides b \mland a \divides c\\
\text{(ii)}  &\qquad& a \divides b + c \mland a \divides \phantom1b - c\phantom2  &\implies a \divides b                    \\
\text{(iii)} &\qquad& a \divides b + c \mland a \divides \phantom1b + 2c          &\implies a \divides b                    \\
\text{(iv)}  &\qquad& a \divides b + c \mland a \divides 2b + 2c                  &\implies a \divides b                    \\
\text{(v)}   &\qquad& a \divides b + c \mland a \divides 2b + 3c                  &\implies a \divides 3b + 2c\,.
\endalignat
$$

\solution
A (ii) é falsa: um contraexemplo seria o $a = 2$, $b = c = 1$.
Realmente, temos
$$
2 \divides 1 + 1 = 2 \mland 2 \divides 1 - 1 = 0,\qquad \text{mas}\qquad 2\ndivides 1.
$$
\endgraf
A (iii) é verdadeira:
$$
\left.
\aligned
        a\divides b + c \implies a\divides 2b + 2c\\
                                 a\divides \phantom1b + 2c
\endaligned
\right\}
\implies
a \divides \underbrace{(2b + 2c) - (b + 2c)}_{\dsize b}.
$$

\endexercise

\definition Primo.
\label{prime}
\tdefined{primo}
\tdefined{composto}
\iiseealso{composto}{primo}
\iiseealso{primo}{composto}
Seja $p\in\nats$, com $p \geq 2$.
Chamamos o $p$ \dterm{primo} sse $p$ é divisível apenas por $\pm 1$ e $\pm p$.
Caso contrario, ele é \dterm{composto}.

\exercise.
O $0$ é primo?  Composto?
O $1$ é primo?  Composto?

\solution
Nenhum dos dois é nem primo nem composto: a definição começa declarando
$p$ como um natural tal que $p \geq 2$.
Logo, não é aplicável nem para o 0 nem para o 1.

\endexercise

\exercise.
\label{2_is_the_only_even_prime}
2 é o único primo par.

\endexercise

\exercise.
Usando uma fórmula de lógica, defina diretamente o que significa que
um $n\in\nats\setminus\set 1$ é composto.

\solution
Seja $n\in\nats$.
$$
\namedpred{Composite}(n) \defiff
\lexists {a,b\in\nats\setminus\set{0,1}} { n = ab }.
$$

\endexercise

{\input knuthprimes
\example.
\label{first_primes}%
Os primeiros 31 primos são os \primes{31}.
\endexample
}

\exercise.
\label{in_primes_divides_means_equals}
Sejam $p$, $q$ primos, com $p\divides q$.
Mostre que $p=q$.

\endexercise

\exercise.
\label{every_composite_number_is_divisible_by_a_prime}
Seja $b$ composto.  Prove que $b$ tem um divisor primo $d\leq \sqrt b$.

\endexercise

\codeit factor.
\label{program_factor}
Use o~\ref{every_composite_number_is_divisible_by_a_prime}
para melhorar teu programa do~\ref{program_factor_naive}.
\endcodeit

\endsection
%%}}}

%%{{{ The division algorithm 
\section O algoritmo da divisão.

\lemma Divisão de Euclides.
\label{euclidean_division}%
\Euclid[lema da divisão]%
\iisee{divisão}{Euclides}%
Dados inteiros $a$ e $b$ com $b>0$, existem inteiros $q$ e $r$ tais que:
$$
a = bq + r,
\qquad
0\leq r < b.
\eqdef{division}
$$
Alem disso, os $q$ e $r$ são \emph{determinados unicamente}.
\sketch.
\endgraf
\proofstyle{Existência:}
Considera a seqüência infinita:
$$
\ldots,~
-3b + r,~
-2b + r,~
-b + r,~
r,~
b + r,~
2b + r,~
3b + r,~
\ldots
$$
Observe que ela tem elementos não-negativos e, aplicando a PBO, considera o menor deles.
\endgraf
\proofstyle{Unicidade:}
    Suponha que $a=bq+r=bq'+r'$
    para alguns $q,r,q',r'\in\ints$ tais que satisfazem as restrições
    $0\leq r < b$ e $0\leq r' < b$.
    Mostre que $r=r'$ e $q=q'$.
\qes

Por causa dessa existência e unicidade, podemos definir:

\definition Devisão.
\tdefined{divisão}
\tdefined{quociente}[divisão]
\sidx[see]{quociente}{divisão} 
Dados $a,b\in\ints$ com $b>0$, chamamos os inteiros $q$ e $r$ que satisfazem
a~\eqref{division}
Chamamos o $q$ o \dterm{quociente} e o $r$ o
\dterm{resto}\tdefined{resto}[divisão] da divisão de $a$ por $b$.

\exercise.
\label{n_divides_exactly_one_of_n_consecutive_integers}
Seja $n\in\nats$ positivo.  Se $a_0,a_1,\dotsc,a_{n-1}$ são $n$ inteiros consecutivos,
então $n\divides a_i$ para um único $i\in\set{0,\dots,n-1}$.

\hint
Seja $a=a_0$.  Assim $a_i = a + i$.

\hint
Divida o $a$ por $n$ e, olhando para o resto $r$,
ache o certo $i$ tal que $n\divides a_i$.

\hint
Para a unicidade, ache o resto da divisão de $n$ por o aleatorio $a_j$.

\endexercise

\exercise.
Prove que para todo $n\in\ints$, se $3\ndivides n$
então $3 \divides n^2 - 1$.

\hint
Ou fatorize o $n^2-1$ e use o~\ref{n_divides_exactly_one_of_n_consecutive_integers},
ou considere os dois casos possíveis dependendo dos restos da divisão de $n$ por $3$.

\endexercise

\definition Conjunto fechado sobre $+$.
\label{closed_under_addition}%
    Seja $A$ conjunto de números que satisfaz a propriedade:
    $$
    a,b \in A \implies a+b \in A.
    $$
    Digamos que $A$ é \dterm{fechado} sobre $+$.

\exercise.
O $\emptyset$ e o $\set{0}$ são fechados sobre $+$.

\endexercise

\exercise.
Generalize a noção em cima para definir a noção
``conjunto fechado sobre $f$'',
onde $f$ é uma operação no $A$ de aridade qualquer.

\endexercise

\exercise.
Quais dos conjuntos abaixo são fechados sobre $+$, quais sobre $-$, e quais sobre $\cdot$?
$$
\aligned
    U &= \set{0}\\
    W &= \set{0,1}\\
    P &= \set{1,2,4,8,16,\dotsc}\\
    I &= \set{1,1/2,1/4,1/8,1/16,\dotsc}
\endaligned
\quad
\aligned
    A &= \set{0,4,6,10,12,16,18,22,24,\dotsc}\\
    B &= \set{0,2,4,6,8,10,12,\dotsc}\\
    C &= \set{\dotsc,-9,-6,-3,0,3,6,9,\dotsc}\\
    D &= \set{\dotsc,-8,-5,-2,1,4,7,10,\dotsc}.
\endaligned
$$

\endexercise

\exercise.
Defina os conjuntos infinitos do exercísio anterior sem usar ``$\ldots$\!''.

\endexercise

\exercise.
O $\rats$ é fechado sobre quais das quatro operações binárias: $+$, $-$, $\cdot$, $\div$?
E o $\rats\setminus\set0$?

\endexercise

\exercise.
Prove que se um conjunto é fechado sobre $-$, ele tem que ser fechado sobre $+$ também, mas não o converso!

\endexercise

Um conjunto de inteiros fechado sobre $-$ (e $+$), não tem muita liberdade na ``forma'' dele.
O teorema seguinte mostra a forma geral de todos eles.

\theorem.
\label{form_of_closed_under_minus}
Seja $S\subseteq\ints$, com $S\neq\emptyset$, $S$ fechado sobre $-$ (e logo, sobre $+$ também).
Prove que $S=\set 0$ ou $S$ possui um mínimo elemento positivo $m$ e $S = \set{ km \st k \in\ints}$.

\exercise.
Sem usar disjunção, escreva uma frase equivalente com a conclusão do
\ref{form_of_closed_under_minus}, e explique porque ela é equivalente.

\hint Existe $n\in\nats$ tal que $S = \set{ kn \st k\in\ints}$.

\endexercise

\endsection
%%}}}

%%{{{ The greatest common divisor 
\section O máximo divisor comum.

\definition.
\label{a_gcd}%
Sejam $a,b\in\ints$.
O inteiro $d$ é \dterm{um máximo divisor comum (m.d.c.)} dos $a$~e~$b$,
sse $d$~é~um divisor comum e um multiplo de todos os divisores comuns.
Em símbolos:
\endgraf\vskip 1ex\noindent
\line{\hfill
$
d = \dsym{\gcd a b}
\defiff
\underbrace{
d \divides a
\;\land\;
d \divides b
}_{\text{divisor comun}}
\;\land\;
\underbrace{
\lforall c {c\divides a \;\land\; c\divides b \implies c\divides d}
}_{\text{``máximo''}}.
$
\hfil\mistake}

\exercise.
A~\ref{a_gcd} tem um erro: o símbolo $\gcd a b$ não foi bem definido!
O que precisamos provar para definir realmente o símbolo $\gcd a b$?

\solution
Duas coisas:
\endgraf
\proofstyle{Existência:}
\emph{para todos inteiros $a$, $b$, existe inteiro $d$ que satisfaz as relações em cima.}
\endgraf
\proofstyle{Unicidade:}
\emph{se $d$, $d'$, são inteiros que satisfazem essas relações, então $d=d'$.}

\endexercise

\proposition.
Sejam $a,b\in\ints$.  Se $d,d'\in\ints$ são máximos divisóres comum
dos $a$ e $b$, então $d=\pm d'$.
\sketch.
Aplicamos a definição de m.d.c.~para cada um dos $d$ e $d'$,
para chegar em $d\divides d'$ e $d'\divides d$.
\qes
\proof.
Suponha que $d,d'$ são m.d.c.'s de $a$ e $b$.
Como $d$ é um m.d.c., todos os divisores em comum dos $a$ e $b$ o dividem.
Mas, como $d'$ é um divisor em comum, então $d'\divides d$.
Simetricamente concluimos que $d\divides d'$.
Logo $d = \pm d'$ (por~\ref{divides_is_almost_antisymmetric}).
\qed

\corollary.
Sejam $a,b\in\ints$.
Existe único $d\in\nats$ tal que $d$ é um m.d.c.~de $a$ e~$b$.

Agora podemos definir o símbolo $\gcd a b$:

\definition O máximo divisor comum.
\label{gcd}
\sidx[see]{máximo divisor comum}{mdc}
\tdefined{mdc}
Sejam $a,b\in\ints$.
$$
d = \dsym{\gcd a b}
\defiff
\underbrace{
d \divides a
\;\land\;
d \divides b
}_{\text{divisor comun}}
\;\land\;
\underbrace{
d\in\nats
\;\land\;
\lforall c {c\divides a \;\land\; c\divides b \implies c\divides d}
}_{\text{o máximo}}
\sdefined{\gcd {\holed a} {\holed b}}{o máximo divisor comum de \holed a e \holed b}
$$

\theorem.
\label{gcd_as_linear_combination}
\label{gcd_divides_any_linear_combination}
Sejam $a,b\in\ints$.
Então existe $d\in\nats$ que satisfaz a definição de $\gcd a b$,
ele pode ser escrito como \dterm{combinação linear}%
\tdefined{combinação linear}
dos $a$ e $b$, com coeficientes inteiros; ou seja,
$$
d = sa + tb,
\qquad
\text{para alguns $s,t\in\ints$}.
$$
Alem disso, o $\gcd a b$ divide qualque combinação linear dos $a$ e $b$.
\sketch.
Considere o conjunto $L$ de todas as combinações lineares.
Prove que existe $n_0\in\nats$ tal que $L = \set{ kn_0 \st k\in\ints }$.
Verifique que $\gcd a b = n_0$.
\qes

\exercise.
Verifique se a forma do $\gcd a b$ como combinação linear dos $a$ e $b$
é unicamente determinada.

\hint
Não é.  Mostre um contraexemplo.

\solution
Toma $3$ e $4$.  Temos $\gcd 4 3 = 1$, mas:
$$
\aligned
1 &= (-2)\ntimes 4 + 3\ntimes 3\\
1 &= \phantom{(-}4\phantom{)}\ntimes 4 - 5\ntimes 3.
\endaligned
$$

\endexercise

\proposition.
\label{gcd_of_comparable}
$a\divides b \implies \gcd a b = a$.
Especificamente, $\gcd a 0 = a$.
\sketch.
Precisamos apenas verificar as condições da definição de m.d.c., que seguem por
as propriedades de $\divides$ que já provamos.
\qes

\exercise.
\label{gcd_signs}
$
\gcd a b
=
\gcd a {-b}
=
\gcd {-a} {b}
=
\gcd b a
$.

\endexercise

\exercise.
\label{gcd_of_two_is_gcd_of_one_plus_sum}
Sejam $a,b\in\ints$.
Prove que
$$
\gcd a b = \gcd a {a+b}.
$$

\hint
Lembre a definição de m.d.c.\,.

\hint
Talvez as propriedades seguintes são úteis:
\beginul
\li para todos $x,y\in\nats$, $x\divides y \mland y\divides x \implies x = y$;
\li para todos $a,x,y\in\ints$, $a\divides x \mland a\divides y \implies a \divides {x+y}$.
\endul

\solution
\proofstyle{Idéia 1:}
Sejam $d = \gcd a b$ e $d' = \gcd a {a + b}$.
Pela definição, $d$ é divisor comum dos $a$ e $b$,
então $d\divides a $ e $d\divides b$, e logo $d \divides a + b$.
Temos então que $d$ é um divisor comum dos $a$ e $a + b$, e, pela definição de mdc,
como $d' = \gcd a {b+c}$, temos $d'\divides d$.
Similarmente, $d\divides d'$.
Então $|d| = |d'|$ e como ambós são naturais (parte da definição de $\gcd x y$), temos:
$$
\gcd a b = d = d' = \gcd a {a + b}.
$$
\endgraf
\proofstyle{Idéia 2:}
Seja $d = \gcd a b$.  Vamos mostrar que $d = \gcd a {a + b}$.
Pela definição, $d\geq0$ e é divisor comum dos $a$ e $b$,
então $d\divides a$ e $d\divides b$, e logo $d \divides a + b$.
Temos então que $d\geq0$ e é um divisor comum dos $a$ e $a + b$, e, pela definição de mdc,
falta so provar que cada divisor comum deles divide o $d$.
Seja $c\in\ints$ tal que $c\divides a$ e $c \divides {a+b}$.
Logo, $c\divides a - (a+b)=-b$.  Concluimos que $c\divides b$, então
$c$ é um divisor comum dos $a$ e $b$, então $c\divides d$ pela definição do $d$
como m.d.c.~dos $a$ e $b$.

\endexercise

\lemma Lema de Euclides.
\label{euclid_lemma}
\Euclid[lema]
Sejam $a,b\in\ints$ e $p$ primo.
Se $p \divides ab$, então $p\divides a$ ou $p\divides b$.
\sketch.
Suponha que $p\ndivides a$.  Logo o $\gcd a p = 1$ pode ser escrito como combinação linear de $a$ e $p$:
$$
1 = as + pt,    \qquad\text{para uns $s,t\in\ints$}.
$$
Multiplica os dois lados por $b$, e explica por que necessariamente $p\divides b$.
\qes
\proof.
Suponha que $p\ndivides a$.  Logo o $\gcd a p = 1$ pode ser escrito como combinação linear de $a$ e $p$,
ou seja, existem $s,t\in\ints$ tais que:
$$
\align
1 &= as + pt.
\intertext{Multiplicando os dois lados por $b$, temos:}
b &= asb + ptb.
\endalign
$$
Observe que como $p\divides ab$, segue que $p\divides asb$ (por~\ref{divides_properties}).
Obviamente $p\divides ptb$ também.
Logo, $p\divides asb + ptb = b$.
\qed

\theorem.
\label{coprime_with_a_part_of_a_product_must_divide_the_other_part_to_divide_the_product}
Se $\gcd d a = 1$ e $d \divides ab$, então $d \divides b$.
\sketch.
A prova é praticamente a mesma com aquela do~\ref{euclid_lemma}:
lá nos precisamos a primalidade do $p$ apenas para
concluir que $\gcd a p = 1$.
Aqui temos diretamente a hipótese $\gcd a d = 1$.
\qes
\proof.
Seja $a,b,d\in\ints$, tais que
$\gcd d a = 1$ e $d \divides ab$.
Como $\gcd a d = 1$, podemos o escrever como combinação linear dos $a$ e $d$:
$$
1 = sa + td,\qquad\text{para alguns $s,t\in\ints$}.
$$
Multiplicando por $b$, ganhamos
$$
b = sab + tdb,
$$
e argumentando como na prova do~\ref{euclid_lemma}
concluimos que $d \divides b$.
\qed

\corollary.
\label{product_of_coprimes_divides_common_multiple}
Se $a\divides m$, $b\divides m$, e $\gcd a b = 1$, então $ab\divides m$.
\proof.
Como $a\divides m$, temos $m = au$ para algum $u\in\ints$.
Mas $b\divides m=au$, e como $\gcd b a = 1$, temos $b\divides u$
(por~\ref{coprime_with_a_part_of_a_product_must_divide_the_other_part_to_divide_the_product}).
Então $bv = u$ para algum $v\in\ints$.
Substituindo, $m = au = a(bv) = (ab)v$, ou seja, $ab \divides m$.
\qed

\theorem Euclid.
\label{primes_is_infinite}%
\Euclid[infinidade de primos]%
Existe uma infinidade de primos.
\sketch.
Para qualquer conjunto finito de primos
$P = \set{p_1,\dotsc,p_n}$, considere o número
$p_1\dotsb p_n + 1$ e use-lo para achar um primo fora do $P$.
\qes
\proof.
Para qualquer conjunto finito de primos
$P = \set{p_1,\dotsc,p_n}$,
observe que o número $p=p_1\dotsb p_n + 1$ não é divisivel por nenhum dos $p_i$'s
(porque $p_i\divides p$ e $p_i\divides p_1\dotsb p_n$ implicam
$p\divides 1$, absurdo).
Então, como $p\geq 2$, ou o $p$ é primo e $p\not\in P$,
ou $p$ é divisivel por algum primo $q \not\in P$
(por~\ref{every_composite_number_is_divisible_by_a_prime}).
Nos dois casos, existe pelo menos um primo fora do $P$.
\qed

\endsection
%%}}}

%%{{{ The Euclidean algorithm 
\section O algoritmo de Euclides.

\note Idéia.
Sejam $a,b$ inteiros positivos.
Como achamos o $\gcd a b$?
Nos vamos aplicar o lemma da divisão~(\refn{euclidean_division}) repetetivamente,
até chegar em resto $0$:
$$
\matrix
\format
\r\;    &\;\c\;  & \; \c \;                   & \l      & \;\c\; & \c      & \qquad\qquad\l          \\
a       &   =    & b                          & q_0     &  +     & r_0,    & 0\leq r_0 < b           \\
b       &   =    & r_0                        & q_1     &  +     & r_1,    & 0\leq r_1 < r_0         \\
r_0     &   =    & r_1                        & q_2     &  +     & r_2,    & 0\leq r_2 < r_1         \\
r_1     &   =    & r_2                        & q_3     &  +     & r_3,    & 0\leq r_3 < r_2         \\
        & \vdots &                            &         &        &         & \hfil\vdots\hfil        \\
r_{n-3} &   =    & r_{n-2}                    & q_{n-1} &  +     & r_{n-1},& 0\leq r_{n-1} < r_{n-2} \\
r_{n-2} &   =    & \boxed{\mathstrut r_{n-1}} & q_n     &  +     & \underbrace{r_n}_0,      & 0 = r_n < r_{n-1}.      
\endmatrix
$$
O $\gcd a b$ estará na posição marcada em cima.
Vamos agora descrever o algoritmo formalmente, e provar
(informalmente e formalmente) sua corretude!

\algorithm Euclides.
\label{euclidean_algorithm}%
\tdefined{algoritmo}[de Euclides]%
\Euclid[algoritmo]%
\endgraf
\noindent%
{\def\Euclid{\algorithmstyle{Euclid}}%
\centerline{$\Euclid(a,b)$}
{\hrule width \hsize height 1pt\relax}
\vskip4pt
\algospec
 INPUT: $a,b\in\nats$
OUTPUT: $\gcd a b$
\endspec
\beginol
\li Se $b=0$, retorna $a$.
\li Retorna $\Euclid(b,r)$, onde $r = a \bmod b$.
\endol
{\hrule width \hsize height 1pt\relax}
}

\example.
\label{euclidean_algorithm_example}
Ache o $\gcd {101} {73}$ com o algoritmo de Euclides.

\solution
Sabendo que os dois números são primos, fica imediato que eles são coprimos entre si também:
a reposta é 1.

Aplicando o algoritmo de Euclides, para achar o $\gcd {101} {73}$, dividimos o $101$ por $73$:
$$
\alignat 2
101 &= 73 \ntimes \underbrace{\phantom 1}_{\text{quociente}} + \underbrace{\phantom{28}}_{\text{resto}},      &\qquad&0 \leq \underbrace{\phantom{28}}_{\text{resto}} < 73
\intertext{e sabemos que assim reduziremos o problema para o alvo de achar o $\gcd {73} {\text{resto}}$.  Pensando, achamos os valores:}
101 &= 73 \ntimes \underbrace{1}_{\text{quociente}} + \underbrace{28}_{\text{resto}},      &\quad&0 \leq \underbrace{28}_{\text{resto}} < 73 
\intertext{ou seja, $\gcd {101} {73} = \gcd {73} {28}$.  Então, repetimos:}
73 &= 28 \ntimes \underbrace{{2}}_{\text{quociente}} + \underbrace{{17}}_{\text{resto}},      &&0 \leq \underbrace{17}_{\text{resto}} < 28
\intertext{ou seja, $\gcd {73} {28} = \gcd {28} {17}$.  Repetimos:}
28 &= 17 \ntimes \underbrace{1}_{\text{quociente}} + \underbrace{11}_{\text{resto}},      &&0 \leq \underbrace{11}_{\text{resto}} < 17
\intertext{ou seja, $\gcd {28} {17} = \gcd {17} {11}$.  Repetimos:}
17 &= 11 \ntimes \underbrace{1}_{\text{quociente}} + \underbrace{6}_{\text{resto}},      &&0 \leq \underbrace{6}_{\text{resto}} < 11
\intertext{ou seja, $\gcd {17} {11} = \gcd {11} 6$.  Repetimos:}
11 &= 6 \ntimes \underbrace{1}_{\text{quociente}} + \underbrace{5}_{\text{resto}},      &&0 \leq \underbrace{5}_{\text{resto}} < 6
\intertext{ou seja, $\gcd {11} 6 = \gcd 6 5$.  Repetimos:}
6 &= 5 \ntimes \underbrace{1}_{\text{quociente}} + \underbrace{1}_{\text{resto}},      &&0 \leq \underbrace{1}_{\text{resto}} < 5
\intertext{ou seja, $\gcd 6 5 = \gcd 5 1$.  Como $\gcd 5 1 = 1$, nem precisamos repetir, mas vamos mesmo assim:}
5 &= \boxed 1 \ntimes \underbrace{5}_{\text{quociente}} + \underbrace{0}_{\text{resto}}.      &&
\endalignat
$$

Mais compactamente, os passos são:
$$
\matrix
\format
\r  & \c    & \c       & \l        & \l & \c    & \l  & \c     & \c & \c       & \c & \c    & \c & \c     & \r            & \c    & \l          \\
101 & {}={} & 73       & {}\ntimes{} & 1  & {}+{} & 28, & \qquad & 0  & {}\leq{} & 28 & {}<{} & 73 & \qquad & \gcd{101}{73} & {}={} & \gcd{73}{28}\\
73  & {}={} & 28       & {}\ntimes{} & 2  & {}+{} & 17, &        & 0  & {}\leq{} & 17 & {}<{} & 28 &        & \gcd{73 }{28} & {}={} & \gcd{28}{17}\\
28  & {}={} & 17       & {}\ntimes{} & 1  & {}+{} & 11, &        & 0  & {}\leq{} & 11 & {}<{} & 17 &        & \gcd{28 }{17} & {}={} & \gcd{17}{11}\\
17  & {}={} & 11       & {}\ntimes{} & 1  & {}+{} & 6,  &        & 0  & {}\leq{} & 6  & {}<{} & 11 &        & \gcd{17 }{11} & {}={} & \gcd{11}{6 }\\
11  & {}={} & 6        & {}\ntimes{} & 1  & {}+{} & 5,  &        & 0  & {}\leq{} & 5  & {}<{} & 6  &        & \gcd{11 }{6 } & {}={} & \gcd{6 }{5 }\\
6   & {}={} & 5        & {}\ntimes{} & 1  & {}+{} & 1,  &        & 0  & {}\leq{} & 1  & {}<{} & 5  &        & \gcd{6  }{5 } & {}={} & \gcd{5 }{1 }\\
5   & {}={} & \boxed 1 & {}\ntimes{} & 5  & {}+{} & 0   &        &    &          &    &       &    &        & \gcd{5  }{1 } & {}={} & \gcd{1 }{0 } = \boxed 1.
\endmatrix
$$
\moveqedup
\endexample

\remark.
\def\Euclid{\algorithmstyle{Euclid}}%
Podemos utilizar o algoritmo de Euclides para achar o $\gcd a b$
onde $a,b\in\ints$ também,
graças o~\ref{gcd_signs}:
$(a,b) = (\abs a, \abs b) = \Euclid(\abs a, \abs b)$.

\exercise.
Usando o algoritmo de Euclides, ache os:
(i) $\gcd {108} {174}$; 
(ii) $\gcd {2016} {305}$.

\solution
(i) Para o $\gcd {108} {174} = \gcd {174} {108}$ calculamos:
$$
\matrix
\format
\r  & \c    & \c       & \l        & \l & \c    & \l    & \c     & \c & \c       & \c & \c    & \c & \c     & \r             & \c    & \l            \\
174 & {}={} & 108      & {}\ntimes{} & 1  & {}+{} &{66},  & \qquad & 0  & {}\leq{} & 66 & {}<{} & 108& \qquad & \gcd{174}{108} & {}={} & \gcd{108}{66 }\\
108 & {}={} & 66       & {}\ntimes{} & 1  & {}+{} &{42},  &        & 0  & {}\leq{} & 42 & {}<{} & 66 &        &                & {}={} & \gcd{66 }{42 }\\
66  & {}={} & 42       & {}\ntimes{} & 1  & {}+{} &{24},  &        & 0  & {}\leq{} & 24 & {}<{} & 42 &        &                & {}={} & \gcd{42 }{24 }\\
42  & {}={} & 24       & {}\ntimes{} & 1  & {}+{} &{18},  &        & 0  & {}\leq{} & 18 & {}<{} & 24 &        &                & {}={} & \gcd{24 }{18 }\\
24  & {}={} & 18       & {}\ntimes{} & 1  & {}+{} &{6 },  &        & 0  & {}\leq{} & 6  & {}<{} & 18 &        &                & {}={} & \gcd{18 }{6  }\\
18  & {}={} & \boxed 6 & {}\ntimes{} & 3  & {}+{} &{0 }   &        &    &          &    &       &    &        &                & {}={} & \gcd{6  }{0  } = \boxed 6.
\endmatrix
$$
\endgraf
\noindent
(ii) Calculamos:
$$
\matrix
\format
\r  & \c    & \c       & \l        & \l & \c    & \l     & \c     & \c & \c       & \c & \c    & \c & \c     & \r              & \c    & \l            \\
2016& {}={} & 305      & {}\ntimes{} & 6  & {}+{} &{186},  & \qquad & 0  & {}\leq{} &186 & {}<{} &305 & \qquad & \gcd{2016}{305} & {}={} & \gcd{305}{186}\\
305 & {}={} & 186      & {}\ntimes{} & 1  & {}+{} &{119},  &        & 0  & {}\leq{} &119 & {}<{} &186 &        &                 & {}={} & \gcd{186}{119}\\
186 & {}={} & 119      & {}\ntimes{} & 1  & {}+{} &{67 },  &        & 0  & {}\leq{} &67  & {}<{} &119 &        &                 & {}={} & \gcd{119}{67 }\\
119 & {}={} & 67       & {}\ntimes{} & 1  & {}+{} &{52 },  &        & 0  & {}\leq{} &52  & {}<{} &67  &        &                 & {}={} & \gcd{67 }{52 }\\
67  & {}={} & 52       & {}\ntimes{} & 1  & {}+{} &{15 },  &        & 0  & {}\leq{} &15  & {}<{} &52  &        &                 & {}={} & \gcd{52 }{15 }\\
52  & {}={} & 15       & {}\ntimes{} & 3  & {}+{} &{7  },  &        & 0  & {}\leq{} &7   & {}<{} &15  &        &                 & {}={} & \gcd{15 }{7  }\\
15  & {}={} & 7        & {}\ntimes{} & 2  & {}+{} &{1  },  &        & 0  & {}\leq{} &1   & {}<{} &7   &        &                 & {}={} & \gcd{7  }{1  }\\
7   & {}={} & \boxed 1 & {}\ntimes{} & 7  & {}+{} &{0  }   &        &    &          &    &       &    &        &                 & {}={} & \gcd{1  }{0  } = \boxed 1.
\endmatrix
$$
%$$
%\matrix
%\format
%\r  & \c    & \c       & \l        & \l & \c    & \l     & \c     & \c & \c       & \c & \c    & \c & \c     & \r              & \c    & \l            \\
%    & {}={} &          & {}\ntimes{} &    & {}+{} &{   },  & \qquad & 0  & {}\leq{} &    & {}<{} &    & \qquad & \gcd{    }{   } & {}={} & \gcd{   }{   }\\
%    & {}={} &          & {}\ntimes{} &    & {}+{} &{   },  &        & 0  & {}\leq{} &    & {}<{} &    &        &                 & {}={} & \gcd{   }{   }\\
%    & {}={} & \boxed 1 & {}\ntimes{} &    & {}+{} &{   }   &        &    &          &    &       &    &        &                 & {}={} & \gcd{   }{   } = .
%\endmatrix
%$$

\endexercise

\codeit.
\label{implement_euclidean_algorithm}
Implemente o algoritmo de Euclides e verifique tuas soluções nos exercísios anteriores.
\endcodeit

\codeit.
\label{implement_verbose_euclidean_algorithm}
Implemente um modo ``verbose'' no teu programa
do~\ref{implement_euclidean_algorithm},
onde ele mostra todas as equações e desigualdades, e não apenas o resultado final.
\endcodeit

\lemma Euclides.
\label{euclid_gcd_lemma}
Se $a,b\in\ints$ com $b > 0$, então $\gcd a b = \gcd b r$,
onde $r$ o resto da divisão de $a$ por $b$.
\wrongproof.
Dividindo o $a$ por $b$, temos $a = bq + r$.
Vamos mostrar que qualquer inteiro $d$ satisfaz a equivalência:
$$
\align
d \divides a
\mland
d \divides b
&\iff
d \divides b
\mland
d \divides r.\\
\intertext{Realmente, usando as propriedades~\refn{divides_properties}, temos:}
d \divides a
\mland
d \divides b
&\implies
d\divides \overbrace{a - bq}^{\dsize r}\\
d\divides \underbrace{bq + r}_{\dsize a}
&\impliedby
d \divides b
\mland
d \divides r.
\endalign
$$
Isso mostra que os divisores em comum dos $a$ e $b$, e dos $b$ e $r$ são os mesmos,
ou, formalmente:
$$
\set{c\in\ints\st c\divides a \mland c\divides b}
=
\set{c\in\ints\st c\divides b \mland c\divides r}.
$$
Logo,
$$
\alignat 2
\gcd a b
&= \max\set{c\in\ints\st c\divides a \mland c\divides b}    \qqby{def.~$\gcd a b$}\\
&= \max\set{c\in\ints\st c\divides b \mland c\divides r}    \qqby{provado em cima}\\
&= \gcd b r                                                 \qqby{def.~$\gcd b r$}
\endalignat
$$
que mostra a corretude do algoritmo.
\mistaqed

\exercise.
Qual é o problema com a prova do~\ref{euclid_gcd_lemma}?

\solution
A definição do maior divisor comum $(a,b)$ não foi ``o maior divisor comum dos $a$ e $b$''.
Lembre-se~\ref{gcd}.  Veja também o~\ref{gcd_alternative_definition}.

\endexercise

\theorem Algorítmo de Euclides.
\label{euclidean_algorithm_correctness}
O~\ref{euclidean_algorithm} é correto.
\proof.
Precisamos provar duas coisas: \emph{terminação\/}\/ e \emph{corretude}.
\endgraf
\proofstyle{Corretude.}
Se o algoritmo precisou $n$ passos, temos que verificar:
Temos
$$
\gcd a b
= \gcd b {r_0}
= \gcd {r_0} {r_1}
= \gcd {r_1} {r_2}
= \dotsb
= \gcd {r_{n-1}} {r_n}
= \gcd {r_n} 0
= r_n.
$$
Todas as igualdades exceto a última seguem por causa do~\ref{euclid_gcd_lemma};
a última por causa da~\ref{gcd_of_comparable}.
\endgraf
\proofstyle{Terminação.}
Note que a seqüência de restos $r_0, r_1, \ldots$ é estritamente
decrescente, e todos os seus termos são não negativos:
$$
0\leq \dotsb < r_2 < r_1 < r_0 < b.
$$
Logo, essa seqüência não pode ser infinita.
Realmente, o tamanho dela não pode ser maior que $b$,
então depois de no máximo $b$ passos, o algorítmo terminará.
\qed

\exercise.
\label{euclidean_algorithm_proof_why_informal}
Nenhuma das duas partes da prova do~\ref{euclidean_algorithm_correctness}
é completamente formal.  Explique porque.
(Veja também os~\refs{euclidean_algorithm_correctness_formal_proof_by_induction}--\refn{euclidean_algorithm_correctness_formal_proof_by_wop}.)

\hint ${}\dotsb{}$

\hint ${}\dotsb{}$

\solution
São os ``${}\dotsb{}$'' mesmo!

\endexercise

\note Passos do algoritmo de Euclides.
Como o exercísio seguinte mostra, o algoritmo de Euclides
é bem mais eficiênte do que nos mostramos em cima:
depois dois passos, as duas entradas, nos piores dos casos,
são reduzidas para metade.

\exercise.
Se $a \geq b$, então $r < a/2$, onde $r$ o resto da divisão de $a$ por $b$.

\hint Separe os casos: ou $b > a/2$ ou $b \leq a/2$.

\hint Qual seria o resto em cado caso?

\hint Num caso, dá para achar exatamente o resto.
No outro, use a restrição que o resto satisfaz.

\solution
\casestyle{Caso $b > a/2$:}
Então $r = a-b < a/2$.
\casestyle{Caso $b < a/2$:}
Então $r < b < a/2$.

\endexercise

\endsection
%%}}}

%%{{{ The extended Euclidean algorithm 
\section O algoritmo estendido de Euclides.

\note.
Nos já provamos que o m.d.c.~$\gcd a b$ de dois inteiros $a$ e $b$
pode ser escrito como uma combinação linear deles, mas como podemos
realmente \emph{achar}\/ inteiros $s,t\in\ints$ que satisfazem a
$$
\gcd a b = as + bt,     \qquad s,t\in\ints?
$$
Supresamente a resposta já está ``escondida'' no mesmo algoritmo de Euclides!

\algorithm Algoritmo estendido de Euclides.
\label{extended_euclidean_algorithm}
\Euclid[algoritmo estendido]
\tdefined{algoritmo}[estendido de Euclides]
\algospec
 INPUT: $a,b\in\ints$, $b>0$
OUTPUT: $s,t\in\ints$ tais que $\gcd a b = as + bt$.
\endspec

\example.
Escreva o $\gcd {101} {73}$ como combinação linear dos $101$ e $73$.

\solution
Primeiramente, precisamos aplicar o algoritmo de Euclides para achar o m.d.c.,
como no~\ref{euclidean_algorithm_example},
mas vamos também resolver cada equação por o seu resto:
$$
\matrix
\format
\r  & \c    & \c       & \l          & \l & \c    & \l & \c           & \r & \c    & \c & \c    &\c & \c         & \c  \\
101 & {}={} & 73       & {}\ntimes{} & 1  & {}+{} & 28 & \qquad\qquad &28  & {}={} &101 & {}-{} &   &{}       {} & 73 \\
73  & {}={} & 28       & {}\ntimes{} & 2  & {}+{} & 17 &              &17  & {}={} &73  & {}-{} & 2 &{}\ntimes{} & 28 \\
28  & {}={} & 17       & {}\ntimes{} & 1  & {}+{} & 11 &              &11  & {}={} &28  & {}-{} &   &{}       {} & 17 \\
17  & {}={} & 11       & {}\ntimes{} & 1  & {}+{} & 6  &              &6   & {}={} &17  & {}-{} &   &{}       {} & 11 \\
11  & {}={} & 6        & {}\ntimes{} & 1  & {}+{} & 5  &              &5   & {}={} &11  & {}-{} &   &{}       {} & 6  \\
6   & {}={} & 5        & {}\ntimes{} & 1  & {}+{} & 1  &              &1   & {}={} &6   & {}-{} &   &{}       {} & 5  \\
5   & {}={} & \boxed 1 & {}\ntimes{} & 5  & {}+{} & 0. &              &    &       &    &       &    &          &     \\
\endmatrix
$$
Utilizando as equações no lado direto, de baixo para cima, calculamos:
$$
\def\hl#1{\underline{#1}}
\alignat 2
1 &= \hl6 - \hl5                                           &\quad&\text{(6 e 5)   }\\
  &= \hl6 - (\hl{11} - \hl6)
   = \hl6 - \hl{11} + \hl6
   = -\hl{11} + 2\ntimes \hl6                                   &&\text{(11 e 6)  }\\
  &= -\hl{11} + 2\ntimes (\hl{17} - \hl{11})
   = -\hl{11} + 2\ntimes \hl{17} - 2\ntimes \hl{11}
   = 2\ntimes \hl{17} - 3\ntimes \hl{11}                        &&\text{(17 e 11) }\\
  &= 2\ntimes \hl{17} - 3\ntimes (\hl{28} - \hl{17})
   = 2\ntimes \hl{17} - 3\ntimes \hl{28} + 3\ntimes 17
   = -3\ntimes \hl{28} + 5\ntimes \hl{17}                       &&\text{(28 e 17) }\\
  &= -3\ntimes \hl{28} + 5\ntimes (\hl{73} - 2\ntimes \hl{28})
   = -3\ntimes \hl{28} + 5\ntimes \hl{73} - 10\ntimes \hl{28}
   = 5\ntimes \hl{73} -13\ntimes \hl{28}                        &&\text{(73 e 28) }\\
  &= 5\ntimes \hl{73} -13\ntimes (\hl{101} - \hl{73})
   = 5\ntimes \hl{73} -13\ntimes \hl{101} + 13\ntimes \hl{73}
   = -13\ntimes \hl{101} + 18\ntimes \hl{73}                    &&\text{(101 e 73)}
\endalignat
$$
No lado direto mostramos nosso progresso, no sentido de ter conseguido
escrever o m.d.c.~como combinação linear de quais dois números.
Sublinhamos os inteiros que nos interessam para não perder nosso foco.
Em cada nova linha, escolhemos o menor dos dois números sublinhados,
e o substituimos por a combinação linear que temos graças o algoritmo de Euclides.
Obviamente, essa notação e metodologia não tem nenhum sentido matematicamente
falando.  Serve apenas para ajudar nossos olhos humanos.
\endgraf
Achamos então $s,t\in\ints$ que satisfazem a equação $1 = sa + tb$:
são os $s = -13$ e $t = 18$.
\endexample

\exercise.
Usando o algoritmo estendido de Euclides, escreve:
(i) o $\gcd {108} {174}$ como combinação linear dos 108 e 174; 
(ii) o $\gcd {2016} {305}$ como combinação linear dos 2016 e 305.

\endexercise

\endsection
%%}}}

%%{{{ The sieve of Eratosthenes 
\section O crivo de Eratostenes.

\question.
Como podemos achar todos os primos até um dado limitante $b\in\nats$?

\Eratosthenes[crivo]%
\noindent
Eratostenes (276--194 a.C.)~conseguiu responder com sua metodo conhecida como
o ``crivo de Eratostenes''.
Antes de descrever o seu algoritmo formalmente, vamos aplicar sua idéia
para achar todos os primos menores ou iguais que $b=128$.

{%
\def\co#1{\phantom{#1}}%
\def\ci#1{\underline{#1}}%
Primeiramente liste todos os números de $2$ até $b=128$:
$$
\matrix
\format
~\r  &~\r  &~\r   &~\r   &~\r   &~\r   &~\r   &~\r   &~\r   &~\r   &~\r   &~\r   &~\r   &~\r   &~\r   &~\r   \\
   {   }&   {  2}&   {   3}&   {   4}&   {   5}&   {   6}&   {   7}&   {   8}&   {   9}&   {  10}&   {  11}&   {  12}&   {  13}&   {  14}&   {  15}&   {  16}\\
   { 17}&   { 18}&   {  19}&   {  20}&   {  21}&   {  22}&   {  23}&   {  24}&   {  25}&   {  26}&   {  27}&   {  28}&   {  29}&   {  30}&   {  31}&   {  32}\\
   { 33}&   { 34}&   {  35}&   {  36}&   {  37}&   {  38}&   {  39}&   {  40}&   {  41}&   {  42}&   {  43}&   {  44}&   {  45}&   {  46}&   {  47}&   {  48}\\
   { 49}&   { 50}&   {  51}&   {  52}&   {  53}&   {  54}&   {  55}&   {  56}&   {  57}&   {  58}&   {  59}&   {  60}&   {  61}&   {  62}&   {  63}&   {  64}\\
   { 65}&   { 66}&   {  67}&   {  68}&   {  69}&   {  70}&   {  71}&   {  72}&   {  73}&   {  74}&   {  75}&   {  76}&   {  77}&   {  78}&   {  79}&   {  80}\\
   { 81}&   { 82}&   {  83}&   {  84}&   {  85}&   {  86}&   {  87}&   {  88}&   {  89}&   {  90}&   {  91}&   {  92}&   {  93}&   {  94}&   {  95}&   {  96}\\
   { 97}&   { 98}&   {  99}&   { 100}&   { 101}&   { 102}&   { 103}&   { 104}&   { 105}&   { 106}&   { 107}&   { 108}&   { 109}&   { 110}&   { 111}&   { 112}\\
   {113}&   {114}&   { 115}&   { 116}&   { 117}&   { 118}&   { 119}&   { 120}&   { 121}&   { 122}&   { 123}&   { 124}&   { 125}&   { 126}&   { 127}&   { 128.}
\endmatrix
$$
Agora começa com o primeiro número na lista, o $2$, e apaga todos os maiores múltiplos dele:
$$
\matrix
\format
~\r &~\r &~\r  &~\r  &~\r  &~\r  &~\r  &~\r  &~\r  &~\r  &~\r  &~\r  &~\r  &~\r  &~\r  &~\r   \\
   {   }&   {  2}&   {   3}&\co{   4}&   {   5}&\co{   6}&   {   7}&\co{   8}&   {   9}&\co{  10}&   {  11}&\co{  12}&   {  13}&\co{  14}&   {  15}&\co{  16}\\
   { 17}&\co{ 18}&   {  19}&\co{  20}&   {  21}&\co{  22}&   {  23}&\co{  24}&   {  25}&\co{  26}&   {  27}&\co{  28}&   {  29}&\co{  30}&   {  31}&\co{  32}\\
   { 33}&\co{ 34}&   {  35}&\co{  36}&   {  37}&\co{  38}&   {  39}&\co{  40}&   {  41}&\co{  42}&   {  43}&\co{  44}&   {  45}&\co{  46}&   {  47}&\co{  48}\\
   { 49}&\co{ 50}&   {  51}&\co{  52}&   {  53}&\co{  54}&   {  55}&\co{  56}&   {  57}&\co{  58}&   {  59}&\co{  60}&   {  61}&\co{  62}&   {  63}&\co{  64}\\
   { 65}&\co{ 66}&   {  67}&\co{  68}&   {  69}&\co{  70}&   {  71}&\co{  72}&   {  73}&\co{  74}&   {  75}&\co{  76}&   {  77}&\co{  78}&   {  79}&\co{  80}\\
   { 81}&\co{ 82}&   {  83}&\co{  84}&   {  85}&\co{  86}&   {  87}&\co{  88}&   {  89}&\co{  90}&   {  91}&\co{  92}&   {  93}&\co{  94}&   {  95}&\co{  96}\\
   { 97}&\co{ 98}&   {  99}&\co{ 100}&   { 101}&\co{ 102}&   { 103}&\co{ 104}&   { 105}&\co{ 106}&   { 107}&\co{ 108}&   { 109}&\co{ 110}&   { 111}&\co{ 112}\\
   {113}&\co{114}&   { 115}&\co{ 116}&   { 117}&\co{ 118}&   { 119}&\co{ 120}&   { 121}&\co{ 122}&   { 123}&\co{ 124}&   { 125}&\co{ 126}&   { 127}&\co{ 128}
\endmatrix
$$
Toma o próximo número que está ainda na lista, o $3$, e faça a mesma coisa:
$$
\matrix
\format
~\r &~\r &~\r  &~\r  &~\r  &~\r  &~\r  &~\r  &~\r  &~\r  &~\r  &~\r  &~\r  &~\r  &~\r  &~\r   \\
   {   }&   {  2}&   {   3}&\co{   4}&   {   5}&\co{   6}&   {   7}&\co{   8}&\co{   9}&\co{  10}&   {  11}&\co{  12}&   {  13}&\co{  14}&\co{  15}&\co{  16}\\
   { 17}&\co{ 18}&   {  19}&\co{  20}&\co{  21}&\co{  22}&   {  23}&\co{  24}&   {  25}&\co{  26}&\co{  27}&\co{  28}&   {  29}&\co{  30}&   {  31}&\co{  32}\\
\co{ 33}&\co{ 34}&   {  35}&\co{  36}&   {  37}&\co{  38}&\co{  39}&\co{  40}&   {  41}&\co{  42}&   {  43}&\co{  44}&\co{  45}&\co{  46}&   {  47}&\co{  48}\\
   { 49}&\co{ 50}&\co{  51}&\co{  52}&   {  53}&\co{  54}&   {  55}&\co{  56}&\co{  57}&\co{  58}&   {  59}&\co{  60}&   {  61}&\co{  62}&\co{  63}&\co{  64}\\
   { 65}&\co{ 66}&   {  67}&\co{  68}&\co{  69}&\co{  70}&   {  71}&\co{  72}&   {  73}&\co{  74}&\co{  75}&\co{  76}&   {  77}&\co{  78}&   {  79}&\co{  80}\\
\co{ 81}&\co{ 82}&   {  83}&\co{  84}&   {  85}&\co{  86}&\co{  87}&\co{  88}&   {  89}&\co{  90}&   {  91}&\co{  92}&\co{  93}&\co{  94}&   {  95}&\co{  96}\\
   { 97}&\co{ 98}&\co{  99}&\co{ 100}&   { 101}&\co{ 102}&   { 103}&\co{ 104}&\co{ 105}&\co{ 106}&   { 107}&\co{ 108}&   { 109}&\co{ 110}&\co{ 111}&\co{ 112}\\
   {113}&\co{114}&   { 115}&\co{ 116}&\co{ 117}&\co{ 118}&   { 119}&\co{ 120}&   { 121}&\co{ 122}&\co{ 123}&\co{ 124}&   { 125}&\co{ 126}&   { 127}&\co{ 128}
\endmatrix
$$
Repeta o processo (o próximo agora seria o $5$) até não tem mais números para tomar.
Os números que ficarão são todos os primos até o $128$:
$$
\matrix
\format
~\r &~\r &~\r  &~\r  &~\r  &~\r  &~\r  &~\r  &~\r  &~\r  &~\r  &~\r  &~\r  &~\r  &~\r  &~\r   \\
   {   }&   {  2}&   {   3}&\co{   4}&   {   5}&\co{   6}&   {   7}&\co{   8}&\co{   9}&\co{  10}&   {  11}&\co{  12}&   {  13}&\co{  14}&\co{  15}&\co{  16}\\
   { 17}&\co{ 18}&   {  19}&\co{  20}&\co{  21}&\co{  22}&   {  23}&\co{  24}&\co{  25}&\co{  26}&\co{  27}&\co{  28}&   {  29}&\co{  30}&   {  31}&\co{  32}\\
\co{ 33}&\co{ 34}&\co{  35}&\co{  36}&   {  37}&\co{  38}&\co{  39}&\co{  40}&   {  41}&\co{  42}&   {  43}&\co{  44}&\co{  45}&\co{  46}&   {  47}&\co{  48}\\
   { 49}&\co{ 50}&\co{  51}&\co{  52}&   {  53}&\co{  54}&\co{  55}&\co{  56}&\co{  57}&\co{  58}&   {  59}&\co{  60}&   {  61}&\co{  62}&\co{  63}&\co{  64}\\
\co{ 65}&\co{ 66}&   {  67}&\co{  68}&\co{  69}&\co{  70}&   {  71}&\co{  72}&   {  73}&\co{  74}&\co{  75}&\co{  76}&   {  77}&\co{  78}&   {  79}&\co{  80}\\
\co{ 81}&\co{ 82}&   {  83}&\co{  84}&\co{  85}&\co{  86}&\co{  87}&\co{  88}&   {  89}&\co{  90}&   {  91}&\co{  92}&\co{  93}&\co{  94}&\co{  95}&\co{  96}\\
   { 97}&\co{ 98}&\co{  99}&\co{ 100}&   { 101}&\co{ 102}&   { 103}&\co{ 104}&\co{ 105}&\co{ 106}&   { 107}&\co{ 108}&   { 109}&\co{ 110}&\co{ 111}&\co{ 112}\\
   {113}&\co{114}&\co{ 115}&\co{ 116}&\co{ 117}&\co{ 118}&   { 119}&\co{ 120}&   { 121}&\co{ 122}&\co{ 123}&\co{ 124}&\co{ 125}&\co{ 126}&   { 127}&\co{ 128}
\endmatrix
$$
Tomando o $7$:
$$
\matrix
\format
~\r &~\r &~\r  &~\r  &~\r  &~\r  &~\r  &~\r  &~\r  &~\r  &~\r  &~\r  &~\r  &~\r  &~\r  &~\r   \\
   {   }&   {  2}&   {   3}&\co{   4}&   {   5}&\co{   6}&   {   7}&\co{   8}&\co{   9}&\co{  10}&   {  11}&\co{  12}&   {  13}&\co{  14}&\co{  15}&\co{  16}\\
   { 17}&\co{ 18}&   {  19}&\co{  20}&\co{  21}&\co{  22}&   {  23}&\co{  24}&\co{  25}&\co{  26}&\co{  27}&\co{  28}&   {  29}&\co{  30}&   {  31}&\co{  32}\\
\co{ 33}&\co{ 34}&\co{  35}&\co{  36}&   {  37}&\co{  38}&\co{  39}&\co{  40}&   {  41}&\co{  42}&   {  43}&\co{  44}&\co{  45}&\co{  46}&   {  47}&\co{  48}\\
\co{ 49}&\co{ 50}&\co{  51}&\co{  52}&   {  53}&\co{  54}&\co{  55}&\co{  56}&\co{  57}&\co{  58}&   {  59}&\co{  60}&   {  61}&\co{  62}&\co{  63}&\co{  64}\\
\co{ 65}&\co{ 66}&   {  67}&\co{  68}&\co{  69}&\co{  70}&   {  71}&\co{  72}&   {  73}&\co{  74}&\co{  75}&\co{  76}&\co{  77}&\co{  78}&   {  79}&\co{  80}\\
\co{ 81}&\co{ 82}&   {  83}&\co{  84}&\co{  85}&\co{  86}&\co{  87}&\co{  88}&   {  89}&\co{  90}&\co{  91}&\co{  92}&\co{  93}&\co{  94}&\co{  95}&\co{  96}\\
   { 97}&\co{ 98}&\co{  99}&\co{ 100}&   { 101}&\co{ 102}&   { 103}&\co{ 104}&\co{ 105}&\co{ 106}&   { 107}&\co{ 108}&   { 109}&\co{ 110}&\co{ 111}&\co{ 112}\\
   {113}&\co{114}&\co{ 115}&\co{ 116}&\co{ 117}&\co{ 118}&\co{ 119}&\co{ 120}&   { 121}&\co{ 122}&\co{ 123}&\co{ 124}&\co{ 125}&\co{ 126}&   { 127}&\co{ 128}
\endmatrix
$$
Tomando o $11$:
$$
\matrix
\format
~\r &~\r &~\r  &~\r  &~\r  &~\r  &~\r  &~\r  &~\r  &~\r  &~\r  &~\r  &~\r  &~\r  &~\r  &~\r   \\
   {   }&   {  2}&   {   3}&\co{   4}&   {   5}&\co{   6}&   {   7}&\co{   8}&\co{   9}&\co{  10}&   {  11}&\co{  12}&   {  13}&\co{  14}&\co{  15}&\co{  16}\\
   { 17}&\co{ 18}&   {  19}&\co{  20}&\co{  21}&\co{  22}&   {  23}&\co{  24}&\co{  25}&\co{  26}&\co{  27}&\co{  28}&   {  29}&\co{  30}&   {  31}&\co{  32}\\
\co{ 33}&\co{ 34}&\co{  35}&\co{  36}&   {  37}&\co{  38}&\co{  39}&\co{  40}&   {  41}&\co{  42}&   {  43}&\co{  44}&\co{  45}&\co{  46}&   {  47}&\co{  48}\\
\co{ 49}&\co{ 50}&\co{  51}&\co{  52}&   {  53}&\co{  54}&\co{  55}&\co{  56}&\co{  57}&\co{  58}&   {  59}&\co{  60}&   {  61}&\co{  62}&\co{  63}&\co{  64}\\
\co{ 65}&\co{ 66}&   {  67}&\co{  68}&\co{  69}&\co{  70}&   {  71}&\co{  72}&   {  73}&\co{  74}&\co{  75}&\co{  76}&\co{  77}&\co{  78}&   {  79}&\co{  80}\\
\co{ 81}&\co{ 82}&   {  83}&\co{  84}&\co{  85}&\co{  86}&\co{  87}&\co{  88}&   {  89}&\co{  90}&\co{  91}&\co{  92}&\co{  93}&\co{  94}&\co{  95}&\co{  96}\\
   { 97}&\co{ 98}&\co{  99}&\co{ 100}&   { 101}&\co{ 102}&   { 103}&\co{ 104}&\co{ 105}&\co{ 106}&   { 107}&\co{ 108}&   { 109}&\co{ 110}&\co{ 111}&\co{ 112}\\
   {113}&\co{114}&\co{ 115}&\co{ 116}&\co{ 117}&\co{ 118}&\co{ 119}&\co{ 120}&\co{ 121}&\co{ 122}&\co{ 123}&\co{ 124}&\co{ 125}&\co{ 126}&   { 127}&\co{ 128}
\endmatrix
$$
E podemos já parar aqui, certos que os números que ainda ficam na lista, são todos os primos desejados.
}

\exercise.
Por quê?

\endexercise

\exercise.
\label{eratosthenian_algorithm}
Escreva formalmente o algoritmo do Eratostenes.

\endexercise

\codeit.
\label{implement_eratosthenian_algorithm}
Implemente o algoritmo de Eratostenes e use-lo para achar todos os primos até o $1024$.
\endcodeit

\endsection
%%}}}

%%{{{ The fundamental theorem of arithmetic 
\section O teorema fundamental da aritmética.

\theorem fundamental da aritmética (Euclides, Gauss).
\ii{teorema}[fundamental da aritmética]%
\Gauss{}%
\Euclid{}%
\label{fundamental_theorem_of_arithmetic}%
Todo $n\in\nats$ com $n > 1$, pode ser escrito como um produtório de primos.
Essa expressão é única se desconsiderar a ordem dos fatores do produtório.
\proof.
Seja $n\in\nats$ com $n > 1$.
\endgraf
\proofstyle{Existência:}
\ii{indução}[forte]
Usamos indução forte (veja~\refn{Strong_induction}).
Caso que $n$ seja primo, trivialmente ele mesmo é o produtório de primos (produtório de tamanho 1).
Caso contrário, $n = ab$, para uns $a,b\in\nats$ com $1<a<n$ e $1<b<n$,
logo podemos assumir (hipoteses indutivas) que cada um deles pode ser
escrito na forma desejada:
$$
\alignat 2
a &= p_1p_2\dotsb p_{k_a},          &\quad& \text{para alguns $p_i$'s primos;}\\
b &= q_1q_2\dotsb q_{k_b},          &\quad& \text{para alguns $q_j$'s primos.}
\endalignat
$$
Então temos
$$
n = ab = (p_1p_2\dotsb p_{k_a})(q_1q_2\dotsb )
       = p_1p_2\dotsb p_{k_a}q_1q_2\dotsb q_{k_b}
$$
que realmente é um produtório de primos.
\endgraf
\proofstyle{Unicidade:}
Suponha que para alguns primos $p_i$'s e $q_j$'s, e uns $s,t\in\nats$, temos:
$$
\align
n &= p_1 p_2 \dotsc p_s,\\
n &= q_1 q_2 \dotsc q_t.
\endalign
$$
Vamos mostrar que $s = t$ e que para todo $i\in\set{1,\dotsc,s}$,  $p_i = q_j$.
Temos
$$
p_1 p_2 \dotsc p_s = q_1 q_2 \dotsc q_t,
$$
e $p_1$ é primo que divide o lado esquerdo, então divide também o lado direito:
$$
p_1 \divides q_1 q_2 \dotsc q_t.
$$
Pelo Lema de Euclides\Euclid~\ref{euclid_lemma}, $p_1\divides q_{j_1}$ para algum $j_1$.
Mas o $q_{j_1}$, sendo um dos $q_j$'s, também é primo.
Logo $p_1 = q_{j_1}$ (veja~\ref{in_primes_divides_means_equals}).
Cancelando o $p_1$, temos:
$$
p_2 \dotsc p_s = q_1 q_2 \dotsc q_{j_1 - 1} q_{j_1 + 1} q_t,
$$
Agora repetimos até um dos dois lados não ter mais fatores primos.
Necessariamente, isso vai acontecer ``simultaneamente'' nos dois lados
(caso contrario teriamos um produtório
de primos igual com 1, impossível), ou seja: $s = t$.
Note que as equações $p_i = q_{j_i}$ mostram a unicidade desejada.
\qed

Graças o teorema fundamental da aritmética podemos definir a:

\definition Representação canónica de inteiros.
\label{canonical_representation_of_ints}
\tdefined{representação canónica}[de inteiros]
Seja $0\neq n\in\ints$.
Sua \dterm{representação canónica} é o produtório
$$
n =
(\pm 1)
\prod_{i=1}^k
p_i^{a_i}
=
(\pm 1)
p_1^{a_1}
p_2^{a_2}
\dotsb
p_k^{a_k},
$$
onde os $p_1 < p_2 < \cdots < p_k$'s são primos, e $a_i\in\nats_{>0}$ para $i=1,\dotsc,k$.
\endgraf
\label{canonical_complete_representation}%
Observe que se relaxar a restrição nos exponentes tal que $a_i\in\nats$,
cada $n\in\ints$ ($n\neq 0$) pode ser representado (também únicamente) como o produtório
$$
n =
(\pm 1)
\prod_{i=0}^k
p_i^{a_i}
=
(\pm 1)
p_0^{a_0}
p_1^{a_1}
\dotsb
p_k^{a_k},
$$
onde agora os $p_0 < p_1 < \cdots < p_k$ são \emph{todos os $k+1$ primeiros primos},
sendo então $p_0 = 2$, e $p_k$ o maior primo divisor do $n$.
Chamamos essa forma a \dterm{representação canónica completa} do $n$.
(Veja também o~\ref{encoding_of_finite_sequences}.)

\exercise.
\label{canonical_representation_with_int_exponents}
O que acontece se relaxar a restrição nos exponentes ainda mais?: $a_i\in\ints$.

\endexercise

\endsection
%%}}}

%%{{{ Open problems 
\section Problemas em aberto.
\label{Open_problems_in_number_theory}%

\conjecture Goldbach.
\label{goldbach_conjecture}%

\conjecture Twin primes.
\label{twin_primes_conjecture}%

\conjecture Legendre.
\label{legendre_conjecture}%

\conjecture Collatz.
\label{collatz_conjecture}%

\endsection
%%}}}

%%{{{ Problems 
\problems.

%%{{{ prob: p_divides_comb_p_r 
\problem.
\label{p_divides_comb_p_r}
Para todo $p$ primo, e todo $r\in\set{1,\dotsc,p-1}$,
$$
p \divides \comb p r.
$$

\hint
$\comb p r \in\ints$, $r < p$, e $p-r < p$.

\endproblem
%%}}}

%%{{{ prob 
\problem.
(Generalização do~\ref{implications_with_divisibility_of_linear_combinations}.)
Para quais $u,v\in\ints$, a afirmação
$$
a \divides b + c \mland a \divides ub + vc \implies a \divides xb + yc \quad \text{para todos $x,y\in\ints$}
$$
é válida?

\endproblem
%%}}}

%%{{{ prob 
\problem.
Prove que para todo $n\in\nats$, $\gcd {F_n} {F_{n+1}} = 1$,
onde $F_n$ é o $n$-ésimo termo da seqüência Fibonacci\Fibonacci[seqüência]{}~(\ref{fibonacci}).

\hint
Tu já resolveu o~\ref{gcd_of_two_is_gcd_of_one_plus_sum}, certo?

\solution
Vamos provar o pedido por indução.
Para $n=0$ temos
$$
    \gcd {F_0} {F_1} = \gcd 0 1 = 1.
$$
Seja $k\in\nats$ tal que $\gcd {F_k} {F_{k+1}} = 1$.
Precisamos mostrar que $\gcd {F_{k+1}} {F_{k+2}} = 1$.
Calculando,
$$
\alignat 2
\gcd {F_{k+1}} {F_{k+2}}
    &= \gcd {F_{k+1}} {F_{k+1} + F_k}   \qqby{pela definição da $F_n$}\\
    &= \gcd {F_{k+1}} {F_k}             \qqby{pelo~\ref{gcd_of_two_is_gcd_of_one_plus_sum}, com $a\asseq F_{k+1},\ b\asseq F_k$}\\
    &= \gcd {F_k} {F_{k+1}}             \qqby{propriedade de mdc}\\
    &= 1                                \qqby{pela hipótese indutiva}.
\endalignat
$$

\endproblem
%%}}}

%%{{{ prob 
\problem.
Seja $n\in\nats$, $n>1$.
Entre $n$ e $n!$ existe primo.

\hint
Olha para o $n!-1$.

\hint
Se $n!-1$ não é primo, toma um dos seus primos divisores, $p$.

\hint
Necessariamente $p > n$.

\endproblem
%%}}}

%%{{{ prob 
\problem.
Seja $n\in\nats$.
Ache $n$ consecutivos números compostos.

\hint
$!$

\hint
$m!+2$.

\hint
$m!+3$\dots

\hint
$m!+m$.

\endproblem
%%}}}

%%{{{ prob 
\problem.
Existe uma infinidade de primos ``da forma $4n+3$'', ou seja, o conjunto
$$
\set{4n+3 \st n\in\nats,\ \text{$4n+3$ primo}}
$$
é infinito.

\hint
Esquecendo o $2$, todos os primos são da forma $4n+1$ ou $4n+3$.
Suponha que $p_1, p_2,\dotsc, p_k$ ($k\in\nats$)
são todos os primos da segunda forma.

\hint
O que Euclides faria?

\hint
Tente achar (criar) um número da mesma forma tal que nenhum dos $p_i$ o divide.

\hint
$N = 4p_1p_2\dotsb p_k - 1$.

\hint
$N$ não pode ter apenas divisores da forma $4n+1$.

\endproblem
%%}}}

%%{{{ prob: WOP_iff_PFI 
\problem.
\label{WOP_iff_PFI}%
$\text{PBO} \iff \text{PIF}$.

\hint
Para provar a \rldir, considere o predicado
``todos os conjuntos $A$ com $k\in A$ para algum natural $k\leq n$ têm mínimo'',
ou o ``todos os conjuntos $A$ com $n\in A$ têm mínimo''.

\hint
Prove que todos os naturais satisfazem o predicado que tu consideraste.
(Para um dos dois, tu precisará indução \emph{forte}.)

\solution
Nos já provamos a direção \lrdir.

\endproblem
%%}}}

%%{{{ gcd_alternative_definition 
\problem {Definição alternativa de m.d.c.}.
\label{gcd_alternative_definition}%
Uma definição alternativa do m.d.c.~é a seguinte:
{\it Sejam $a,b\in\ints$.
O m.d.c.~dos $a$ e $b$ é o maior dos divisores em comum de $a$ e $b$.}
Ache um problema com essa definição, corrige-o, e depois compare
com a~\ref{gcd}.

\hint
O que acontece se $a=b=0$?
O que acontece se pelo menos um dos $a$ e $b$ não é o $0$?

\solution
Se $a=b=0$, o símbolo $\gcd a b$ não é definido,
porque todo $n\in\nats$ é um divisor em comum,
mas como o $\nats$ não tem um elemento máximo,
não existe o maior deles.
\endgraf
Precisamos restringir a definição para ser aplicável
apenas nos casos onde pelo menos um dos $a$ e $b$
não é o $0$ (escrevemos isso curtamente: $ab\neq 0$).
Assim, quando a nova definição é aplicável, ela realmente
defina o mesmo número, fato que segue pelas propriedades:
$$
\align
\gcd x 0 = 0 &\implies x = 0\\
x \divides y \mland y \neq 0 &\implies \abs x \leq \abs y.
\endalign
$$

\endproblem
%%}}}

%%{{{ prob 
\problem Contando os passos.
O que muda no~\ref{partitioning_restricted_best_strategy}
se em cada passo podemos quebrar todos os termos que aparecem?
Qual é a melhor estratégia, e quantos passos são necessários?

\hint
Tente quebrar o $10$ usando várias estratégias.
O que tu percebes?

\hint
Qual é o maior termo e como ele muda depois cada passo?

\solution
Agora a estratégia ótima seria quebrar cada termo ``no meio''.
Assim, para o $10$ temos:
$$
\align
10 &= 5 + 5\\
   &= (2 + 3) + (2 + 3)\\
   &= [(1 + 1) + (1 + 2)] + [(1 + 1) + (1 + 2)]\\
   &= 1 + 1 + 1 + 1 + 1 + 1 + 1 + 1 + 1 + 1
\endalign
$$
em apenas $4$ pássos.
Depois cada passo, se o maior termo fosse o $m$, agora é o $\ceil {\frac m 2}$.
Precisamos tantos passos quantas vezes que podemos dividir o $n$ por $2$ até
chegar na unidade $1$: precisamos $\ceil {\log_2(n)}$ passos.

\endproblem
%%}}}

%%{{{ prob: euclidean_algorithm_correctness_formal_proof_by_induction 
\problem.
\label{euclidean_algorithm_correctness_formal_proof_by_induction}%
\def\Euclid{\algorithmstyle{Euclid}}%
Como tu percebeu resolvendo o~\ref{euclidean_algorithm_proof_why_informal},
nenhuma das duas partes da prova do~\ref{euclidean_algorithm_correctness}
foi formal.
Prove completamente formalmente as duas partes usando indução.

\hint Precisa expressar o alvo na forma $\lforall n {P(n)}$ para um certo predicado $P$.

\hint
\def\Euclid{\algorithmstyle{Euclid}}%
``\emph{Para todo $x\in\ints$, o $\Euclid(x,n)$ termina com o resultado certo}.''

\hint O que significaria $P(0)$?  $P(b)$?

\hint Ser forte é coisa boa.

\solution
\def\Euclid{\algorithmstyle{Euclid}}%
Seja o predicado
$$
P(n) \defiff \lforall {x\in\ints} {\text{$\Euclid(x,n)$ termina com $\gcd x n$}}
$$
Vamos provar que $\lforall {n\in\nats} {P(n)}$ por indução forte.
Seja $k\in\nats$ tal que $P(i)$ é válido para todo $i<k$ (hipótese indutiva).
Precisamos provar o $P(k)$, ou seja, que
\emph{para todo $x\in\ints$, $\Euclid(x,k)$ termina e $\Euclid(x,k) = \gcd x k$}.
Seja $x\in\ints$, e aplica o $\Euclid(x,k)$ para um passo.
Se $k=0$, a computação termina imediatamente com o resultado $x$, que é correto
(\refn{gcd_of_comparable}).
Se $k>0$, o algoritmo manda reduzir sua computação para a computação do $\Euclid(k,r)$,
onde $r = x \bmod k < k$.  Seguindo o~\ref{euclid_gcd_lemma} $\gcd x k = \gcd k r$, então
falta verificar que o $\Euclid(k,r)$ termina mesmo com $\gcd k r$,
que é verdade pela hipótese indutiva porque $r < k$ e logo $P(r)$ é válido.

\endproblem
%%}}}

%%{{{ prob: euclidean_algorithm_correctness_formal_proof_by_wop 
\problem.
\label{euclidean_algorithm_correctness_formal_proof_by_wop}%
Prove completamente formalmente as duas partes
do~\ref{euclidean_algorithm_correctness} usando o princípio da boa ordem.

\hint
Considera as duas partes separamente:
qual conjunto é o não vazio em cada parte?

\hint
Vai pelo absurdo.

\hint
O que seria um contraxemplo para cada parte?
\emph{O que acontece se existem contraexemplos?}

\hint 
\def\Euclid{\algorithmstyle{Euclid}}%
Sobre sua terminação:
Considere o menor $m$ tal que o $\Euclid(x,m)$ não termina
para algum $x\in\ints$.

\hint
\def\Euclid{\algorithmstyle{Euclid}}%
Sobre sua corretude:
Considere o menor $m$ tal que o $\Euclid(x,m)$
termina com resultado errado par algum $x\in\ints$.
Mas mostre primeiro a terminação.

\solution
\def\Euclid{\algorithmstyle{Euclid}}%
Provamos cada parte separamente:
\endgraf
\smallskip
\proofstyle{Terminação.}
Para chegar num absurdo, suponha que existem contraexemplos:
\emph{inteiros $c\geq0$, tais que o $\Euclid(a,c)$
não termina para algum $x\in\ints$.}
Seja $m$ o menor deles (PBO):
$$
m = \min\set{c \in\nats \st \lexists {x\in\ints} {\text{\Euclid(x,c) não termina}}}.
$$
Logo, para algum certo $a\in\ints$, temos que $\Euclid(a,m)$ não termina.
Com certeza $m\neq 0$, porque nesse caso o algoritmo termina imediatamente.
Logo $m > 0$ e aplicando o $\Euclid(a,m)$ para apenas um passo
a sua computação é reduzida no computação do $\Euclid(m,r)$,
onde $r = a \bmod m < m$, \emph{e agora precisamos mostrar que o
$\Euclid(m,r)$ termina para chegar num absurdo}.
Pela escolha do $m$ como \emph{mínimo} dos contraexemplos,
o $r$ não pode ser contraexemplo também.
Em outras palavras, o $\Euclid(x,r)$ realmente termina para qualquer $x$,
então para $x=m$ também, que foi o que queriamos provar.
\endgraf
\smallskip
\proofstyle{Corretude.}
Para chegar num absurdo, suponha que existem contraexemplos:
\emph{inteiros $c\geq0$, tais que o $\Euclid(x,c)$
acha resultado errado para algum $x\in\ints$.}
Seja $m$ o menor desses contraexemplos (PBO):
$$
m = \min\set{c \in\nats \st \lexists {x\in\ints} {\Euclid(x,c) \neq \gcd x c}}.
$$
Logo, para algum certo $a\in\ints$, temos $\Euclid(a,m) \neq \gcd a m$.
Esse $m$ não pode ser $0$, porque nesse caso o algoritmo retorna sua primeira entrada $a$;
resultado correto por causa do~\ref{gcd_of_comparable}.
Então $m>0$.
Aplicamos para um passo o $\Euclid(a,m)$.
Como $m\neq 0$, o algoritmo manda realizar o segundo passo:
retornar o que $\Euclid(m,r)$, onde $r = a \bmod m < m$.
(E sabemos que o $\Euclid(m,r)$ vai retornar algo, porque
já provamos a terminação do algoritmo para todas as suas possíveis entradas!)
Para concluir, observe:
$$
\alignat 2
\Euclid(m,r) 
&= \Euclid(a,m)     \qqby{pelas instruções do algoritmo}\\
&\neq \gcd a m      \qqby{escolha dos $m$ e $a$}\\
&= \gcd m r.        \qqby{pelo~\ref{euclid_gcd_lemma}}
\endalignat
$$
Então $\Euclid(m,r) \neq \gcd m r$ e achamos um contraexemplo (o $r$)
menor que o mínimo (o $m$)---absurdo!

\endproblem
%%}}}

%%{{{ prob: encoding_of_finite_sequences 
\problem Codificação de seqüências finitas.
\label{encoding_of_finite_sequences}%
Seja $S$ o conjunto de seqüências finitas de números naturais.
Descreva uma método para ``codificar'' os elementos de $S$
com os elementos de $\nats\setminus\set0$.
Tua metodo deve ser uma \emph{revertível}, no sentido que
cada seqüência finita
$$
s = \seq{s_0,s_1,\dotsc,s_{k_s}}\in S
$$
deve corresponder em exatamente um número natural $n_s\in\nats\setminus\set0$,
e, dado esse número $n_s\in\nats_{>0}$, deveria ser possível ``extrair''
a seqüência $s$ cuja codificação é o $n_s$.
Não se preocupe se existem naturais que não são codificações de nenhuma
seqüência.

\hint Use o teorema fundamental da aritmética~(\refn{fundamental_theorem_of_arithmetic}).

\hint
Seja $p_i$ o $i$-esimo primo ($p_0 = 2$, $p_1 = 3$, $p_2 = 5$, \dots).
Como podemos escrever o aleatorio $n\in\nats$?

\hint
Olha nos exponentes na forma
$
n =
p_0^{a_0}
p_1^{a_1}
p_2^{a_2}
\cdots
p_{k_n}^{a_{k_n}}
$.

\hint
Teste tua codificação nas seqüências $\seq{1,3}$ e $\seq{1,3,0}$.
Como essas seqüências são diferentes, suas codificações devem ser diferentes também.
E a seqüência vazia $\seq{}\in S$?  

\solution
Seja
$$
p_0 < p_1 < p_2 < p_3 < \dotsb
$$
a seqüência infinita dos primos.  (Assim, $p_0 = 2$, $p_1 = 3$, $p_2 = 5$, etc.)
Seja
$$
\seq{s_0, s_1, \dotsc, s_{n-1}} \in S
$$
uma seqüência de naturais de tamanho $n$.
Vamos a codificar com o inteiro
$$
c_s
= \prod_{i=0}^{n-1} p_i^{s_i + 1}
= p_0^{s_0 + 1} p_1^{s_1 + 1} p_2^{s_2 + 1} \cdots p_{n-1}^{s_{n-1} + 1}.
$$
(Note que a seqüência vazia ($n=0$) correponde no número $1\in\nats_{>0}$.)
\endgraf
Conversamente, dado um número $c\in\nats_{>0}$ que codifica uma seqüência,
como podemos ``decodificar'' a seqüência que corresponda com ele?
Graças o teorema fundamental da aritmética~(\refn{fundamental_theorem_of_arithmetic}),
temos
$$
c = p_0^{a_0} p_1^{a_1} \cdots p_{m-1}^{a_{m-1}}.
$$
Se existe exponente $a_j = 0$, o $c$ não codifica nenhuma seqüência.
Caso contrario, todos os exponentes são positivos, e o $c$ codifica
a seqüência
$$
\seq{a_0-1, a_1-1, \dotsc, a_m-1} \in S.
$$

\endproblem
%%}}}

%%{{{ prob: canonical_representation_of_rats 
\problem Representação canónica de racionais.
\label{canonical_representation_of_rats}%
Generalize a representação canónica de inteiros para racionais.

\hint
Qual foi tua resposta no~\ref{canonical_representation_with_int_exponents}?

\endproblem
%%}}}

\endproblems
%%}}}

%%{{{ Further reading 
\further.

Veja o~\cite[\S\S1.6--1.8]{babybm}.

\cite{elements},
\cite{disquisitiones}.

\cite{andrewsnumber}.

\cite{nivennumbers},
\cite{hardywright}.

\endfurther
%%}}}

\endchapter
%%}}}

%%{{{ chapter: Number theory II: congruences 
\chapter Teoria de números II: congruências.
\label{Congruences}%

%%{{{ The idea behind congruence relation 
\section A idéia da relação de congruência.

\note.
Vamos fixar um inteiro positivo $m\in\nats$.
Graças à divisão de Euclides~(\refn{euclidean_division}),
qualquer inteiro $a\in\ints$ pode ser escrito na forma
$$
a = mk + r,
\qquad
0 \leq r < m,
$$
num jeito único.
\endgraf
Enquanto investigando a (ir)racionalidade dos $\sqrt 2$, $\sqrt 3$, $\sqrt {\vphantom3 m}$,
etc., nos percebemos que foi útil separar os inteiros em classes,
``grupando'' aqueles que compartilham o mesmo resto quando divididos por $m$.
Trabalhando com essa idéia nos encontramos nosso primeiro contato com
\emph{aritmética modular}.%
\footnote{Mentira.  Não foi o primeiro não: nos somos todos acostumados com
aritmética modular mesmo sem perceber.  Um desses contatos é por causa de ter
que contar com horas e relógios, cuja aritmética não parece muito com aquela
dos inteiros.  Por exemplo: 21 + 5 = 26, mas se agora são 21h00, que horas
serão depois de 5 horas?  Nossos relógios não vou mostrar 26h00, mas 02h00.}

\endsection
%%}}}

%%{{{ Two equivalent definitions 
\section Duas definições equivalentes.

\blah.
Precisamos definir formalmente a noção de ``\emph{dois inteiros $a$ e $b$ pertencem na
mesma classe, quando separamos eles em grupos usando o inteiro $m$}''.
A primeira coisa que precisamos perceber é que essa frase é uma afirmação sobre 3 objetos.
Queremos então uma definição e uma notação que captura essa relação \emph{de aridade 3}.

\note Congruência intuitivamente (1).
Chamamos dois inteiros \dterm{congruentes} módulo um terceiro inteiro,
sse eles têm o mesmo resto, quando divididos por ele.

\note Crítica.
Primeiramente, o texto da definição é bem informal e ambíguo.
Para tirar essas ambiguidades, precisamos introduzir variáveis para
referir sobre os ``mesmos restos'':

\note Congruência intuitivamente (2).
\label{congruence_intuitive_definition}
Sejam $a,b,m\in\ints$ com $m>0$, e sejam $q_a$, $r_a$, $q_b$, e $r_b$
os inteiros determinados por as divisões:
$$
\alignat 2
a &= mq_a + r_a     &\qquad& 0 \leq r_a < m\\
b &= mq_b + r_b     &      & 0 \leq r_b < m
\endalignat
$$
Digamos que os $a$ e $b$ são \dterm{congruentes} módulo~$m$,
sse $r_a = r_b$.

\remark.
\label{from_same_remainders_to_divides_the_diference}
Olhando para dois números $a$ e $b$, congruêntes módulo~$m$,
o que podemos dizer sobre a diferença deles?
Observe:
$$
\left.
\aligned
a - b
&= (mq_a + r_a) - (mq_b + r_b)\\
&= mq_a - mq_b + r_a - r_b\\
&= m(q_a - q_b) + (r_a - r_b)\\
&= m(q_a - q_b) + 0\\
&= m(q_a - q_b)
\endaligned
\right\}
\qquad
\aligned
\text{ou seja, $m\divides a - b$.}
\endaligned
$$
Essa observação nos mostra um caminho mais curto e elegante para definir o mesmo
conceito.  É o seguinte:

\definition Congruência (Gauss).
\label{congruence}
\tdefined{congruência}
\tdefined{módulo}
\Gauss[definição de congruência]
\sdefined{\holed a \cong {\holed b} \pmod {\holed m}}{\holed a é congruente com \holed b módulo \holed m}
Sejam $a,b,m\in\ints$ com $m>0$.
Digamos que os $a$ e $b$ são \dterm{congruentes} \dterm{módulo} $m$,
sse $m \divides a - b$.
Em símbolos, escrevemos
$$
a \cong b \pmod m
\defiff m \divides a - b
$$
e lemos: \emph{o $a$ é congruente com $b$ módulo~$m$}.

\beware.
A notação de congruência as vezes iluda de ser enterpredada como se fosse
uma relação entre o lado esquerdo $L$ e o lado direito $R$, assim:
$$
\underbrace{a}_{L} \cong \underbrace{b \pmod m}_{R}.
$$
\emph{Não!}
Principalmente, o lado direito, $b \pmod m$, nem é definido, então não tem significado,
e nem faz sentido afirmar algo sobre ele.
Prestando mais atenção, percebemos que o $\hole \cong \hole$ também não foi definido!
O que nos definimos foi o:
$$
\holed u \cong \holed v \pmod {\holed w}
$$
dados $u,v,w\in\ints$ com $w>1$.

\note Intuição.
Se precisamos para algum motivo pessoal---porque sim---separar mentalmente a
notação de congruência em dois lados, o único jeito que faz algum sentido
seria o:
$$
\underbrace{\mathstrut a \cong b}_L\quad\underbrace{\!\!\pmod m}_{R}.
$$
Assim, entendemos que ``algo acontece'' (lado $L$), ``dentro algo'' (lado $R$),
onde ``algo acontece'' seria ``o $a$ \emph{parece} com $b$'',
e ``dentro algo'' seria ``módulo~$m$''.
Mas, claramente tudo isso é apenas uma guia (caso que queremos)
e nada mais que isso.  Para argumentar sobre a relação de congruência,
usamos \emph{apenas sua definição formal!}

Para ganhar o direito de usar qualquer uma das duas definições, precisamos
mostrar que são equivalentes:

\theorem Equivalência das duas definições.
Sejam $a,b,m\in\ints$ com $m>0$, e sejam $q_a, r_a, q_b, r_b\in\ints$
os números determinados por as divisões:
$$
\alignat 2
a &= mq_a + r_a     &\qquad& 0 \leq r_a < m\\
b &= mq_b + r_b     && 0 \leq r_b < m
\endalignat
$$
Temos a equivalência:
$$
a \cong b \pmod m
\iff
r_a = r_b.
$$
\sketch.
Precisamos mostrar as duas direções do \bidir.
A direção \rldir, é praticamente
a~\ref{from_same_remainders_to_divides_the_diference}.
Para a direção \lrdir, vamos mostrar que $r_a - r_b = 0$.
Usamos a hipótese e propriedades de~$\divides$
para mostrar que $m\divides r_a - r_b$,
e depois as duas disegualdades para confirmar que, com suas restrições,
o único inteiro múltiplo de $m$ que as satisfaça, é o $0$.
\qes
\proof.
Precisamos mostrar as duas direções do \bidir:
\endgraf
\lrdir:
Suponha que
$a \cong b \pmod m$, ou seja,
$m\divides a - b$.
Resolvendo as duas equações das divisões por os restos $r_a$ e $r_b$,
temos:
$$
\left.
\aligned
r_a &= a - mq_a \\
r_b &= b - mq_b 
\endaligned
\right\}
\implies
\aligned
r_a - r_b
&= (a - mq_a) - (b - mq_b)\\
&= (a - b) - (mq_a - mq_b)\\
&= (a - b) - m(q_a - q_b).
\endaligned
$$
Observe que $m\divides a-b$ e $m\divides m(q_a - q_b)$.
Então $m$ tem que dividir a diferença deles também:
$$
m\divides \underbrace{(a - b) - m(q_a - q_b)}_{\dsize r_a - r_b}
$$
Usando as duas disegualdades: $0 \leq \abs{r_a - r_b} < m$.
Como $\abs{r_a-r_b}$ é um múltiplo de $m$, concluimos que necessariamente
$0 = \abs{r_a - r_b}$, ou seja: $r_a = r_b$.
\endgraf
\rldir:
Suponha que $r_a = r_b$.
Temos:
$$
\alignat 2
a - b
&= (mq_a + r_a) - (mq_b + r_b)                 \\
&= (mq_a - mq_b) - (r_a - r_b)                 \\
&= (mq_a - mq_b) - 0            \qqby{hipótese}\\
&= m(q_a - q_b),
\endalignat
$$
e como $q_a - q_b\in\ints$,
concluimos que
$m\divides a - b$, ou seja:
$a \cong b \pmod m$.
\qed

\note A operação binária ``mod''.
Em linguagens de programação é comum encontrar o operador \emph{binário}
``mod'', frequentemente denotado com o símbolo ``\thinspace{\tt \%}\thinspace''.
Em matemática, essa função \emph{binária} (aridade 2) é mais
encontrada como $\bmod$ mesmo.
Cuidado não confudir a \emph{função}
$\bmod : \ints\times\nats_{>0} \to \nats$
com a \emph{relação} $a \cong b \pmod m$.
Faz sentido escrever:
$$
69 \bmod 5 = 4.
$$
Isso significa apenas que o resto da divisão de 69 por 5, é 4.
No outro lado, nenhuma das expressões abaixo tem significado!:
$$
69 \pmod 5 = 4
\qquad\qquad
4 = 69 \pmod 5
$$

\exercise.
Explique o tipo da função $\bmod : \ints\times\nats_{>0} \to \nats$.

\hint Qual seria o valor de $4 \bmod 0$?

\solution
Por~\ref{euclidean_division},
precisamos $a\in\ints$ e $b\in\nats_{>0}$ para definir a divisão
de $a$ por $b$.

\endexercise

\exercise mod vs mod.
\label{mod-vs-mod}
Para cada uma das expressões abaixo uma das 3 opções é valida:
(a) ela denota um termo;
(b) ela denota uma afirmação; ou
(c) ela não tem significado.
Para cada expressão, decida qual é a opção certa e:
se for a (a), ache o seu valor (objeto);
se for a (b), ache o seu valor (de verdade).
\doublecolumns
\beginol
\li $69 \pmod 5 = 4$
\li $12 = 3 \pmod 8 $
\li $12 \cong 20 \pmod 4 $
\li $8 \pmod 3 \cong 12$
\li $108 \cong 208 \pmod {(43 \bmod 30)}$
\li $x \bmod 4 = 2 \implies x \cong 0 \pmod 2$
\li $5^{192 \bmod 3}$
\li $13\pmod 8 \cong 23 \pmod {18}$
\endol
\singlecolumn

\solution
Vamo lá:
\beginol
\li $69 \pmod 5 = 4$ não significa nada.
\li $12 = 3 \pmod 8$ não significa nada.%
\footnote{Se o $=$ fosse $\cong$, então significaria que $8 \divides 12 - 3 = 9$ (que é falso).}
\li $12 \cong 20 \pmod 4$ significa que $4 \divides 12 - 20 = -8$, que é verdade.
\li $8 \pmod 3 \cong 12$ não significa nada.
\li $108 \cong 208 \pmod {(43 \bmod 30)}$ significa que $43\bmod 30 \divides 108 - 208$, e para achar se é verdade ou não, calculamos $43\bmod 30 = 13$, e $108 - 208 = -100$ e substituimos: $13 \divides -100$, que é falso.
\li $\lforall {x\in\ints} {x \bmod 4 = 2 \limplies x \cong 0 \pmod 2}$ denota a afirmação que para todo $x\in\ints$, se $x \bmod 4 = 2$ então $x \cong 0 \pmod 2$, que é verdade:
seja $x\in\ints$ tal que $x \bmod 4 = 2$.  Logo $x = 4k + 2 = 2(2k + 1)$ para algum $k\in\ints$, ou seja, $2 \divides x = x - 0$.
\li $5^{192 \bmod 3}$ é o número 5 elevado ao resto da divisão de 192 por 3.  Como $3 \divides 192$ (por quê?  1 + 9 + 2 = 12; 1 + 2 = 3; veja~\ref{divisibility_criterion_3}), temos $192 \bmod 3 = 0$, então o valor da expressão é o número $5^0 = 1$.
\li $13\pmod 8 \cong 23 \pmod {18}$ não significa nada.
\endol

\endexercise

\note Notação.
Talvez ficaria mais intuitivo (e menos confúso), usar uma notação como a
$$
a \cong_m b \defiff m \divides a - b
$$
que aparece mais fiel na sua semántica.
Mas a notação que é usada internacionalmente, é aquela que nos definimos
e nos vamos ficar consistentemente fieis nela!

\endsection
%%}}}

%%{{{ The relation congruence modulo $m$ 
\section A relação de congruência módulo~$m$.

\note Fixando um $m$.
Se fixar um inteiro $m>0$, então a expressão
$$
\holed a \cong \holed b \pmod m
$$
tem duas variáveis livres: a $a$ e a $b$.
Assim, a ``congruência módulo~$m$'' é uma relação binária,
cujas propriedades investigamos agora.

\theorem.
\ii{relação}[de equivalência]
\label{congruence_is_an_equivalence_relation}
Fixe um inteiro $m > 0$.
Para todos os $a,b\in\ints$,
temos:
$$
\xxalignat 2
(1)\quad&a \cong a \pmod m                             &&\textrm{(reflexividade)}\\
(2)\quad&a \cong b \pmod m \mland b \cong c \pmod m
\implies
a \cong c \pmod m                                      &&\textrm{(transitividade)}\\
(3)\quad&a \cong b \pmod m \implies b \cong a \pmod  m &&\textrm{(simetria)}.
\endxxalignat
$$
\sketch.
Todas são facilmente provadas aplicando diretamente
a definição de congruência módulo~$m$ (\refn{congruence})
e as propriedades básicas da $\divides$.
\qes
\proof.
(1) Obvio por que $m\divides a - a = 0$.
(2) Pela hipótese temos $m\divides a-b$ e $m\divides b-c$.
Então pelo~\ref{divides_properties} $m\divides (a-b) + (b-c) = a-c$.
(3) Pela hipótese temos $m\divides a - b$\thinspace; logo (\refn{divides_properties} de novo) $m\divides -(a - b) = b - a$.
\qed

\endsection
%%}}}

%%{{{ Modular arithmetic 
\section Aritmética modular.

\property.
\label{modular_arithmetic_properties}
Se $a \cong b \pmod m$, então para todo $x\in\ints$ temos:
$$
(1)\quad a + x \cong b + x   \pmod m;
\quad\enspace
(2)\quad ax \cong bx         \pmod m;
\quad\enspace
(3)\quad -a \cong -b         \pmod m.
$$
\sketch.
Todas seguem facilmente pela definição (\refn{congruence}) de congruência módulo~$m$.
\qes

\beware.
\label{wrong_cancellation_law_modulo_m}
O lei de cancelamento, mesmo valido nas igualdades,
não é valido nas congruências em geral.
Por exemplo,
$3\ntimes 2 \cong 3\ntimes 8 \pmod {18}$,
mas não podemos cancelar os $3$ nos dois lados:
$2 \ncong 8 \pmod {18}$.

Felizmente, o teorema seguinte mostra quando realmente podemos cancelar:

\theorem Lei de cancelamento módulo $m$.
\label{cancellation_law_modulo_m}
Seja $c\in\ints$ tal que $\gcd c m = 1$.
$$
ca\cong cb\pmod m
\implies
a \cong b \pmod m.
$$
\sketch.
Multiplicamos tudo por $c^{-1}$, cuja existência (módulo~$m$) é garantida
pela hipótese (aplicando o~\ref{find_inverse_modulo_m}).
\qes
\proof.
Como $\gcd c m = 1$, existe $c^{-1}$ (módulo~$m$) então:
$$
\align
ca\cong cb\pmod m
&\implies          c^{-1}c a\cong          c^{-1}c b\pmod m\\
&\implies \phantom{c^{-1}c}a\cong \phantom{c^{-1}c}b\pmod m.
\endalign
$$
\moveqedup
\qed

\exercise.
Aplicando as definições e propriedades de congruência e da relação $\divides$,
ache uma outra prova do~\refn{cancellation_law_modulo_m}.

\solution
Pela hipótese $m\divides ca - cb = c(a - b)$.
Mas $\gcd m c = 1$, então
(pelo~\ref{coprime_with_a_part_of_a_product_must_divide_the_other_part_to_divide_the_product})
$m\divides a - b$, ou seja, $a \cong b \pmod m$.

\endexercise

\exercise.
\label{from_mod_m_to_mod_am}
Suponha que $x \cong t \pmod m$, e seja $a$ um inteiro positivo.
O que podemos concluir sobre o $x$ módulo~$ma$?

\hint
Use as definições de congruência (\refn{congruence}) e de $\divides$ (\refn{divides})
para escrever o $x$ na forma $x = mk + t$ para algum $k\in\ints$.

\hint
Agora divida o $k$ por $a$.

\solution
Pela definição de congruência (\refn{congruence}) e de $\divides$ (\refn{divides}) temos que:
$$
\align
x &= mk + t, \quad\text{para algum $k\in\ints$}.
\intertext{Dividindo o $k$ por $a$,
temos $k = aq + i$, onde $q,i\in\ints$ e $0\leq i < a$.  Substituindo:}
x &= m(aq + i) + t\\
  &= maq + (mi + t).
\intertext{Logo, chegamos nas $a$ congruências}
x &\cong mi + t \pmod{ma}, \quad\text{para $i=0,\dotsc,a-1$},
\endalign
$$
e $x$ tem que satisfazer (exatamente) uma delas.

\endexercise

\exercise De igualdades para congruências.
\label{from_equality_to_congruence}
Seja $a,b\in\ints$.  Se $a = b$, então $a\cong b\pmod m$ para qualquer inteiro $m\in\nats$.

\endexercise

\endsection
%%}}}

%%{{{ Divisibility criteria 
\section Critéria de divisibilidade.
\ii{divisibilidade}[critéria]%

\criterion Divisibilidade por 10$^k$.
\label{divisibility_criterion_powers_of_10}
Um inteiro $c\neq 0$ é divisível por $10^k$
sse
o $c$ escrito em base decimal
termina com $k$ dígitos 0.

\criterion Divisibilidade por $2$ ou $5$.
\label{divisibility_criterion_2}
\label{divisibility_criterion_5}
Seja $m\in\set{2, 5}$.
Um inteiro $c$ é divisível por $m$
sse
o valor do último dígitos do $c$ (em base decimal) é $m$.

\criterion Divisibilidade por 3 ou 9.
\label{divisibility_criterion_3}
\label{divisibility_criterion_9}
Seja $m\in\set{3, 9}$.
Um inteiro $c$ é divisível por $m$
sse
o somatório dos valores dos dígitos do $c$ (em base decimal) é
divisível por $m$.

\criterion Divisibilidade por 4, 20, 25, 50.
\label{divisibility_criterion_4}
\label{divisibility_criterion_20}
\label{divisibility_criterion_25}
\label{divisibility_criterion_50}
Seja $m\in\set{4, 20, 25, 50}$.
Um inteiro $c$ é divisível por $m$
sse
o número formado por os \oldstyle{2} últimos dígitos do $c$ (em base decimal)
é divisível por $m$.

\criterion Divisibilidade por 11.
\label{divisibility_criterion_11}
Um inteiro $c$ é divisível por $11$
sse
o somatório dos valores dos dígitos do $c$ (em base decimal) em posição par
menos o somatório dos valores dos seus dígitos em posição impar
é divisível por $11$.

\exercise Divisibilidade por 6.
\label{divisibility_criterion_6}
Ache um critério (para o sistema decimal) para divisibilidade por 6.

\hint $6 = 2\ntimes 3$

\solution
Observe que por causa do~\ref{product_of_coprimes_divides_common_multiple}, temos:
$$
6\divides c
\iff
2 \divides c
\mland
3 \divides c.
$$
Logo, aplicamos os critéria de divisibilidade por $2$ e por $3$.

\endexercise

\criterion Divisibilidade por 8.
\label{divisibility_criterion_8_wrong}
Um número $c$ é divisível por 8 sse ele satisfaz
os critéria de divisibilidade por $2$ e $4$.
\wrongproof.
Observe que por causa do~\ref{product_of_coprimes_divides_common_multiple}, temos:
$$
8 \divides c
\iff
2 \divides c
\mland
4 \divides c.
$$
Logo, aplicamos os critéria de divisibilidade por $2$ e por $4$.
\mistaqed

\exercise.
Ache o erro no~\ref{divisibility_criterion_8_wrong},
e compare com a solução do~\ref{divisibility_criterion_6}.

\hint Veja o~\ref{product_of_coprimes_divides_common_multiple}.

\solution
Como $\gcd 4 2 = 4 \neq 1$, não podemos aplicar
o~\ref{product_of_coprimes_divides_common_multiple}.
Um contraexemplo:
$2 \divides 12$ e $4 \divides 12$, mas $2\ntimes 4 = 8 \ndivides 12$.

\endexercise

\exercise.
Ache um critério (no sistema decimal) para divisibilidade por 8,
e generalize para divisibilidade por $2^k$

\hint $8 = 2^3$, e $2 \divides 10$.

\hint $2 \divides 10 \implies 2^3 \divides 10^3$.

\endexercise

\exercise.
Ache um critério (no sistema decimal) para divisibilidade por $2^x5^y$, onde $x,y\in\nats$.

\hint Os $2$ e $5$ são divisores de $10$.

\endexercise

\endsection
%%}}}

%%{{{ Inverses modulo $m$ 
\section Inversos módulo~$m$.

\definition Inverso módulo $m$.
\tdefined{inverso}[multiplicativo módulo~$m$]
\sdefined{{\holed a}^{-1}}{o inverso (multiplicativo) do $\holed a$ (módulo~$m$)}
\label{inverse_modulo_m}
Seja $a,a',m\in\ints$.
Chamamos $a'$ \dterm{um inverso (multiplicativo) de $a$ módulo~$m$},
sse
$$
aa' \cong 1 \pmod m.
$$
Se existe inverso do $a$, o denotamos com $a^{-1}$ (dado um módulo~$m$).\mistake

\exercise.
Qual o problema com a definição do $a^{-1}$?

\solution
Para o símbolo $a^{-1}$ ser bem-definido, precisamos mostrar que, caso que
existe um inverso, ele é único, que nos realmente mostramos
no~\ref{inverse_modulo_m_uniqueness}.

\endexercise

Podemos falar sobre \emph{o} inverso (ao invés de \emph{um} inverso) graças o teorema seguinte:

\theorem Unicidade do inverso.
\label{inverse_modulo_m_uniqueness}
Sejam $a,m\in\ints$.
Se $b,b'\in\ints$ satisfazem a $ax \cong 1 \pmod m$, então $b \cong b' \pmod m$.
\proof.
Como
$$
ab  \cong 1 \pmod m \qquad\mland\qquad ab' \cong 1 \pmod m,
$$
pela transitividade e reflexividade da congruência módulo~$m$, temos:
$$
ab \cong ab' \pmod m.
$$
Pela~\ref{modular_arithmetic_properties}, podemos multiplicar os dois lados por $b$:%
\footnote{Nada especial sobre $b$ contra o $b'$.  Poderiamos multiplicar por qualquer inverso do $a$ aqui.}
$$
bab \cong bab' \pmod m.
$$
Daí,
$(ba)b \cong (ba)b' \pmod m$,
ou seja $b \cong b' \pmod m$.
\qed

\example.
O inverso de $2$ módulo~$9$ é o $5$, porque $2\ntimes 5 = 10 \cong 1 \pmod 9$.
\endexample

Como o exemplo seguinte mostra, os inversos não existem sempre:

\example.
O $4$ não tem inverso módulo~$6$.

\solution
Podemos verificar com força bruta:
$$
\align
4 \ntimes 1 = \phantom04    &\cong 4 \pmod 6\\
4 \ntimes 2 = \phantom08    &\cong 2 \pmod 6\\
4 \ntimes 3 = 12            &\cong 0 \pmod 6\\
4 \ntimes 4 = 16            &\cong 4 \pmod 6\\
4 \ntimes 5 = 20            &\cong 2 \pmod 6
\endalign
$$
\moveqedup
\endexample

O teorema seguinte esclariza a situação:

\theorem Inverso módulo $m$.
\label{find_inverse_modulo_m}
Sejam $a,m\in\ints$.
$$
\text{$a$ tem inverso módulo~$m$}
\iff
\gcd a m = 1.
$$
\proof.
Precisamos mostrar as duas direções do \bidir.
\endgraf
\lrdir:
Escrevemos o $\gcd a m = 1$ como combinação\ii{combinação linear} linear dos $a$ e $m$
(sabemos que é possivel por \ref{gcd_as_linear_combination}, e, até melhor construtível graças o algoritmo estendido de Euclides,~\ref{extended_euclidean_algorithm}):
$$
\alignat 2
1 &= sa + tm, &&\qquad\text{para alguns $s,t\in\ints$.}
\intertext{Então temos:}
1 &\cong sa + tm    &&\pmod m\\
  &\cong sa + 0     &&\pmod m\\
  &\cong sa         &&\pmod m.
\endalignat
$$
Acabamos de achar um inverso de $a$ módulo~$m$: o $s$.
\endgraf
\rldir:
Seja $b$ um inverso de $a$ módulo~$m$; em outras palavras:
$$
ab \cong 1 \pmod m,
$$
ou seja, $m \divides ab - 1$, e $mu = ab - 1$ para algum $u\in\ints$.
Conseguimos escrever
$$
1 = um - ba,
$$
a combinação linear dos $a$ e $m$.
Pelo~\ref{gcd_divides_any_linear_combination}, $\gcd a m \divides 1$,
logo $\gcd a m = 1$.
\qed

\endsection
%%}}}

%%{{{ The Chinese remainder theorem 
\section O teorema chinês do resto.

\theorem Teorema chinês do resto.
\ii{teorema}[chinês do resto]
\iisee{chinês}[teorema do resto]{teorema chinês}
\ii{sistema}[de congruências]
\ii{congruência}[sistema]
\label{chinese_remainder_theorem}
Sejam $a_1,\dotsc,a_k,m_1,\dotsc,m_k\in\ints$,
com os $m_i$'s coprimos dois a dois:
$$
\lforall {i,j\in\set{1,\dotsc,k}} {i\neq j \limplies \gcd {m_i} {m_j} = 1}.
$$
Então existe $x\in\ints$ que satisfaz o sistema de congruências
$$
\align
x &\cong    a_1    \pmod {m_1}\\
x &\cong    a_2    \pmod {m_2}\\
  &\eqvdots                   \\
x &\cong    a_k    \pmod {m_k}.
\endalign
$$
Alem disso, a solução do sistema é única módulo~$m_1\dotsb m_k$.
\sketch.
\proofstyle{Existência:}
Seja
$$
\align
M   &\asseq \prod_{i=1}^k m_i = m_1m_2\dotsb m_k
\intertext{e, para todo $i\in\set{1,\dotsc,k}$, defina o}
M_i &\asseq \prod_{\Sb j=1\\j\neq i\endSb}^k m_j
     = m_1\dotsb m_{i-1}m_{i+1}\dotsb m_k
     = \frac M {m_i}.
\endalign
$$
Observe que $M_i$ é invertível módulo~$m_i$,
então seja $B_i$ o seu inverso.
Verificamos que o inteiro
$$
x =
\sum_{i=1}^k
a_i M_i B_i
$$
satisfaz todas as $k$ congruências,
e é então uma solução do sistema.
\endgraf
\proofstyle{Unicidade:}
Suponha que $x'\in\ints$ é uma solução do sistema.
Usando a definição de congruência e propriedades
de $\divides$, mostramos que $x' \cong x \pmod M$.
\qes

\example.
Ache todos os inteiros $x\in\ints$ que satisfazem o sistema de congruências:
$$
\align
x &\cong 2 \pmod 9\\
x &\cong 1 \pmod 5\\
x &\cong 2 \pmod 4.
\endalign
$$

\solution
Observamos primeiramente que os módulos $9$, $5$, e $4$ realmente são coprimos
dois a dois.  Então, pelo teorema chinês do resto,
o sistema realmente tem solução.
Seguindo sua método---e usando os mesmos nomes para as variáveis como
no~\ref{chinese_remainder_theorem} mesmo---calculamos os:
$$
\xalignat2
M_{\phantom0}   &= 9\ntimes5\ntimes4 = 180  &      &               \\
M_1 &= \phantom{9\ntimes{}}5\ntimes4 = 20   &  B_1 &\cong 5 \pmod 9\\
M_2 &= 9\phantom{{}\ntimes5}\ntimes4 = 36   &  B_2 &\cong 1 \pmod 5\\
M_3 &= 9\ntimes5\phantom{{}\ntimes4} = 45   &  B_3 &\cong 1 \pmod 4, 
\endxalignat
$$
onde os invérsos $B_i$'s podemos calcular usando o algoritmo estendido de
Euclides~(\refn{extended_euclidean_algorithm}) como na prova
do~\ref{find_inverse_modulo_m}), mas nesse caso, sendo os módulos tão pequenos
fez mas sentido os achar testando, com ``força bruta''.
Então, graças o teorema chinês, as soluções são exatamente os inteiros $x$ que satisfazem
$$
\alignat2
x &\cong a_1M_1B_1 + a_2M_2B_2 + a_3M_3B_3                                      &&\pmod M\\
  &\cong 2 \ntimes 20\ntimes 5 + 1 \ntimes 36\ntimes 1 + 2 \ntimes 45\ntimes 1  &&\pmod {180}\\
  &\cong 200 + 36 + 90                                                          &&\pmod {180}\\
  &\cong 146                                                                    &&\pmod {180}.
\endalignat
$$
Para resumir, as soluções do sistema são todos os elementos do
$\set{ 180k + 146 \st k\in\ints }$.
\endexample

\example.
Ache todos os inteiros $x\in\ints$ com $\abs x < 64$ que satisfazem o sistema de congruências:
$$
\align
x  &\cong 1 \pmod 3\\
3x &\cong 1 \pmod 4\\
4x &\cong 2 \pmod 5.
\endalign
$$

\solution
Para aplicar o teorema chinês do resto precisamos os $3$, $4$, e $5$ coprimos dois a dois,
que realmente são.
Mas observe que o sistema não está na forma do teorema;
aí, não podemos o aplicar diretamente.
Nosso primeiro alvo então seria transformar a segunda e a terceira congruência para
equivalentes, na forma necessária para aplicar o teorema.
Na segunda vamos nos livrar do fator $3$, e na terceira do fator $4$.
Como $\gcd 3 4 = 1$,%
\footnote{Qual ``4'' foi esse?}
o $3$ é invertível módulo~$4$.
Como o $3\cong -1 \pmod 4$, temos diretamente que $3^{-1} \cong -1 \pmod 4$.
Similarmente achamos o inverso $4^{-1} \cong -1 \pmod 4$.
Então temos:
$$
\left.
\aligned
x  &\cong 1 \pmod 3\\
3x &\cong 1 \pmod 4\\
4x &\cong 2 \pmod 5
\endaligned
\right\}
\iff
\left\{
\aligned
x  &\cong \phantom{1^{-1}}1 \pmod 3\\
3^{-1}3x &\cong 3^{-1}1 \pmod 4\\
4^{-1}x &\cong 4^{-1}2 \pmod 5
\endaligned
\right\}
\iff
\left\{
\aligned
x &\cong \phantom{-}1 \pmod 3\\
x &\cong -1 \pmod 4\\
x &\cong -2 \pmod 5.
\endaligned
\right.
$$
Agora sim, podemos aplicar o teorema chinês.
Usando os mesmos nomes para as variáveis
como no~\ref{chinese_remainder_theorem}, calculamos:
$$
\xalignat2
M_{\phantom0}   &= 3\ntimes4\ntimes5              = 60 &     &               \\
M_1             &= \phantom{3\ntimes{}}4\ntimes5  = 20 & B_1 &\cong 2 \pmod 3\\
M_2             &= 3\phantom{{}\ntimes4}\ntimes5  = 15 & B_2 &\cong 3 \pmod 4\\
M_3             &= 3\ntimes4\phantom{{}\ntimes5}  = 12 & B_3 &\cong 3 \pmod 5, 
\endxalignat
$$
onde os invérsos $B_1$ e $B_2$ calculamos percebendo que
$20\cong-1 \pmod 3$ e $15\cong-1 \pmod 4$, e o $B_3$ com força bruta mesmo.
Pronto: as soluções do sistema são exatamente os inteiros $x$ que satisfazem:
$$
\alignat 3
x &\cong a_1M_1B_1 + a_2M_2B_2 + a_3M_3B_3                            &&\pmod M\\
  &\cong 2\ntimes20\ntimes1 + 3\ntimes15\ntimes3 + 3\ntimes12\ntimes3 &&\pmod {60}\\
  &\cong 40 + 135 + 108                                               &&\pmod {60}\\
  &\cong 40 + 15 + 108                                                &&\pmod {60}\qqby{$135\cong15\pmod{60}$}\\
  &\cong 55 + 48                                                      &&\pmod {60}\qqby{$108\cong48\pmod{60}$}\\
  &\cong -5 + 48                                                      &&\pmod {60}\qqby{$\phantom055\cong-5\pmod{60}$}\\
  &\cong 43                                                           &&\pmod {60}.
\endalignat
$$
Logo, o conjunto de todas as soluções do sistema é o $\set{ 60k + 43 \st k\in\ints }$.
Facilmente verificamos que os únicos dos seus elementos que satisfazem nossa
restrição $\abs x < 64$ são os inteiros obtenidos pelos valores de $k = 0$ e $-1$:
$x_1 = 43$, $x_2 = -17$.
\endexample

\exercise ``entre si'' vs ``dois a dois''.
\label{coprime_vs_pairwise_coprime}
Considere as frases:
\beginol
\li Os inteiros $a_1, a_2, \dotsc, a_n$ são coprimos entre si.
\li Os inteiros $a_1, a_2, \dotsc, a_n$ são coprimos dois a dois.
\endol
\noindent São equivalentes?
Se sim, prove as duas direções da equivalência; se não, ache um contraexemplo.

\hint
Não são equivalentes.

\hint
Procure um contraexemplo com três inteiros.

\solution
Os $2$, $4$, $5$ são coprimos entre si (têm m.d.c.~1) mas mesmo assim não são coprimos dois a dois:
$\gcd 2 4 = 2$.

\endexercise

\exercise.
Acha as soluções dos seguinte sistemas de congruências:
$$
\xalignat2
\text{(1)}\quad&
\left\{
\aligned
x  &\cong 3\phantom0 \pmod 4\\
5x &\cong 1\phantom0 \pmod 7\\
x  &\cong 2\phantom0 \pmod 9
\endaligned
\right.
&
\text{(2)}\quad&
\left\{
\aligned
x  &\cong 3\phantom0 \pmod 3\\
3x &\cong 3\phantom0 \pmod 4\\
4x &\cong 2\phantom0 \pmod 5\\
5x &\cong 1\phantom0 \pmod 7.
\endaligned
\right.
\endxalignat
$$

\endexercise

O exercísio seguinte te convida descobrir que o teorema chinês pode ser aplicado
em casos mais gerais do que aparece inicialmente!

\exercise.
Resolva os sistemas de congruências:
$$
\xalignat2
\text{(1)}\quad&
\left\{
\aligned
5x &\cong 2\phantom0  \pmod 6\\
x  &\cong 13 \pmod {15}\\
x  &\cong 2\phantom0  \pmod 7
\endaligned
\right.
&
\text{(2)}\quad&
\left\{
\aligned
x  &\cong 2\phantom0  \pmod 6\\
x  &\cong 13 \pmod {15}\\
x  &\cong 2\phantom0  \pmod 7.
\endaligned
\right.
\endxalignat
$$

\hint Observe que não podes aplicar o teorema chinês diretamente: $\gcd 6 {15} > 1$.

\hint Tente substituir a congruência $5x \cong 2  \pmod 6$ com um sistema equivalente, de \emph{duas} congruências.

\hint Mostre que para qualquer $a\in\ints$,
$$
\left.
a \cong 2 \pmod 6
\right\}
\iff
\left\{
\aligned
a &\cong 0 \pmod 2\\
a &\cong 2 \pmod 3.
\endaligned
\right.
$$

\hint
Não todos os sistemas de congruências tem soluções.
(Acontéce quando temos restrições contraditórias.)
Nesse exercísio um dos dois sistemas não tem solução.

\endexercise

\endsection
%%}}}

%%{{{ Euler's totient function 
\section A função totiente de Euler.

\definition Função totiente de Euler.
\iisee{totiente}{função totiente}
\iisee{Euler}[função]{função totiente}
\Euler[função totiente]
\tdefined{função}[totiente de Euler]
\sdefined{\totsym} {a função totiente de Euler}
\label{euler_phi_function}
Seja inteiro $n>0$.
Definimos
$$
\tot n \defeq \sizeof {\set { i \in \set{1,\dotsc,n} \st \gcd i n = 1 } }.
$$
Em palavras, $\tot n$ é o número dos inteiros entre $1$ e $n$ que são coprimos com $n$.

\exercise.
Calcule os valores da $\tot n$ para $n=1,2,3,4,8,11,12,16.$

\endexercise

\property.
\label{tot_of_prime}
$\text{$p$ primo} \implies \tot p = p-1.$
\proof.
Como $p$ é primo, ele é coprimo com todos os $1,\dotsc,p-1$.
E como $\gcd p p = p \neq 1$, pela definição da $\totsym$ temos
$\tot p = p-1$.
\qed

\exercise.
\label{number_of_multiples_of_a_until_ai}
Quantos multiplos de $a$ existem no $\set{1,2,\dotsc,a^i}$?

\endexercise

\exercise.
\label{tot_of_power_of_prime}
$
\text{$p$ primo} \implies
\tot {p^a} = p^a - p^{a-1} = p^{a-1} (p-1)
$.

\hint Use a definição de $\totsym$ e o~\ref{number_of_multiples_of_a_until_ai}.

\solution
Seguindo a definição de $\totsym$, vamos contar todos os números
no $\set{1,2,\dotsc,p^i}$ que são coprimos com $p^i$.
Quantos não são?
Observe que como $p$ é primo,
os únicos que não são coprimos com ele,
são os múltiplos de $p$.
Pelo~\ref{number_of_multiples_of_a_until_ai} temos a resposta.

\endexercise

\exercise.
\label{sum_of_tots_of_powers_of_prime}
Sejam $p$ primo e $k\in\ints$.  Calcule o valor do somatório
$\sum_{i=0}^k \tot {p^i}$.

\hint Calcule o valor de cada termo aplicando o~\ref{tot_of_power_of_prime}.

\endexercise

\exercise.
\label{tot_of_product_of_primes}
$\text{$p,q$ primos, $p\neq q$} \implies \tot {pq} = (p-1)(q-1)$.

\endexercise

\theorem.
\iisee{multiplicativa}{função multiplicativa}
\tdefined{função}[multiplicativa]
\label{tot_is_multiplicative}
A função $\totsym$ é \dterm{multiplicativa}:
$$
\gcd m n = 1 \implies \tot {mn} = \tot m \tot n.
$$

\corollary.
Se $n\geq 2$, então
$$
\tot n
= n \!\!\prod_{\Sb\text{$p$ primo}\\p \divides n\endSb}\!\!\left(1-\frac 1 p\right)
= n
\left(1-\frac 1 {p_1}\right)
\left(1-\frac 1 {p_2}\right)
\dotsb
\left(1-\frac 1 {p_k}\right),
$$
onde os $p_i$'s são todos os primos divisores de $n$.
\sketch.
Escrevemos o $n$ na sua representação canónica pelo teorema fundamental da aritmética~(\refn{fundamental_theorem_of_arithmetic}), e aplicamos repetetivamente o~\ref{tot_is_multiplicative}.
\qes
\proof.
Pelo teorema fundamental da aritmética~(\refn{fundamental_theorem_of_arithmetic}),
seja $n \eqass p_0^{a_0} p_1^{a_1} \dotsb p_k^{a_k}$ a representação canónica
\ii{representação canónica} do $n$.  Calculamos:
$$
\alignat2
\tot n
&= \tot {p_0^{a_0} p_1^{a_1} \dotsb p_k^{a_k}}              \\
&= \tot {p_0^{a_0}} \tot{p_1^{a_1}} \dotsb \tot{p_k^{a_k}}  \qqby{\ref{tot_is_multiplicative}}\\
&=
p_0^{a_0}\paren{1 - \frac 1 {p_0}}
p_1^{a_1}\paren{1 - \frac 1 {p_1}}
\dotsb
p_k^{a_k}\paren{1 - \frac 1 {p_k}}                          \qqby{\ref{tot_of_power_of_prime}}\\
&=
p_0^{a_0}p_1^{a_1}\dotsb p_k^{a_k}
\paren{1 - \frac 1 {p_0}}\paren{1 - \frac 1 {p_1}}\dotsb\paren{1 - \frac 1 {p_k}}\\
&=
n
\paren{1 - \frac 1 {p_0}}\paren{1 - \frac 1 {p_1}}\dotsb\paren{1 - \frac 1 {p_k}}.
\endalignat
$$
\moveqedup
\qed

\exercise.
$a\divides b \implies \tot a \divides \tot b$.

\endexercise

\exercise.
$
\tot {2n} =
\knuthcases{
2\tot n,& $n$ é par \cr
\tot n, & $n$ é impar.
}
$

\endexercise

\exercise.
\label{tot_is_par_for_n_greater_than_2}
$\tot n$ é par para todo $n\geq 3$.

\hint Os inverso aparecem ``em pares''.  Mas uns são emparelhados com eles
mesmo, sendo seus próprios inversos (por exemplo, o inverso de 1, é o 1, módulo
qualquer $m$).

\hint Escreva o $n$ na sua representação canónica pelo teorema fundamental da aritmética~(\refn{fundamental_theorem_of_arithmetic}).

\hint
Aplica o~\ref{tot_is_multiplicative} e o~\ref{tot_of_power_of_prime}.

\hint
Ou o $2$ aparece nos fatores primos do $n$, ou não.

\endexercise

\endsection
%%}}}

%%{{{ Theorems of Euler and Fermat 
\section Teoremas de Euler e de Fermat.

\theorem Euler.
\label{euler_congruence_theorem}
\Euler[teorema]
Sejam $a,m\in\ints$ com $\gcd a m = 1$.
Então
$$
a^{\tot m} \cong 1 \pmod m.
$$
\sketch.
Considere o conjunto
$$
\align
 R &=\set{r_1, r_2, \dotsc, r_{\tot m}}
\intertext{de todos os inteiros $r$ com $1 \leq r \leq m$, e $\gcd r m = 1$, e o conjunto}
aR &= \set{ar \st r \in R}\\
   &= \set{ar_1, ar_2, \dotsc, ar_{\tot m}}.
\endalign
$$
Observamos agora que (módulo~$m$) os $ar_1,ar_2,\dotsc,ar_{\tot m}$
são apenas uma permutação dos $r_1,r_2,\dotsc,r_{\tot m}$.
Logo os seus produtórios são congruentes:
$$
(ar_1)(ar_2)\dotsb (ar_{\tot m})
\cong
r_1r_2\dotsb r_{\tot m}
\pmod m.
$$
Trabalhando na última congruência chegamos na congruência desejada.
\qes

\corollary Fermat.
\label{fermat_little_theorem_1}%
\Fermat[teorema]%
\ii{teorema}[pequeno Fermat]%
Sejam $p$ primo e $a\in\ints$ com $\gcd a p = 1$.
Então
$$
a^{p-1} \cong 1 \pmod p.
$$
\sketch.
O resultado é imediato usando o teorema de Euler
(\refn{euler_congruence_theorem}) e a~\ref{tot_of_prime}.
\qes
\proof.
Como $p$ é primo, sabemos que $\tot p = p-1$, então temos:
$$
\alignedat 2
a^{p-1}
&= a^{\tot m}       \qqby{\ref{tot_of_prime}} \\
&\cong 1 \pmod m.   \qqby{\ref{euler_congruence_theorem}}
\endalignedat
$$
\qed

\corollary Fermat.
\label{fermat_little_theorem_2}%
\Fermat[teorema]%
\ii{teorema}[pequeno Fermat]%
Sejam $p$ primo e $a\in\ints$.
Então
$$
a^p \cong a \pmod p.
$$
\sketch.
Considere dois casos: $\gcd a p = 1$ ou $\gcd a p > 1$.
No primeiro usamos o~\ref{fermat_little_theorem_1}.
No segundo, necessariamente temos $0\cong a^p\cong a \pmod p$.
\qes
\proof.
Temos dois casos:
\endgraf
\casestyle{Caso $\gcd a p$ = 1}:
Pelo~\ref{fermat_little_theorem_1} temos:
$$
\align
a^{p-1} &\cong 1 \pmod p
\intertext{e multiplicando por $a$,}
a^p     &\cong a \pmod p.
\endalign
$$
\endgraf
\casestyle{Caso $\gcd a p$ > 1}:
Nesse caso, como $p$ é primo, necessariamente $p\divides a$,
ou seja, $a \cong 0 \pmod p$,
logo $a^p \cong 0 \pmod p$.
Agora pela transitividade e
simetria da $congruência$ módulo~$m$, finalmente temos:
$a^p \cong a \pmod p$
\qed

\note Resumindo os três teoremas.
Esquematicamente:
$$
\alignat 4
\text{$\gcd a m = 1$}
&\implies{}
&a^{\tot m} &&{}\cong 1 & \pmod m
&\qqqquad&\text{Euler, (\refn{euler_congruence_theorem})}
\\
\text{$p$ primo}
&\implies{}
&a^{p-1} &&{}\cong 1& \pmod p
&\qqqquad&\text{Fermat, (\refn{fermat_little_theorem_1})}
\\
\left.
\aligned
\text{$p$ primo}\\
\gcd a p = 1
\endaligned
\right\}
&\implies{}
&a^p &&{}\cong a &\pmod p
&\qqqquad&\text{Fermat, (\refn{euler_congruence_theorem})}.
\endalignat
$$

\example.
\label{last_digit_of_big_number_example}
Ache o último dígito do $2^{800}$.

\solution
Procuramos um $y$ tal que $2^{800}\cong y \pmod {10}$
(por quê?).
Usando o teorema de Fermat~(\ref{fermat_little_theorem_1}) temos:
$$
\align
2^4 &\cong 1 \pmod 5,
\intertext{logo}
2^{800} = (2^4)^{200} &\cong 1 \pmod 5.
\endalign
$$
Então módulo~$10$ temos duas possibilidades (por quê?):
$$
2^{800} \cong
\knuthcases{
1 \pmod {10}\cr
6 \pmod {10}.\cr
}
$$
Podemos já eliminar a primeira porque $2^{800}$ é par.
Finalmente, o último dígito de $2^{800}$ é o $\digit6$.
\endexample

\exercise.
Responda no primeiro ``por quê?'' do~\ref{last_digit_of_big_number_example}.

\hint Escreva o inteiro como somatório baseado na sua forma decimal.

\hint Divide ele por 10.

\solution
Para o número $n$ escrito em base decimal como $\delta_k\delta_{k-1}\dotsb \delta_1\delta_0$,
temos:
$$
\align
n
&= d_0 + 10d_1 + 100d_2 + \dotsb 10^{k-1}d_{k-1} + 10^kd_k\\
&= \underbrace{d_0}_{\text{resto}} + 10\underbrace{(d_1 + 10d_2 + \dotsb 10^{k-2}d_{k-1} + 10^{k-1}d_k)}_{\text{quociente}},
\endalign
$$
onde $d_i$ é o correspondente valor do dígito $\delta_i$.
(Evitamos aqui confudir o ``dígito'' com seu valor usando notação diferente
para cada um, para enfatisar a diferença entre os dois conceitos.)

\endexercise

\exercise.
Responda no segundo também, e ache um outro caminho para chegar no resultando, usando
o teorema chinês do resto (\ref{chinese_remainder_theorem}).

\hint Qual a solução do sistema
$$
\align
x &\cong 1 \pmod 5\\
x &\cong 0 \pmod 2?
\endalign
$$

\solution
Podemos ou aplicar o teorema chinês (\ref{chinese_remainder_theorem}) no sistema de congruências
$$
\align
x &\cong 1 \pmod 5\\
x &\cong 0 \pmod 2,
\endalign
$$
ou, como $5\divides 10$, usar diretamente o~\ref{from_mod_m_to_mod_am},
para concluir que $x = 10k + 5i + 1$, onde $i=0,1$.

\endexercise

\exercise.
Ache o resto da divisão de $41^{75}$ por $3$.

\hint Procure $y$ tal que $41^{75} \cong y \pmod 3$.

\hint $\gcd {41} 3 = 1$.

\hint $41^{75} = 41^{74}\ntimes 41$.

\hint Fermat.

\solution
Como $\gcd {41} 3 = 1$, pelo teorema de Fermat (\ref{fermat_little_theorem_1})
temos
$$
41^{\tot 3} = 41^2 \cong 1 \pmod 3,
$$
e agora dividindo o $75$ por $2$, temos $75 = 2\ntimes 37 + 1$, então:
$$
\alignat 3
41^2 \cong 1 \pmod 3 &\implies (41^2)^{37}  &&\cong \phantom01  &&\pmod 3    \\
                     &\implies 41^{74}      &&\cong \phantom01  &&\pmod 3    \\
                     &\implies 41^{74}41    &&\cong 41 &&\pmod 3    \\
                     &\implies 41^{75}      &&\cong 41 &&\pmod 3    \\
                     &\implies 41^{75}      &&\cong \phantom02  &&\pmod 3.
\endalignat
$$
Outro jeito para escrever exatamente a mesma idéia,
mas trabalhando ``de fora pra dentro'', seria o seguinte:
$$
\alignat 2
41^{75} &= 41^{2\ntimes 37 + 1}\\
        &= 41^{2 \ntimes 37} 41\\
        &= (41^2)^{37} 41\\
        &\cong  1^{37} 41       &&\pmod 3\\
        &\cong  41              &&\pmod 3\\
        &\cong  2               &&\pmod 3.
\endalignat
$$
Então $41^{75} \bmod 3 = 2$.

\endexercise

\endsection
%%}}}

%%{{{ Exponentiation 
\section Exponenciação.

\endsection
%%}}}

%%{{{ Cryptography 
\section Criptografía.

%%{{{ The idea of cryptography 
\TODO A idéia da criptografia.
%%}}}

%%{{{ Public-key cryptography 
\TODO Criptografia ``public-key''.
%%}}}

%%{{{ RSA encryption and decryption 
\TODO RSA criptografia e descriptografia.

%%{{{ thm 
\theorem.
Sejam $e, M\in\ints$ com $\gcd e {\tot M} = 1$,
e seja $d$ um inverso de $e$ módulo~$\tot M$: $ed \cong 1 \pmod {\tot M}$.
Para cada $m$ com $\gcd m M = 1$,
$$
\paren{m^e}^d \cong m \pmod M.
$$
\proof.
Observe primeiramente que:
$$
\alignat 2
ed \cong 1 \pmod {\tot M}
&\iff \tot M \divides ed - 1                    \qqby{\ref{congruence}}\\
&\iff \lexists {k\in\ints} {k\tot M = ed - 1}.  \qqby{\ref{divides}}
\endalignat
$$
Seja $k\in\ints$ então um tal $k$, e agora resolvendo por $ed$:
$$
ed = k\tot M + 1.
\eqdef{rsa_ed_equality}
$$
Calculamos:
$$
\alignat 2
\paren{m^e}^d
&= m^{ed}                   \\
&= m^{k\tot M + 1}          \qqby{por~\eqref{rsa_ed_equality}}\\
&= m^{k\tot M} m            \qqby{def.~de exponenciação}\\
&= \paren{m^k}^{\tot M} m   \\
&\cong m  \pmod M,          \qqby{por teorema de Euler~\refn{euler_congruence_theorem}}
\endalignat
$$
onde no último pásso precisamos a hipótese que $m$ e $\tot M$ são coprimos
e logo, $m^k$ e $\tot M$ também são: $\gcd m {\tot M} = \gcd {m^k} {\tot M} = 1$.
\qed
%%}}}

%%{{{ x 
\exercise.
O que acontece se $m$ e $\tot M$ não são coprimos?

\endexercise
%%}}}
%%}}}

\endsection
%%}}}

%%{{{ Digital signatures 
\section Assinaturas digitais.

\endsection
%%}}}

%%{{{ Problems 
\problems.

%%{{{ prob 
\problem.
Prove numa linha o~\ref{odd_to_any_power_is_odd}:
{\proclaimstyle
para todo $n\in\nats$ e todo impar $k\in\ints$, $k^n$ é impar.}

\endproblem
%%}}}

%%{{{ prob 
\problem.
(Generalização do~\ref{odd_to_any_power_is_odd}.)
Sejam $a\in\ints$ e $m\in\nats$.
Prove numa linha que para todo $n\in\nats$, existe $b\in\ints$ tal que $(am + 1)^n = bm + 1$.

\endproblem
%%}}}

%%{{{ prob 
\problem.
Prove que para todo $m\in\nats$, o produto de $m$ consecutivos inteiros é divisível por $m!$.

\endproblem
%%}}}

%%{{{ prob: freshmans_dream 
\problem O sonho do calouro.
\label{freshmans_dream}%
\ii{sonho do calouro}%
Seja $p$ primo, $x,y\in\ints$.
$$
(x + y)^p \cong x^p + y^p \pmod p.
$$

\hint
Use o teorema binomial~\refn{binomial_theorem}.

\hint
Use o \ref{p_divides_comb_p_r}.

\solution
Temos
$$
\alignat 2
(x + y)^p
&= \sum_{i=0}^p \binom p i x^{p-i}y^i    \qqby {por~\ref{binomial_theorem}}\\
&= \binom p 0 x^p + \sum_{i=1}^{p-1} \binom p i x^{p-i}y^i + \binom p p y^p   \qqby{$p\geq2$}  \\
&= x^p + \sum_{i=1}^{p-1} \binom p i x^{p-i}y^i + y^p     \\
&= x^p + \sum_{i=1}^{p-1} p c_i x^{p-i}y^i + y^p, \quad\text{para algum $c_i\in\ints$}  \qqby{por~\ref{p_divides_comb_p_r}}  \\
&= x^p + p \sum_{i=1}^{p-1} c_i x^{p-i}y^i + y^p      \\
&\cong x^p + y^p \pmod p.
\endalignat
$$

\endproblem
%%}}}

%%{{{ prob: little_fermat_new_proof 
\problem.
\label{little_fermat_new_proof}%
Ache uma nova prova, \emph{por indução}, do teorema de Fermat~\refn{fermat_little_theorem_2}:
\endgraf\noindent
{\sl Para todo primo $p$ e todo $a\in\ints$},
$$
a^p \cong a \pmod p.
$$
(Note que $a\in\ints$ e não $a\in\nats$.)

\hint
Use o sonho de calouro (\ref{freshmans_dream}).

\endproblem
%%}}}

\ignore{

%%{{{ prob 
\problem.
Prove que
$$
\tot {mn}
=
{\tot m \tot n}
\frac
d
{\tot d}
,\qquad\text{onde $d={\gcd m n}$}.
$$
Note quantas e quais das propriedades que já provamos são casos especiais dessa!

\endproblem
%%}}}

%%{{{ prob 
\problem.
Sejam $p,q$ primos com $p\neq q$.  Prove que
$$
p^{q-1} + q^{p-1} \cong 1 \pmod {pq}.
$$

\hint
O que seria o $p^{q-1} + q^{p-1}$ módulo~$p$?  E módulo~$q$?

\hint
China.

\endproblem
%%}}}

%%{{{ prob 
\problem.
Ache uma generalização do problema anterior,
aplicável para inteiros $a,b$ com $\gcd a b = 1$:

\hint Nesse caso \emph{não} temos
$$
a^{b-1} + b^{a-1} \cong 1 \pmod {ab}.
$$

\hint $p-1 = \tot p$ para qualquer primo $p$.

\hint
Prove que:
$$
a^{\tot b} + b^{\tot a} \cong 1 \pmod {ab}.
$$

\endproblem
%%}}}

}% ignore

\endproblems
%%}}}

%%{{{ Further reading 
\further.

Veja o~\cite[\S1.9]{babybm}.

\cite{disquisitiones}.

\cite{nivennumbers},
\cite{hardywright}.

\endfurther
%%}}}

\endchapter
%%}}}

%%{{{ chapter: Sets 
\chapter Conjuntos.
\label{Sets}%

%%{{{ chapintro: sets like assembly 
\chapintro
Nesse capítulo estudamos um ``tipo'' de objeto, o tipo de \emph{conjuntos}.
Como nos vamos apreciar no~\ref{Axiomatic_set_theory},
os conjuntos e sua linguagem têm um papel importante para a fundação
de matemática.
Podemos traduzir todas as definições e relações matemáticas nessa
linguagem.  Nesse sentido, parece como uma ``assembly'', uma linguagem
``low-level'' onde podemos ``compilar'' toda a matemática, em tal modo
que cada definição, cada afirmação, cada teorema que provamos,
no final das contas, todos podem ser traduzidos para definições,
afirmações, teoremas e provas, que envolvem apenas conjuntos
e a relação primitiva de \emph{pertencer}.
E \emph{nada} mais!
%%}}}

%%{{{ Concept, notation, equality 
\section Conceito, notação, igualdade.
\tdefined{conjunto}%

%%{{{ Q: What does it mean to be a set? 
\question.
O que significa ser conjunto?
%%}}}

\blah.
\Cantor{}Cantor deu a seguinte resposta:

%%{{{ pseudodf: set 
\pseudodefinition.
\label{set_pseudodefinition}%
\tdefined{conjunto}[definição intuitiva]%
Um \dterm{conjunto} $A$ é a colecção numa totalidade
de certos objetos (definidos e separados) da nossa intuição ou mente, que chamamos de \dterm{elementos} de $A$.
%%}}}

%%{{{ x: set_pseudodefined 
\exercise.
Qual é o problema principal com a definição em cima?

\solution
O que é uma ``colecção (numa totalidade)''?
Cuidado: para responder nessa pergunta
tu não podes usar a palavra ``conjunto'', pois assim teria uma
definição circular.
Em outras palavras, definimos a palavra ``conjunto'' em termos da palavra
``colecção'', que no final das contas, é algo sinónimo.

\endexercise
%%}}}

%%{{{ Primitive notions 
\note Noções primitivas.
\ii{noção primitiva}%
Aceitamos apenas duas \emph{noções primitivas}\/:
$$
\gathered
\text{ser conjunto}\\
\Set(\dhole)
\endgathered
\qqqquad
\gathered
\text{pertencer}\\
\dhole\in\dhole
\endgathered
$$
%%}}}

%%{{{ Convention: family, colection, agglomerate, and fonts 
\note Convenção.
\tdefined{família}%
\tdefined{colecção}%
\tdefined{aglomerado}%
Usamos os termos \dterm{família}, \dterm{colecção}, \dterm{aglomerado}, etc.\ como
sinónimos da palavra ``conjunto''.
Mesmo assim, seguindo uma prática comum, quando temos conjuntos de conjuntos
dizemos ``família de conjuntos'', e depois ``colecção de famílias'', etc.
Com o mesmo motivo (de facilitar nossos olhos ou ouvidos humanos), as vezes
mudamos a font que usamos para denotar esses conjuntos.
Por exemplo, usando $A,B,C$ para denotar certos conjuntos,
usamos $\scr A, \scr B, \scr C$ para denotar famílias deles,
etc.
%%}}}

%%{{{ df: set_notation 
\definition Notação de conjuntos.
\label{set_notation}%
\tdefined{conjunto}[notação]%
\tdefined{conjunto}[homogêneo]%
\tdefined{conjunto}[heterogêneo]%
\tdefined{homogeneidade}%
\tdefined{heterogeneidade}%
A notação mais simples para denotar um conjunto é usar
``chaves'' (os símbolos~`$\{$'~e~`$\}$') e listar todos os seus elementos dentro.
Por exemplo:
$$
\xalignat2
A &=\set {0,1}                          &E &=\set {2,3,\set{5,7}, \set{\set{2}}}\\
B &=\set {\nats, \ints, \rats, \reals}  &F &=\set {\mathrm{Thanos}}\\
C &=\set {2}                            &G &=\set {1,2,4,8,16,31,A,B,\nats}\\
D &=\set {2,3,5,7}                      &H &=\set {\mathrm{Thanos}, \mathrm{Natal}, \set{E,\set{F,G}}}
\endxalignat
$$
Chamamos os conjuntos cujos elementos são ``do mesmo tipo'' \dterm{homogêneos},
e os outros \dterm{heterogêneos}.
Deixamos ambíguo o que significa ``do mesmo tipo'', mas, naturalmente, consideramos
os $A, B, C, D, F$ homogêneos e os $E, G, H$ heterogêneos.
%%}}}

%%{{{ Dealing with more complicated sets 
\blah.
Todos os conjuntos que acabamos de escrever aqui são \emph{finitos},
seus elementos são conhecidos, e ainda mais são poucos e conseguimos os listar todos.
Nenhuma dessas três propriedades é garantida!
Se não temos a última fica impráctico listar todos elementos,
e quando não temos uma das duas primeiras, é plenamente impossível.
Considere por exemplo os conjuntos seguintes:
$$
\align
X &= \text{o conjunto de todos os números reais entre 0 e 1}\\
Y &= \text{o conjunto dos assassinos do Richard \Montague{}Montague}\\
Z &= \text{o conjunto de todos os números naturais menores que $2^{256!}$}.
\endalign
$$
%%}}}

%%{{{ Set comprehension 
\definition Set comprehension.
\label{set_comprehension}%
\label{definite_condition}%
\tdefined{set comprehension}%
\tdefined{condição definitiva}%
\sdefined {\setst {\holed x} {\text{\thole}}} {o conjunto de todos os $\holed x$ tais que \thole}%
\iisee{set-builder}{set comprehension}%
Uma notação diferente e bem útil é chamada
\dterm{set comprehension} ou notação \dterm{set-builder},
onde escrevemos
$$
\setst x {\text{\thole$x$\thole}}
$$
para denotar <<o conjunto de todos os objetos $x$ tais que \thole$x$\thole>>.
\footnote{Na literatura aparece também o símbolo~`\,$:$\,'~em vez
do~`$\st$'~que usamos aqui.}
Entendemos que no lado direito escrevemos o \dterm{filtro},
uma \emph{condição definitiva},
e não algo ambíguo ou algo subjetivo.  Por exemplo, não podemos escrever 
algo do tipo
$$
\setst p { \text{$p$ é uma pessoa linda} }.
$$
Mas como podemos formalizar o que é uma \dterm{condição definitiva}?
Bem, concordamos escrever apenas algo que podemos (se precisarmos e se quisermos)
descrever usando uma fórmula de FOL $\phi(x)$ onde possivelmente aparece
a variável $x$ livre e todos os símbolos da FOL tem interpretações bem definidas.%
\footnote{Pouco mais sobre isso no~\ref{fol_filter_by_fraenkel_and_skolem}.}
Chegamos então na forma
$$
\setst x {\phi(x)}.
$$
A notação set-builder é bem mais poderosa do que acabamos de mostrar,
pois nos permite utilisar \emph{termos} mais complexos na sua parte esquerda,
e não apenas uma variável.
Por exemplo
$$
\setst {p^n + x} {\text{$p$ é primo, $n$ é ímpar, e $x\in[0,1)$}}
$$
seria o conjunto de todos os números reais que podem ser escritos na forma
$p^n + x$ para algum primo $p$, algum ímpar $n$, e algum real $x$ com $0 \leq x < 1$.
\endgraf
Finalmente, mais uma extensão dessa notação é que usamos
$$
\setst {x \in A} {\text{\thole$x$\thole}}
\defeq
\setst {x} {x \in A \mland \text{\thole$x$\thole}}.
$$
%%}}}

%%{{{ x: div_mul_pow_of_12_and_m_setbuilder_practice 
\exercise.
\label{div_mul_pow_of_12_and_m_setbuilder_practice}%
Usando a notação set-builder defina os conjuntos
$D_{12}$, $M_{12}$, e $P_{12}$
de todos os divisores, todos os múltiplos, e todas as potências de $12$.
Generalize para um inteiro $m$.
Identifique quais variáveis que aparecem na tua resposta são livres e quais são ligadas.

\solution
$$
\xalignat2
D_{12} &\defeq \setst {n \in \ints} {n \divides 12}  &
D_m    &\defeq \setst {n \in \ints} {n \divides m}   \\
M_{12} &\defeq \setst {12n}  {n \in \ints}           &
M_m    &\defeq \setst {mn}   {n \in \ints}           \\
P_{12} &\defeq \setst {12^n} {n \in \nats}           &
P_m    &\defeq \setst {m^n}  {n \in \nats}
\endxalignat
$$
Onde aparece a variável `$m$', ela está livre, e
onde aparece a variável `$n$', ela está ligada.

\endexercise
%%}}}

%%{{{ x: set_builder_map_size_limit 
\exercise.
\label{set_builder_map_size_limit}%
Seja $T = \set{u,v}$ um conjunto com dois elementos $u,v$.
Definimos um conjunto $A$ pela
$$
A \defeq \setst {f(n,m)} {n,m \in T}
$$
Quantos elementos tem o $A$?

\hint
Primeiramente calcule a extensão do $A$:
$$
A = \set{ \dots?\dots }.
$$

\solution
Calculando a extensão de $A$ achamos:
$$
A = \set{ f(u,u), f(u,v), f(v,u), f(v,v) }.
$$
Então o $A$ tem \emph{no máximo} $4$ elementos---mas pode acontecer que tem menos (veja~\ref{set_builder_map_exercise}).

\endexercise
%%}}}

%%{{{ x: set_builder_map_exercise 
\exercise.
\label{set_builder_map_exercise}%
Escreve a extensão do conjunto
$$
B \defeq \setst {n^2 + m^2} {n,m \in \set{1,3}}.
$$

\solution
Calculamos:
$$
\align
B
&= \setst {n^2 + m^2} {n,m \in \set{1,3}}\\
&= \set{ 1^2 + 1^2, 1^2 + 3^2, 3^2 + 1^2, 3^2 + 3^2 }\\
&= \set{ 2, 10, 10, 18 }\\
&= \set{ 2, 10, 18 }.
\endalign
$$

\endexercise
%%}}}

%%{{{ x: rich_set_builder_sugar 
\exercise.
Mostre como a notação ``mais rica'' de
$$
\setst {t(x_1, \dotsc, x_n)} {\phi(x_1,\dotsc,x_n)}
$$
pode ser definida como açúcar sintáctico se temos já a notação de
comprehensão que permite apénas uma variável no lado esquerdo.
Aqui considere que o $t(x_1,\dotsc, x_n)$ é um termo que pode ser
bem complexo, formado por outros termos complexos, etc., e onde
possivelmente aparecem as variáveis $x_1,\dotsc,x_n$.

\hint
Precisa descrever (definir) o conjunto
$$
\align
\setst {t(x_1, \dotsc, x_n)} {\phi(x_1,\dotsc,x_n)} &\\
\intertext{com uma notação que já temos:}
\setst {t(x_1, \dotsc, x_n)} {\phi(x_1,\dotsc,x_n)}
&\defeq
\dots?
\endalign
$$

\hint
$$
\setst {t(x_1, \dotsc, x_n)} {\phi(x_1,\dotsc,x_n)}
\defeq
\setst x {\text{\thole?\thole}}
$$

\solution
$
\setst {t(x_1, \dotsc, x_n)} {\phi(x_1,\dotsc,x_n)}
\defeq
\setst x { \exists x_1\dotsb\exists x_n \paren{ x = t(x_1,\dotsc,x_n) \land \phi(x_1,\dotsc,x_n)}}.
$

\endexercise
%%}}}

%%{{{ Diagramas de Venn 
\note Diagramas de Venn.
Suas limitações.
%%}}}

%%{{{ What we need to specify for each new type 
\note Tipos.
Cada vez que introduzimos um novo tipo de objetos,
devemos especificar:
\beginol
\li Quando dois objetos desse tipo são \emph{iguais}?
\li Qual é o ``interface'' desses objetos?
\endol
%%}}}

%%{{{ Black boxes 
\note Black boxes.
\label{blackbox_set}%
\tdefined{black box}%
\tdefined{white box}%
\iiseealso{white box}{black box}%
\iiseealso{black box}{white box}%
\tdefined{black box}[de conjunto]%
\iisee{conjunto!como black box}{black box}%
O conceito de \dterm{black box} é muito útil para descrever certos tipos diferentes.
A idéia é que queremos descrever o que realmente determina um objeto desse tipo,
e \emph{esconder} os detalhes ``de implementação'', apresentado apenas seu ``interface''.
Podemos então pensar que um conjunto $A$ é um black box que tem apenas uma
entrada onde podemos botar qualquer objeto $x$ desejamos, e tem como única
``saida'' uma luz que pode piscar ``sim'' ou ``não'' (correspondende nos casos
$x\in A$ e $x\notin A$ respectivamente).
\endgraf
Usamos o termo \emph{black} box para enfatisar que não temos como ``olhar dentro''
desse aparelho, dessa caixa, e ver o que acontece assim que botar uma entrada;
nossa única informação será a luz da caixa que vai piscar ``sim'', ou ``não''.
Quando temos acesso nos ``internals'' da caixa a gente chama de \dterm{white box}
ou \dterm{transparent box}; mas não vamos precisar o uso desse conceito nesse texto.
\endgraf
A única \emph{estrutura interna} de um conjunto é a capabilidade de
\emph{decidir se dois elementos $x,y$ do conjunto são iguais ou não}.
%%}}}

\question.
Quando dois conjuntos são iguais?
\spoiler.

%%{{{ pseudodf: set_eq_pseudodefinition 
\pseudodefinition.
\label{set_eq_pseudodefinition}%
Consideramos dois conjuntos $A,B$ \dterm{iguais} sse não tem como diferenciar
eles como black boxes.  Em outras palavras, para cada objeto $x$
que vamos dar como entrada para cada um deles, eles vão concordar:
ou os dois vão piscar ``sim'', ou os dois vão piscar ``não''.
%%}}}

%%{{{ defining_sets 
\note Definindo conjuntos.
Para determinar então um conjunto $A$, precisamos dizer exatamente
quando um objeto arbitrário $x$ pertence nele.
As notações que viermos até agora realmente deixam isso claro;
mas um outro jeito muito útil para \emph{definir} um certo conjunto $A$,
seria apenas preencher o
$$
x \in A  \defiff  \text{\xlthole}.
$$
com alguma \ii{condição definitiva}condição definitiva
(veja~\ref{definite_condition}).
\endgraf
Concluimos que as duas formas seguintes de definir um conjunto $A$,
são completamente equivalentes:
$$
\xalignat2
x &\in A \defiff \text{\lthole}
&
A &\defeq \setst x {\text{\lthole}}.
\endxalignat
$$
As duas afirmações tem exatamente o mesmo efeito:
definir o mesmo conjunto $A$.
Qual das duas usamos, será mais questão de gosto ou de contexto.
%%}}}

%%{{{ note: order_and_multiplicity 
\note Ordem e multiplicidade.
\label{order_and_multiplicity}%
\ii{conjunto}[ordem de membros]%
\ii{conjunto}[multiplicidade]%
Considere os conjuntos seguintes:
$$
\xalignat3
A&=\set{2,3},&
B&=\set{3,2},&
C&=\set{3,2,2,2,3}
\endxalignat
$$
Observe que $A = B = C$.
Ou seja, esses não são três conjuntos, mas apenas \emph{um} conjunto
denotado em três jeitos diferentes.
O ``dispositivo'' conjunto não sabe nem de \dterm{ordem}
nem de \dterm{multiplicidade} dos seus membros.
Não podemos perguntar a um conjunto
<<qual é teu \emph{primeiro} elemento?>>, nem 
<<quantas vezes o tal elemento pertence em ti?>>.
Lembre o conjunto como black box!
Seu único interface aceita qualquer objeto,
e responda apenas com um pleno ``sim'' ou ``não''.
Logo encontramos outros tipos de ``recipientes'',
onde as informações de ordem e de multiplicidade são preservadas:
multisets~(\refn{Multisets}), tuplas~(\refn{Tuples}), e seqüências~(\refn{Sequences}).
Bem depois vamos estudar \emph{conjuntos ordenados} (\ref{Posets}).
%%}}}

\blah.
Logo vamos formalizar a~\ref{set_eq_pseudodefinition}.
Mas antes disso, vamos discutir sobre dois conceitos de igualdade diferentes.

\endsection
%%}}}

%%{{{ Intension vs. Extension 
\section Intensão vs{.}~extensão.

%%{{{ Consider four extensionally equal sets 
\blah.
Condidere os conjuntos
$$
\align
P &= \setst d {\text{$d$ é um divisor primo de $2^{256!}$}}\\
Q &= \setst p {\text{$p$ é primo e par}}\\
R &= \setst x {\text{$x$ é raiz real do polinómio $x^3 - 8$}}\\
S &= \set {2}.
\endalign
$$
%%}}}

%%{{{ Q: what_are_the_extensions_of_these_four_sets} 
\question.
\label{what_are_the_extensions_of_these_four_sets}%
Quais são os membros de cada um dos conjuntos em cima?
%%}}}

%%{{{ A: thinking a bit, they're all the same set 
\note Resposta.
\emph{Pensando um pouco} percebemos que esses quatro conjuntos
consistem em exatamente os mesmos membros, viz.~o número $2$ e nada mais.
Lembrando na idéia de black box, realmente não temos como diferenciar
entre esses black boxes.
Começando com o $S$, é direto que ele responda ``sim'' apenas no número $2$
e ``não'' em todos os outros objetos.
Continuando com os $P,Q,R$, realmente temos que o único objeto que satisfaz
seus fíltro é o número $2$.
%%}}}

%%{{{ Intension description 
\note.
\label{intension_description}%
\tdefined{extensão}%
\tdefined{intensão}%
\tdefined{igualdade}[extensional]%
\tdefined{igualdade}[intensional]%
Como comporta o $S$?
Recebendo sua entrada $a$, ele a compara com o $2$ para ver se $a=2$ ou não,
e responde ``sim'' ou ``não'' (respectivamente) imediatamente.
E o $R$?
Recebendo sua entrada $a$, ele verifica se $a$ é uma raiz do $x^3 - 8$.
Substituindo então o $x$ por $a$, elevando o $a$ ao $3$ e subtraindo $8$,
se o resultado for $0$ responda ``sim''; caso contrario, ``não''.
E o $Q$?
Recebendo sua entrada $a$, ele verifica se $a$ é primo, e se $a$ e par.
Se as duas coisa acontecem, ele responda ``sim''; caso contrario, ``não''.
E o $P$?
Recebendo sua entrada $a$, ele verifica se $a\divides 2^{256!}$ e se $a$ é
um número primo.
Se as duas coisa acontecem, ele responda ``sim''; caso contrario, ``não''.
\endgraf
Acabamos de descrever a \dterm{intensão} de cada conjunto, tanto aqui
quanto na~\ref{what_are_the_extensions_of_these_four_sets}.
Mas, sendo black boxes, dados esses conjuntos $P,Q,R,S$,
não conseguimos os diferenciar, pois a única interação que sua interface
permite é botar objetos $a$ como entradas, e ver se pertencem ou não.
Falamos então que \dterm{extensionalmente} os quatro conjuntos
são iguais, mas \dterm{intensionalmente}, não.
Usamos os termos \dterm{igualdade extensional} e \dterm{igualdade intensional}.
E para abusar a idéia de black box:
provavelmente o black box $Q$ demora mais para responder, ou fica mais quente,
ou faz mais barulho, etc., do que o $S$.
\endgraf
Quando definimos um conjunto simplesmente listando todos os seus membros,
estamos escrevendo sua \dterm{extensão}.  E nesse caso, a intensão é a mesma.
Quando usamos a notação builder com um fílter, estamos mostrando a \dterm{intensão}
do conjunto.  Nesse caso as duas noções podem ser tão diferentes, que nem
sabemos como achar sua extensão!
%%}}}

%%{{{ eg: number_theory_conjectures_set_intension 
\example.
\label{number_theory_conjectures_set_intension}%
Revise a~\ref{Open_problems_in_number_theory} e considere os conjuntos
$$
\align
T &\defeq
\setstt { p } {$p$ e $p+2$ são prímos}\\
L &\defeq
\setstt { n \in \nats_{>0} } {não existe primo entre $n^2$ e $(n+1)^2$}\\
G &\defeq
\setstt { n \in \nats_{>1} } {$2n=p+q$ para alguns primos $p,q$}\\
C &\defeq
\setstt { n \in \nats_{>0} } {a seqüência Collatz começando com $n$ nunca pega o valor $1$}.
\endalign
$$
Sobre o $T$ não sabemos se são finitos ou não!
Sobre o $G$, não sabemos se $G = \nats_{>1}$ ou não!
E, sobre os $L$ e $C$ nem sabemos se eles têm elementos ou não,
ou seja não sabemos nem se $L=\emptyset$ nem se $C=\emptyset$!
\endexample
%%}}}

\blah.
Fechando essa secção lembramos que em conjuntos (e em matemática em geral)
usamos igualdade `$=$' como igualdade extensional.

\endsection
%%}}}

%%{{{ Relations between sets and how to define them 
\section Relações entre conjuntos e como as definir.

%%{{{ df: set_eq 
\definition Igualdade de conjuntos.
\label{set_eq}%
Definimos
$$
A = B \defiff \forall x \paren{ x\in A \liff x\in B }.
$$
%%}}}

%%{{{ df: subconjunto 
\definition.
\label{subconjunto}%
\tdefined{subconjunto}%
\tdefined{subconjunto}[próprio]%
\sdefined{\holed A \subseteq \holed B} {$\holed A$ é um subconjunto de $\holed B$}%
O conjunto $A$ é um \dterm{subconjunto} de $B$ sse todos os membros de
$A$ pertencem em $B$.
Em símbolos:
$$
\align
A \subseteq B
&\defiff
\lforall {x\in A} {x\in B}\\
&\abbriff
\forall x \paren{ x \in A \limplies x \in B }
\endalign
$$
Se $B$ tem elementos que não pertencem no $A$, chamamos o $A$
\dterm{subconjunto próprio} de $B$, e escrevemos $A \subsetneq B$.
%%}}}

%%{{{ x: eq_implies_subseteq 
\exercise.
\label{eq_implies_subseteq}%
$A = B \implies A \subseteq B$.

\endexercise
%%}}}

%%{{{ x: eq_using_subsets 
\exercise.
\label{eq_using_subsets}%
$A = B \iff A \subseteq B \land B \subseteq A$

\endexercise
%%}}}

%%{{{ x: define_subsetneq 
\exercise.
\label{define_subsetneq}%
Defina com uma fórmula o $A\subsetneq B$.

\hint
Já definimos o $\subseteq$, então podemos o usar!

\solution
$
A \subsetneq B
\defiff
A \subseteq B \land A \neq B.
$
\quad
(Lembre-se que $A \neq B \abbreq \lnot(A = B)$.)

\endexercise
%%}}}

%%{{{ notation of subsets 
\beware.
O uso dos símbolos $\subseteq$, $\subset$, e $\subsetneq$ não é muito padronizado:
encontramos textos onde usam $\subseteq$ e $\subset$ para ``subconjunto''
e ``subconjunto próprio'' respectivamente; outros usam $\subset$ e $\subsetneq$.
Assim o símbolo $\subset$ é usado com dois significados diferentes.
Por isso usamos $\subseteq$ e $\subsetneq$ aqui, evitando completamente
o uso do ambíguo $\subset$.
%%}}}

%%{{{ notation: supset_sugar 
\note Notação.
\label{supset_sugar}%
Seguindo uma prática comum que involve símbolos ``direcionais'' de relações
binárias como os $\rightarrow$, $\leq$, $\subseteq$, etc., introduzimos os:
$$
\xalignat2
A \supseteq B  & \defiff B \subseteq A &
A \supsetneq B & \defiff B \subsetneq A
\endxalignat
$$
%%}}}

\endsection
%%}}}

%%{{{ The empty set and the universal set 
\section O conjunto vazio e o conjunto universal.

%%{{{ df: empty 
\definition Vazio.
\label{empty}%
\tdefined{vazio}%
Um conjunto é \dterm{vazio} sse ele não contem nenhum elemento.
Formalmente, definimos o predicado unário
$$
\Empty(A) \defiff \forall x (x \notin A).
$$
%%}}}

%%{{{ df: singleton 
\definition Singleton.
\label{singleton}%
Um conjunto é \dterm{singleton} (ou \dterm{unitário}) sse ele contem
apenas um elemento.
%%}}}

%%{{{ x: define_singleton_formally 
\exercise.
\label{define_singleton_formally}%
Defina formalmete o predicado unário $\Singleton(A)$.

\endexercise
%%}}}

%%{{{ df: emptyset 
\definition.
\sdefined {\emptyset} {o conjunto vazio}
\label{emptyset_symbol}%
Denotamos o conjunto vazio por $\emptyset$.
\mistake
%%}}}

%%{{{ x: what_is_wrong_with_the_emptyset_definition 
\exercise.
\label{what_is_wrong_with_the_emptyset_definition}%
Na~\ref{emptyset_symbol} roubamos!
Resolva o crime.

\hint
``o''

\solution
Como não provamos a unicidade do vazio não podemos usar o artigo
definido ``o'', e conseqüentemente, não podemos usar um símbolo
para \emph{o} denotar.  Seria mal-definido.

\endexercise
%%}}}

%%{{{ Parallelism with Singleton(-) to emphasize error 
\note.
Para apreciar ainda mais a gravidade do erro:
se apenas a definição de $\Empty(\dhole)$ fosse suficiente
para introduzir a notação $\emptyset$ para denotar ``o conjunto vazio'',
poderiamos também escolher um símbolo para denotar ``o conjunto unitário'':
$$
\align
\text{$\Empty(\dhole)$ definido}
&\quad\leadsto\quad \text{<<Denotamos o conjunto vazio por $\emptyset$.>>}\\
\text{$\Singleton(\dhole)$ definido}
&\quad\leadsto\quad \text{<<Denotamos o conjunto unitário por $\cancel{1}$.>>}
\endalign
$$
Qual de todos---quantos são?---os conjuntos unitários seria o $\cancel{1}$?
Essa ambiguidade não é permitida em matemática.
%%}}}

%%{{{ x: how_many_singletons 
\exercise.
\label{how_many_singletons}%
Quantos são mesmo?

\solution
Infinitos!  Pois, para cada objeto $x$ já temos um singleton $\set{x}$.
E agora o singleton dele $\set{\set{x}}$, e dele, e dele, \dots

\endexercise
%%}}}

%%{{{ x: naive_uniqueness_of_emptyset 
\exercise Unicidade do vazio.
\label{naive_uniqueness_of_emptyset}%
Supondo que existe pelo menos um conjunto vazio, mostre sua unicidade.
Em outras palavras, prove que:
$$
\text{\emph{se $A,B$ são vazios, então $A = B$.}}
$$
Não use reductio ad absurdum.

\hint
Suponha que $A,B$ são vazios.
Qual é teu alvo agora?

\hint
Teu alvo é mostrar que $A = B$.
Para fazer isso é necessário lembrar a definição de $=$ nos conjuntos (\refn{set_eq}).

\solution
Suponha que $A,B$ são vazios.
Preciso mostrar $A=B$, ou seja, que $\forall x( x\in A \liff x \in B )$.
Seja $x$ um objeto arbitrário.
Pela definição de vazio, as duas afirmações $x \in A$ e $x \in B$ são
falsas, e logo a equivalência $(x\in A \liff x\in B)$ verdadeira,
que foi o que queremos provar.

\endexercise
%%}}}

%%{{{ x: naive_uniqueness_of_emptyset_absurdum 
\exercise.
\label{naive_uniqueness_of_emptyset_absurdum}%
Ache uma prova diferente, essa vez usando reductio ad absurdum.

\hint
Suponha que $A,B$ são vazios.
Queremos mostrar que $A = B$.
Então suponha o contrario ($A \neq B$) pera chegar num absurdo.

\hint
Qual é teu alvo agora?
Achar um absurdo qualquer!
Como podemos usar o fato de $A \neq B$?
Lembre-se que $A \neq B$ é apenas uma abreviação para $\lnot(A = B)$.

\solution
Suponha que $A,B$ são vazios.
Queremos provar que $A = B$.
Para chegar num absurdo, suponha que $A \neq B$.
Logo, pela definição de igualdade, temos
que existe $x$ tal que:
$x \in A$ mas $x \notin B$; ou $x \in B$ mas $x \notin A$.
As duas alternativas chegam num absurdo:
a primeira pois $A$ é vazio e logo $x \notin A$, e similarmente
a segunda pois $B$ é vazio e logo $x \notin B$.

\endexercise
%%}}}

%%{{{ df: universal 
\definition Universal.
\tdefined{universal}[conjunto]%
\iiseealso{universo}{universal}%
\label{universal}%
Um conjunto é \dterm{universal} sse todos os objetos pertencem nele.
Formalmente,
$$
\Universal(A) \defiff \forall x(x \in A).
$$
%%}}}

%%{{{ x: naive_uniqueness_of_universet 
\exercise.
\label{naive_uniqueness_of_universet}
Supondo que existe um conjunto universal, mostre sua unicidade.

\endexercise
%%}}}

%%{{{ df: universet 
\definition.
\label{universet_symbol}%
\sdefined {\universet} {o conjunto universal}%
Denotamos o conjunto universal por $\universet$.
%%}}}

%%{{{ Q: How do we use a fact that a set is nonempty? 
\question.
Como podemos usar um fato do tipo $D\neq \emptyset$ em nossas provas?
O que ganhamos realmente?
%%}}}
\spoiler.

%%{{{ A: We can pick an element from it! 
\blah Resposta.
Ganhamos o direito de escrever ``Seja $d\in D$.''
Em outras palavras: de tomar um elemento de $D$;
de declarar uma variável (não usada) para denotar um elemento de $D$.
%%}}}

%%{{{ warning: letting_in_empty_like_division_by_0 
\warning.
\label{letting_in_empty_like_division_by_0}%
Quando temos um conjunto $A$, escrever ``seja $x \in A$''
seria errado se não sabemos que $A \neq \emptyset$.
É um erro parecido quando dividimos uma expressão de aritmética
por $x$, ou apenas escrever uma expressão como a $a/x$,
sem saber que $x\neq 0$.
Como em aritmética precisamos separar em casos
(caso $x = 0$ e caso $x\neq 0$) e os tratar em formas diferentes,
precisamos fazer a mesma coisa trabalhando com conjuntos:
caso $A \neq \emptyset$, achamos uma prova onde podemos
realmente declarar uma variável não-usada para declarar
um elemento de $A$;
caso $A = \emptyset$, achamos uma prova
diferente---e na maioria das vezes essa caso vai ser trivial
para provar.
%%}}}

\endsection
%%}}}

%%{{{ Some simple proofs 
\section Umas provas simples.

%%{{{ x: yes_no_depends_emptyset_arbitraryset 
\exercise.
\label{yes_no_depends_emptyset_arbitraryset}
Seja $A$ um conjunto.
Responda para cada uma das afirmações em baixo com
{``sim''},
{``não''}, ou
{``depende''}:
$$
\xalignat8
\emptyset &\in       \emptyset;&
\emptyset &\in       A        ;&
\emptyset &\subseteq \emptyset;&
\emptyset &\subseteq A        ;&
A         &\in       \emptyset;&
A         &\in       A        ;&
A         &\subseteq \emptyset;&
A         &\subseteq A        . 
\endxalignat
$$

\endexercise
%%}}}

%%{{{ beware: claiming yes/no/depends without proof isn't much 
\beware.
Nossa intuição muitas vezes nos engana, e por isso apenas responder no jeito
que o~\ref{yes_no_depends_emptyset_arbitraryset} pediu não vale muita coisa.
Precisamos provar todas essas respostas.  Para cada uma das oito afirmações
então, precisamos dizer:
\beginul
\li {``sim''} e \emph{provar} a afirmação;
\li {``não''} e \emph{refutar} a afirmação;
\li {``depende''} e \emph{mostrar} pelo menos dois casos:
um onde a afirmação é verdadeira, e outro onde ela é falsa.
Idealmente nesse último caso queremos determinar quando a afirmação é verdadeira,
achando condições \emph{suficientes} e/ou \emph{necessárias}.
\endul
\noindent
Vamos fazer tudo isso agora.
%%}}}

%%{{{ x: attack_order 
\exercise.
\label{attack_order}%
Em qual ordem tu escolharia ``atacar'' essas afirmações?

\endexercise
%%}}}

%%{{{ prop: A_notin_emptyset 
\proposition.
\label{A_notin_emptyset}
Para todo conjunto $A$, $A \notin \emptyset$.
\proof.
Seja $A$ conjunto.  Agora diretamente pala definição de vazio
tomando $x\asseq A$, temos $A \notin \emptyset$.
\qed
%%}}}

%%{{{ cor: emptyset_notin_emptyset 
\corollary.
\label{emptyset_notin_emptyset}%
$\emptyset \notin \emptyset$.
\proof.
Essa afirmação é apenas um \emph{caso especial} de~\ref{A_notin_emptyset}:
tome $A \asseq \emptyset$.
\qed
%%}}}

%%{{{ x: emptyset_notin_emptyset_direct_proof 
\exercise.
\label{emptyset_notin_emptyset_direct_proof}%
Prove o~\ref{emptyset_notin_emptyset} diretamente, sem usar a~\ref{A_notin_emptyset}.

\hint
Quais noções são involvidas nessa afirmação?
Quais são as definições delas?

\endexercise
%%}}}

%%{{{ prop: A_subseteq_A 
\proposition.
\label{A_subseteq_A}%
Para todo conjunto $A$, temos $A \subseteq A$.
\proof.
Seja $A$ conjunto.  Suponha que $a \in A$.
Agora precisamos mostrar $a\in A$, algo que já temos.
\qed
%%}}}

%%{{{ x: A_subseteq_A_quickest_proof 
\exercise.
Pode achar uma prova com menos passos?

\solution
A fórmula que queremos provar é uma tautologia lógica!

\endexercise
%%}}}

%%{{{ cor: emptyset_subseteq_emptyset 
\corollary.
\label{emptyset_subseteq_emptyset}%
$\emptyset\subseteq\emptyset$.
\proof.
Caso especial da~\ref{A_subseteq_A} tomando $A\asseq \emptyset$,
pois $\emptyset$ é um conjunto.
\qed
%%}}}

%%{{{ prop: emptyset_subseteq_A 
\proposition.
\label{emptyset_subseteq_A}%
Para todo conjunto $A$, temos $\emptyset \subseteq A$.
\sketch.
Para chegar num absurdo suponha que tem um contraexemplo: um conjunto $A$ tal que $\emptyset\nsubseteq A$.
Daí achamos rapidamente o absurdo desejado lembrando a definição de $\nsubseteq$.
Sem usar reductio ad absurdum, vamos acabar querendo provar que uma implicação é verdadeira.
Mas cuja premissa é falsa, algo que garanta a vericidade da implicação!
\qes
%%}}}

%%{{{ x: emptyset_subseteq_A_does_not_imply_A_subseteq_A_and_vv 
\exercise.
\label{emptyset_subseteq_A_does_not_imply_A_subseteq_A_and_vv}%
Podemos ganhar a~\ref{A_subseteq_A} como corolário da~\refn{emptyset_subseteq_A}
ou vice-versa?  Explique.

\solution
Não; nenhuma das duas proposições implica a outra.
Da $A\subseteq A$ não podemos substituir o $A$ com nenhum conjunto para chegar na $\emptyset\subseteq A$.
Nem como o $\emptyset$, pois ele teria sido substituito selectivamente em apenas na sua primeira instância, algo que obviamente não podemos fazer.
(Similarmente $x = x$ para todos os números $x$ mas não podemos concluir disso que $0 = x$ para todos os $x$.)
Da $\emptyset\subseteq A$ não podemos chegar na $A \subseteq A$, pois precisamos substituir a constante $\emptyset$ por a variável $A$.
(Similarmente $0 + x = x$ para todos os números $x$, mas não podemos concluir que $x + x = x$.)

\endexercise
%%}}}

%%{{{ prop: emptyset_in_A_sometimes 
\proposition.
\label{emptyset_in_A_sometimes}%
Existe uma infinidade de conjuntos $A$ que satisfazem a $\emptyset \in A$
e uma infinidade de conjuntos $A$ que nco a satisfazem.
%%}}}

%%{{{ prop: A_subseteq_emptyset_iff_A_eq_emptyset 
\proposition.
\label{A_subseteq_emptyset_iff_A_eq_emptyset}%
O único subconjunto do $\emptyset$ é ele mesmo.
Em outras palavras:
$$
A \subseteq \emptyset
\iff
A = \emptyset.
$$
%%}}}

\blah.
Agora falta apenas uma afirmação para examinar: $A \in A$?

%%{{{ x: can_you_find_an_irregular_set 
\exercise.
\label{can_you_find_an_irregular_set}%
Consegues mostrar algum conjunto com a propriedade que ele pertence nele mesmo?
Ou seja, podes achar um conjunto $A$ tal que $A\in A$?

\endexercise
%%}}}

\endsection
%%}}}

%%{{{ Operations on sets and how to define them 
\section Operações entre conjuntos e como as definir.

%%{{{ Operações 
\note Operações.
Lembramos que uma operação num tipo de objetos
mapeia certos objetos desse tipo (suas entradas)
para \emph{exatamente um} objeto desse tipo (sua saida).
%%}}}

%%{{{ Definindo operações 
\note Definindo operações.
Então como podemos \emph{definir uma operação} nos conjuntos?
O que precisamos deixar claro?
Como a saida (ou o ``resultado'') duma operação, no final das contas,
é um conjunto, basta determinar esse conjunto para quaisquer entradas aceitáveis pela operação.
E como determinamos um conjunto?
Para começar, podemos usar um dos jeitos que já encontramos para definir um conjunto $A$:
$$
\xalignat2
A       &\defeq \setst x {\text{\thole$x$\thole}} &
x \in A &\defiff \text{\thole$x$\thole}.
\endxalignat
$$
Bora definir umas operações conhecidas para aquecer.
%%}}}

%%{{{ df: union_def 
\definition.
\label{union_def}%
\tdefined{união}%
\sdefined {\holed A \union \holed B} {a união dos $A$ e $B$}%
Sejam $A,B$ conjuntos.  Definimos
$$
A \union B \defeq \setst x {\text{$x \in A$ ou $x\in B$}}
$$
Alternativamente, podemos definir a mesma operação na seguinte forma equivalente:
$$
x \in A \union B \defiff \text{$x \in A$ ou $x\in B $}.
$$
Chamamos o $A \union B$ a \dterm{união} dos $A$ e $B$.
%%}}}

%%{{{ df: inter_def 
\definition.
\label{inter_def}%
\tdefined{intersecção}%
\sdefined {\holed A \inter \holed B} {a intersecção dos $A$ e $B$}%
Sejam $A,B$ conjuntos.  Definimos
$$
x \in A \inter B \defiff x \in A \mland x\in B.
$$
Chamamos o $A\inter B$ a \dterm{intersecção} dos $A$ e $B$.
%%}}}

%%{{{ x: inter_altdef
\exercise.
\label{inter_altdef}%
Defina a operação $\inter$ usando a notação set-builder.

\solution
Qualquer uma das definições seguintes serve:
$$
\align
A \inter B &\defeq \set {x} {x \in A \mland x\in B}\\
A \inter B &\defeq \set {x\in A} {x\in B}\\
A \inter B &\defeq \set {x\in B} {x\in A}
\endalign
$$

\endexercise
%%}}}

%%{{{ df: disjoint_sets 
\definition.
\label{disjoint_sets}%
\tdefined{disjuntos}%
Sejam $A,B$ conjuntos.
Chamamos os $A$ e $B$ \dterm{disjuntos}
sse não tem nenhum elemento em comum.
Em símbolos,
$$
\text{$A,B$ disjuntos} \defiff A \inter B = \emptyset.
$$
%%}}}

%%{{{ beware: type_errors 
\beware Type errors.
\label{type_errors}%
Não confuda o uso dos $\defeq$ e $\defiff$ (nem dos $=$ e $\iff$).
Usamos $=$ para denotar \emph{igualdade} entre dois \emph{objetos},
e usamos $\iff$ para denotar que as \emph{afirmações} que
aparecem nos dois lados são \emph{equivalentes}.
No mesmo jeito que não podemos escrever
$$
\xalignat3
2 + 3 &\iff 5 &&\text{nem} & x \leq y &\;=\; x + 1 \leq y + 1\\
\intertext{não podemos escrevemos também}
A \setminus B &\iff A \inter \compl B &&\text{nem} & A \subsetneq B &\;=\; A \subseteq B \land A \neq B.\\
\intertext{O que queriamos escrever nesses casos seria:}
2 + 3 &= 5                           &&& x \leq y &\iff x+1 \leq y + 1\\
A \setminus B &= A \inter \compl B &&& A \subsetneq B &\iff A \subseteq B \land A \neq B.
\endxalignat
$$
%%}}}

\blah.
Continuamos com mais operações, incluindo nossa primeira operação unária.

%%{{{ df: complement_def 
\definition.
\label{complement_def}%
\tdefined{complemento}%
\sdefined {\compl{\holed A}} {o complemento de $A$}%
Seja $A$ conjunto.  Definimos
$$
\compl A \defeq \setst x {x\notin A}
$$
Chamamos o $\compl A$ o \dterm{complemento} de $A$.
%%}}}

%%{{{ df: setminus_def 
\definition.
\label{setminus_def}%
\tdefined{complemento}[relativo]%
\sdefined {\holed A \setminus \holed B} {o complemento relativo de $B$ no $A$}%
Sejam $A,B$ conjuntos.
$$
A \setminus B
\defeq
\setst { x\in A } {x \notin B}
$$
Chamamos o conjunto $A\setminus B$ o \dterm{complemento relativo} de $B$ no $A$,
e pronunciamos o $A\setminus B$ como <<$A$ \emph{menos} $B$>> ou <<$A$ \emph{fora} $B$>>.
%%}}}

%%{{{ x: setminus_practice 
\exercise.
\label{setminus_practice}%
Calcule (a extensão d)os conjuntos:
\doublecolumns
\beginol
\li $\set{0,1,2,3,4} \setminus \set{4,1}$
\li $\set{0,1,2,3,4} \setminus \set{7,6,5,4,3}$
\li $\set{0,1,2} \setminus \nats$
\li $\nats \setminus \set{0,1,2}$
\li $\nats \setminus \ints$
\li $\set{\set{0,1}, \set{1,2}, \set{0,2}} \setminus \set{0,1}$
\li $\set{\set{0,1}, \set{1,2}, \set{0,2}} \setminus \set{0,1,2}$
\li $\set{\set{0,1}, \set{1,2}, \set{0,2}} \setminus \set{\set{0,1,2}}$
\li $\set{\set{0,1}, \set{1,2}, \set{0,2}} \setminus \set{\set{1,2}}$
\li $\set{\set{0,1}, \set{1,2}} \setminus \set{\set{1}}$
\li $\set{7,\emptyset} \setminus \emptyset$
\li $\set{7,\emptyset} \setminus \set{\emptyset}$
\li $\reals \setminus 0$
\li $\reals \setminus \set{0}$
\li $\set{1,\set{1}, \set{\set{1}}, \set{\set{\set{1}}}} \setminus 1$
\li $\set{1,\set{1}, \set{\set{1}}, \set{\set{\set{1}}}} \setminus \set{\set{1}}$
\endol
\singlecolumn

\hint
Não siga tua intuição; siga fielmente a definição!

\solution
\beginol
\li $\set{0,1,2,3,4} \setminus \set{4,1} = \set{0,2,3}$
\li $\set{0,1,2,3,4} \setminus \set{7,6,5,4,3} = \set{0,1,2}$
\li $\set{0,1,2} \setminus \nats = \emptyset$
\li $\nats \setminus \set{0,1,2} = \set{3,4,5,6,\dotsc} = \setst {n \in \nats} {n \geq 3}$
\li $\nats \setminus \ints = \emptyset$
\li $\set{\set{0,1}, \set{1,2}, \set{0,2}} \setminus \set{0,1} = \set{\set{0,1}, \set{1,2}, \set{0,2}}$
\li $\set{\set{0,1}, \set{1,2}, \set{0,2}} \setminus \set{0,1,2} = \set{\set{0,1}, \set{1,2}, \set{0,2}}$
\li $\set{\set{0,1}, \set{1,2}, \set{0,2}} \setminus \set{\set{0,1,2}} = \set{\set{0,1}, \set{1,2}, \set{0,2}}$
\li $\set{\set{0,1}, \set{1,2}, \set{0,2}} \setminus \set{\set{1,2}} = \set{\set{0,1}, \set{0,2}}$
\li $\set{\set{0,1}, \set{1,2}} \setminus \set{\set{1}} = \set{\set{0,1}, \set{1,2}}$
\li $\set{7,\emptyset} \setminus \emptyset = \set{7,\emptyset}$
\li $\set{7,\emptyset} \setminus \set{\emptyset} = \set{7}$
\li $\reals \setminus 0 = \reals$
\li $\reals \setminus \set{0} = (-\infty,0)\union(0,+\infty)$
\li $\set{1,\set{1}, \set{\set{1}}, \set{\set{\set{1}}}} \setminus 1 = \set{1,\set{1}, \set{\set{1}}, \set{\set{\set{1}}}}$
\li $\set{1,\set{1}, \set{\set{1}}, \set{\set{\set{1}}}} \setminus \set{\set{\set{1}}} = \set{1,\set{1}, \set{\set{\set{1}}}}$
\endol

\endexercise
%%}}}

%%{{{ x: setminus_equalities_practice 
\exercise.
\label{setminus_equalities_practice}%
Sejam $A,B$ conjuntos.
Para cada uma das afirmações em baixo: prove se é verdadeira;
refuta se é falsa; mostre um exemplo e um contraexemplo
se depende nos $A,B$ (e tente determinar exatamente quando é verdadeira).
$$
\xalignat3
(1)  \quad& \emptyset \setminus \emptyset = \emptyset       &
(5)  \quad& A \setminus A = A                               &
(9)  \quad& A \setminus B = B                               \\
(2)  \quad& A \setminus \emptyset         = A               &
(6)  \quad& A \setminus A = \emptyset                       &
(10) \quad& A \setminus B                   = B \setminus A \\
(3)  \quad& \emptyset \setminus A         = \emptyset       &
(7)  \quad& A \setminus B = \emptyset                       &
(11) \quad& A \setminus \set{B}             = A             \\
(4)  \quad& \set{\emptyset} \setminus A   = \emptyset       &
(8)  \quad& A \setminus B = A                               &
(12) \quad& \set{A,B} \setminus (A\union B) = \emptyset     
\endxalignat
$$

\endexercise
%%}}}

%%{{{ df: symdiff_def 
\definition.
\label{symdiff_def}%
\tdefined{diferença simétrica}%
\sdefined {\holed A \symdiff \holed B} {a diferença simétrica dos $A$ e $B$}%
Sejam $A,B$ conjunto.  Definimos
$$
x \in A \symdiff B \defiff \text{$x$ pertence em exatamente um dos $A,B$}.
$$
Chamamos o $A \symdiff B$ a \dterm{diferença simétrica} dos $A$ e $B$.
%%}}}

%%{{{ eg: symdiff_example 
\example.
\label{symdiff_example}%
Calculamos os conjuntos:
\beginul
\li $\set{0,1,2,3} \symdiff \set{1,2,4,8} = \set{0,3,4,8}$
\li $\set{\set{0,1}, \set{1,2}} \symdiff \set{0,1,2} = \set{ \set{0,1}, \set{1,2}, 0, 1, 2 }$
\li $\set{\set{0,1}, \set{1,2}} \symdiff \set{\set{0,1},\set{0,2}} = \set{ \set{0,2}, \set{1,2} }$
\li $(-2,1) \symdiff (-1,2) = (-2,-1] \union [1,2)$
\endul
\endexample
%%}}}

%%{{{ x: what_is_A_symdiff_A_and_A_symdiff_emptyset 
\exercise.
\label{what_is_A_symdiff_A_and_A_symdiff_emptyset}%
Dado $A$ conjunto, calcule os conjuntos $A\symdiff A$ e $A \symdiff \emptyset$.

\endexercise
%%}}}

\endsection
%%}}}

%%{{{ Proving equalities and inclusions 
\section Provando igualdades e inclusões.

%%{{{ x: simple_inclusion_practice 
\exercise.
Prove ou refuta a afirmação:
$$
\text{\emph{para todos os conjuntos $A,B,C$,
se $A \subseteq B$ e $A \subseteq C$, então $A \subseteq B \inter C$}.}
$$

\solution
Vou provar a afirmação.
\endgraf
Suponha que $A \subseteq B$ e $A \subseteq C$.
Tome um $a \in A$.  Precisamos mostrar que $a \in B \inter C$.
Como $a \in A$ e $A \subseteq B$, temos $a \in B$;
e como $a \in A$ e $A \subseteq C$, temos $a \in C$.
Logo $a \in B\inter C$, pela definição de $B\inter C$.

\endexercise
%%}}}

%%{{{ prop: set_de_morgan 
\proposition.
\label{set_de_morgan}%
\ii{dualidade}%
Para todos os conjuntos $A,B,C$,
$$
\align
C \setminus (A \union B) &= (C \setminus A) \inter (C \setminus B)\\
C \setminus (A \inter B) &= (C \setminus A) \union (C \setminus B).
\endalign
$$
\proof Prova (usando fórmulas).
Sejam $A,B,C$ conjuntos.  Temos:
$$
\alignat2
x \in C \setminus (A \union B)
&\iff x \in C \land \lnot (x \in A \union B)                        \qqby{def.~$\setminus$}\\
&\iff x \in C \land \lnot (x \in A \lor x \in B)                    \qqby{def.~$\union$}\\
&\iff x \in C \land (x \notin A \land x \notin B)                   \qqby{De Morgan}\\
&\iff (x \in C \land x \notin A) \land (x \in C \land x \notin B)   \qqby{lógica}\\
&\iff (x \in C \setminus A) \land (x \in C \setminus B)             \qqby{def.~$\setminus$}\\
&\iff x \in (C \setminus A) \inter (C \setminus B)                  \qqby{def.~$\inter$}
\endalignat
$$
Logo
$
C \setminus (A \union B)
=
(C \setminus A) \inter (C \setminus B)
$
pela definição de igualdade de conjuntos.
A prova da outra igualdade é similar:
trocamos apenas os $\union$ com os $\inter$, e os $\lor$ com os $\land$!
\qed
%%}}}

%%{{{ x: set_de_morgan_nat_lang 
\exercise.
Escreva uma prova em linguagem natural da~\ref{set_de_morgan}.

\solution
Mostramos as duas inclusões separadamente:
\endgraf
\lrdirset:
Tome $x \in C \setminus (A \union B)$.
Daí $x \in C$ e $x \notin (A \union B)$, ou seja $x\notin A$ e $x \notin B$.
Como $x \in C$ e $x\notin A$, temos $x\in C\setminus A$, e,
similarmente $x\in C\setminus B$.
Logo chegamos no desejado $x\in (C\setminus A) \inter (C\setminus B)$.
\endgraf
A inclusão inversa \rldirset\ é similar.

\endexercise
%%}}}

%%{{{ prop: symdiff_altdef1 
\proposition.
Sejam $A,B$ conjuntos.  Logo,
$$
A \symdiff B
=
(A \setminus B) \union (B \setminus A)
$$
\sketch.
Antes de começar, traduzimos os dois lados:
<<em exatamente um dos dois>> na esquerda,
<<no primeiro mas não no segundo ou no segundo mas não no primeiro>> na direita.
Faz sentido que os dois conjuntos são iguais, pois as duas frases são equivalentes!
Mas para provar formalmente a afirmação, mostramos as duas direções
$$
\xalignat3
A \symdiff B &\subseteq (A \setminus B) \union (B \setminus A) && \mland &
A \symdiff B &\supseteq (A \setminus B) \union (B \setminus A)
\endxalignat
$$
separadamente, usando as definições de $\subseteq$ e $\supseteq$.
\qes
%%}}}

%%{{{ prop: symdiff_altdef2 
\proposition.
Sejam $A,B$ conjuntos.  Logo,
$$
A \symdiff B
=
(A \union B) \setminus (A \inter B).
$$
\sketch.
Novamente, começamos pensando nos dois lados e suas intensões:
<<em exatamente um dos dois>> na esquerda;
<<em pelo menos um dos dois, mas não nos dois>> na direita.
As duas frases são equivalentes, mas vamos mostrar formalmente
a igualdade desses conjuntos, mostrando novamente as
``$\subseteq$'' e ``$\supseteq$'' separadamente.
\qes
%%}}}

\note.
No~\refn{symdiff_def} eu dei uma definição elementária, usando a
notação set-builder para determinar o conjunto $A\symdiff B$.
De fato, seria até melhor \emph{definir} a operação $\symdiff$ usando uma das
duas expressões em cima.

\endsection
%%}}}

%%{{{ Cardinality 
\section Cardinalidade.

%%{{{ df: naive_cardinality 
\definition.
\label{naive_cardinality}%
\tdefined{cardinalidade}[ingenuamente]%
\sdefined {\card {\holed A}} {a cardinalidade de $\holed A$}%
Seja $A$ um conjunto.
A \dterm{cardinalidade} de $A$ é a quantidade de elementos de $A$.
Denotamos-la por $\card A$:
$$
\card A \defeq \knuthcases{
    n,      & se $A$ é finito com exatamente $n$ membros distintos\cr
    \infty, & se $A$ é infinito.
}
$$
%%}}}

\blah.
Nos capítulos~\refn{Cantors_paradise} e~\refn{Axiomatic_set_theory}
vamos \emph{refinar} essa notação pois como \Cantor{}Cantor percebeu,
o segundo caso na~\ref{naive_cardinality} é \emph{bem}, \emph{bem}, \emph{bem}
mais rico do que aparece!

%%{{{ prop: cardinality_of_union_of_finite_sets 
\property.
\label{cardinality_of_union_of_finite_sets}%
Sejam $A,B$ conjuntos finitos.
Logo
$$
\card{A\union B} \leq \card A + \card B.
$$
\sketch.
Pelo princípio da inclusão--exclusão
(\refn{Inclusion_exclusion_principle})
temos
$$
\card{A\union B} = \card A + \card B - \card{A \inter B}
$$
e como uma cardinalidade não pode ser negativa, segue disegualdade desejada.
\qes
%%}}}

%%{{{ x: cardinality_of_union_of_disjoint_finite_sets 
\exercise.
\label{cardinality_of_union_of_disjoint_finite_sets}%
Determine quando temos $=$ na desigualdade da~\ref{cardinality_of_union_of_finite_sets}.

\solution
Exatamente quando $A,B$ são disjuntos.
Em símbolos,
$$
\card{A\union B} = \card A + \card B
\iff
A \inter B = \emptyset.
$$

\endexercise
%%}}}

\endsection
%%}}}

%%{{{ Powerset 
\section Powerset.

%%{{{ df: powerset_def 
\definition.
\label{powerset_def}%
\tdefined{powerset}%
\iisee{conjunto}[de partes]{powerset}%
\sdefined {\pset{\holed A}} {o powerset de $\holed A$ (conjunto de partes)}%
Seja $A$ conjunto.
Chamamos o \dterm{powerset} (ou \dterm{conjunto de partes}, ou
\dterm{conjunto potência}) de $A$,
denotado por $\pset A$, é o conjunto de todos os subconjuntos de $A$.
Formalmente:
$$
X \in \pset A \defiff X \subseteq A.
$$
%%}}}

%%{{{ x: cardinality_of_poweset_of_finite_set 
\exercise.
\label{cardinality_of_poweset_of_finite_set}%
Seja $A$ conjunto finito.  Qual a cardinalidade do $\pset A$?

\solution
Já resolvemos isso na~\ref{Number_of_subsets}!
Concluimos que:
$$
\card{\pset A} = 2^{\card A}.
$$

\endexercise
%%}}}

\endsection
%%}}}

%%{{{ Tuples 
\section Tuplas.
\label{Tuples}%

\TODO Par ordenado como black box.

\blah.
Naturalmente, podemos generalizar a idéia de par ordenado, cujo tamanho é $2$, para um conceito
correspondente de tamanho $n$, onde $n$ é qualquer número natural.  Oi, tuplas!

\TODO Tuplas como black boxes.

\endsection
%%}}}

%%{{{ Cartesian products 
\section Produtos cartesianos.

%%{{{ df: cartesian_product 
\definition.
\label{cartesian_product}%
Sejam $A,B$ conjuntos.
$$
A \times B \defeq \setst {\tup{a,b}} {a\in A,\ b\in B}
$$
%%}}}

%%{{{ x: when_cartesian_commutes 
\exercise.
\label{when_cartesian_commutes}%
Suponha que $A,B\neq\emptyset$.
Escreva uma prova direta (sem usar reductio ad absurdum) do
$A \times B = B\times A \iff A = B$.

\hint
Uma direção é trivial---por quê?

\hint
Como usamos os fatos $A\neq\emptyset$ e $B\neq\emptyset$?

\endexercise
%%}}}

%%{{{ x: when_cartesian_commutes_absurdum 
\exercise.
Prove usando reductio ad absurdum a direção não-trivial
do~\ref{when_cartesian_commutes}.

\endexercise
%}}}

%%{{{ pseudodf: set_exp 
\pseudodefinition.
Sejam $A$ conjunto e $n\in\nats$ com $n\geq 2$.
$$
A^n \pseudodefeq \setstt {\tup{a_1,\dotsc,a_n}} {$a_i \in A$ para todo $2\leq i\leq n$}.
$$
%%}}}

%%{{{ messy quantifier 
\note.
Na definição em cima aparece a frase ``para todo $2\leq i\leq n$''.
Qual é a variável ligada com esse ``para todo''?
Bem, $2$ é um constante, e $n$ já foi declarado, então entendemos que a
frase corresponde na quantificação:
$$
\lforall {i\in\nats} {2\leq i \leq n \limplies a_i \in A}.
$$
%%}}}

%%{{{ df: set_exp_first_recursive_definition 
\definition.
\label{set_exp_first_recursive_definition}%
Seja $A$ conjunto.
Para $n\in\nats$ com $n\geq 2$ definimos os conjuntos:
$$
\align
A^2     &= A\times A\\
A^{n+1} &= A^n \times A.
\endalign
$$
%%}}}

\TODO As duas definições são equivalentes?

\TODO O produto cartesiano é associativo?

\endsection
%%}}}

%%{{{ Sequences 
\section Seqüências.
\label{Sequences}%

\TODO Seqüências como black boxes.

\TODO Seqüências de conjuntos.

\endsection
%%}}}

%%{{{ Indexed families 
\section Famílias indexadas.

\TODO De tupla para família de conjuntos indexada por algum conjunto $\cal I$: $\family {A_i} {i\in \cal I}$.

%%{{{ eg: anscestors_and_children 
\example.
\label{ancestors_and_children}%
Seja $\cal P$ o conjunto de todas as pessoas do mundo.
Definimos para cada pessoa $p\in \cal P$, os conjuntos $A_p$ e $C_p$ de todos
os ancestrais e todos os filhos de $p$, respeitivamente.
\endexample
%%}}}

%%{{{ eg: airports_direct_flights 
\example.
\label{airports_direct_flights}%
Seja $\cal A$ o conjunto de todos os aeroportos.
Para cada aeroporto $a\in\cal A$, seja
$$
D_a \defeq \setst { b \in \cal A } { \text{existe vôo direto de $a$ para $b$} }.
$$
\endexample
%%}}}

%%{{{ eg: books_and_authors 
\example.
\label{books_and_authors}%
Seja $\cal B$ o conjunto de todos os livros.
Para cada livro $b\in\cal B$, sejam
$$
\align
A_b &\defeq \setstt a {$a$ é um autor do livro $b$}\\
W_b &\defeq \setstt w {$w$ é uma palavra que aparece no texto do livro $b$}.
\endalign
$$
\endexample
%%}}}

%%{{{ df: cartesian_product_of_indexed_family_of_sets 
\definition Produto cartesiano (família).
\label{cartesian_product_of_indexed_family_of_sets}%
Seja $\family {A_i} {i\in \cal I}$ uma família de conjuntos
indexada por um conjunto de índices $\cal I$.
$$
\Prod_{i\in \cal I}\defeq
\setstt {\family {a_i} {i\in\cal I}} {$a_i\in A_i$ para todo $i\in\cal I$}.
$$
%%}}}

\blah.
Assim que ganhar pouca experiência com funções, vamos dar uma definição mais simples de produto cartesiano (\ref{Cartesian_product_of_indexed_family_of_sets_revisited}).

\endsection
%%}}}

%%{{{ Big union; big intersections 
\section União grande; intersecção grande.

%%{{{ intro 
\blah.
Generalizamos agora as operações binárias de união e intersecção para suas versões arbitrárias, as operações unitárias $\Union\dhole$ e $\Inter\dhole$.
Antes de dar uma definição, mostramos uns exemplos.
Esses operadores são mais interessantes e úteis quando são aplicados em
conjunto cujos membros são conjuntos também, pois---coloquiamente
falando---eles correspondem na união e na intersecção dos seus membros.
%%}}}

%%{{{ eg: Union_Inter_example 
\example.
\label{Union_Inter_example}%
Aplicamos as operações $\Union$ e $\Inter$ nos conjuntos seguintes:
$$
\align
\Union \set{ \set{1,2,4,8}, \set{0,2,4,6}, \set{2,10} } &= \set{0,1,2,4,6,8,10}\\
\Inter \set{ \set{1,2,4,8}, \set{0,2,4,6}, \set{2,10} } &= \set{2}\\
\Union \set{ \nats, \ints, \rats, \reals }              &= \reals\\
\Inter \set{ \nats, \ints, \rats, \reals }              &= \nats\\
\Union \set{ 2, 3, \set{4,5}, \set{4,6} }               &= \set{4,5,6}\\
\Inter \set{ 2, 3, \set{4,5}, \set{4,6} }               &= \emptyset
\endalign
$$
\endexample
%%}}}

%%{{{ df: Union_def 
\definition.
\label{Union_def}%
\label{Inter_def}%
\tdefined{união}[grande]%
\tdefined{intersecção}[grande]%
\sdefined {\Union \holed {\scr A}} {a união de $\holed{\scr A}$}%
\sdefined {\Inter \holed {\scr A}} {a intersecção de $\holed{\scr A}$}%
Seja $\scr A$ um conjunto.
$$
\align
x \in \Union \scr A
&\defiff
\lexists {A \in \scr A} {x \in A}\\
x \in \Inter \scr A
&\defiff
\lforall {A \in \scr A} {x \in A}.
\endalign
$$
Chamamos o $\Union\scr A$ a \dterm{união} de $\scr A$, e o $\Inter\scr A$
a \dterm{intersecção} de $\scr A$.
%%}}}

%%{{{ x: union_and_inter_from_Union_and_Inter_sugar 
\exercise.
\label{union_and_inter_from_Union_and_Inter_sugar}%
Defina os operadores binários $\union$ e $\inter$ como açúcar sintáctico
definido pelos operadores unários $\Union$ e $\Inter$ respeitivamente.

\solution
Sejam $A,B$ conjuntos.
Botamos
$$
\xalignat2
A \union B &\defeq \Union\set{A,B} &
A \inter B &\defeq \Inter\set{A,B}.
\endxalignat
$$

\endexercise
%%}}}

%%{{{ x: iterate_pset_and_Union_on_emptyset 
\exercise.
\label{iterate_pset_and_Union_on_emptyset}%
Calcule os conjuntos:
$$
\pset\emptyset,
\qquad \pset\pset\emptyset,
\qquad \pset\pset\pset\emptyset,
\qquad \Union \emptyset,
\qquad \Union\Union \emptyset.
$$

\solution
Calculamos pelas definições de $\pset$ e $\Union$:
$$
\align
\pset\emptyset &= \set { \emptyset }\\
\pset\pset\emptyset &= \set { \emptyset, \set{\emptyset} }\\
\endalign
$$

\endexercise
%%}}}

%%{{{ x: Inter_emptyset 
\exercise.
\label{Inter_emptyset}%
Calcule o $\Inter\emptyset$.

\hint
Siga a definição e nada mais!

\endexercise
%%}}}

%%{{{ note: intuition_about_Inter_Union_powerset_and_set_braces 
\note.
\label{intuition_about_Inter_Union_powerset_and_set_braces}
Bem, bem, bem informalmente podemos dizer que: a operação $\Union$ tire o nivel
mais externo de chaves; a $\Inter$ também mas jogando fora bem mais elementos
(aqueles que não pertencem em todos os membros do seu argumento); e o $\pset$
bote todas as chaves em todas as combinações possíveis para ``o nivel mais
próximo''.
%%}}}

%%{{{ x: card_of_set_its_Union_and_its_Inter_comparison 
\exercise.
\label{card_of_set_its_Union_and_its_Inter_comparison}%
Ache conjuntos finitos $A,B$ tais que
$$
\xalignat2
&0 < \card { \Inter A } < \card A < \card { \Union A } &
&0 < \card B < \card { \Inter B } < \card { \Union B }.
\endxalignat
$$

\solution
Tome
$$
\xalignat2
A &\asseq \set{ \set{ 1,2,3 },   \set{ 3,4,5 }   } &&(0 < 1 < 2 < 4)\\
B &\asseq \set{ \set{ 1,2,3,4 }, \set{ 2,3,4,5 } } &&(0 < 2 < 3 < 5).
\endxalignat
$$

\endexercise
%%}}}

%%{{{ Q: how_would_you_define_Union_of_sequence 
\question.
\label{how_would_you_define_Union_of_sequence}%
Imagina que temos, para cada $n\in\nats$, um conjunto $A_n$.%
\footnote{Isso é chamado uma \dterm{seqüência} de conjuntos.
Mais sobre isso:~\ref{Sequences}.}
Como tu definiria os conjuntos
$\Union_{n=0}^{\infty} A_n$
e 
$\Inter_{n=0}^{\infty} A_n$
?
%%}}}
\spoiler.

%%{{{ df: Union_Inter_of_sequence 
\definition.
\label{Union_Inter_of_sequence}%
Definimos os operadores unários
$\Union_{n=0}^{\infty}$ e
$\Inter_{n=0}^{\infty}$ assim:
suponha que $A_n$ é um conjunto para cada $n\in\nats$; definimos
$$
\xalignat2
\Union_{n=0}^{\infty} A_n &\defeq \lexists {n \in \nats} {x \in A_n}&
\Inter_{n=0}^{\infty} A_n &\defeq \lforall {n \in \nats} {x \in A_n}.
\endxalignat
$$
%%}}}

%%{{{ Union_Inter_of_sequences_as_sugar 
\exercise.
\label{Union_Inter_of_sequences_as_sugar}%
Na~\ref{Union_Inter_of_sequence} definimos \emph{elementariamente}
as operações $\Union_{n=0}^{\infty}$ e $\Inter_{n=0}^{\infty}$.
Defina-las usando as operações de $\Union$ e $\Inter$.

\solution
Seja $A_n$ uma seqüência de conjuntos.
Defina
$$
\Union_{n=0}^{\infty} \defeq \Union\setst {A_n} {n\in\nats}
\qqqquad
\mland
\qqqquad
\Inter_{n=0}^{\infty} \defeq \Inter\setst {A_n} {n\in\nats}.
$$

\endexercise
%%}}}

%%{{{ df: Union_Inter_of_family 
\definition.
\label{Union_Inter_of_family}%
Definimos os operadores unários
$\Union_{i\in\cal I}$ e
$\Inter_{i\in\cal I}$ assim:
suponha que $\famil F i {\cal I}$ é uma família de conjuntos indexada por um
conjunto de índices $\cal I$.
Definimos
$$
\xalignat2
\Union_{i\in\cal I} A_i &\defeq \lexists {i\in \cal I} {x \in A_i}&
\Inter_{i\in\cal I} A_i &\defeq \lforall {i\in \cal I} {x \in A_i}.
\endxalignat
$$
%%}}}

%%{{{ prop: set_de_morgan_gen 
\proposition.
\label{set_de_morgan_gen}%
\ii{dualidade}%
Para todo conjunto $C$ e cada seqüência de conjuntos
$\set{ A_n }_n$,
$$
\align
C \setminus {\Union_{n=0}^\infty A_n} &= \Inter_{n=0}^{\infty} \paren{C \setminus A_n}\\
C \setminus {\Inter_{n=0}^\infty A_n} &= \Union_{n=0}^{\infty} \paren{C \setminus A_n}.
\endalign
$$
\proof Prova (usando fórmulas).
Sejam $C$ conjunto e $\set{ A_n }_n$ seqüência de conjuntos.
Calculamos
$$
\alignat2
x \in C \setminus \Union_{n=0}^{\infty}A_n
&\iff x \in C \land \lnot \biggparen{x \in \Union_{n=0}^{\infty}A_n}   \qqby{def.~$\setminus$}\\
&\iff x \in C \land \lnot (\exists n\in\nats)[x \in A_n]  \qqby{def.~$\Union_{n=0}^{\infty}$}\\
&\iff x \in C \land (\forall n\in\nats)[x \notin A_n]     \qqby{De Morgan}\\
&\iff (\forall n\in\nats)[x \in C \land x \notin A_n]     \qqby{lógica}\\
&\iff (\forall n\in\nats)[x \in C \setminus A_n]          \qqby{def.~$\setminus$}\\
&\iff x \in \Inter_{n=0}^{\infty}(C \setminus A_n).       \qqby{def.~$\Inter_{n=0}^{\infty}$}
\endalignat
$$
A prova da outra igualdade é similar:
é só trocar os $\Union$ com os $\Inter$, e os $\exists$ com os $\forall$!
\qed
%%}}}

\blah.
Observe a semelhança entre essa prova e a prova da~\ref{set_de_morgan}:
até nas justificativas de cada passo!
Realmente, a~\ref{set_de_morgan_gen} é uma generalização da~\refn{set_de_morgan},
algo que tu provarás no exercício seguinte:

%%{{{ x: set_de_morgan_gen_indeed 
\exercise.
\label{set_de_morgan_gen_indeed}%
O que precisamos fazer para ganhar a~\ref{set_de_morgan} como um corolário
da~\refn{set_de_morgan_gen}?

\hint
Quais são os dados que precisamos ter per aplicar a~\ref{set_de_morgan_gen}?

\solution
Dados conjuntos $A,B,C$, precisamos mostrar que:
$$
C \setminus (A \union B)
=
(C \setminus A) \inter (C \setminus B).
$$
Seja $\set{A_n}_n$ a seqüência $A,B,B,B,\dotsb$, (ou seja, a seqüência definida pelas:
$A_0 = A$; e $A_i = B$ para $i > 0$).
Pela~\ref{set_de_morgan_gen} temos então:
$$
C \setminus \munderbrace{\Union_{n=0}^\infty A_n} {A \union B}
=
\munderbrace{\Inter_{n=0}^{\infty} (C \setminus A_n)} {(C\setminus A) \inter (C\setminus B)}.
$$

\endexercise
%%}}}

%%{{{ x: set_de_morgan_gen_nat_lang 
\exercise.
Escreva uma prova em linguagem natural da~\ref{set_de_morgan_gen}.

\solution
Mostramos as duas inclusões separadamente:
\endgraf
\lrdirset:
Tome $x \in C \setminus \Union_{n=0}^{\infty}A_n$,
ou seja, $x \in A_k$ para algum $k\in \nats$,
e logo $x \notin C \setminus A_k$ e daí
$x \notin \Inter_n \paren{C \setminus A_n}$
(pois não pertence no seu $k$-ésimo conjunto)
\endgraf
A inclusão inversa \rldirset\ é similar.

\endexercise
%%}}}

%%{{{ df: pairwise_disjoint 
\definition.
\label{pairwise_disjoint}%
Seja $\scr A$ uma família de conjuntos.
Chamamos seus membros \dterm{disjuntos dois a dois} sse nenhum deles
tem elementos em comum com nenhum outro deles.
Em símbolos:
$$
\align
\text{os conjuntos da $\scr A$ são disjuntos dois a dois}
&\defiff
\lforall {A,B \in \scr A} {A \neq B \limplies A\inter B = \emptyset }.
\intertext{Similarmente para famílias indexadas:}
\text{os conjuntos da $\famil A i {\cal I}$ são disjuntos dois a dois}
&\defiff
\lforall {i,j \in \cal I} {i\neq j \limplies A_i \inter A_j = \emptyset}.
\intertext{E logo para seqüências também:}
\text{os conjuntos da $\sequence A n$ são disjuntos dois a dois}
&\defiff
\lforall {i,j \in \nats} {i\neq j \limplies A_i \inter A_j = \emptyset}.
\endalign
$$
%%}}}

%%{{{ x: disjoint_not_pairwise_disjoint 
\exercise.
\label{disjoint_not_pairwise_disjoint}%
Ache uma família de conjuntos $\scr A$ com $\Inter \scr A = \emptyset$
mas cujos membros não são disjuntos dois a dois.

\solution
Tome
$$
\scr A \asseq \set{ \set{0,1}, \set{1,2}, \set{7} }.
$$

\endexercise
%%}}}

%%{{{ warning: big_setops_vs_sum 
\warning.
\label{big_setops_vs_sum}%
Uma diferença importantíssima entre os operadores $\sum{}$ e $\Union$ e seus indices:
um ``somatório infinito'' não é nem associativo nem comutativo!
O seguinte teorema de \Riemann{}Riemann é bastante impressionante!
%%}}}

%%{{{ thm: riemann_rearrangement 
\theorem Riemann's rearrangement.
\label{riemann_rearrangement}%
Se uma serie infinita $\sum a_i$ de reais é \emph{condicionalmente convergente},%
\footnote{Uma série $\sum a_i$ é \dterm{condicionalmente convergente}
sse ela é convergente, mas a $\sum |a_i|$ não é.}
então podemos apenas permutando seus termos criar uma nova serie infinita que
converga em \emph{qualquer} $x\in[-\infty,\infty]$ que desejamos.
\sketch.
Separe as metas: (i) mostrar como criar uma série que converge num dado número $\ell$;
(ii) mostrar como criar uma série divergente.
\endgraf
Observe que como $\sum a_i$ é condicionalmente convergente, ela contem uma infinidade
de termos positivos, e uma infinidade de termos negativos.
Para o (i), fixe um $\ell\in\reals$.
Fique tomando termos positivos da serie $a_i$ até seu somatório supera o $\ell$.
Agora fique tomando termos negativos da $a_i$ até seu somatório cai embaixo do $\ell$.
Agora fique tomando termos positivos até superar o $\ell$, etc.~etc.
Continuando assim conseguimos construir uma serie feita por termos da $\sum a_i$ que converge no $\ell$.
\qes
\proof.
Veja~\cite[\S10.21~\&~Teorema~10.22]{apostol2}.
\qed
%%}}}

\endsection
%%}}}

%%{{{ Translating propositions from and to set relations 
\section Traduzindo proposições de e para relações de conjuntos.

\blah.
Lembra os exemplos~\refn{ancestors_and_children},~\refn{airports_direct_flights},
e~\refn{books_and_authors}?
Bom.  Vamos praticar nossa fluência em cojuntos usando esses exemplos agora.

%%{{{ x: ancestors_and_children_translations 
\exercise.
\label{ancestors_and_children_translations}%
Para toda pessoa $p\in\cal P$ defina $A_p$ e $C_p$ como
no~\ref{ancestors_and_children}.
Suponha que $p,q,r$ denotam pessoas.  Traduza os de linguagem de conjuntos para
linguagem natural e vice versa as afirmações e os objetos seguintes.
\beginol
\li $p$ e $q$ são irmãos.
\li $C_p \neq \emptyset$.
\li $p$ é filho único.
\li $p$ e $q$ são parentes.
\li $p$ e $q$ são primos de primeiro grau.
\li $r$ é filho dos $p$ e $q$.
\li $C_p \inter C_q \neq \emptyset$.
\li $A_p \subseteq A_q$.
\li $C_p \subsetneq C_q$.
\endol

\endexercise
%%}}}

\TODO More exercises.

\endsection
%%}}}

%%{{{ Intervals on the real line 
\section Intervalos na reta real.

%%{{{ x: running_intervals_on_reals 
\exercise.
\label{running_intervals_on_reals}%
Para $n=1,2,3,\dotsc$ defina os intervalos de reais
$$
\xalignat2
F_n &= \ival[ {-1+\frac1n \,,\; 1-\frac1n} ] &
G_n &= \ival( {-1-\frac1n \,,\; 1+\frac1n} ).
\endxalignat
$$
Calcule os conjuntos $\Union_n F_n$ e $\Inter_n G_n$.

\endexercise
%%}}}

%%{{{ x: running_intervals_on_nats 
\exercise.
\label{running_intervals_on_nats}%
Para $n\in\nats$, defina os conjuntos de naturais
$$
\xalignat2
A_n &= \setst {i\in\nats} {i \leq n} &
B_n &= \nats\setminus A_n.
\endxalignat
$$
Calcule os conjuntos $\Inter_n A_n$ e $\Inter_n B_n$.

\endexercise
%%}}}

\endsection
%%}}}

%%{{{ Multisets 
\section Multisets.
\label{Multisets}%
\iisee{bag}{multiset}%

\note Multisets como black boxes.

\endsection
%%}}}

%%{{{ Problems 
\problems.

%%{{{ sequence_of_sets_inclusions_tricky_indices 
\problem.
\label{sequence_of_sets_inclusions_tricky_indices}%
Sejam $\sequence A n$ e $\sequence B n$ duas seqüências de conjuntos,
tais que para todo $n\in\nats$,
$A_n \subsetneq B_{n+1}$.
Prove que:
$$
\Union_{n=0}^\infty A_n \subseteq \Union_{n=0}^{\infty} B_n.
$$

\solution
Suponha $x \in \Union_{n=0}^\infty A_n$.
Preciso mostrar que $x \in \Union_{n=0}^{\infty} B_n$,
ou seja, mostrar que $x$ pertence em pelo menos um dos $B_n$'s.
[ou seja, procuro um $k\in \nats$ tal que $x \in B_k$].
Seja $m\in \nats$ tal que $x \in A_m$
(pela definição de $\Union_{n=0}^\infty A_n$).
Pela hipótese, $x \in B_{m+1}$, algo que mostra que
$x \in \Union_{n=0}^{\infty} B_n$
[tome $k \asseq m+1$].

\endproblem
%%}}}

%%{{{ sequence_of_sets_inclusions_counterexample 
\problem.
\label{sequence_of_sets_proper_inclusions_counterexample}%
Sejam $\sequence A n$ e $\sequence B n$ duas seqüências de conjuntos,
tais que para todo $n\in\nats$,
$A_n \subsetneq B_n$.
Podemos concluir que
$$
\Union_{n=0}^\infty A_n \subsetneq \Union_{n=0}^{\infty} B_n\,?
$$

\hint
Procure um contraexemplo:
defina duas seqüências $\sequence A n$ e $\sequence B n$
que satisfazem a condição do problema, e mesmo assim,
não é valida a conclusão proposta.

\solution
Vamos construir um contraexemplo.
\endgraf
Considere as seqüências de conjuntos seguintes:
para todo $n\in\nats$ define
$$
A_n \eqass (-n,n)
\qqqquad
\aligned
B_0     &\asseq \emptyset\\
B_{n+1} &\asseq [-n,n].
\endaligned
$$
Observe que, realmente, para todo $n\in\nats$ temos $A_n \subsetneq B_{n+1}$:
$$
A_n = (-n,n) \subsetneq [-n,n] = B_{n+1}.
$$
Mesmo assim,
$$
\Union_{n=0}^\infty A_n = \reals = \Union_{n=0}^{\infty} B_n.
$$

\endproblem
%%}}}

%%{{{ arbitrary_finite_symdiff 
\problem.
\label{arbitrary_finite_symdiff}%
Sejam $n\in\nats$ com $n\geq 2$ e $n$ conjuntos
$A_1, A_2, \dotsc, A_n$.
Seja
$$
A \asseq A_1 \symdiff A_2 \symdiff \dotsb \symdiff A_n.
$$
Observe que como a operação $\symdiff$ é associativa e comutativa,
o $A$ é bem definido.
Prove que:
$$
A = \set{ a \st \text{$a$ pertence numa quantidade ímpar de $A_i$'s}}.
$$

\hint
Indução.

\solution
Provamos por indução que para todo inteiro $n \geq 2$,
$$
A_1 \symdiff A_2 \symdiff \dotsb \symdiff A_n
= \set{ a \st \text{$a$ pertence numa quantidade ímpar de $A_i$'s}}.
$$
\casestyle{Base ($n=2$):}
$x \in A_1 \symdiff A_2 \iff \text{$x$ pertence numa quantidade ímpar dos $A_1,A_2$}$ (óbvio).
\endgraf
Seja $k\in\nats$ tal que 
$$
A_1 \symdiff A_2 \symdiff \dotsb \symdiff A_k
= \set{ a \st \text{$a$ pertence numa quantidade ímpar de $A_i$'s}}.\tag{H.I.}
$$
Precisamos mostrar que:
$$
A_1 \symdiff \dotsb \symdiff A_{k+1}
= \set{ a \st \text{$a$ pertence numa quantidade ímpar dos $A_1,\dotsc,A_{k+1}$'s}}.
$$
\endgraf
(\lrdirset):
Suponha que $x\in A_1 \symdiff A_2 \symdiff \dotsb \symdiff A_{k+1}$,
ou seja, $x \in (A_1 \symdiff A_2 \symdiff \dotsb \symdiff A_k) \symdiff A_{k+1}$.
Pela definição de $\symdiff$, temos dois casos:
\endgraf
\casestyle{Caso 1:}
$x \in A_1 \symdiff A_2 \symdiff \dotsb \symdiff A_k \mland x\notin A_{k+1}$.\CR
Pela H.I., $x$ pertence numa quantidade ímpar dos $A_1,\dotsc,A_k$, e não no $A_{k+1}$, então numa quantidade ímpar dos $A_1,\dotsc,A_{k+1}$.
\endgraf
\casestyle{Caso 2:}
$x \notin A_1 \symdiff A_2 \symdiff \dotsb \symdiff A_k \mland x\in A_{k+1}$.\CR
Pela H.I., $x$ pertence numa quantidade par dos $A_1,\dotsc,A_k$ e no
$A_{k+1}$, então numa quantidade ímpar dos $A_1,\dotsc,A_{k+1}$.
\endgraf
\bigskip
\endgraf
(\rldirset):
Suponha que $x$ pertence numa quantidade ímpar dos $A_1,\dotsc,A_{k+1}$.
Separamos em dois casos:
\endgraf
\casestyle{Caso 1:} $x \in A_{k+1}$.\CR
Logo $x$ pertence numa quantidade par dos $A_1,\dotsc,A_k$
e logo
$x\notin A_1\symdiff\dotsb\symdiff A_k$ (pela H.I.). 
Ou seja,
$x \in (A_1 \symdiff A_2 \symdiff \dotsb \symdiff A_k) \symdiff A_{k+1}$.
\endgraf
\casestyle{Caso 2:} $x \notin A_{k+1}$.\CR
Nesse caso $x$ pertence numa quantidade ímpar dos $A_1,\dotsc,A_k$
e pela H.I.~temos que $x\in A_1\symdiff\dotsb\symdiff A_k$.
De novo,
$x \in (A_1 \symdiff A_2 \symdiff \dotsb \symdiff A_k) \symdiff A_{k+1}$.

\endproblem
%%}}}

%%{{{ arbitrary_finite_symdiff_gen 
\problem.
\label{arbitrary_finite_symdiff_gen}%
O que devemos mudar (e como) no~\ref{arbitrary_finite_symdiff} e sua resolução,
se apagar o ``$n \geq 2$''?

\hint
O que significa $A_1\symdiff\dotsb\symdiff A_n$ nesse caso?
Por quê?

\solution
Precisamos verificar que a expressão $A_1\symdiff\dotsb\symdiff A_n$ faz
sentido no caso que $n=0$, ou seja, definir razoavelmente a diferença simétrica
de uma seqüência vazia de conjuntos.
Formalmente verificamos que
$\emptyset\symdiff C = C = C\symdiff\emptyset$ para qualquer conjunto $C$:
$\emptyset$ é o elemento neutro da operação $\symdiff$, e logo o valor próprio
da expressão em cima.
\endgraf
Na prova, a base muda para $n=0$, onde devemos apenas provar que nenhum $a$ pertence numa quantidade ímpar dos (zero) $A_i$'s, que é óbvio.

\endproblem
%%}}}

\endproblems
%%}}}

%%{{{ Further reading 
\further.

Podes achar boas explicações, dicas, muitos exemplos resovidos,
e exercícios e problemas para resolver no \cite[\S1.3~\&~\S2.4]{velleman}.

\endfurther
%%}}}

\endchapter
%%}}}

%%{{{ chapter: Functions 
\chapter Funções.
\label{Functions}%

%%{{{ intro 
\chapintro
Nesse capítulo estudamos mais um \emph{tipo} importantíssimo em
matemática: a \emph{função}.
Nosso objetivo aqui é familiarizar-nos com funções,
entender como podemos as definir, as usar, as combinar para criar novas,
operar nelas, etc.
\endgraf
Bem depois, no~\ref{Axiomatic_set_theory}, vamos nos preocupar com
a questão de \emph{como implementar} esse tipo, \emph{como fundamentar}
esse conceito---mas esqueça isso por enquanto.
Primeiramente precisamos entender bem o que é uma função e como se comporta.
%%}}}

%%{{{ Concept, notation, equality 
\section Conceito, notação, igualdade.

\TODO Conceito.

\TODO Examples.

%%{{{ df: function_pseudodefinition 
\pseudodefinition Função.
\label{function_pseudodefinition}%
\iisee{domínio}{função, domínio}%
\iisee{codomínio}{função, codomínio}%
\tdefined{função}[definição intuitiva]%
\tdefined{função}[domínio]%
\tdefined{função}[codomínio]%
\tdefined{função}[valor]%
Sejam $A,B$ conjuntos.
Chamamos a $f$ uma \dterm{função de $A$ para $B$}, sse
para todo $x \in A$, o símbolo $f(x)$ é definido e $f(x)\in B$.
O $f(x)$ é o \dterm{valor} da $f$ no $x$.
O \dterm{domínio} da $f$ é o conjunto $A$,
e seu \dterm{codomínio} é o conjunto $B$.
Consideramos então que a função $f$ associa \emph{para todo} elemento $a \in A$,
\emph{exatamente um} elemento $f(a) \in B$.
%%}}}

%%{{{ df: function_notation 
\definition.
\label{function_notation}%
\sdefined {\holed f : \holed A \to \holed B} {$f$ é uma função de $A$ para $B$}%
\sdefined {\holed A \toarrow {\holed f} \holed B} {$f$ é uma função de $A$ para $B$}%
\sdefined {\dom {\holed f}} {o domínio da função $f$}%
\sdefined {\cod {\holed f}} {o codomínio da função $f$}%
Escrevemos
$$
f : A \to B
\qqqquad\text{e sinonimamente}\qqqquad
A \toarrow f B
$$
para dizer que $f$ é uma função de $A$ para $B$.
Definimos também as operações $\dom$ e $\cod$
que retornam o domínio e o codomínio do seu argumento.
Resumindo:
$$
f : A \to B
\quad\defiff\quad
\text{$f$ é uma função}\;\mland\;
\dom f = A\;\mland\;
\cod f = B.
$$
Escrevemos também $f\,x$ em vez de $f(x)$,
e em certos casos (mais raros) até $x\,f$.
%%}}}

%%{{{ f : A -> B  and  A --f--> B  notations 
\note.
Escrevemos ``sejam $f,g : A \to B$'' e entendemos como:
``sejam funções $f$ e $g$ de $A$ para $B$'', e (abusando) se não temos já
declarados os conjuntos $A,B$, a mesma frase entendemos com um implicito
``sejam conjuntos $A,B$ e funções \dots''.
Do mesmo jeito, a frase
$$
\text{<<{Sejam $A \toarrow f B \toarrow g C$.}>>}
$$
pode se equivalente com a:
$$
\text{<<{Sejam conjuntos $A,B,C$, e funções $f:A\to B$ e $g:B\to C$.}>>}
$$
Espero que dá para apreciar a laconicidade dessa notação.
%%}}}

%%{{{ Conditions: functionhood_conditions 
\note Condições.
\label{functionhood_conditions}%
Na~\ref{function_pseudodefinition} aparece a frase
``o símbolo $f(x)$ é definido''.
Com isso entendemos que não existe ambigüidade, ou seja, para uma entrada $x$,
a $f$ não pode ter mais que uma saida.
E graças a outra frase, ``para todo $x\in A$'', sabemos que tem
\emph{exatamente uma} saida.  Esta saida é o que denotamos por $f(x)$.
Resumindo:
\beginil
\item{(1)} Para todo $x\in\dom f$, $f(x)$ é definido (totalidade);
\item{(2)} Exite no máximo um $y\in\cod f$ com $f(x) = y$ (determinicidade).
\endil
\noindent
Quando o (1) acontece dizemos que \emph{a função é definida em todo o seu domínio}.
%%}}}

\note Aridade e tuplas.
\ii{aridade}%
No jeito que ``definimos'' o que é uma função, ela só pode depender
em apenas um objeto, apenas uma entrada.
Isso parece bastante limitante, pois estamos já acostumados
com funções com aridades diferentes.%
\footnote{Lembra-se que \dterm{aridade} é a quantidade de
argumentos-entradas.}

\TODO Função como ``black box''.

%%{{{ Synonyms 
\note Sinônimos.
\tdefined{função!sinônimos}%
\iisee{mapeamento}{função}%
\iisee{mapa}{função}%
\iisee{map}{função}%
\iiseealso{operação}{função}%
\iiseealso{operador}{função}%
\iiseealso{procedimento}{função}%
Lembra que usamos várias palavras como sinônimos de ``conjunto''?
(Quais?)
Pois é, para funções a situação é parecida.
\emph{Dependendo do contexto e da ênfase},
as palavras seguintes podem ser usadas como
sinônimos de ``função'':
mapeamento,
mapa,
map,
operação,
operador,
procedimento,
etc.
%%}}}

%%{{{ df: f_eq_g 
\definition Igualdade.
\label{f_eq_g}%
Sejam $f,g$ funções.
Digamos que $f=g$ sse \emph{quaisquer duas} das afirmações seguintes são válidas:
\beginol
\li $\dom f = \dom g$;
\li para todo $x \in \dom f$, $f(x) = g(x)$;
\li para todo $x \in \dom g$, $f(x) = g(x)$.
\endol
%%}}}

\endsection
%%}}}

%%{{{ Intension vs. Extension 
\section Intensão vs{.}~extensão.

%%{{{ df: function_graph 
\definition.
\label{function_graph}%
\tdefined{função!gráfico}%
\sdefined{\graph{\holed f}} {o gráfico da função $\holed f$}%
Dado função $f : X\to Y$, o \dterm{gráfico da $f$}, é o conjunto
$$
\graph f \defeq \setst {(x,f(x))} {x \in X}.
$$
%%}}}

\endsection
%%}}}

%%{{{ How to define and how not to define functions 
\section Como definir e como não definir funções.

\question.
\emph{O que precisamos fazer para definir corretamente uma função?}

\TODO Definição por expressão.

%%{{{ eg 
\example.
Seja $f : \reals \to \reals$ definida pela
$$
f(x) = x^2 + x + 1, \qquad \text{para todo $x\in \reals$}.
$$
\endexample
%%}}}

\TODO Descrição definitiva.

\TODO riota.

%%{{{ eg: mother_and_child_function_and_wannabe 
\example.
\label{mother_and_child_function_and_wannabe}%
Seja $\cal P$ o conjunto de todas as pessoas.
\emph{Queremos} definir as funções $m,c : \cal P \to \cal P$ pelas equações
$$
\align
m(p) &= \text{a mãe de $p$}\\
c(p) &= \text{a prole de $p$}
\endalign
$$
(onde ``mãe'' significa ``mãe biológica'').
Mas\dots
\endexample
%%}}}

%%{{{ x: mother_and_child_function_and_wannabe 
\exercise.
Ache o problema no~\ref{mother_and_child_function_and_wannabe} em cima.

\hint
Lembe-se as condições no~\refn{functionhood_conditions}.

\endexercise
%%}}}

\note Definição por casos (branching).
As vezes os valores $f(x)$ duma função $f$ não seguem o mesmo ``padrão'',
a mesma ``regra'' para todos os $x\in\dom f$.

%%{{{ eg: branching_example_1 
\example.
\label{branching_example_1}%
Seja $f : \reals \to \reals$ definida pela
$$
f(x)
=
\knuthcases{
x^2,     & se $x\in\rats$;\cr
0,       & se $x = \sqrt p$ para algum primo $p$;\cr
2x + 1,  & caso contrário.
}
$$
\endexample
%%}}}

%%{{{ beware: function_definition_by_cases_mistakes 
\beware.
\label{function_definition_by_cases_mistakes}%
Cada vez que definimos uma função por casos, precisamos verificar que:
\beginol
\li contamos para todos os casos possíveis da entrada;
\li não existe sobreposição inconsistente em nossos casos.
\endol
Seguem uns exemplos que demonstram esses erros.
%%}}}

%%{{{ eg 
\example.
Definimos a função $f : \nats \to \nats$, pela
$$
f(n) = \knuthcases{
0, & se $n$ pode ser escrito como $3k$ para algum $k\in\nats$;\cr
k, & se $n$ pode ser escrito como $3k+1$ para algum $k\in\nats$.\cr
}
$$
\endexample
%%}}}

%%{{{ eg 
\example.
Definimos a função $g : \nats \to \nats$, pela
$$
g(n) = \knuthcases{
0,  & se $n$ é primo;\cr
1,  & se $n$ é par;\cr
12, & caso contrário.
}
$$
\endexample
%%}}}

%%{{{ By extension 
\note Por extensão.
Outro jeito para definir uma função $f : A \to B$, é
determinar completamente quando é que $f(x) = v$,
para todo $x\in A$ e $v \in B$:
$$
f(x) = v \defiff \tunderbrace {\phi(x,v)} {function-like}.
$$
Onde a fórmula (ou a afirmação) $\phi(x,v)$ deve ser \dterm{function-like}
no $A$, ou seja, $\phi$ é tal que para todo $x\in A$, exatamente um $v \in B$
satisfaz a $\phi(x,v)$.
\footnote{Pode olhar também na definição formal disso,~\refn{functionlike}.}
%%}}}

\TODO Explicar e comparar.

\TODO $f(x,y)$~vs.~$f(\tup{x,y})$.

\blah.
Deixamos a notação mais interessante para definir funções para depois.
Ela merece uma secção própria (\refn{Lambda_notation})!

\endsection
%%}}}

%%{{{ Recursive definitions 
\section Definições recursivas.

\TODO Definições recursivas.

\TODO Recursão mutual.

\endsection
%%}}}

%%{{{ Internal and external diagrams 
\section Diagramas internos e externos.

\TODO Mostre internos e externos diagramas.

%%{{{ df: endomapping 
\definition Endomapa.
\label{endomapping}%
\tdefined{endomapa}%
Uma função $f$ é um \dterm{endomapa} no $A$ sse $\dom f = \cod f = A$.
%%}}}

\TODO Mostre internos diagramas especiais de endomapas.

\endsection
%%}}}

%%{{{ Composition 
\section Composição.

%%{{{ df: fcompose 
\definition.
\label{fcompose}%
\tdefined{função}[composição]%
\iisee{composição!de funções}{função, composição}%
Sejam
$A \toarrow f B \toarrow g C$.
Definimos a função $g\of f : A\to C$ pela
$$
\paren{g\of f}(x) = g\paren{f(x)}.
$$
Chamamos a $g\of f$ a \dterm{composição} da $g$ \emph{seguindo} a $f$
(ou ``da $g$ com $f$'', ou ``da $g$ de $f$'').
%%}}}

%%{{{ beware: gof_not_fog 
\beware.
\label{gof_not_fog}%
Escrevemos $g\of f$ e não $f\of g$ para a composição na~\ref{fcompose}!
Então quando temos $A\toarrow f B\toarrow g C$, a composição é
$g\of f : A \to C$, que parece o oposto da ordem das setinhas.
Definimos a notação alternativa
$$
f;g = g\of f,
$$
mas vamos principalmente usar a $g\of f$ mesmo.
%%}}}

%%{{{ x: when_is_fog_defined 
\exercise.
\label{when_is_fog_defined}%
Sejam $A \toarrow f B \toarrow g C$.
Qual função é a $f \of g$?

\solution
Nem é definida a $f \of g$ no caso geral!
Para ser definida, é necessário e suficiente ter
$A = C$.

\endexercise
%%}}}

\TODO Mais um jeito de definir função: seja $f = g \of h$.

\TODO pointless vs pointfull.

\endsection
%%}}}

%%{{{ Constructions and operations on functions 
\section Construções e operações em funções.

\TODO Identidade.

%%{{{ x: id_compose_practice
\exercise.
\label{id_compose_practice}%
Sejam $A,B$ conjuntos diferentes, e $f : A \to B$.
Para cada uma das igualdades em baixo,
decida se ela é válida ou não, justificando tua resposta.
$$
\xalignat4
(1)\quad f &= {f \compose \id_A} &
(2)\quad f &= {f \compose \id_B} &
(3)\quad f &= {\id_A \compose f} &
(4)\quad f &= {\id_B \compose f}.
\endxalignat
$$

\solution
\noindent (1)
Válida: a composição é definida, e se $a\in A$ então:
$$
\alignat2
(f \compose \id_A)(a)
&= f(\id_A(a))        \qqby{def.~$\compose$}\\
&= f(a)               \qqby{def.~$\id_A$}
\endalignat
$$
\endgraf
\noindent (2) e (3): as composições não são definidas
\endgraf
\noindent (4) Similar com (1): a composição é definida e se $a \in A$ então:
$$
\alignat2
(\id_B \compose f)(a)
&= \mathrm{id}_B(f(a)) \qqby{def.~$\compose$}\\
&= f(a)                \qqby{def.~$\id_B$}
\endalignat
$$

\endexercise
%%}}}

\TODO Caraterística.

\TODO Constante.

%%{{{ x: constant_implies_idempotent 
\exercise.
\label{constant_implies_idempotent}%
Seja $f : A \to A$.
Prove ou refuta a implicação:
$$
\text{$f$ constante}
\implies
\text{$f$ idempotente}
$$

\solution
Vamos provar a implicação.
Suponha que $f:A\to A$ é constante.
Seja $c\in A$ tal que para todo $x\in A$, $f(x) = c$\fact1.
Queremos mostrar que $f\of f = f$, ou seja, que para todo $a\in A$,
$$(f\of f)(a) = f(a)$$
(definição de igualdade de funções~\refn{f_eq_g}).
Calculamos no lado esquerdo:
$$
\alignat2
(f\of f)(a)
&= f\paren{f(a)}    \qqby{def.~$\fcom$}\\
&= f(c)             \qqby{pelo \byfact1, com $x \asseq a$}\\
&= c                \qqby{pelo \byfact1, com $x \asseq c$}
\intertext{e no lado direito:}
f(a)
&= c                \qqby{pelo \byfact1, com $x \asseq a$}.
\intertext{Logo, $f \of f = f$ como desejamos.
\endgraf
Alternativamente, podemos nos livrar dum passo no calculo do lado esquerdo assim:
}
(f\of f)(a)
&= f\paren{f(a)}    \qqby{def.~$\fcom$}\\
&= c.               \qqby{pelo \byfact1, com $x \asseq f(a)$}
\endalignat
$$

\endexercise
%%}}}

%%{{{ x: gof_constant_does_not_imply 
\exercise.
\label{gof_constant_does_not_imply}%
Prove ou refuta a afirmação seguinte:
\emph{se $(g\compose f)$ é constante, então pelo menos uma das $f,g$ também é}.

\hint
Tente achar contraexemplo desenhando diagramas internos.

\solution
Falso.
Um contraexemplo é o seguinte:
$$
\tikzpicture
\draw (0,0) ellipse (1cm and 2cm);
\draw (3,0) ellipse (1cm and 2cm);
\draw (6,0) ellipse (1cm and 2cm);
\draw (0,0)  node {$\bullet$};
\draw (0,-1) node {$\bullet$};
\draw (3,1)  node {$\bullet$};
\draw (3,0)  node {$\bullet$};
\draw (3,-1) node {$\bullet$};
\draw (6,1)  node {$\bullet$};
\draw (6,-1) node {$\bullet$};
\draw[|->] (0.2,0) -- (2.8,-0.9);
\draw[|->] (0.2,-.9) -- (2.8,0);
\draw[|->] (3.2,0) -- (5.8,-0.9);
\draw[|->] (3.2,-1) -- (5.8,-1);
\draw[|->] (3.2,1) -- (5.8,1);
\draw[->]  (0.5,2.5) -- (2.5,2.5);
\draw[->]  (3.5,2.5) -- (5.5,2.5);
\draw (0,2.5) node {$A$};
\draw (3,2.5) node {$B$};
\draw (6,2.5) node {$C$};
\draw (1.5,2.20) node {$f$};
\draw (4.5,2.20) node {$g$};
\endtikzpicture
$$

\endexercise
%%}}}

\TODO Iteração de função: $f^n$ (idéia e definição recursiva).

\TODO Aplicação parcial e ``buracos'': com $-$, $\cdot$, etc.; com lambdas ($\lambda$).

\TODO Aplicação parcial como black box.

\TODO Restrição: definição e propriedades, notações alternativas.

\TODO União: hipoteses necessárias para definir.

\TODO Embedding.

\endsection
%%}}}

%%{{{ Empty (co)domains 
\section (Co)domínios vazios.

%%{{{ x: f_from_A_to_emptyset 
\exercise.
\label{f_from_A_to_emptyset}%
Seja $f : A \to \emptyset$.
O que podemos concluir sobre o $A$?
Quantas funções têm esse tipo?

\hint
O que significa ser função?

\solution
$A=\emptyset$, pois, caso contrario, pegando um $a\in A$,
chegamos na contradição $f(a) \in \emptyset$.
Quantas funções $f$ têm esse tipo?
Vamos resolver isso logo: veja~\ref{emptyfun} e~\ref{uniqueness_of_emptyfun}.

\endexercise
%%}}}

%%{{{ x: f_from_emptyset_to_A 
\exercise.
\label{f_from_emptyset_to_A}%
Seja $f : \emptyset \to A$.
O que podemos concluir sobre o $A$?
Quantas funções têm esse tipo?

\endexercise
%%}}}

%%{{{ df: emptyfun 
\definition Função vazia.
\label{emptyfun}%
\tdefined{função}[vazia]%
\iisee{vazia!função}{função, vazia}%
Uma função $f$ com $\dom f = \emptyset$ é chamada \dterm{função vazia}.
%%}}}

%%{{{ x: uniqueness_of_emptyfun 
\exercise.
\label{uniqueness_of_emptyfun}%
<<Uma>> ou <<a>>?

\solution
<<A>> mesmo!
Tome $f,g$ funções vazias.
Logo $f : \emptyset \to A$ e $g : \emptyset \to B$ para alguns conjuntos $A,B$.
Vacuamente temos que para todo $x\in \emptyset$, $f(x) = g(x)$.

\endexercise
%%}}}

\endsection

%%}}}

%%{{{ Composition laws 
\section Leis de composição.

\TODO Intuição com tarefas: comutatividade não, associatividade sim.

%%{{{ thm: associativity_of_fcom 
\theorem.
\label{associativity_of_fcom}%
Sejam
$$
A \toarrow f B \toarrow g C \toarrow h D.
$$
Então
$$
h \of (g \of f) = (h \of g) \of f.
$$
\sketch.
Mostramos que as duas funções são iguais seguindo a definição
de igualdade~\refn{f_eq_g}:
mostrando que elas tem o mesmo domínio $A$,
e que comportam igualmente para o arbitrário $a \in A$.
Pegamos cada lado da igualdade e aplicando a definição de $\of$
chegamos no mesmo valor.
\qes
\proof.
Primeiramente vamos verificar que as duas funções tem o mesmo domínio:
$$
\alignat2
\dom\paren{ h \of (g \of f) }
&= \dom (g \of f)               \qqby {def.~$\of$} \\
&= \dom f                       \qqby {def.~$\of$} \\
&= A                            \qqby {hipótese}
\intertext{e do lado direito}
\dom\paren{ (h \of g) \of f }
&= \dom f                       \qqby {def.~$\of$} \\
&= A                            \qqby {hipótese}
\endalignat
$$
Agora precisamos mostrar que as duas funções comportam no mesmo jeito
para todos os elementos no $A$.
Suponha $a\in A$ então, e calcule:
$$
\alignat2
\paren{h \of (g \of f)}(a)
&= h\paren{\paren{g \of f}(a)}  \qqby{def.~$\of$}\\
&= h\paren{g \paren{f(a)}}      \qqby{def.~$\of$}
\intertext{e o lado direito:}
\paren{(h \of g) \of f}(a)
&= \paren{h \of g}\paren{f(a)}  \qqby{def.~$\of$}\\
&= h\paren{g \paren{f(a)}}      \qqby{def.~$\of$}
\endalignat
$$
\qed
%%}}}

%%{{{ x: non_commutativity_of_fcom 
\exercise.
\label{non_commutativity_of_fcom}%
Existem funções $f,g$ tais que $f\of g$ e $g\of f$ são ambas definidas,
mas mesmo assim
$$
f\of g \neq g\of f.
$$

\endexercise
%%}}}

\TODO Composição como black box.

\TODO Composição como operação: comparação com multiplicação nos inteiros.

%%{{{ Q: what_is_the_identity_of_composition 
\question.
\label{what_is_the_identity_of_composition}%
Chamamos o $1$ a \dterm{identidade} da operação $\ntimes$ nos reais, pois:
$$
1\ntimes x = x = x\ntimes 1, \qquad \text{para todo $x\in \reals$}
$$
Tentando achar uma similaridade entre funções e números num lado, e composição e multiplicação no outro,
qual seria nosso $1$ aqui?
Ou seja,
procuramos objeto \holed?\ tal que
$$
\holed? \of f = f.
$$
para toda função $f : A \to B$.
%%}}}
\spoiler.

%%{{{ df: idempotent_function 
\definition Idempotente.
\label{idempotent_function}%
\tdefined{função}[idempotente]%
\iisee{idempotente!função}{função idempotente}%
Seja $f : A \to A$ um endomapping.
Chamamos a $f$ \dterm{idempotente} sse
$$
f \of f = f.
$$
%%}}}

%%{{{ x: how_many_idempotents_on_AtoA_for_A_triset 
\exercise.
\label{how_many_idempotents_on_AtoA_for_A_triset}%
Seja $A$ conjunto com $\card A = 3$.
Quantas funções idempotentes podemos definir no $A$?

\hint
Use um diagrama interno para endomapas.

\hint
Observe que em geral, aparecem certos ``redemoinhos''
num diagrama interno de endomapa quando ela é idempotente.

\hint
Separe as funções idempotêntes dependendo na quantidade
de ``redemoinhos'' que aparecem nos seus diagramas.

\endexercise
%%}}}

\endsection
%%}}}

%%{{{ Commutative diagrams 
\section Diagramas comutativos.

\TODO Como expressar o lei da identidade usando apenas a comutatividade de um diagrama.

\endsection
%%}}}

%%{{{ λ-notation and the mapsto arrow 
\section A notação $\lambda$ e a setinha barrada $\mapsto$.
\label{Lambda_notation}%

\TODO De palavras para a notação lambda.

\TODO Função anônima.

\TODO Convenções.

%%{{{ x: eta_conversion_first_encounter 
\exercise.
\label{eta_conversion_first_encounter}%
Sejam $f : A \to B$.
Qual nome tu daria para a função $\lam x {f\, x}$?

\solution
$f$.

\endexercise
%%}}}

\endsection
%%}}}

%%{{{ Function spaces 
\section Espaços de funções.

%%{{{ df: function_space 
\definition.
\tdefined{espaço de funções}%
\label{function_space}%
Sejam $A,B$ conjuntos.
O \dterm{espaço de funções} de $A$ para $B$ é o conjunto de todas as funções de
$A$ para $B$.  Denotamos-lo por $(A \to B)$ ou por $B^A$:
$$
(A \to B) \defeq B^A \defeq \setst f {f: A \to B}.
$$
%%}}}

%%{{{ df: pointwise_operation 
\definition Operação pointwise.
\label{pointwise_operation}%
\tdefined{pointwise!operação}%
\iisee{função!pointwise operação}{pointwise, operação}%
Sejam $A,B$ conjuntos e $\ast$ uma operação binária no $B$.
Dadas funções $f,g : A \to B$, definimos a função
$f \ast g : A \to B$
pela
$$
(f \ast g)(x) = f(x) \ast g(x) , \qquad \text{para todo $x\in A$}.
$$
Chamamos a $(\dhole \ast \dhole)$ a \dterm{operação pointwise}
da $\ast$ no $(A \to B)$.
%%}}}

\endsection
%%}}}

%%{{{ Higher-order functions 
\section Funções de ordem superior.

%%{{{ Holes again 
\note Buracos de novo.
Já encontramos a idéia de \emph{abstrair} certas partes de uma expressão,
botando \emph{buracos}, criando assim funções de várias aridades.
Além de buracos, trabalhamos com \emph{$\lambda$-abstraição} que nos
permitiu dar nomes para esses buracos, identificar certos buracos, etc.
Considere novamente uma expressão como a:
$$
\cos(1 + 5\cbrt2)^{2a} + \sin(5)
$$
onde $\cos : \reals\to\reals$, $\sin : \reals\to\reals$, e $a\in \reals$.
Botando uns buracos, criamos, por exemplo as funções:
$$
\alignat2
f_1&=\cos(1 + 5\cbrt2)^{\bhole a} + \sin(5)            &&\eqtype \reals   \to \reals\\
f_2&=\cos(\bhole + 5\cbrt{\bhole})^{2a} + \sin(\bhole) &&\eqtype \reals^3 \to \reals\\
f_3&=\cos(\bhole + \bhole\cbrt2)^{2\bhole} + \sin(5)   &&\eqtype \reals^3 \to \reals\\
f_4&=\cos(\bhole + \bhole)^{\bhole} + \sin(5)          &&\eqtype \reals^3 \to \reals\\
f_5&=\bhole + \sin(\bhole)                                &&\eqtype \reals^2 \to \reals\\
f_6&=\lam x {\cos(1 + 5\cbrt2)^{2x} + \sin(5)}         &&\eqtype \reals   \to \reals\\
f_7&=\lam {x,y} {\cos(1 + y\cbrt x)^{2a} + \sin(y)}    &&\eqtype \reals^2 \to \reals.
\endalignat
$$
%%}}}

%%{{{ x: verify_abstractions_of_expression 
\exercise.
\label{verify_abstractions_of_expression}%
Para quais entradas cada uma dessas funções retorna o valor da expressão inicial?

\hint
Respeite as aridades!

\solution
Temos
$$
\align
f_1&(2)\\
f_2&(1,2,5)\\
f_3&(1,5,a)\\
f_4&(1,5,2a)\\
f_5&(\cos(1 + 5\cbrt2)^{2a}, 5)\\
f_6&(a)\\
f_7&(2, 5).
\endalign
$$

\endexercise
%%}}}

%%{{{ Higher-order holes 
\note Buracos de ordem superior.
Em todos os exemplos em cima, botamos os buracos para substituir
apenas termos cujos valores seriam números (reais).
Expressões tanto como as $2$, $1$, $5$, e $a$,
quanto como as $5\cbrt2$, $2a$, e $cos(1 + 5\cbrt2)^{2a}$,
denotam, no final das contas, números reais.
Que tal botar um buraco assim:
$$
\cos(1 + 5\cbrt2)^{2a} + \bhole(5)
$$
O que substituimos aqui?  O próprio $\sin$!  Por que não?
E o que tipo de objetos podemos botar nesse buraco?
Um real, não serve: a expressão
$$
\cos(1 + 5\cbrt2)^{2a} + 7(5)
$$
não faz sentido: ela é \dterm{mal-tipada}.
O $7$ não pode receber um argumento como se fosse uma função, pois não é.
Que tipo de coisa então caiba nesse buraco?
Funções!
E não funções quaisquer, mas precisam ter um tipo compatível,
com a aridade certa, etc.
Esses buracos são ``de ordem superior''.
E com buracos de ordem superior, vêm funções de ordem superior.

%%}}}

%%{{{ pseudodf: higher_order_function 
\pseudodefinition.
\label{higher_order_function}%
\tdefined{função}[de ordem superior]%
Dizemos que uma função $f : A \to B$ é de \dterm{ordem superior}
se ela recebe ou retorna funções.
%%}}}

%%{{{ x: type_higher_order_holed_expressions 
\exercise.
\label{type_higher_order_holed_expressions}%
Escreva os tipos das funções seguintes:
$$
\align
F_1 &= \cos(1 + 5\cbrt2)^{2a} + \bhole(5)\\
F_2 &= \bhole(1 + 5\cbrt2)^{2a} + \sin(5)\\
F_3 &= \bhole(1 + 5\cbrt2)^{2a} + \bhole(5)\\
F_4 &= \cos(\bhole(1,5\cbrt2)^{2a} + \sin(5)\\
F_5 &= \cos(\bhole(1,5\cbrt2)^{2a} + \bhole(\bhole)\\
F_6 &= \lam {r,t,u} {\cos(1 + r(u,\cbrt2))^{r(t,a)} + \sin(u)}
\endalign
$$

\solution
Temos
$$
\alignat2
F_1 &= \cos(1 + 5\cbrt2)^{2a} + \bhole(5)                          &&\eqtype (\reals\to\reals) \to \reals\\
F_2 &= \bhole(1 + 5\cbrt2)^{2a} + \sin(5)                          &&\eqtype (\reals\to\reals) \to \reals\\
F_3 &= \bhole(1 + 5\cbrt2)^{2a} + \bhole(5)                           &&\eqtype \paren{(\reals\to\reals)\times(\reals\to\reals)} \to \reals\\
F_4 &= \cos(\bhole(1,5\cbrt2)^{2a} + \sin(5)                    &&\eqtype (\reals^2\to\reals)\to\reals\\
F_5 &= \cos(\bhole(1,5\cbrt2)^{2a} + \bhole(\bhole)                &&\eqtype \paren{(\reals^2\to\reals)\times(\reals\to\reals)\times\reals}\to\reals\\
F_6 &= \lam {r,t,u} {\cos(1 + r(u,\cbrt2))^{r(t,a)} + \sin(u)}  &&\eqtype \paren{(\reals^2\to\reals)\times\reals\times\reals}\to\reals
\endalignat
$$

\endexercise
%%}}}

%%{{{ Returning functions 
\note Retornando funções.
Até agora encontramos exemplos onde funções recebem como argumentos outras funções,
mas aindo não conhecemos alguma função que \emph{retorna função}.
Ou sera que conhecemos?
Seguem uns exemplos de operadores de ordem superior que você talves já
encontrou e até usou na tua vida.
%%}}}

\TODO Exemplos: composição, derivação, integração, somatório, produtório, etc.

%%{{{ Defining higher-order functions 
\note Definindo funções de ordem superior.
Considere a função $F : \reals \to (\reals \to \reals)$ definda pela
$$
F(w) = \text{aquela função $f : \reals \to \reals$ que quando recebe um $x$, retorna $w + x$}.
$$
%%}}}

%%{{{ Q: what type of thing is F(25)? 
\question.
Que tipo de coisa é o $F(25)$?
%%}}}
\spoiler.

%%{{{ Answer 
\note.
Antes de responder nessa pergunta, vamos responder numa outra, ainda mais específica:
o que \emph{é} o $F(25)$?
Ou seja:
$$
F(25) = \dots?\dots
$$
Não precisamos pensar nada profundo!
Vamos apenas \emph{copiar fielmente} sua definição (no lado depois do ``$=$''), substituindo cada ocorrência de $w$, por $25$:
$$
F(25) = \text{aquela função $f : \reals \to \reals$ que quando recebe um $x\in\reals$, retorna o número $25 + x$}.
$$
Ou seja,
$$
F(25) : \reals\to\reals.
$$
Sendo função, podemos a chamar com um argumento do certo tipo, por exemplo como o $3\in\reals$, e a evaluar:
$$
\paren{F(25)}(3) = 28.
$$
%%}}}

%%{{{ x: where_did_28_come_from 
\exercise.
\label{where_did_28_come_from}%
De onde chegou esse $28$?

\solution
Seguindo a definição da $F(25)$ ela é a função que recebendo um valor (agora tá recebendo o $3$), retorna a soma de $25$ e esse valor:
$25 + 3 = 28$.

\endexercise
%%}}}

%%{{{ With lambdas 
\note Com lambdas.
\ii{função!anônima}%
Observe que na definição de $F(w)$, apareceu a frase ``aquela função $f$\dots''.
Assim baptizamos temporariamente essa função com um nome (``$f$'') à toa:
apenas para referir a ela e a retornar.
Usando a $\lambda$-notação, essa definição fica mais direta, mais elegante,
e (logo) mais legível:
$$
F(w) = \lam x {w + x} \quad \eqtype \reals \to \reals
$$
Observe que no lado direito aparece uma função anônima.
%%}}}

\endsection
%%}}}

%%{{{ Currying 

\section Currificação.

\note Ha-ha! Minha linguagem é melhor que a tua.
Imagine que um amigo definiu uma função $f : \ints^2\to\ints$
trabalhando numa linguagem de programação que não permite
definições de funções de ordem superior.
Queremos escrever um programa equivalente ao programa do nosso amigo
numa outra linguagem que permite sim definir funções de ordem superior
mas não funções de aridade maior que $1$.
Mesmo assim nossas funções podem \emph{chamar} a função $f$ do nosso amigo nos seus corpos.

\question.
Como fazer isso?
\spoiler.

\blah.
Aqui uma resposta, escrita em Python.

\sourcecode currying.py;

\TODO Explicar o que acontece.

\exercise.
Resolva o problema converso:
começando com a função ``currificada'' de ordem superior $F$,
defina sua versão ``decurrificada'' $f$ (de aridade 2).

\solution
Dados a $F : \ints\to(\ints\to\ints)$,
é só definir a $f : \ints^2\to\ints$ pela $f(x,y) = F(x)(y)$.

\endexercise

\endsection
%%}}}

%%{{{ Injections, Surjections, Bijections 
\section Funções injetoras, sobrejetoras, e bijetoras.

\TODO Ideia e sketches.

\question.
Como tu definiria os conceitos de função injetiva e sobrejetiva?
\spoiler.

%%{{{ df: starjective_function 
\definition.
\label{starjective_function}%
\tdefined{função}[injetora]%
\tdefined{função}[sobrejetora]%
\tdefined{função}[bijetora]%
\sdefined {\holed f : \holed A \injto \holed B} {$\holed f : \holed A\to \holed B$ é injetora}%
\sdefined {\holed f : \holed A \surto \holed B} {$\holed f : \holed A\to \holed B$ é sobrejetora}%
\sdefined {\holed f : \holed A \bijto \holed B} {$\holed f : \holed A\to \holed B$ é bijetora}%
Seja $f : A \to B$.
Chamamos a $f$ \dterm{injetora} (ou \dterm{injetiva}) sse
$$
\text{para todo $x,y \in A$, se $x\neq y$ então $f(x) \neq f(y)$}.
$$
Chamamos a $f$ \dterm{sobrejetora} (ou \dterm{sobrejetiva}) sse
$$
\text{para todo $b \in B$, existe $a\in A$ tal que $f(a) = b$}.
$$
Chamamos a $f$ \dterm{bijetora} (ou \dterm{bijetiva}, ou \dterm{correspondência}), sse
$f$ é injetora e sobrejetora.
Usamos as notações
$$
\align
f : A \injto B &\defiff \text{$f:A\to B$ é injetora}\\
f : A \surto B &\defiff \text{$f:A\to B$ é sobrejetora}\\
f : A \bijto B &\defiff \text{$f:A\to B$ é bijetora}
\endalign
$$
%%}}}

\TODO Exemplos e nãœxemplos.

\TODO Quantas funções entre dois conjuntos tais que\dots?

%%{{{ x: fcom_respects_jections 
\exercise Composição respeita ``-jetividade''.
\label{fcom_respects_jections}%
Sejam $f : A\to B$ e $g : B \to C$.
Prove:
\item{(1)} Se $f$ e $g$ são injetoras, então $g\fcom f$ também é;
\item{(2)} Se $f$ e $g$ são sobrejetoras, então $g\fcom f$ também é;
\item{(3)} Se $f$ e $g$ são bijetoras, então $g\fcom f$ também é.

\endexercise
%%}}}

%%{{{ x: converse_to_gof_bij_conclusions 
\exercise.
Se $g\compose f$ é bijetora, o que podemos concluir sobre as $f,g$?  Justifique.

\endexercise
%%}}}

%%{{{ x: converse_to_gof_bij_conclusions 
\exercise.
Se $f$ é injetora e $g$ sobrejetora, a $g\compose f$ é necessariamente bijetora?

\solution
Não.
Aqui um contraexemplo:
$$
\tikzpicture
\draw (0,0) ellipse (-.75cm and 1.25cm);
\draw (3,0) ellipse (-.75cm and 1.25cm);
\draw (6,0) ellipse (-.75cm and 1.25cm);
\draw (0,-.5) node {$\bullet$};
\draw (3,.5)  node {$\bullet$};
\draw (3,-.5) node {$\bullet$};
\draw (6,.5)  node {$\bullet$};
\draw (6,-.5) node {$\bullet$};
\draw[|->] (0.2,-.5) -- (2.8,-.5);
\draw[|->] (3.2,-.5) -- (5.8,-.5);
\draw[|->] (3.2,.5) -- (5.8,.5);
\draw[->]  (0.5,1.5) -- (2.5,1.5);
\draw[->]  (3.5,1.5) -- (5.5,1.5);
\draw (0,1.5) node {$A$};
\draw (3,1.5) node {$B$};
\draw (6,1.5) node {$C$};
\draw (1.5,1.20) node {$f$};
\draw (4.5,1.20) node {$g$};
\endtikzpicture
$$

\endexercise
%%}}}

%%{{{ x: x_mapsto_singleton_x_properties 
\exercise.
Sejam $A\neq\emptyset$ um conjunto e $f : A \to \pset A$ definida pela equação
$$
f(a) = \set a.
$$
\beginol
\li A $f$ é injetora?
\li A $f$ é sobrejetora?
\endol

\solution
\noindent (1) Sim: se $x,y\in A$, então
$$
x\neq y
    \implies \set x \neq \set y
    \implies f(x) \neq f(y).
$$
\endgraf
\noindent (2) Não: para todo $a\in A$ temos
$f(a) = \set a \neq \emptyset \in \pset A$.

\endexercise
%%}}}

%%{{{ df: finverse 
\definition Função inversa.
\label{finverse}%
\tdefined{função}[inversa]%
\sdefined {\finv {\holed f}} {a função inversa da $f$}%
Seja função bijetora $f : A \bijto B$.
Definimos a função $\finv f : B \to A$ pela
$$
\finv f (y) \defeq \text{aquele $x\in A$ tal que $f(x) = y$}.
$$
Chamamos a $\finv f$ a \dterm{função inversa} da $f$.
%%}}}

%%{{{ x: justify_finverse 
\exercise.
\label{justify_finverse}%
Justifique a definição em cima.
O que precisas provar?

\solution
Pelo menos um tal $x\in A$ existe, pois a $f$ é sobrejetora.
E como $f$ é injetora, existe no máximo um.
Provamos assim a unicidade, algo que nos permite
\emph{definir a função} no jeito que definimos
na~\ref{finverse}.

\endexercise
%%}}}

%%{{{ x: finv_is_bij 
\exercise.
\label{finv_is_bij}%
Prove que quando a função inversa $\finv f$ é definida,
ela é bijetiva.

\endexercise
%%}}}

%%{{{ x: double_or_reverse_string_inj_or_surj 
\exercise.
\label{double_or_reverse_string_inj_or_surj}%
Seja~$S$ o conjunto de todos os strings \emph{não vazios}
dum alfabeto $\Sigma$, com $\card{\Sigma} \geq 2$.
Considere a função
$f : S \times \set{0,1} \to S$ definida pela:
$$
f(w,i) =
\knuthcases{
ww, &se $i = 0$\cr
w', &se $i = 1$
}
$$
onde $w'$ é o string reverso de $w$,
e onde denotamos a concatenação de strings por juxtaposição.
\beginil
\item{(i)} A $f$ é injetora?
\item{(ii)} A $f$ é sobrejetora?
\endil

\hint
Quais são o domínio e o codomínio da $f$?

\solution
(i) Não, $f$ não é injetora.
Tome uma letra do alfabeto $a\in\Sigma$ e observe que
$$
f(a, 0) = aa = f(aa,1).
$$
Como $(a,0) \neq (aa,1)$, a $f$ não é injetora.
\endgraf
(ii)
Sim, $f$ é sobrejetora.
Tome um aletorio string $w\in S$, e seja $w'$ o string reverso de $w$.
Temos
$$
f(w',1) = (w')' = w.
$$

\endexercise
%%}}}

\endsection
%%}}}

%%{{{ Images and pre-images 
\section Imagens e pre-imagens.
\label{Images_and_preimages}%

%%{{{ Two important subsets 
\note Dois subconjuntos importantes.
Sejam $A,B$ conjuntos e $f:A \to B$.
$$
\tikzpicture[node distance=0mm]
\tikzi imgpreimgbase;
\endtikzpicture
$$
Vamos associar, com qualquer subconjunto $X$ de $A$, um certo subconjunto de $B$, que vamos o chamar a \emph{imagem} de $X$ através da $f$.
Similarmente, com qualquer subconjunto $Y$ de $B$, vamos associar um certo subconjunto de $A$, que vamos o chamar a \emph{preimagem} de $Y$ através da $f$.
Bem informalmente, parece que estamos ``elevando'' a função $f$ do nível ``membros'' para para o nível ``subconjuntos''.
Vamos ver como.
%%}}}

%%{{{ eg: sketches of image and preimage 
\example.
Nas figuras seguintes mostramos com azul o subconjunto com que começamos,
e com vermelho o subconjunto que associamos com ele.
Considere o $X\subseteq A$ como aparece no desenho em baixo,
e veja qual é o subconjunto $\img f X$ de $B$ que vamos associar com o $X$:
$$
\gathered
\tikzpicture[scale=0.75,node distance=0mm]
\draw [fill=blue!25] (0,0) ellipse (0.9cm and 1.6cm);
\tikzi imgpreimgbase;
\node [color=blue] (subset-X) at (-0.7,-1.6) {$X$};
\endtikzpicture
\endgathered
\quad
\leadsto
\quad
\gathered
\tikzpicture[scale=0.75,node distance=0mm]
\draw [fill=blue!25] (0,0) ellipse (0.9cm and 1.6cm);
\draw [fill=red!25] (4.9,-0.45) ellipse (0.7cm and 2cm);
\tikzi imgpreimgbase;
\node [color=blue] (subset-X) at (-0.7,-1.6) {$X$};
\node [color=red] (subset-fX) at (5.9,-2.05) {$\img f X$};
\endtikzpicture
\endgathered
$$
Na direção oposta, comece por exemplo com esse $Y\subseteq B$
e observe qual é o subconjunto $\pre f Y$ de $A$ que queremos associar com o $Y$:
$$
\gathered
\tikzpicture[scale=0.75,node distance=0mm]
\draw [fill=blue!25] (4.9,1.45) ellipse (1.1cm and 1.2cm);
\tikzi imgpreimgbase;
\node [color=blue] (subset-Y) at (5.95,0.4) {$Y$};
\endtikzpicture
\endgathered
\quad
\leadsto
\quad
\gathered
\tikzpicture[scale=0.75,node distance=0mm]
\draw [fill=blue!25] (4.9,1.45) ellipse (1.1cm and 1.2cm);
\draw [fill=red!25]  (0,1.5) ellipse (0.8cm and 2.0cm);
\tikzi imgpreimgbase;
\node [color=blue] (subset-Y)     at (5.95,0.4)  {$Y$};
\node [color=red]  (subset-fpreY) at (-0.9,-0.6) {$\pre f Y$};
\endtikzpicture
\endgathered
$$
\endexample
%%}}}

\note.
Observe que na discussão em cima não supusemos \emph{nada mais} além da
existência de uma função $f$ dum conjunto $A$ para um conjunto $B$.

\question.
Como podemos definir formalmente os conjuntos indicados nos desenhos em cima?
\spoiler.

%%{{{ df: img_and_pre 
\definition.
\label{img_and_pre}%
\tdefined{função!imagem}%
\tdefined{função!preimagem}%
\sdefined {\img {\holed f} {\holed X}} {a imagem de $X$ através da $f$}%
\sdefined {\pre {\holed f} {\holed Y}} {a preimagem de $Y$ através da $f$}%
Seja $f : A \to B$, e sejam subconjuntos $X\subseteq A$ e $Y\subseteq B$.
Definimos:
$$
\align
\img f X &\defeq \setst {f(x)} {x \in X}\\
\pre f Y &\defeq \setst {a \in A} {f(a) \in Y}
\endalign
$$
%%}}}

%%{{{ x: type_of_img_f_hole 
\exercise.
\label{type_of_img_f_hole}%
Qual o tipo da $\img f {\dhole}$?

\endexercise
%%}}}

%%{{{ x: type_of_pre_f_hole 
\exercise.
\label{type_of_pre_f_hole}%
Qual o tipo da $\pre f {\dhole}$?

\endexercise
%%}}}

%%{{{ x: erroneous_definition_of_pre 
\exercise.
\label{erroneous_definition_of_pre}%
Podemos definir a preimagem de $Y$ através de $f$ assim?:
$$
\pre f Y \defeq \setst {\finv f (y)} {y \in Y}
$$

\solution
Não!
O símbolo $\finv f (y)$ nem é definido no caso geral, pois
o definimos \emph{apenas para funções bijetoras}.

\endexercise
%%}}}

%%{{{ x: when_erroneous_definition_of_pre_is_valid 
\exercise.
\label{when_erroneous_definition_of_pre_is_valid}%
Seja $f: A \bijto B$.
Mostre que:
$$
\pre f Y = \setst {\finv f (y)} {y \in Y}.
$$

\endexercise
%%}}}

%%{{{ x: pre_notation_problem 
\exercise.
\label{pre_notation_problem}%
Qual o problema com a~\ref{img_and_pre}?

\hint
Quando $f$ \emph{não} é bijetora, nenhum.

\hint
Se $f$ é bijetora, o símbolo $\pre f Y$ pode ter duas interpretações diferentes.

\solution
Se $f$ é bijetora, o símbolo $\pre f Y$ pode ter duas interpretações diferentes:
$$
\pre f Y \askeq
\knuthcases{
\pre {{\color{alert} f}} Y
&= a preimagem de $Y$ através da função $f:A\to B$\cr
& \cr
\img {{\color{alert} {(\finv f)}}} Y
&= a imagem de $Y$ através da função $\finv f:B\to A$
}
$$
onde usamos cores e parentese para enfatisar o ``parsing'' diferente
de cada interpretação.
Observe que a segunda alternativa não é possível quando $f$ não é bijetora,
pois a função $\finv f$ nem é definida nesse caso!

\endexercise
%%}}}

%%{{{ x: correctness_of_pre_notation 
\exercise.
\label{correctness_of_pre_notation}
Depois de resolver o~\ref{pre_notation_problem}, justifique a corretude
da~\ref{img_and_pre}:
explique o que precisas provar, e prove!

\hint
Precisamos provar que no caso que $f:A\bijto B$ as duas interpretações do
símbolo $\pre f Y$ \emph{denotam o mesmo objeto}.

\hint
Temporariamente mude tua notação para ajudar teus olhos distinguir entre as duas interpretações melhor.
Use, por exemplo, $f_{-1}$ para denotar a função inversa da $f$, ou, alternativamente, bota parenteses
$(f^{-1})$ para enfatisar que estás denotando a função inversa da $f$.

\endexercise
%%}}}

%%{{{ x: composition_with_inverse 
\exercise.
\label{composition_with_inverse}%
Sejam conjuntos $X$ e $Y$, $f : X\to Y$, e $A\subseteq X$ e $B\subseteq Y$.
Compare os conjuntos:
$$
\align
A &\askeq \pre f {\img f A}\\
B &\askeq \img f {\pre f B}.
\endalign
$$

\solution
Mostramos dois contraexemplos, um para cada afirmação.
$$
\tikzpicture
\draw (0,0) ellipse (1cm and 15mm);
\draw (3,0) ellipse (1cm and 12mm);
\draw (0,0.666) ellipse (5mm and 5mm);
\draw (-0.6,0.1) node {$A$};
\draw (0,0.666)  node {$1\,\bullet$};
\draw (0,-0.666) node {$2\,\bullet$};
\draw (3,0)  node {$\bullet\,3$};
\draw[|->] (0.3,0.6) -- (2.7,0.1);
\draw[|->] (0.3,-0.6) -- (2.7,-0.1);
\draw (0,2.0) node {$X$};
\draw (3,2.0) node {$Y$};
\draw (1.5,1.70) node {$f$};
\draw[->]  (0.5,2.0) -- (2.5,2.0);
\endtikzpicture
\qqqquad
\tikzpicture
\draw (0,0) ellipse (1cm and 12mm);
\draw (3,0) ellipse (1cm and 15mm);
\draw (3,0) ellipse (8mm and 10mm);
\draw (2.4,0.1) node {$B$};
\draw (3,0.666)  node {$\bullet\,2$};
\draw (3,-0.666) node {$\bullet\,3$};
\draw (0,0)  node {$1\,\bullet$};
\draw[|->] (0.35,0.05) -- (2.7,0.6);
\draw (0,2.0) node {$X$};
\draw (3,2.0) node {$Y$};
\draw (1.5,1.70) node {$f$};
\draw[->]  (0.5,2.0) -- (2.5,2.0);
\endtikzpicture
$$
No primeiro contraexemplo temos:
$A=\set{1}$;
$\img f A = \set {3}$; e
$\pre f {\set 3} = \set {1,2}$.
Logo
$$
A = \set{ 1 } \neq \set {1, 2} = \pre f {\img f A}.
$$
No segundo contraexemplo temos:
$B = \set{2,3}$;
$\pre f B = \set {1}$; e
$\img f {\set 1} = \set {2}$.
Logo
$$
B = \set{ 2,3 } \neq \set {2} = \img f {\pre f B}.
$$

\endexercise
%%}}}

%%{{{ x: composition_with_inverse_conditions 
\exercise.
\label{composition_with_inverses}%
Caso que alguma das igualdades não é válida em geral, que mais seria necessário e suficiente supor sobre a $f$ para a garantir
cada uma das igualdades do~\ref{composition_with_inverse}?
No caso geral, é alguma das duas inclusões ({\lrdirset} ou {\rldirset}) de cada $\askeq$ válida?

\endexercise
%%}}}

%%{{{ x: img_and_pre_of_emptyset 
\exercise.
\label{img_and_pre_of_emptyset}%
Seja $f : A \to B$.
Calcule os conjuntos
$\img f \emptyset$ e $\pre f \emptyset$.

\endexercise
%%}}}

%%{{{ x: operations_respected_by_img 
\exercise.
\label{operations_respected_by_img}%
Sejam $f : X \to Y$, $A,B\subseteq X$.
Mostre que:
$$
\align
\img f {A\union B} &=       \img f A \union \img f B\\
\img f {A\inter B} &\askeq  \img f A \inter \img f B\\
\img f {A\minus B} &\askeq  \img f A \minus \img f B
\endalign
$$
onde nas $\askeq$ prove que a igualdade em geral não é válida,
mas uma das~\lrdirset~e~\rldirset, é.

\endexercise
%%}}}

%%{{{ x: operations_respected_by_img_of_inj 
\exercise.
\label{operations_respected_by_img_of_inj}%
Sejam $f : X \injto Y$ injetora, e $A,B\subseteq X$.
Mostre que:
$$
\align
\img f {A\inter B} &= \img f A \inter \img f B\\
\img f {A\minus B} &= \img f A \minus \img f B.
\endalign
$$

\endexercise
%%}}}

\blah.
A preimagem comporta bem melhor que a imagem: ela respeita
todas essas operações mesmo quando $f$ não é injetora,
algo que tu provarás agora.

%%{{{ x: operations_respected_by_pre 
\exercise.
\label{operations_respected_by_pre}%
Sejam $f : X \to Y$, $A,B\subseteq Y$.
Mostre que:
$$
\align
\pre f {A\union B} &= \pre f A \union \img f B\\
\pre f {A\inter B} &= \pre f A \inter \img f B\\
\pre f {A\minus B} &= \pre f A \minus \img f B
\endalign
$$

\endexercise
%%}}}

\blah.
Essas propriedades generalizam naturalmente:
veja~\ref{big_operations_respected_by_img_and_pre}.

\endsection
%%}}}

%%{{{ Partial functions 
\section Funções parciais.

\blah.
A restrição que uma função $f : A \to B$ precisa ser \emph{totalmente} definida no $A$,
não nos permite considerer naturalmente situações onde um certo programa, por exemplo,
retorna valores para certas entradas aceitáveis, e para as outras não: talvez
ele fica num \emph{loop infinito}, ele faz a maquina explodir, ou simplesmente
não termina com uma saida---não importa o porquê.
Definimos então o conceito de \emph{funções parciais}, que são exatamente isso:
funções que para certas entradas aceitáveis delas, possivelmente \emph{divergem}.

%%{{{ df: partial_function 
\definition.
\label{partial_function}%
Sejam $A,B$ conjuntos.
Chamamos a $f$ uma \dterm{função parcial} de $A$ para $B$ sse:
$$
\text{para todo $x\in A$, se $f(x)$ é definido então $f(x) \in B$.}
$$
Nesse caso, chamamos \dterm{domínio} o conjunto
$$
\dom f \defeq \setstt {x \in A} {$f(x)$ é definido}
$$
e \dterm{codomínio} o $B$ mesmo.
Escrevemos $f : A \parto B$ para ``$f$ é uma função parcial de $A$ para $B$''.
Naturalmente denotamos o conjunto de todas as funções parciais de $A$ para $B$ por
$$
(A \parto B) \defeq \setst f {f : A \parto B}.
$$
%%}}}

\endsection
%%}}}

%%{{{ Product of family of sets 
\section Produto de família de conjuntos.
\label{Cartesian_product_of_indexed_family_of_sets_revisited}%

\TODO Definição, idéia, exercícios.

\TODO CD of tuples, cartesian product, functions, generalized.

\endsection
%%}}}

%%{{{ Fixpoints 
\section Fixpoints.

%%{{{ df: fixpoint 
\definition Fixpoint.
\label{fixpoint}%
\tdefined{fixpoint}%
\iisee{função!fixpoint}{fixpoint}%
Seja $f : A \to A$ um endomapa num conjunto $A$.
Chamamos \dterm{fixpoint} da $f$ qualquer $x\in A$ tal que
$f(x) = x$.
%%}}}

%%{{{ x: fix_f_properties 
\exercise.
\label{fix_f_properties}%
Seja $f : A \to A$,
e seja $F$ o conjunto de todos os fixpoints da $f$.
$$
F = \setstt {x \in A} {$x$ é um fixpoint da $f$}
$$
Decida quais das igualdades seguintes são validas:
prove aquelas que são; refuta aquelas que não.
$$
\img f F = F
\qqqquad
\pre f F = F
\qqqquad
f \resto F = \idof F
$$
O que muda se $f$ é injetora?

\endexercise
%%}}}

\endsection
%%}}}

%%{{{ Problems 
\problems.

\TODO Aplicação como operação.

%%{{{ how_many_idempotents_on_AtoA_for_A_finite 
\problem.
\label{how_many_idempotents_on_AtoA_for_A_finite}%
Generalize o~\ref{how_many_idempotents_on_AtoA_for_A_triset} para
um arbitrário conjunto finito $A$.

\hint
Seja $n = \card A$.

\endproblem
%%}}}

%%{{{ type_of_restriction_op_with_hole 
\problem.
\label{type_of_restriction_op_with_hole}%
Sejam $f : A\to B$ e $X\subseteq A$.
Ache o tipo dos:
$$
\align
f\resto X       &\eqtype \dots?\dots\\
f\resto \dhole  &\eqtype \dots?\dots\\
(\dhole\resto X)\resto(A \to B)
                &\eqtype \dots?\dots
\endalign
$$

\solution
Temos
$$
\align
f\resto X       &\eqtype X \to B\\
f\resto \dhole  &\eqtype \pset A \to \Union_{X\in\pset A}(X\to B)\\
(\dhole\resto X)\resto (A \to B)
                &\eqtype \paren{(A \to B) \to (X \to B)}
\endalign
$$

\endproblem
%%}}}

%%{{{ first_contact_with_initial_and_terminal_objects 
\problem Objetos iniciais e terminais.
\label{first_contact_with_initial_and_terminal_objects}%
(1) Quais conjuntos $S$ (se algum) têm a propriedade seguinte?:
$$
\text{Para todo conjunto $A$, existe única função $f:S \to A$.}
$$
(2) Quais conjuntos $T$ (se algum) têm a propriedade seguinte?:
$$
\text{Para todo conjunto $A$, existe única função $f:A \to T$.}
$$

\solution
(1) Apenas o conjunto vazio:
a única função que existe de $\emptyset$ para $A$,
é a função vazia.
Se $S \neq \emptyset$ tome $s\in S$ e considere
o conjunto $A = \set{0,1}$.
Já temos duas funções diferentes $f,g : S \to A$:
basta apenas diferenciar elas no $s$.
Tome por exemplo $f = \lam x 0$ e $g = \lam x 1$.
Como $f(s) = 0 \neq 1 = g(s)$, temos $f\neq g$.
(2) Todos os singletons.
Se $T = \emptyset$, tome $A\neq\emptyset$ e observe
que não existe nenhuma função $f : A \to T$.
E se $\card T > 1$, tome $u,v\in T$ com $u\neq v$
e considere o $A = \set 0$.
Já temos duas funções $f,g : A \set T$:
a $f = \lam x u$ e a $g = \lam x v$.
Elas são realmente distintas,
pois $f(0) = u \neq v = g(0)$, e logo $f\neq g$.

\endproblem
%%}}}

%%{{{ big_operations_respected_by_img_and_pre 
\problem.
\label{big_operations_respected_by_img_and_pre}%
Sejam $f : A \to B$, e duas famílias de conjuntos indexadas:
$\famil A i {\cal I}$ feita por subconjuntos de $A$, e
$\famil B j {\cal J}$ feita por subconjuntos de $B$.
Ou seja, para todo $i\in \cal I$, e todo $j \in \cal J$,
temos $A_i \subseteq A$ e $B_j \subseteq B$.
Mostre que:
$$
\align
\img f {\Unionl_{i \in \cal I} A_i} &=      \Unionl_{i \in \cal I} \img f {A_i}\\
\img f {\Interl_{i \in \cal I} A_i} &\askeq \Interl_{i \in \cal I} \img f {A_i}\\
\pre f {\Interl_{j \in \cal J} B_j} &=      \Unionl_{j \in \cal J} \pre f {B_j}\\
\pre f {\Interl_{j \in \cal J} B_j} &=      \Interl_{j \in \cal J} \pre f {B_j}
\endalign
$$
onde na $\askeq$ prove que a igualdade em geral não é válida,
mas uma das~\lrdirset~e~\rldirset, é, e supondo que $f$ é injetora,
prove a outra também.

\endproblem
%%}}}

\endproblems
%%}}}

%%{{{ Further reading 
\further.

O \cite[Cap.~5]{velleman} defina e trata funções
como casos especiais de relações (veja~\ref{Relations}),
algo que não fazemos nesse texto.
Muitos livros seguem essa abordagem, então o leitor é conselhado
tomar o cuidado necessário enquanto estudando esses assuntos.

Um livro excelente para auto-estudo é o~\cite[Par.~I~\&~II]{babylawvere}.
\emph{Não pule seus exercícios e problemas!}

\endfurther
%%}}}

\endchapter
%%}}}

%%{{{ chapter: Relations 
\chapter Relações.
\label{Relations}%

%%{{{ Concept, notation, equality 
\section Conceito, notação, igualdade.

\note.
Se pensamos em funções como construtores (ou ``apontadores'') de objetos,
então as relações são \emph{construtores de afirmações sobre objetos}.
Podemos pensar que uma relação é como um \emph{verbo}, ou um \emph{predicado}
duma afirmação.

%%{{{ eg 
\example.
Nos numeros reais, estamos bem acostumados com as relações de ordem
($\leq$, $\geq$, $<$, $>$).
Nos inteiros já estudamos bastante a relação de ``divide'' $\divides$.
\endexample
%%}}}

%%{{{ eg 
\example.
No conjunto $\cal P$ de pessoas, conhecemos várias relações também.
Por exemplo
$$
\align
\namedpred{Mother}(x,y)     &\defiff \text{$x$ é a mãe de $y$}\\
\namedpred{Parents}(x,y,z)  &\defiff \text{$x$ e $y$ são os pais de $z$}\\
\namedpred{Love}(x,y)       &\defiff \text{$x$ ama $y$}
\endalign
$$
\endexample
%%}}}

\blah.
As relações podem ``relacionar'' tipos diferentes.

%%{{{ eg 
\example.
Considere as relações seguintes, cujos argumentos não têm o mesmo típo.
$$
\align
\namedpred{Born}(x,w)       &\defiff \text{$x$ nasceu no ano $w$}\\
\namedpred{Author}(x,k)     &\defiff \text{$x$ escreveu o livro $k$}
\endalign
$$
O primeiro argumento da primeira relação é uma pessoa
mas o segundo é um ano; e a relação $\namedpred{Author}$
é entre pessoas e livros.
\endexample
%%}}}

%%{{{ eg 
\example.
Para cada tipo, sua igualdade $=$ é uma relação, de aridade $2$.
Nos números naturais por exemplo, se $n,m\in\nats$, $n = m$ é uma afirmação:
$$
\align
n = m &\pseudodefiff \text{os $n$ e $m$ denotam o mesmo número natural}.
\intertext{Similarmente nos conjuntos: se $A,B$ são conjuntos, $A = B$ é a afimação seguinte:}
A = B &\pseudodefiff \text{os $A$ e $B$ denotam o mesmo conjunto}.
\endalign
$$
Etc., etc.
Observe que podemos fazer uma \ii{aplicação!parcial}\emph{aplicação parcial},
nas relações como fazemos nas funções.
Fixando um objeto de nosso tipo, por exemplo o natural $0\in\nats$
em qualquer um dos dois lados da igualdade (vamos fixar na direita nesse exemplo),
chegamos numa relação de aridade $1$:
$$
\bhole = 0
$$
onde aplicando a relação para qualquer $x\in\nats$ chegamos na afirmação
$$
x = 0.
$$
\endexample
%%}}}

%%{{{ Black boxes 
\note Black boxes.
\label{blackbox_rel}%
\tdefined{black box}[de relação]%
\iisee{relação!como black box}{black box}%
Visualizamos uma relação $R$ de aridade $n$ como um black box com $n$ entradas
e uma lâmpada que pisca ou não, dependendo de se os objetos-entradas
$x_1,\dotsc,x_n$ são relacionados pela $R$.
Nesse caso dizemos que os $x_1,\dotsc,x_n$ \dterm{satisfazem} a $R$.
%%}}}

\blah.
Cada vez que introduzimos um tipo novo, precisamos definir quando dois objetos
desse tipo são iguais.  Vamos fazer isso agora.

%%{{{ df: R_eq_S_bin_on_single_set_case 
\definition Igualdade.
\label{R_eq_S_bin_on_single_set_case}%
Sejam $R,S$ relações binárias num conjunto $A$.
Definimos
$$
R=S
\defiff
\text {para todos $x,y \in A$, teemos:\ \ $x \rel R y \iff x \rel S y$}.
$$
%%}}}

\blah.
Principalmente vamos trabalhar com relações binárias definidas num conjunto só,
então a definição de igualdade~\refn{R_eq_S_bin_on_single_set_case} que
acabamos de ver nos serve bem.  No~\ref{R_eq_S_bin_on_different_sets_case}
e no~\ref{R_eq_S} tu vai extender essa definição para os casos mais gerais.

%%{{{ x: R_eq_S_bin_on_different_sets_case 
\exercise Igualdade.
\label{R_eq_S_bin_on_different_sets_case}%
Como tu extenderia a~\ref{R_eq_S_bin_on_single_set_case} para o caso onde as
$R,S$ não são relações num conjunto só?
Ou seja, tendo relações binárias~$R,S$, a~$R$ de~$A$ para~$B$, e a~$S$ de~$C$
para~$D$, como tu definaria a igualdade~$R=S$ nesse caso?

\solution
Digamos que $R=S$ sse para todo $x\in A\union C$, e todo $y \in B\union D$, temos $x \rel R y \iff x \rel S y$.

\endexercise
%%}}}

%%{{{ notation: funlike_notation_for_rels 
\note Notação.
\label{funlike_notation_for_rels}%
Como nas relações também temos a idéia de <<relação \emph{de} \dots\ \emph{para} \dots>>,
vamos criar uma notação parecida com aquala das funções para dizer que $R$ é uma relação do conjunto $A$ para o conjunto $B$.
Escrevemos, equivalentemente:
$$
R : A \relto B
\qqqquad
R : B \relfrom A
\qqqquad
A \reltoarrow R B
\qqqquad
B \relfromarrow R A.
$$
%%}}}

\beware.
A notação definida no~\refn{funlike_notation_for_rels} \emph{não} é padrão.

\endsection
%%}}}

%%{{{ Intension vs. Extension 
\section Intensão vs{.}~extensão.

%%{{{ df: relation_graph 
\definition.
\label{relation_graph}%
\label{truth_set}%
\tdefined{relação!gráfico}%
\tdefined{truth set}%
\iisee{gráfico}{relação, gráfico}%
\iisee{gráfico}{função, gráfico}%
\sdefined{\graph{\holed R}} {o gráfico da relação $\holed R$}%
Dado relação $R$ de $A$ para $B$, o \dterm{gráfico da $R$}
é o conjunto
$$
\graph R \defeq \setst {(x,y)} {x \rel R y},
$$
também conhecido como \dterm{truth set} da $R$.
%%}}}

\endsection
%%}}}

%%{{{ Defining relations 
\section Definindo relações.

\endsection
%%}}}

%%{{{ Internal diagrams 
\section Diagramas internos.

%%{{{ Relation as a generalization of function 
\note Relação como uma generalização de função.
Um jeito de olhar para uma relação é como uma ``função'' sem as restrições
de totalidade e de determinicidade que encontramos no~\refn{functionhood_conditions}.
Então: lembra dos conjuntos $A,B$ que encontramos na~\ref{Images_and_preimages}?
Aqui são duas relações $R,Q$ de $A$ para $B$, determinadas por seus
diagramas internos:
$$
\tikzpicture[node distance=0mm,scale=0.8]
\tikzi imgpreimgpointsetsbase;
\draw[-{Latex[length=6pt,open]}]  (0.5,4.5) -- (4.5,4.5);
\node (arrow-R) at (2.5,4) {$R$};
\draw[->] (elem-b) -- (elem-1);
\draw[->] (elem-b) -- (elem-2);
\draw[->] (elem-c) -- (elem-2);
\draw[->] (elem-e) -- (elem-5);
\draw[->] (elem-e) -- (elem-6);
\draw[->] (elem-e) -- (elem-5);
\draw[->] (elem-e) -- (elem-2);
\endtikzpicture
\qquad
\tikzpicture[node distance=0mm,scale=0.8]
\tikzi imgpreimgpointsetsbase;
\draw[-{Latex[length=6pt,open]}]  (0.5,4.5) -- (4.5,4.5);
\node (arrow-Q) at (2.5,4) {$Q$};
\draw[->] (elem-a) -- (elem-1);
\draw[->] (elem-b) -- (elem-1);
\draw[->] (elem-c) -- (elem-1);
\draw[->] (elem-d) -- (elem-3);
\draw[->] (elem-d) -- (elem-4);
\draw[->] (elem-d) -- (elem-5);
\draw[->] (elem-d) -- (elem-6);
\draw[->] (elem-f) -- (elem-6);
\endtikzpicture
$$
%%}}}

\blah.
A situação sobre relações binárias definidas num conjunto $A$
é mais divertida.

%%{{{ Relation as a directed graph 
\note Relação como grafo direcionado.
Seja $A$ um conjunto e $R$ uma relação binária nele.
Podemos representar a $R$ como um \emph{grafo direcionado},
onde, para todos $x,y\in A$, desenhamos uma setinha
$x\longrightarrow y$ sse $x \rel R y$.
%%}}}

\TODO Terminar o exemplo seguinte.

%%{{{ eg: first_internal_diagrams_for_rel 
\example.
\label{first_internal_diagrams_for_rel}%
Seja $A = \set{1, 2, 3, 4, 5, 6, 7, 8}$.
Desenhamos:
$$
\tikzpicture[>=stealth, scale=0.84]
\tikzi reldiaginvestbase;
\draw[->] (elem-1) to [bend left=20]  (elem-2);
\draw[->] (elem-2) to [bend left=20]  (elem-1);
\draw[->] (elem-8) to                 (elem-6);
\draw[->] (elem-5) to                 (elem-7);
\draw[->] (elem-5) to                 (elem-8);
\node (elem-4h) at (-1.5 ,-1.3 ) {};
\draw[->] (elem-4h) arc (10:290:0.25);
\node (elem-5h) at ( 0.1 , 0.7 ) {};
\draw[->] (elem-5h) arc (10:290:0.25);
\draw (0,-3.0) node {$R$};
\endtikzpicture
\quad
\tikzpicture[>=stealth, scale=0.84]
\tikzi reldiaginvestbase;
\draw[->] (elem-1) to                 (elem-2);
\draw[->] (elem-8) to                 (elem-6);
\draw[->] (elem-7) to                 (elem-5);
\draw[->] (elem-5) to                 (elem-8);
\draw[->] (elem-6) to                 (elem-5);
\draw (0,-3.0) node {$S$};
\endtikzpicture
\quad
\tikzpicture[>=stealth, scale=0.84]
\tikzi reldiaginvestbase;
\draw[->] (elem-1) to [bend left=20]  (elem-2);
\draw[->] (elem-2) to [bend left=20]  (elem-1);
\draw[->] (elem-5) to [bend left=20]  (elem-8);
\draw[->] (elem-8) to [bend left=20]  (elem-5);
\draw[->] (elem-3) to [bend left=20]  (elem-4);
\draw[->] (elem-4) to [bend left=20]  (elem-3);
\draw[->] (elem-7) to [bend left=10]  (elem-6);
\draw[->] (elem-6) to [bend left=10]  (elem-7);
\node (elem-5h) at ( 0.1 , 0.7 ) {};
\draw[->] (elem-5h) arc (10:290:0.25);
\draw (0,-3.0) node {$T$};
\endtikzpicture
$$
\endexample
%%}}}

%%{{{ x: we_cannot_draw_two_parallel_arrows_on_a_rel_diagram 
\exercise.
\label{we_cannot_draw_two_parallel_arrows_on_a_rel_diagram}%
Para quais relações podemos ter \emph{duas} setinhas do objeto $x$
para o objeto $y$?

\solution
Para nenhuma!
Exatamente como um conjunto não tem noção de \emph{quantas vezes}
um certo objeto pertence nele, uma relação também não tem noção
de \emph{quantas vezes} um certo objeto relaciona com um certo outro.

\endexercise
%%}}}

\endsection
%%}}}

%%{{{ Properties of relations 
\section Propriedades de relações.

Aqui aumentamos nossa terminologia, identificando certas propriedades
interessantes que uma relação binária $R$ no $X$ pode ter.

%%{{{ Reflection 
\note Reflexão.
\label{reflexion_terminology}
\tdefined{relação!reflexiva}
\tdefined{relação!irreflexiva}
Olhamos como cada elemento do $X$ relaciona com ele mesmo.
Dois casos notáveis aparecem:
(i) pode ser que para todo $x$ temos $x\rel R x$;
(ii) pode ser que para nenhum $x$ temos $x\rel R x$.
No primeiro caso, chamamos $R$ \dterm{reflexiva}; no segundo, \dterm{irreflexiva}.
Observe que ``irreflexiva'' não significa ``não reflexiva'', etc:
$$
\alignat 2
\text{$R$ é reflexiva}      &\iff \phantom\lnot\forall x R(x,x)        &&\iff \lnot\exists x \lnot R(x,x)\\
\text{$R$ não é reflexiva}  &\iff \lnot\forall x R(x,x)                &&\iff \phantom\lnot\exists x \lnot R(x,x)\\
\text{$R$ é irreflexiva}    &\iff \phantom\lnot\forall x \lnot R(x,x)  &&\iff \lnot\exists x R(x,x)\\
\text{$R$ não é irreflexiva}&\iff \lnot\forall x \lnot R(x,x)          &&\iff \phantom\lnot\exists x R(x,x)
\endalignat
$$
onde os quantificadores quantificam sobre o $X$.
%%}}}

%%{{{ eg 
\example.
As relações $=$, $\leq$, $\geq$, nos números e $=$, $\subseteq$, $\supseteq$ nos conjuntos são todas reflexivas.
Também reflexivas são as relações
\trel{\thole\ nasceu no mesmo pais que \thole},
\trel{\thole\ tem o mesmo primeiro nome com \thole}, etc.,
definidas entre pessoas.
Típicos exemplos de irreflexivas são as $\neq$, $<$, $>$, $\subsetneq$, $\supsetneq$,
\trel{\thole\ é mais baixo que \thole},
\trel{\thole\ e \thole\ nunca estiveram em distância de 2 metros entre si},
etc.
\endexample
%%}}}

%%{{{ Symmetry 
\note Simetria.
\label{symmetry_terminology}
\tdefined{relação!simétrica}
\tdefined{relação!assimétrica}
\tdefined{relação!antissimétrica}
Agora examinamos a relação $R$ com respeito à ordem dos seus argumentos.
Novamente, certos casos notáveis aparecem:
(i) $R$ pode comportar sempre no mesmo jeito independente da ordem dos seus argumentos; nesse caso a chamamos \dterm{simétrica}.
(ii) O $R$-relacionamento dum objeto $x$ com outro $y$ pode garantir que o $y$ não esta $R$-relacionado com o $x$; a chamamos \dterm{asimmétrica}.
(iii) O único caso onde a $R$ relaciona os mesmos argumentos com as duas possíveis órdens, é quando os dois argumentos são iguais; chamamos a $R$ \dterm{antisimmétrica}.
%%}}}

%%{{{ eg 
\example.
Simétricas: $=$, $\neq$, composicionalidade de funções, \trel{\thole\ e \thole\ são irmãos}, \trel{\thole\ e \thole\ são cidades do mesmo país}, etc.
\endgraf\noindent
Asimétricas: $<$, $\subsetneq$, $>$, $\supsetneq$, \trel{\thole\ deve dinheiro para \thole}, \trel{\thole\ está andando no lado esquerdo de \thole}, \trel{\thole\ é a mãe de \thole}, etc.
\endgraf\noindent
Antisimétricas: $\leq$, $\subseteq$, $\geq$, $\supseteq$, $=$, \trel{a palavra \thole\ aparece, mas não depois da palavra \thole\ no dicionário}, etc.
\endexample
%%}}}

%%{{{ x: not_symmetric_notequiv_asymmetric 
\exercise.
\label{not_symmetric_notequiv_asymmetric}%
Verifique que ``não simétrica'' não significa nem ``assimétrica''
nem ``antissimétrica'', escrevendo todas as fórmulas envolvidas e suas negações,
como no~\refn{reflexion_terminology}.

\endexercise
%%}}}

%%{{{ x: check_reflexion_and_symmetry 
\exercise.
\label{check_reflexion_and_symmetry}%
Decida a ``reflexão'' e a ``simetria'' das relações seguintes:
$$
\align
R(A, B) &\letiff \text{os conjuntos $A$ e $B$ são disjuntos}\\
S(A, B) &\letiff |A\setminus B| > 1\\
T(A, B) &\letiff A\symdiff B \neq \emptyset.
\endalign
$$

\endexercise
%%}}}

%%{{{ x: asymmetric_implies_irreflexive 
\exercise.
\label{asymmetric_implies_irreflexive}%
Mostre que:
$$
\text{$R$ assimétrica} \implies \text{$R$ irreflexiva}.
$$

\hint
Contrapositivo.

\solution
Mostramos o contrapositivo.
Suponha que $R$ não é irreflexiva.
Então existe $s$ com $R(s,s)$,
e logo é impossível que a $R$ seja assimétrica,
pois achamos $x$ e $y$ ($x,y\asseq s$) que satisfazem ambas
$R(x,y)$ e $R(y,x)$.

\endexercise
%%}}}

%%{{{ x: asymetric_implies_antisymmetric 
\exercise.
\label{asymetric_implies_antisymmetric}%
Uma das duas direções abaixo é válida:
$$
\text{$R$ assimétrica} \askiff \text{$R$ antissimétrica}.
$$
Prove-la, e mostre que a oposta não é.

\hint
Como uma relação assimétrica poderia não ser antissimétrica?
(O que significa ``não ser antissimétrica''?)

\hint
Procure contraexemplo nos exemplos típicos de relação antissimétrica.

\solution
Para provar a \lrdir, observe que $R$ não é antissimétrica
sse existem $x$ e $y$ tais que:
$$
\underbrace{R(x,y)
\land
R(y,x)}_{\text{impossível por assimetria}}
{}\land\ \ 
{x\neq y}.
$$
Para refutar a \rldir, considere o contraexemplo da antissimétrica $\leq$
no $\nats$, que não é assimétrica, pois é reflexiva.

\endexercise
%%}}}

\note Totalidade.

\TODO Total.

\TODO Tricotomia.

\note Transições.

\TODO Transitividade.

\TODO Circular

\TODO Left-euclidean

\TODO Right-euclidean

\TODO Explicar os nomes ``euclidean''.

%%{{{ Glossary: relations_glossary 
\note Glossário.
\label{relations_glossary}%
Resumimos aqui as propriedades que encontramos.
Seja $X$ conjunto e $R$ uma relação binária nele.
Definimos as seguintes propriedades:
$$
\alignat 2
&x \rel R x                                         &\qqqquad&\textrm{(reflexiva)}\\
&x \not\rel R x                                     &\qqqquad&\textrm{(irreflexiva)}\\
&x \rel R y  \implies  y \rel R x                   &\qqqquad&\textrm{(simétrica)}\\
&x \rel R y  \implies  y \not\rel R x               &\qqqquad&\textrm{(assimétrica)}\\
&x \rel R y  \mland y \rel R x \implies x = y       &\qqqquad&\textrm{(antissimétria)}\\
&x \rel R y  \mland  y \rel R z \implies x \rel R z &\qqqquad&\textrm{(transitiva)}\\
&x \rel R y  \mland  y \rel R z \implies z \rel R x &\qqqquad&\textrm{(circular)}\\
&x \rel R y  \mland  x \rel R z \implies y \rel R z &\qqqquad&\textrm{(right-euclidean)}\\
&x \rel R z  \mland  y \rel R z \implies x \rel R y &\qqqquad&\textrm{(left-euclidean)}\\
&x \rel R y  \mlor   y \rel R x                     &\qqqquad&\textrm{(total)}\\
\text{exatamente uma das:}\quad
&x \rel R y\, ; \ \  y \rel R x\, ; \ \  x = y      &\qqqquad&\textrm{(tricotômica)}\\
\endalignat
$$
%%}}}

%%{{{ x: investigate_rel_properties_of_three_diags 
\exercise.
\label{investigate_rel_properties_of_three_diags}%
Para cada uma das relações $R,S,T$ do~\refn{first_internal_diagrams_for_rel} decida
se ela têm ou não, cada uma das propriedades do glossário no~\refn{relations_glossary}.

\hint
Cuidado: nas propriedades que acabam ser implicações, as variáveis que aparecem
nas suas premissas não denotam obrigatoriamente objetos distintos!

\endexercise
%%}}}

%%{{{ prop: wrong_property_of_sym_and_trans_implies_refl 
\proposition.
\label{wrong_property_of_sym_and_trans_implies_refl}
Seja $X\neq\emptyset$ e $\sim$ uma relação no $X$.
Se $\sim$ é simétrica e transitiva, então ela é reflexiva.
\wrongproof.
Como ela é simétrica, de $x\sim y$ concluimos que $y\sim x$ também.
E agora usando a transitividade, de $x\sim y$ e $y\sim x$, concluimos a $x\sim x$,
que mostra que $\sim$ é reflexiva também.
\mistaqed
%%}}}

%%{{{ x: find the error and prove that the proposition is false 
\exercise.
Ache o erro na prova em cima e \emph{prove} que a proposição é falsa!

\endexercise
%%}}}

\endsection
%%}}}

%%{{{ Equivalence relations and partitions 
\section Relações de equivalência e partições.

%%{{{ Equivalence 
\note Equivalência.
Considere um conjunto $A$, onde queremos ``identificar'' certos elementos deles,
talvez porque ligamos apenas sobre uma propriedade, e queremos ignorar os
detalhes irrelevantes que nos obrigariam distinguir uns deles.
Por exemplo, se $A$ é um conjunto de pessoas, podemos focar apenas na
``nacionalidade''.  Esquecendo todos os outros detalhes então, vamos
considerar todos os copatriotas como se fossem ``iguais'':
o termo certo é \dterm{equivalentes}.
Outra propriedade poderia ter sido o ano que cada pessoa nasceu,
ou o primeiro nome,  ou até quem é a mãe de cada pessoa.
Nesse último caso por exemplo, duas pessoas seriam equivalentes sse elas são
irmãos ``maternais''.
Queremos identificar as propriedades que uma relação desse tipo tem que ter:
\item{(i)} Reflexividade: não importa qual foi o critério que escolhemos
para ``equivaler'' os objetos, cada objeto com certeza vai ``concordar''
com ele mesmo nesse critério.
\item{(ii)} Simetría: pela natureza da nossa intuição é claro que
para decidir se dois elementos serão equivalentes ou não, não precisamos
os considerar numa certa ordem.
\item{(iii)} Transitividade:
Se $a$ e $b$ viram equivalentes, e $b$ e $c$ também, isso quis dizer
que não podemos distinguir entre $a$ e $b$, nem entre $b$ e $c$;
segue que $a$ e $c$ também serão equivalentes.
\endgraf
\noindent Chegamos assim na definição seguinte:
%%}}}

%%{{{ df: equivalence_relation 
\definition.
\label{equivalence_relation}
\tdefined{relação!de equivalência}
Seja $A$ conjunto e $\sim$ uma relação binária no $A$.
Chamamos $\sim$ uma \dterm{relação de equivalência} sse ela é
reflexiva, simétrica, e transitiva.
%%}}}

%%{{{ eg: equivalent_relations_on_euclidean_plane 
\example.
\label{equivalent_relations_on_euclidean_plane}
No $\reals^2$ considere as relações:
$$
\align
\tup{x,y} \sim_1 \tup{x',y'}
&\iff x = x'\\
\tup{x,y} \sim_2 \tup{x',y'}
&\iff y = y'\\
\tup{x,y} \sim_{\text N} \tup{x',y'}
&\iff \norm{ \tup{x,y} } = \norm{ \tup{x',y'} }
\endalign
$$
Facilmente todas são relações de equivalência.
\endexample
%%}}}

%%{{{ eg: equivalent_relations_on_euclidean_space 
\example.
\label{equivalent_relations_on_euclidean_space}
No $\reals^3$ considere as relações:
$$
\align
\tup{x,y,z} \sim_1 \tup{x',y',z'}
&\iff z = z'\\
\tup{x,y,z} \sim_2 \tup{x',y',z'}
&\iff x = x' \mland y = y'
\endalign
$$
Facilmente ambas são relações de equivalência.
\endexample
%%}}}

%%{{{ x: cannot replace and with or 
\exercise.
Mudamos o ``e'' para ``ou'' na última relação do~\ref{equivalent_relations_on_euclidean_space}:
$$
\tup{x,y,z} \sim_3 \tup{x',y',z'} \iff x = x' \mlor y = y'
$$
A $\sim_3$ é uma relação de equivalência?

\hint
Ache um contraexemplo que refuta sua transitividade.

\endexercise
%%}}}

%%{{{ x: equivalent_properties_to_eqrel 
\exercise.
\label{equivalent_properties_to_eqrel}%
Seja $R$ uma relação binária num conjunto $A$.
O.s.s.e.:
\item{(i)} $R$ é uma relação de equivalência;
\item{(ii)} $R$ é reflexiva e circular;
\item{(iii)} $R$ é reflexiva e left-euclideana.
\item{(iv)} $R$ é reflexiva e right-euclideana.

\endexercise
%%}}}

%%{{{ x: how_many_equivalent_relations_on_3 
\exercise.
\label{how_many_equivalent_relations_on_3}%
Seja $A$ conjunto com $\card A = 3$.
Quantas relações de equivalência podemos definir no $A$?

\hint
Deixe para responder junto com o~\ref{how_many_partitions_on_3}.

\endexercise
%%}}}

%%{{{ remark: conventions_for_internal_diagrams_of_rel 
\remark Convenções para diagramas internos.
\label{conventions_for_internal_diagrams_of_rel}%

%%}}}

\TODO Elaborar.

%%{{{ Partition 
\note Partição.
Voltamos de novo para nosso conjunto $A$, mas essa vez sem uma predeterminada
propriedade para focar.  Essa vez vamos dividir os elementos do $A$ em
\dterm{classes}, tais que cada membro do $A$ pertencerá em
\emph{exatamente uma delas}, e cada uma delas terá pelo menos um membro do $A$.
E nem vamos justificar essa separação, explicando o como ou o porquê.
Esse tipo de colecção de classes vamos definir agora.
%%}}}

%%{{{ df: partition 
\definition.
\label{partition}%
\tdefined{partição}%
Seja $A$ conjunto e $\scr A$ uma família de subconjuntos de $A$:
$$
\scr A \subseteq \powerset A.
$$
$\scr A$ é uma \dterm{partição} de $A$, sse:
\item{(i)} $\Union \scr A = A$;
\item{(ii)} os membros de $\scr A$ são disjuntos dois a dois;
\item{(iii)} $\emptyset \notin \scr A$;
\endgraf\noindent
Chamamos de \dterm{classes} os elementos de $\scr A$.
%%}}}

%%{{{ x: partitions_of_seven 
\exercise.
\label{partitions_of_seven}
Seja $A=\set{0,1,2,3,4,5,6}$.
Quais das colecções seguintes são partições do $A$?:
$$
\alignat 2
\scr A_1 &= \big\{ \set{0,1,3}, \set 2, \set{4,5}, \set 6 \big\}                &\quad\scr A_5 &= \big\{ \set{0,1,2}, \set{2,3,4}, \set{4,5,6} \big\} \\
\scr A_2 &= \big\{ \set{0,1,2,3}, \emptyset, \set{4,5,6} \big\}                 &\scr A_6 &= \big\{ \set{1,2}, \set{0,3}, \set 5, \set 6 \big\}  \\
\scr A_3 &= \big\{ \set{0,1,2,3,4,5,6} \big\}                                   &\scr A_7 &= \big\{ \set{0,1,2}, \set 3, \set{4,5,6,7} \big\}    \\
\scr A_4 &= \big\{ \set 0, \set 1, \set 2, \set 3, \set 4, \set 5, \set 6 \big\}&\scr A_8 &= \big\{ \set 0, \set{1,2}, \set{6,5,4,3} \big\}        
\endalignat
$$

\endexercise
%%}}}

%%{{{ x: pairwise_necessary_for_partition 
\exercise.
\label{pairwise_necessary_for_partition}%
Podemos trocar o~(ii) da~\ref{partition} por
$$
\text{(ii$'$)}\quad\Inter \scr A = \emptyset\  ?
$$

\hint
Não: nossa definição vai permitir mais colecções ser chamadas ``partição'' do que deveriam.

\hint
Uma das colecções não-partições do~\ref{partitions_of_seven} vai
acabar sendo partição.

\solution
A $\scr A_5\subseteq\powerset A$ do~\ref{partitions_of_seven} satisfaz as
$$
\Union \scr A_5 = A,
\qqqquad
\Inter \scr A_5 = \emptyset,
$$
mas não é uma partição: o $2\in A$ por exemplo, pertenceria em duas classes
diferentes, algo contra da nossa idéia de ``partição''.

\endexercise
%%}}}

%%{{{ Two sides of the same coin 
\note Dois lados da mesma moeda.
Os dois conceitos de ``relação de equivalência'' e ``partição'',
parecem diferentes mas realmente são apenas duas formas diferentes de
expressar a mesma idéia.
Cada relação de equivalência determina uma partição;
e vice-versa: cada partição determina uma relação de equivalência.
Bora provar isso.
%%}}}

%%{{{ x: wrong_partition_of_eqrel_def 
\exercise Cuidado!.
\label{wrong_partition_of_eqrel_def}%
Tentando resolver esse problema, começando com uma relação de equivalência $R$,
um aluno tentou definir a partição $\scr A_{R}$ assim:
$$
\align
C \in \scr A_{R}
\defiff
&C \subseteq A\\
&\mland  C \neq \emptyset\\
&\mland  \lforall {c,d \in C} {c \rel R d}.
\endalign
$$
Qual o erro na definição do aluno?
O que faltou escreverer para virar uma definição correta?

\hint
Considere o conjunto $A$ com a relação de equivalência $R$ como no diagrama
interno seguinte:%
\footnote{Lembra-se que como já declaramos a~$R$ de ser uma relação de
equivalência, não precisamos botar todas as setinhas, apenas as necessárias
para ``gerar'' a~$R$.
Veja~\ref{conventions_for_internal_diagrams_of_rel}.}
$$
\tikzpicture
\tikzi eqrel2partitionbase;
\draw (elem-1) -- (elem-2) -- (elem-3) -- (elem-4);
\draw (elem-5) -- (elem-6);
\endtikzpicture
$$
Seguindo fielmente sua definição, qual é o conjunto $\scr A_{R}$?

\hint
\emph{Não} é o conjunto dos seguintes subconjuntos de $A$:
$$
\tikzpicture
\tikzi eqrel2partitionbase;
\tikzi eqrel2partitiondesired;
\endtikzpicture
$$
Ache um $C\subseteq A$ tal que $C \in \scr A_{R}$ mas \emph{não deveria}.

\hint
A definição do aluno garanta que todos os elementos numa classe realmente
relacionam entre si através da $R$;
mas não garanta que cada classe $C$ é feita por \emph{todos} os elementos
de~$A$ que relacionam com os membros da $C$ mesmo.
Por exemplo, ela \emph{corretamente exclue} conjuntos como $\set{2,5}$;
mas ela \emph{incorretamente inclue} conjuntos como o $\set{2,3}$.
$$
\tikzpicture
\draw [rounded corners=3mm, fill=blue!20] (-1.6,0.4)--(-1.6,1.4)--(-0.3,1.4)--(-0.3,0.4)--cycle;
\tikzi eqrel2partitiondesired;
\tikzi eqrel2partitionbase;
\endtikzpicture
\qqqquad
\tikzpicture
\draw [rounded corners=2mm, fill=red!20] (-1.5,1.4)--(-1.5,-1.0)--(-0.9,-1.0)--(-0.9,1.4)--cycle;
\tikzi eqrel2partitiondesired;
\tikzi eqrel2partitionbase;
\endtikzpicture
$$

\solution
Seguindo as dicas, faltou escrever a ultima linha:
$$
\align
C \in \scr A_{R}
\defiff
&C \subseteq A\\
&\mland  C \neq \emptyset\\
&\mland  \lforall {c,d \in C} {c \rel R d}\\
&\mland  \paren{\forall a \in A} \lforall {c \in C} {a \rel R c \implies a \in C}.
\endalign
$$

\endexercise
%%}}}

%%{{{ x: from_eqrel_to_partition 
\exercise.
\label{from_eqrel_to_partition}%
Escreva claramente como definir uma partição dum conjunto $A$,
dada uma relação de equivalência $\sim$ no $A$.
Prove que realmente é uma partição
(a partição \dterm{induzida} pela $\sim$).

\endexercise
%%}}}

%%{{{ x: from_partition_to_eqrel 
\exercise.
\label{from_partition_to_eqrel}%
Escreva claramente como definir uma relação de equivalência num conjunto $A$,
dada uma partição $\scr A$ do $A$.
Prove que realmente é uma relação de equivalência
(a relação de equivalência \dterm{induzida} pela $\scr A$).

\endexercise
%%}}}

%%{{{ x: why_all_of_partition_properties_are_needed 
\exercise.
\label{why_all_of_partition_properties_are_needed}%
Explique onde precisou cada uma das condições (i)--(iii) da~\ref{partition}
na tua resolução do~\ref{from_partition_to_eqrel}.

\endexercise
%%}}}

%%{{{ x: how_many_partitions_on_3 
\exercise.
\label{how_many_partitions_on_3}%
Seja $A$ conjunto finito com $\card A = 3$.
Quantas partições de $A$ existem?

\endexercise
%%}}}

%%{{{ df: equivalent_class and quotset 
\definition.
\label{equivalent_class}%
\label{quotset}%
\tdefined{classe de equivalência}%
\tdefined{conjunto}[quociente]%
\sdefined {\eqclass {\holed a} {\holed R}} {classe de equivalência do $\holed a$ através da $\holed R$}%
\sdefined {\eqclassimp {\holed a}} {classe de equivalência do $\holed a$ (relação implícita pelo contexto)}%
\sdefined {\quotset {\holed A} {\holed R}} {conjunto quociente do $\holed A$ por $\holed R$}%
\iisee{quociente!conjunto}{conjunto quociente}%
Seja $A$ um conjunto e $\sim$ uma relação de equivalência no $A$.
Para cada $a\in A$, definimos a \dterm{classe de equivalência do} $a$
como o conjunto
$$
\eqclass a {\sim} \defeq \set {x\in A \st x\sim a},
$$
e quando a relação de equivalência é implicita pelo contexto
denotamos apenas por $\eqclassimp a$.
Definimos também o \dterm{conjunto quociente} de $A$ por $\sim$:
$$
\quotset A {\sim} \defeq \set {\eqclass a {\sim} \st a \in A }.
$$
%%}}}

%%{{{ eg: some_eqclasses_geometrically_and_algebrically
\example.
\label{some_eqclasses_geometrically_and_algebrically}%
Vamos descrever geometricamente as classes de equivalência das
relações do~\ref{equivalent_relations_on_euclidean_plane} no $R^2$,
ou seja, determinar os conjuntos quocientes correspondentes.
Lembramos as relações:
$$
\align
\tup{x,y} \sim_1 \tup{x',y'}
&\iff x = x'\\
\tup{x,y} \sim_2 \tup{x',y'}
&\iff y = y'\\
\tup{x,y} \sim_{\text N} \tup{x',y'}
&\iff \norm{ \tup{x,y} } = \norm{ \tup{x',y'} }
\endalign
$$
\endexample
%%}}}

\TODO Terminar o exemplo em cima e o exercício em baixo com gráficos.

%%{{{ x: more_eqclasses_geometrically_and_algebrically
\exercise.
\label{more_eqclasses_geometrically_and_algebrically}%
Descreva geometricamente e algebricamente os conjuntos quocientes
das relações de equivalência
do~\ref{equivalent_relations_on_euclidean_space}.

\endexercise
%%}}}

%%{{{ x: more_eqclasses_geometrically_and_algebrically
\exercise.
\label{guaranteed_eqrels}%
Seja $A$ um conjunto qualquer.
Quais relações de equivalência podes já definir nele, sem saber
absolutamente nada sobre seus elementos?

\solution
A identidade $=$, a trivial $\True$, e a vazia $\False$.

\endexercise
%%}}}

%%{{{ x: equivalent_statements_to_x_equiv_y 
\exercise.
Sejam $\sim$ uma relação de equivalência num conjunto $X$, e $x,y\in X$.
Mostre que as afirmações seguintes são equivalentes:
\item{(i)} $x\sim y$
\item{(ii)} $\eqclassimp x = \eqclassimp y$
\item{(iii)} $\eqclassimp x \inter \eqclassimp y \neq \emptyset$.

\solution
Vamos provar o ``(i)\bidir(iii)''.
\endgraf
\lrdir.
Suponha que $a \sim b$.
Precisamos achar um elemento que pertence nos dois conjuntos
$\eqclassimp a$ e $\eqclassimp b$.
Tome o proprio $a$.
Temos $a\in \eqclassimp a$ pois $a \sim a$ (pela reflexividade da $\sim$).
Tambem temos $a \in \eqclassimp b$, pois $a \sim b$ (hipótese).
Logo $a \in \eqclassimp a \inter \eqclassimp b\neq\emptyset$.
\endgraf
\rldir.
Suponha que $\eqclassimp a\inter\eqclassimp b \neq \emptyset$
e tome $w \in \eqclassimp a\inter\eqclassimp b$.
Logo $w \in \eqclassimp a$ e $w \in \eqclassimp b$,
ou seja $w \sim a$ e $w \sim b$ pela definição de classe de equivalência.
Pela simetría da $\sim$ temos $a \sim w$.
Agora como $a\sim w$ e $w \sim b$, pela transitividade da $\sim$ ganhamos
o desejado $a \sim b$.

\endexercise
%%}}}

%%{{{ x: congruence_mod_m_is_an_eqrel 
\exercise.
\label{congruence_mod_m_is_an_eqrel}%
Seja $m\in\nats$.
Prove que a relação binária nos inteiros definida pela
$$
a \sim_m b \defiff a \cong b \pmod m
$$
é uma relação de equivalência.
Qual é a partição dos inteiros correspondente?

\endexercise
%%}}}

%%{{{ df: kernel_coimage 
\definition.
\label{kernel_coimage}
\tdefined{kernel}
\tdefined{coimagem}
\sdefined {\ker {\holed f}} {kernel da \holed f}
\sdefined {\coim {\holed f}} {coimagem da \holed f}
Seja $f : X \to Y$ e defina a relação binária $\frel f$ no $X$ pela
$$
x_1 \frel f x_2 \defiff f(x_1)=f(x_2).
$$
Chamamos a $\frel f$ o \dterm{kernel} da $f$, e seu conjunto quociente
a \dterm{coimagem} da $f$.
Usamos também os símbolos $\ker f$ e $\coim f$ respectivamente.
%%}}}

%%{{{ x: frel_is_an_eqrel 
\exercise.
\label{frel_is_an_eqrel}%
Com os dados da~\ref{kernel_coimage}
mostre que $\frel f$ é uma relação de equivalência e descreva os elementos da coimagem.
O que podemos dizer se $f$ é injetora?  Se ela é sobrejetora?

\endexercise
%%}}}

%%{{{ x: immediate applications 
\exercise.
Mostre que todas as relações dos
exemplos~\refn{equivalent_relations_on_euclidean_plane}
e~\refn{equivalent_relations_on_euclidean_space}
são casos especiais do~\ref{frel_is_an_eqrel}
(e logo são relações de equivalência ``gratuitamente'').

\hint
Para cada uma delas, precisa definir completamente a
$f : A \to B$ (esclarecendo também quais são os $A,B$).

\endexercise
%%}}}

%%{{{ x: distance_like_not_transitive 
\exercise.
\label{distance_like_not_transitive}%
Seja real $\epsilon\in(0,1)$, e defina a relação $\approx_\epsilon$:
$$
x \approx_\epsilon y \defiff (x-y)^2 < \epsilon.
$$
A $\approx_\epsilon$ é uma relação de equivalência?

\hint
Primeiramente resolve o mesmo problema mas para a relação:
$$
x \sim_\epsilon y \defiff |x-y| < \epsilon
$$
onde $\epsilon > 0$.

\hint
Reflexividade e simetria das $\sim_\epsilon$ e $\approx_\epsilon$ são imediatas.
Sobre a transitividade, um desenho na linha real ajudaria.

\hint
Como contraexemplo, tome os reais $0$, $\epsilon/2$, e $\epsilon$ e
observe que $0 \sim_\epsilon \epsilon/2$ e $\epsilon/2 \sim_\epsilon \epsilon$
mas mesmo assim não temos $0\sim_\epsilon \epsilon$.

\hint
Como podes usar a não-transitividade da $\sim_\epsilon$
para deduzir a não-transitividade da $\approx_\epsilon$?

\hint
Para todo $\alpha\in\reals$ com $\alpha\geq0$, temos:
$$
\cdots
\iff
\sqrt{\alpha}\in(0,1)
\iff
\alpha\in(0,1)
\iff
\alpha^2\in(0,1)
\iff
\cdots
$$

\endexercise
%%}}}

\endsection
%%}}}

%%{{{ Constructions and operations on relations 
\section Construções e operações em relações.

%%{{{ df: rinverse 
\definition.
\label{rinverse}%
Seja $R$ uma relação de $A$ para $B$.
Definimos a relação $\rinv R$ pela:
$$
x \rinv R y \defiff y \rel R x.
$$
%%}}}

\blah.
Todas as relações que consideramos nessa secção serão binárias.

\note Composição.

\TODO Ideia.

%%{{{ eg: persons_books_words 
\example.
\label{persons_books_words}%
\DefPred{Author}%
\DefPred{Read}%
\DefPred{Contains}%
Sejam os conjuntos $\cal P$ de pessoas, $\cal B$ de livros, e $\cal W$ de palavras.
Considere as relações:
$$
\align
\Author(x,y)    &\defiff \text{$x$ é um escritor do livro $y$}\\
\Read(x,y)      &\defiff \text{$x$ leu o livro $y$}\\
\Contains(x,y)  &\defiff \text{a palavra $y$ aparece no livro $x$}
\endalign
$$
Observe que $\Author$ e $\Read$ são relações de $\cal P$ para $\cal B$,
e $\Contains$ de $\cal B$ para $\cal W$.
O que seria a relação $\Author\rcom\Read$, o que a $\Author\rcom\Contains$,
e o que a $\Read\rcom\Contains$?
Antes de defini-las, vamos primeiramente pensar se faz sentido compor essas
relações.
Como a $\Author$ é uma relação de $\cal P$ para $\cal B$,
e a $\Read$ de $\cal P$ para $\cal B$ também,
podemos realmente as combinar:
Similarmente, a $\Author$ é compatível com a $\Contains$.
são compatíveis (gráças ao $\cal B$ ``no meio'').
No outro lado, não faz podemos compor a $\Author$ com a $\Read$!
Bem, então $\Author\rcom\Contains$ e $\Read\rcom\Contains$ são ambas
relações de $\cal P$ para $\cal W$.
Mas quais?
Lembre que para definir uma relação, precisamos determinar
completamente quando dois arbitrários $x,y$ são relacionados pela
relação.
Precisamos então completar as:
$$
\align
x \relp{\Author\rcom\Contains} y &\defiff \text{\dots?\dots}\\
x \relp{\Read\rcom\Contains}   y &\defiff \text{\dots?\dots}
\endalign
$$
mas como?
\spoiler.
Bem, botamos:
$$
\align
x \relp{\Author\rcom\Contains} y &\defiff \text{a pessoa $x$ escreveu algum livro que contem a palavra $y$}\\
x \relp{\Read\rcom\Contains}   y &\defiff \text{a pessoa $x$ leu algum livro que contem a palavra $y$}
\endalign
$$
\endexample
%%}}}

%%{{{ x: comparison_of_statements_about_reading_books 
\exercise.
\DefPred{Author}%
\DefPred{Read}%
\label{comparison_of_statements_about_reading_books}%
A relação $R$ de $\cal P$ para $\cal W$ definida pela
$$
R(x,y) \defiff \text{a pessoa $x$ leu a palavra $y$ num livro}
$$
é a mesma relação com a $\Author\rcom\Read$?
Em outras palavras:
$$
R \askeq \Author\rcom\Read
$$

\hint
As afirmações:
$$
\gather
\text{a pessoa $x$ leu a palavra $y$ num livro}\\
\text{a pessoa $x$ leu um livro onde a palavra $y$ aparece}
\endgather
$$
estão afirmando a mesma coisa?

\endexercise
%%}}}

\blah.
Vamos ver mais um exemplo, essa vez usando apenas um conjunto---e
logo todas as relações são gratuitamente compatíveis para composição.

%%{{{ x: grandparents_grandchildren_siblings_and_couples_with_children 
\exercise.
\label{grandparents_grandchildren_siblings_and_couples_with_children}%
\DefPred{Parent}%
\DefPred{Child}%
Seja $\cal P$ o conjunto de todas as pessoas, e considere as relações
$$
\align
\Parent(x,y) &\defiff \text{$x$ é a mãe ou o pai de $y$}\\
\Child(x,y)  &\defiff \text{$x$ é filho ou filha de $y$}.
\endalign
$$
Como tu definaria diretamente as relações seguintes?:
$$
\Parent\rcom\Parent
\qqquad
\Child\rcom\Child
\qqquad
\Parent\rcom\Child
\qqquad
\Child\rcom\Parent
$$

\solution
{%
\DefPred{Parent}%
\DefPred{Child}%
Temos:
$$
\align
x \relp{\Parent\rcom\Parent} y &\defiff \text{$x$ é um avô ou uma avó de $y$}\\
x \relp{\Child\rcom\Child}   y &\defiff \text{$x$ é um neto ou uma neta de $y$}\\
x \relp{\Parent\rcom\Child}  y &\defiff \text{$x$ e $y$ são irmã(o)s ou a mesma pessoa}\\
x \relp{\Child\rcom\Parent}  y &\defiff \text{$x$ e $y$ tem um filho ou uma filha juntos}
\endalign
$$
}

\endexercise
%%}}}

\question.
Como podemos definir a composição de relações?
\spoiler.

%%{{{ df: rcompose 
\definition.
\label{rcompose}%
\tdefined{relação}[composição]%
\iisee{composição!de relações}{relação, composição}%
Sejam conjuntos $A,B,C$ e as relações~$R$ de~$A$ para~$B$
e~$S$ de~$B$ para~$C$.
Definimos a relação~$R\rcom S$ de~$A$ para~$C$ pela
$$
a\rel{(R \rcom S)}c
\defiff
\text{existe $y \in B$ tal que $a \rel R y$ \& $y \rel S c$}.
$$
Chamamos a $R \rcom S$ a \dterm{composição} da $R$ \emph{com} a $S$.
%%}}}

%%{{{ beware: RoS_not_SoR 
\beware.
\label{RoS_not_SoR}%
Não existe um consensus para a ordem de escrever esses $R,S$
na~\ref{rcompose}.  Tome cuidado então enquanto lendo a notação
$R \rcom S$, pois nosso $R \rcom S$ pode ser escrito como $S \rcom R$
por outros autores, e vice versa.
Quando a composição é denotada por $;$ a ordem concorda com nossa:
$$
a\rel{(R ; S)}c
\defiff
\text{existe $y \in B$ tal que $a \rel R y$ \& $y \rel S c$}.
$$
Veja também a~\ref{gof_not_fog}.
%%}}}

%%{{{ prop: associativity_of_rcom 
\property.
\label{associativity_of_rcom}%
Sejam as relações binárias:
$R$ de $A$ para $B$;
$S$ de $B$ para $C$; e
$T$ de $C$ para $D$.
Então
$$
(R\rcom S)\rcom T = R \rcom (S\rcom T),
$$
e logo podemos escrever apenas $R\rcom S\rcom T$.
\sketch.
Supomos $a \in A$ e $d\in D$, e mostramos a equivalência
$$
a \rel{((R\rcom S)\rcom T)} d
\iff
a \rel{(R\rcom (S\rcom T))} d.
$$
aplicando a definição de $\rcom$ e leis da lógica.
\qes
\proof.
Suponha $a\in A$ e $d\in D$.  Temos:
$$
\alignat 2
a \rel{((R\rcom S)\rcom T)} d
&\iff \lexists {c\in C} { a \rel{(R\rcom S)} c \;\land\; c \rel{T} d } \qqby{def.~de $\rcom$} \\
&\iff \lexists {c\in C} { \lexists {b\in B} {x \rel{R} b \;\land\; b \rel{R} c } \;\land\; c \rel{T} d } \qqby{def.~de $\rcom$} \\
&\iff \lexists {c\in C} { \lexists {b\in B} { \paren{ a \rel{R} b \;\land\; b \rel{R} c } \;\land\; c \rel{T} d } } \qqby{lógica} \\
&\iff \lexists {b\in B} { \lexists {c\in C} { \paren { a \rel{R} b \;\land\; b \rel{R} c } \;\land\; c \rel{T} d } } \qqby{lógica} \\
&\iff \lexists {b\in B} { \lexists {c\in C} { a \rel{R} b \;\land\; \paren{ b \rel{R} c \;\land\; c \rel{T} d } } } \qqby{lógica} \\
&\iff \lexists {b\in B} { a \rel{R} b \;\land\; \lexists {c\in C} {  b \rel{R} c \;\land\; c \rel{T} d } } \qqby{lógica} \\
&\iff \lexists {b\in B} { a \rel{R} b \;\land\; { b \rel{(R\rcom T)} d } } \qqby{def.~de $\rcom$} \\
&\iff a \rel{(R\rcom(S\rcom T))} d \qqby{def.~de $\rcom$}.
\endalignat
$$
\qed
%%}}}

%%{{{ x: associativity_of_rcom_nat_lang 
\exercise.
\label{associativity_of_rcom_nat_lang}%
Escreva uma prova em linguagem natural da~\ref{associativity_of_rcom}.

\solution
\lrdir:
Suponha $a \in A$ e $d \in D$ tais que
$a \rel{((R\rcom S)\rcom T)} d$.
Logo, para algum $c\in C$, temos $a \rel{(R\rcom S)} c$\fact1\ e
$c \rel{T} d$\fact2,
e usando a \byfact1\ ganhamos um $b\in B$ tal que $a \rel{R} b$\fact3
e $b \rel{S} c$\fact4.
Juntando as \byfact4~e~\byfact2 temos $b \rel{(S \rcom T)} d$, e agora
junto com a \byfact3~chegamos em
$a \rel{((R\rcom S)\rcom T)} d$.
\endgraf
A direção \rldir\ é similar.

\endexercise
%%}}}

%%{{{ x: only_if_incest 
\exercise.
\label{only_if_incest}%
\DefPred{Parent}%
\DefPred{Child}%
Prove ou refuta:
$$
\Child\rcom\Child
\askeq
\Parent\rcom\Child.
$$

\endexercise
%%}}}

%%{{{ x: id_of_rcom 
\exercise.
\label{id_of_rcom}%
Considere a $\rcom$ como uma operação nas relações binárias num conjunto $A$.
Ela tem \dterm{identidade}?  Ou seja, existe alguma relação binária $E$ no $A$,
tal que
$$
\text{para toda relação $R$ no $A$,}\quad
E \rcom R = R = R \rcom E\,?
$$
Se sim, defina essa relação $E$ e prove que realmente é.
Se não, prove que não existe.

\hint
Sem pensar, dois candidatos prováveis para considerar seriam a igualdade e a relação trivial $\True$ satisfeita por todos os pares de elementos de $A$:
$$
\text{ou}
\knuthcases{
x \rel E y \defiff \True\cr
x \rel E y \defiff x = y
}
$$

\solution
Existe sim: a $E$ é a igualdade $=$.
Vamos mostrar que para todos $a,b \in A$
$$
a \rel R b \iff a \rel{(E \rcom R)} b.
$$
Tratamos cada direção separadamente.
\endgraf
\lrdir.
Suponha que $a \rel R b$.
Precisamos mostrar que
existe $w\in A$ tal que $a \rel E w$ e $w \rel R b$.
Tome $w \asseq a$.  Realmente temos $a \rel E a$ (pois $E$ é a igualdade),
e $a \rel R b$ que é nossa hipótese.
\endgraf
\rldir.
Suponha que $a \rel {(E \rcom R)} b$.
Logo, existe $w\in A$ tal que $a \rel E w$\fact1 e $w \rel R b$\fact2.
Mas, como $E$ é a igualdade, o único $w$ que satisfaz a~\reffact1 é o proprio~$a$.
Ou seja, $w = a$.
Substituindo na~\byfact2, ganhamos o desejado $a \rel R b$.
\endgraf
A outra equivalência,
$$
a \rel R b \iff a \rel{(R \rcom E)} b,
$$
é similar.

\endexercise
%%}}}

\TODO Interações.

\TODO De composição e identidade para iteração.

%%{{{ x: R_exp_n 
\exercise Iterações.
\label{R_exp_n}%
Define formalmente a ``exponenciação'' $R^n$ duma dada relação binária $R$
num conjunto, informalmente definida por:
$$
x\mathrel{\paren{R^n}}y \pseudodefiff
x\mathrel{\Big(\underbrace{R\rcompose \dotsb\rcompose R}_{\text{$n$ vezes}}\Big)}y,
$$
valida \emph{para todo $n\in\nats$}.

\hint
Questão: o que precisa achar para tua definição servir para o caso $n=0$ também?

\hint
Resposta: precisa achar o elemento neutro da operação $\rcompose$.
Já fez o~\ref{id_of_rcom}, né?

\solution
Definimos:
$$
\align
x\mathrel{\paren{R^{0}}}y   &\defiff x=y\\
x\mathrel{\paren{R^{n+1}}}y &\defiff x\mathrel{\paren{R\rcompose {R^n}}}y
\endalign
$$
Ou, direitamente, em estilo ``point-free'':
$$
\align
R^0     &\defeq \Eq\\
R^{n+1} &\defeq R^n \rcom R,
\endalign
$$
onde escrevemos $\Eq$ para a relação de igualdade $=$.

\endexercise
%%}}}

%%{{{ x: operation_with_inverse_does_not_yield_identity 
\exercise.
\label{operation_with_inverse_does_not_yield_identity}%
Prove ou refuta:
\emph{para toda relação binária $R$ num conjunto $A$, $R \rcom \rinv R = \Eq = \rinv R \rcom R$}.

\hint
\ref{grandparents_grandchildren_siblings_and_couples_with_children}.

\hint
\ref{only_if_incest}

\hint
\DefPred{Parent}%
\DefPred{Child}%
Temos $\Parent = \rinv \Child$ (e logo $\Child = \rinv \Parent$ também).

\solution
\DefPred{Parent}%
\DefPred{Child}%
Não, como o contraexemplo do~\ref{only_if_incest} mostra, pois
$$
{\Parent} = {\rinv \Child} \qquad\mland\qquad {\Child} = {\rinv \Parent}.
$$

\endexercise
%%}}}

\TODO Quais propriedades ``familiares'' tem a inversa nas relações?.

\endsection
%%}}}

%%{{{ Closures 
\section Fechos.

\TODO Lembre o que é conjunto fechado pela operação.

\TODO Intuição com diagramas.

%%{{{ pseudodf: fecho_pseudodefinition 
\pseudodefinition.
\label{fecho_pseudodefinition}%
\tdefined{fecho!de relação}%
Seja $R$ uma relação binária num conjunto $A$,
e fixe uma propriedade \emph{razoável}
daquelas que aparecem no~glossário~\refn{relations_glossary}.
Definimos o fecho da $R$ pela propriedade para ser a relação que criamos se
botar num jeito \emph{justo} todas as \emph{necessárias}
setinhas no diagrama da $R$ até ela virar uma relação com a propriedade
desejada.
%%}}}

\TODO Examples for fecho reflexivo, simétrico, transitivo.

\TODO Por que não falamos de fecho total, asimétrico, etc.

\TODO Explique o ``justo''.

\TODO Explique o ``necessárias''.

%%{{{ df: rclosure 
\definition Fecho reflexivo.
\label{rclosure}%
\tdefined{fecho!reflexivo}%
\sdefined {\rclosure {\holed R}} {o fecho reflexivo da \holed R}%
\sdefined {\rcl {\holed R}} {o fecho reflexivo da \holed R}%
Seja $R$ relação num conjunto $A$.
Definimos a relação $\rclosure R$ pela
$$
x \rclosure R y \defiff x \rel R y \mlor x = y
$$
Chamamos a $\rclosure R$ o \dterm{fecho reflexivo} da $R$.
Também usamos a notação $\rel {R^=}$.
%%}}}

%%{{{ x: wrong_sclosure_definition 
\exercise.
Alguém definiu o fecho simétrico assim:
\endgraf
\emph{Seja $R$ relação binária num conjunto $A$.
Seu fecho simétrico é a relação $\sclosure R$ definida pela}
$$
x \sclosure R y \pseudodefiff x \rel R y \mland y \rel R y.
$$
\endgraf
Ache o erro na definição e mostre que a definição realmente é errada.

\endexercise
%%}}}

%%{{{ df: sclosure 
\definition Fecho simétrico.
\label{sclosure}%
\tdefined{fecho!simétrico}%
\sdefined {\sclosure {\holed R}} {o fecho simétrico da \holed R}%
\sdefined {\scl {\holed R}} {o fecho simétrico da \holed R}%
Seja $R$ relação num conjunto $A$.
Definimos a relação $\sclosure R$ pela
Chamamos a $\sclosure R$ o \dterm{fecho simétrico} da $R$.
Também usamos a notação $\rel {R^\leftrightarrow}$.
%%}}}

%%{{{ x: wrong_tclosure_definition 
\exercise.
Alguém definiu o fecho transitivo assim.
\emph{Seja $R$ relação binária num conjunto $A$.
Seu fecho transitivo é a relação $\tclosure R$ definida pela}
$$
x \tclosure R y \pseudodefiff \text{existe $w\in A$ tal que $x \rel R w \mland w \rel R y$}.
$$
Mas isso não é o fecho transitivo da $R$.  O que é mesmo?

\solution
A relação definida é a $R^2$, ou seja, a $R \rcom R$.

\endexercise
%%}}}

%%{{{ df: tclosure and rtclosure 
\definition Fecho transitivo.
\label{tclosure}%
\label{rtclosure}%
\tdefined{fecho!transitivo}%
\tdefined{fecho!reflexivo-transitivo}%
\sdefined {\tclosure {\holed R}} {o fecho transitivo da \holed R}%
\sdefined {\tcl {\holed R}} {o fecho transitivo da \holed R}%
\sdefined {\rtclosure {\holed R}} {o fecho reflexivo-transitivo da \holed R}%
\sdefined {\rtcl {\holed R}} {o fecho reflexivo-transitivo da \holed R}%
Seja $R$ relação num conjunto $A$.
Definimos as relações $\tclosure R$ e $\rtclosure R$ pelas
$$
\xalignat2
x \tclosure R y &\defiff x \rel {R^n} y \ \ \text{para algum $n\in\nats_{>0}$}.
&&\text{(\dterm{fecho transitivo} da $R$)}\\
x \rel {R^*} y &\defiff x \rel {R^n} y \ \ \text{para algum $n\in\nats$}
&&\text{(\dterm{fecho reflexivo-transitivo} da $R$)}\\
\endxalignat
$$
Chamamos a $\tclosure R$ o \dterm{fecho transitivo} da $R$,
e a $\rtclosure R$ o \dterm{fecho reflexivo-transitivo} da $R$.
Também usamos as notações $\tcl R$ para o $\tclosure R$
e $\rtcl R$ para o $\rtclosure R$.
%%}}}

%%{{{ x: order_of_closures_matters 
\exercise.
\label{order_of_closures_matters}%
Seja $R$ relação num conjunto $A$.
Podemos concluir alguma das afirmações seguintes?:
\beginil
\item{(i)}  $t(r(R)) \askeq r(t(R))$
\item{(ii)} $t(s(R)) \askeq s(t(R)$
\endil
\noindent
Onde $r, s, t$ são os fechos reflexivo, simétrico, transitivo respectivamente.

\hint
Use diagramas internos.

\hint
Tente achar um contraexemplo para o (ii).
Qual a dificuldade de achar contraexemplo para o (i)?

\endexercise
%%}}}

\blah.
Deixamos as definições de outros fechos para os problemas.

%%{{{ x: isPred 
\exercise.
\label{isPred}
Considere a relação $\to$ no $\nats$, definida pela:
$$
x\to y \defiff x + 1 = y.
$$
Descreva as relações seguintes:
\item{$\transcl{\to}$}: seu fecho transitivo;
\item{$\rtranscl{\to}$}: seu fecho reflexivo transitivo;
\item{$\rtranscl{\leftrightarrow}$}: seu fecho reflexivo transitivo simétrico.
\endgraf\noindent
Descreva o conjunto quociente $\quotset{\nats}{\rtranscl{\leftrightarrow}}$.

\endexercise
%%}}}

%%{{{ x: isPred_in_reals 
\exercise.
\label{isPred_in_reals}%
Considere a relação $\to$ definida pela mesma equação como
no~\ref{isPred}, mas essa vez no conjunto $\reals$.
Descreva os mesmos fechos e o
$\quotset{\reals}{\rtranscl{\leftrightarrow}}$.

\endexercise
%%}}}

\endsection
%%}}}

%%{{{ Order relations 
\section Relações de ordem.

%%{{{ df: order_relation 
\definition Ordem.
\label{order_relation}%
Seja $R$ uma relação binária num conjunto $A$.
Chamamos a $R$ \dterm{ordem parcial} sse ela é reflexiva, transitiva, e antisimétrica.
Se ela também é total, chamamos-la \dterm{ordem total}.
%%}}}

%%{{{ beware: total-partial default, relations vs functions 
\beware.
Quando usamos apenas o termo \emph{ordem}, entendemos como \emph{ordem parcial}.
Observe que esta convenção é a oposta que seguimos nas funções, onde um pleno
\emph{função} quis dizer \emph{função total}.
%%}}}

\TODO Examples.

%%{{{ divides_is_not_an_order_on_ints 
\exercise.
\label{divides_is_not_an_order_on_ints}%
A relação $\divides$ nos inteiros é uma relação de ordem?

\hint
O que podemos concluir se $a\divides b$ e $b\divides a$?

\endexercise
%%}}}

%%{{{ df: preorder_relation 
\definition Preordem.
\label{preorder_relation}%
\tdefined{preordem}%
\iisee{quasiordem}{preordem}%
Uma relação binária $R$ num conjunto $A$ é chamada \dterm{preordem} (ou \dterm{quasiordem}) sse ela é reflexiva e transitiva.
%%}}}

\blah.
No~\ref{why_called_preorder} tu vai justificar o nome ``preordem'',
mostrando que cada preordem $R$ fornece uma ordem $R'$.

\TODO Ordens estritas.

\TODO De fraca para estrita, ida e volta.

\endsection
%%}}}

%%{{{ Recursive definitions 
\section Relações recursivas.

\TODO Definições recursivas.

\example.
No $\nats$ definimos:
$$
\xalignat2
\Even(0).      &                       &\lnot\Odd(0).&\\
\Even(n+1)     &\defiff \lnot \Even(n) &\Odd(n+1)&\defiff \lnot \Odd(n)
\intertext{Ou, alternativamente, usando duas bases:}
\Even(0).      &                       &\lnot\Odd(0).&\\
\lnot\Even(1). &                       &\phantom\lnot\Odd(1).&\\
\Even(n+2)     &\defiff \Even(n)       &\Odd(n+2)&\defiff \Odd(n)
\endxalignat
$$
\endexample

\TODO Recursão mutual.

\example.
$$
\xalignat2
\Even(0).  &                  &\lnot\Odd(0).&\\
\Even(n+1) &\defiff \Odd(n)   &\Odd(n+1)&\defiff \Even(n)
\endxalignat
$$
\endexample

\endsection
%%}}}

%%{{{ Emulating functions with relations and vice-versa 
\section Emulando funções com relações e vice versa.

\endsection
%%}}}

%%{{{ Higher-order relations 
\section Relações de ordem superior.

\TODO HOL.

\endsection
%%}}}

%%{{{ Problems 
\problems.

%%{{{ prob 
\problem.
Define no $\ints\times(\ints\setminus\set0)$ a relação
$$
\tup{a,b} \approx \tup{c,d}
\defiff
ad = bc
$$
Mostre que $\approx$ é uma relação de equivalência
e descreva suas classes de equivalência.

\endproblem
%%}}}

%%{{{ prob 
\problem.
Define no $\rats$ a relação
$$
r \sim s \defiff r-s\in\ints.
$$
Prove que $\sim$ é uma relação de equivalência e descreva as classes do $\quotset {\rats} {\sim}$.

\endproblem
%%}}}

%%{{{ prob 
\problem.
Define no $\reals$ a relação
$$
x \sim y \defiff x-y\in\rats.
$$
Prove que $\sim$ é uma relação de equivalência e descreva as classes do $\quotset {\reals} {\sim}$.

\endproblem
%%}}}

%%{{{ prob 
\problem.
Define no $(\nats\to\nats)$ as relações seguintes:
$$
\align
f \rel{\buildrel{{}_{\exists\forall}} \over =} g &\defiff (\exists n\in\nats) (\forall x \geq n) [ f(x) = g(x) ]\\
f \rel{\buildrel{{}_{\forall\exists}} \over =} g &\defiff (\forall n\in\nats) (\exists x \geq n) [ f(x) = g(x) ]
\endalign
$$
Para cada uma das relações em cima, decida se ela é relação de equivalência (prove ou refuta).
Se é, descreva seu conjunto quociente.

\endproblem
%%}}}

%%{{{ prob 
\problem.
Define no $(\reals\to\reals)$ as relações seguintes:
$$
\align
f \rel{\buildrel{{}_{\exists\forall}} \over \leq} g &\defiff (\exists n\in\nats) (\forall x \geq n) [ f(x) \leq g(x) ]\\
f \rel{\buildrel{{}_{\forall\exists}} \over \leq} g &\defiff (\forall n\in\nats) (\exists x \geq n) [ f(x) \leq g(x) ]
\endalign
$$
Para cada uma das relações em cima, decida se ela tem ou não cada uma das propriedades de uma ordem total.

\endproblem
%%}}}

%%{{{ prob: simz_sime_simo_simi 
\problem.
\label{simz_sime_simo_simi}%
\def\simz{\rel{\stackrel{{}_{\mathrm z}}=}}%
\def\sime{\rel{\stackrel{{}_{\mathrm e}}=}}%
\def\simo{\rel{\stackrel{{}_{\mathrm o}}=}}%
\def\simi{\rel{\stackrel{\infty}=}}%
Defina as relações seguintes no $(\nats\to\nats)$ assim:
$$
\align
f\simz g&\defiff f(0)    = g(0)\\
f\sime g&\defiff f(2n)   = g(2n)  \ \text{para todo $n\in\nats$}\\
f\simo g&\defiff f(2k+1) = g(2k+1)\ \text{para todo $k\in\nats$}\\
f\simi g&\defiff f(n)    = g(n)   \ \text{para uma infinidade de $n\in\nats$.}
\endalign
$$
\beginil
\item{(i)}
Para cada uma da $\simz,\sime,\simo,\simi$, decida se é uma relação de
equivalência ou não.
\item{(ii)}
Prove ou refuta a afirmação seguinte:
\emph{a relação $(\sime\rcom\simo)$ é a relação trivial $\mathsf{True}$.}
\endil

\hint
\def\simi{\rel{\stackrel{\infty}=}}%
Ache um contraexemplo para a $\simi$.  As outras, são.

\solution
\def\simz{\rel{\stackrel{{}_{\mathrm z}}=}}%
\def\sime{\rel{\stackrel{{}_{\mathrm e}}=}}%
\def\simo{\rel{\stackrel{{}_{\mathrm o}}=}}%
\def\simi{\rel{\stackrel{\infty}=}}%
(i)
A $\simi$ não é.  Considere o seguinte contraexemplo.
Sejam as $\alpha, \beta, \gamma : \nats\to\nats$ (como seqüências):
$$
\align
\alpha &= \tup{0,1,0,1,0,1,\dotsc}\\
\beta  &= \tup{0,2,0,2,0,2,\dotsc}\\
\gamma &= \tup{1,2,1,2,1,2,\dotsc}
\endalign
$$
Trivialmente, $\alpha\simi\beta$ e $\beta\simi\gamma$ mas $\alpha\not\simi\gamma$.
\endgraf\noindent
(ii)
Correto.
Sejam $f, g \in (\nats\to\nats)$.
Vamos mostrar que $f(\sime\rcom\simo)g$.
Pela definição da $\rcom$ temos:
$$
f \rel{(\sime\rcom\simo)} g
\iff \text{existe $h \in (\nats\to\nats)$ tal que $f \sime h$ e $h\simo g$}.
$$
A função $h : \nats\to\nats$ definida pela
$$
h(n) = \knuthcases{
f(n), &se $n$ par\cr
g(n), &se $n$ ímpar
}
$$
satisfaz as $f \sime h\simo g$ pela sua construção.
Logo, $f \rel{(\sime\rcom\simo)} g$.

\endproblem
%%}}}

\ignore{
%%{{{ prob: simh_simv 
\problem.
\def\simh{\sim_{\mathrm h}}
\def\simv{\sim_{\mathrm v}}
Defina as relações $\simh$ e $\simv$ no $(\ints\to\ints)$:
$$
\align
f \simh g &\iffdef (\exists u\in\ints)(\forall x \in\ints)[f(x) = g(x+u) ]\\
f \simv g &\iffdef (\exists v\in\ints)(\forall x \in\ints)[f(x) = g(x)+v ]
\endalign
$$
\endproblem
%%}}}
}

%%{{{ R_eq_S 
\problem Igualdade.
\label{R_eq_S}%
Defina a igualdade para o caso mais geral onde as relações $R$ e $S$ são relações de aridade qualquer,
em conjuntos quaisquer.

\endproblem
%%}}}

%%{{{ why_called_preorder 
\problem Por que preordem?.
\label{why_called_preorder}%
Justifique o nome ``preordem'': mostre como começando com uma preordem
$R$ num conjunto $A$, podemos construir uma relação $R'$ consultando a $R$.
Pode provar essa afirmação em vários jeitos, mas o objetivo é achar
a ordem $R'$ mais natural e a mais \emph{justa}, seguindo a preordem $R$.

\hint
A ordem $R'$ não vai ser uma ordem no $A$, pois não podemos
decidir como ``resolver conflitos'' do tipo $R(a,b)$ e $R(b,a)$.
Não podemos arbitrariamente escolher um dos $a,b$ como ``menor'',
botando assim por exemplo $R'(a,b)$ e $\lnot R'(b,a)$.
Isso não seria justo!
Então, em qual conjunto $A'$ faz sentido definir nossa ordem $R'$?

\endproblem
%%}}}

%%{{{ prob: how_many_partitions 
\problem Números Bell.
\label{how_many_partitions}
Seja $A$ conjunto finito.
Quantas partições de $A$ existem?

\hint
Conte ``manualmente'' os casos com $\card A = 0, 1, 2, 3$.

\hint
(Os números que tu achou na dica anterior devem ser: $1$, $1$, $2$, e $5$, respectivamente.)
Chame $B_n$ o número de partições dum conjunto finito com $n$ elementos.
Use recursão para definir o $B_n$.

\hint
Já temos umas bases desde a dica anterior:
$$
\align
B_0 &= 1\\
B_1 &= 1\\
B_2 &= 2\\
B_3 &= 5
\intertext{Para a equação recursiva,}
B_{n+1} &= \dots
\endalign
$$
lembra-se que podes considerar conhecidos \emph{todos} os números
$B_k$ para $k \leq n$.

\hint
Sejam $a_0,\dots,a_n$ os $n+1$ elementos de $A$
e considere uma partição arbitrária $\scr A$ dele.
Sendo partição, existe exatamente um conjunto-classe $A_0$ no $\scr A$
tal que $a_0\in A_0$.
Influenciados por a notação de classes de equivalência, denotamos
o $A_0$ por $\eqclassimp {a_0}$.
Tirando esse conjunto da partição $\scr A$ chegamos no
$$
\scr A \setminus \set {\eqclassimp {a_0}}
$$
que é (certo?) uma partição do conjunto
$$
\Union\paren{\scr A \setminus \set {\eqclassimp {a_0}}}.
$$
Seja $k$ o número de elementos desse conjunto.
Quais são os possíveis valores desse $k$?

\hint
Vamos melhorar nossa notação para nos ajudar raciocinar.
O conjunto 
$$
\Union\paren{\scr A \setminus \set {\eqclassimp {a_0}}}.
$$
da dica anterior, depende de quê?
Como a gente fixou uma enumeração dos elementos do $A$,
ele depende apenas na partição $\scr A$.
Introduzimos então a notação
$$
R_{\scr A} \asseq
\Union\paren{\scr A \setminus \set {\eqclassimp {a_0}}}.
$$
E denotamos o $k$ da dica anterior com
$k_{\scr A} \asseq \card {R_{\scr A}}$.
O $k_{\scr A}$ da dica anterior pode ter qualquer um dos valores
$k_{\scr A}=0,\dotsc,n$.
E agora?

\hint
Agora separe todas as partições $\scr A$ de $A$ em grupos dependendo
no valor de $k_{\scr A}$, ache o tamanho de cada grupo separadamente,
e use o princípio da adição para achar a resposta final.

\hint
Todas as partições do $A$ são separadas assim em:
\beginul
\li as partições $\scr A$ de $A$ tais que $k_{\scr A} = 0$.
\li as partições $\scr A$ de $A$ tais que $k_{\scr A} = 1$.
\li \dots
\li as partições $\scr A$ de $A$ tais que $k_{\scr A} = n$.
\endul
Seja
$N_i$
o número das partições $\scr A$ de $A$ tais que $k_{\scr A} = i$.
Graças ao princípio da adição, procura-se o somatório $\sum_{i=0}^n N_i$.
Ache o valor do arbitrário $N_i$.

\hint
De quantas maneiras pode acontecer que o
$$
R_{\scr A} = \Union\paren{\scr A \setminus \set {\eqclassimp {a_0}}}
$$
tem $i$ elementos?

\hint
Sabemos que o $a_0$ não pode ser um deles,
então precisamos escolher $i$ elementos dos $n$ seguintes: $a_1, \dotsc, a_n$.
Ou seja, de $\comb n i$ maneiras.
Cada escolha $A_i$ corresponde numa colecção de partições:
$$
\Big\{
\eqclassimp {a_0}\ 
,
\underbrace{\quad\dots\quad}_{\hbox{partição do $A_i$}}
\Big\}
$$

\hint
Sabemos a quantidade de partições de qualquer conjunto de tamanho $i$
com $i\leq n$: são $B_i$.

\solution
Seguindo todas as dicas, basta definir:
$$
\align
B_0 &= 1\\
B_{n+1}
&= \sum_{i=0}^n N_i
= \sum_{i=0}^n \comb n i B_i
\endalign
$$
\tdefined{Bell numbers}
Essa seqüência de números é conhecida como números~\Bell[números]{}Bell.

\endproblem
%%}}}

%%{{{ prob: same_limits_eqrel 
\problem.
\label{same_limits_eqrel}%
No $(\nats\to\reals)$ defina a relação
$$
a\sim b
\defiff
\lim\nolimits_n a_n = \lim\nolimits_n b_n
\mlor
\text{nenhum dos dois limites é definido.}
$$
descreva o $\quotset {(\nats\to\reals)} {\sim}$.

\endproblem
%%}}}

%%{{{ prob 
\problem.
Sejam conjunto $A$ com $\card A=1$, e $n\in\nats$ com $n\geq 2$.
No $A^n$ defina:
$$
a \sim b
\defiff
\card{ \set{ i\in\finord n \st a_i = b_i } } \geq n/2,
$$
onde
$a \eqass \tup{ a_0, \dotsc, a_{n-1}}$
e
$b \eqass \tup{ b_0, \dotsc, b_{n-1}}$.
A $\sim$ é uma relação de equivalência?

\solution
Não é.
Sejam $s,t,u\in A$ distintos dois a dois.
Tome
$$
\align
a &\leteq \tup{ s,s,\dots,s,t,t,\dotsc,t }\\
b &\leteq \tup{ s,s,\dots,s,u,u,\dotsc,u }\\
c &\leteq \tup{ t,t,\dots,t,u,u,\dotsc,u }
\endalign
$$
como contraexemplo, pois temos
$a \sim b$ e $b \sim c$ mas $a\not\sim c$.

\endproblem
%%}}}

%%{{{ prob: cyclic_and_euclidean_closures 
\problem Fecho cíclico, fechos euclideanos.
\label{cyclic_and_euclidean_closures}%
\tdefined{fecho!cíclico}%
\tdefined{fecho!left-euclideano}%
\tdefined{fecho!right-euclideano}%
\sdefined {\ccl {\holed R}} {o fecho cíclico da \holed R}%
\sdefined {\recl {\holed R}} {o fecho right-euclideano da \holed R}%
\sdefined {\lecl {\holed R}} {o fecho left-euclideano da \holed R}%
Seja $R$ uma relação num conjunto $A$.
Defina seus fechos: cíclico~($\ccl R$), left-euclideano~($\lecl R$), right-euclideano~($\recl R$).

\endproblem
%%}}}

\endproblems
%%}}}

%%{{{ Further reading 
\further.

O \cite[Cap.~4]{velleman} defina e trata relações
diretamente como conjuntos, algo que não fazemos nesse texto.
De novo: muitos livros seguem essa abordagem, então o leitor é conselhado
tomar o cuidado necessário enquanto estudando esses assuntos.
Nos vamos ver relações como conjuntos apenas no~\ref{Axiomatic_set_theory}.

\endfurther
%%}}}

\endchapter
%%}}}

%%{{{ chapter: Logic programming 
\chapter Programação lógica.
\label{Logic_programming}%

%%{{{ Problems 
\problems.

\endproblems
%%}}}

%%{{{ Further reading 
\further.

\cite{lpbook},
\cite{artofprolog},
\cite{holpbook}.

\endfurther
%%}}}

\endchapter
%%}}}

%%{{{ chapter: Structured sets 
\chapter Conjuntos estruturados.
\label{Structured_sets}%

%%{{{ Concept, notation, equality 
\section Conceito, notação, igualdade.
Já encontramos a idéia de estrutura (interna) dum
conjunto~(foi no~\ref{blackbox_set}).

\TODO Elaborar.

\note Abuso notacional.
\label{notational_abuse_stuctured_sets}
Suponha $\ssetfont A = \sset A {\dotsc}$ é algum conjunto estruturado.
Tecnicamente falando, escrever ``$a\in \ssetfont A$'' seria errado.
Mesmo assim escrevemos sim $a\in \ssetfont A$ ao inves de $a\in A$,
similarmente falamos sobre ``os elementos de~$\ssetfont A$''
quando na verdade estamos se referendo aos elementos de~$A$,
etc.
Em geral, quando aparece um conjunto estruturado~$\ssetfont A$ num contexto
onde deveria aparecer algum conjunto, identificamos o~$\ssetfont A$
com seu carrier set~$A$.
As vezes usamos até o mesmo símbolo na sua definição, escrevendo
$A = \sset A {\dotsc}$.

\endsection
%%}}}

%%{{{ Structures sets with constants 
\section Conjuntos estruturados com constantes.

\endsection
%%}}}

%%{{{ Structures sets with operations 
\section Conjuntos estruturados com operações.

%%{{{ df: closed_associative_identity_inverse 
\definition.
\label{closed_associative_identity_inverse}%
Sejam conjunto $A$,
uma operação binária $\ast$ no $A$,
e um $g \in A$.
Dizemos que:
$$
\align
\text{$A$ é $\ast$-fechado}                     &\defiff \lforall {a,b \in A}   {a \ast b \in A}\\
\text{$\ast$ é associativa}                     &\defiff \lforall {a,b,c \in A} {(a \ast b) \ast c = a \ast (b \ast c)}\\
\text{$\ast$ é comutativa}                      &\defiff \lforall {a,b \in A}   {a \ast b = b \ast a}\\
\text{$u$ é uma identidade esquerda da $\ast$}  &\defiff \lforall {a \in A}     {u \ast a = a}\\
\text{$u$ é uma identidade direita da $\ast$}   &\defiff \lforall {a \in A}     {a \ast u = a}\\
\text{$u$ é uma identidade do $\ast$}           &\defiff \lforall {a \in A}     {u \ast a = a = a \ast u}\\
\text{$y$ é um $\ast$-inverso esquerdo de $g$}  &\defiff \text{$y \ast g = e$, \ onde $e$ é uma identidade da $\ast$}\\
\text{$y$ é um $\ast$-inverso direito de $g$}   &\defiff \text{$g \ast y = e$, \ onde $e$ é uma identidade da $\ast$}\\
\text{$y$ é um $\ast$-inverso de $g$}           &\defiff \text{$y \ast g = e = g \ast y$, \ onde $e$ é uma identidade da $\ast$}
\endalign
$$
onde não escrevemos os ``$\ast$-'' quando são implícitos pelo contexto.
%%}}}

\endsection
%%}}}

%%{{{ Structures sets with relations 
\section Conjuntos estruturados com relações.

\endsection
%%}}}

%%{{{ Problems 
\problems.

\endproblems
%%}}}

%%{{{ Further reading 
\further.

\endfurther
%%}}}

\endchapter
%%}}}

%%{{{ chapter: Group theory 
\chapter Teoria de grupos.

%%{{{ Permutations 
\section Permutações.

%%{{{ Introduce S_3 
\note.
\ii{permutação}%
Vamos começar considerando o conjunto $S_n$ de todas as permutações
dum conjunto com $n$ elementos.  Tomamos o $\set{1,2,\dotsc,n}$ como
nosso conjunto mas sua escolha é inessencial.
Logo, temos por exemplo
$$
S_3 \defeq (\set{1,2,3}\bijto\set{1,2,3}).
$$
Quantos elementos o $S_3$ tem?
Lembramos%
\footnote{Se não lembramos, veja a~\ref{total_permutations} e a~\refn{Permutations_and_combinations} em geral.  Depois disso, lembramos!}
que são
$$
\card{S_3} = \totperm 3 = 3! = 3\ntimes 2\ntimes 1 = 6.
$$
Nosso primeiro objetivo é achar todos esses $6$ membros de $S_3$.
\endgraf
Primeiramente, a identidade $\idof {\set{1,2,3}} \in S_3$ pois é bijetora.
Para continuar, introduzimos aqui uma notação bem práctica para
trabalhar com permutações:
%%}}}

%%{{{ Notation 
\note Notação.
\label{notation_with_cycles}%
Denotamos a bijeção $f \in S_n$ assim:
$$
f = \permf{
1    & 2    & \dotsb & n\\
f(1) & f(2) & \dotsb & f(n)
}
$$
Por exemplo, a identidade de $S_3$,
e a permutação $\phi$ que troca apenas o primeiro com o segundo elemento
são denotadas assim:
$$
\xalignat2
\id
&= \permf{
1 & 2 & 3\\
1 & 2 & 3
}
&
\phi
&\asseq \permf{
1 & 2 & 3\\
2 & 1 & 3
}
\endxalignat
$$
Considere agora uma permutação do $S_8$ e uma do $S_{12}$ por exemplo as
$$
\align
\permf{
1 & 2 & 3 & 4 & 5 & 6 & 7 & 8\\
2 & 3 & 1 & 4 & 6 & 5 & 7 & 8
}
\quad&\text{e}\quad
\permf{
1 & 2 & 3 & 4 & 5 & 6 & 7 & 8 & 9 & 10 & 11 & 12\\
2 & 3 & 1 & 4 & 5 & 10 & 6 & 11 & 9 & 12 & 8 & 7
}.
\intertext{Podemos as quebrar em ``ciclos'' escrevendo}
\permc{1 & 2 & 3}
\permc{5 & 6}
\quad&\text{e}\quad
\permc{1 & 2 & 3}
\permc{6 & 10 & 12 & 7}
\permc{8 & 11}
\endalign
$$
respectivamente.
Entendemos o ciclo $\permc{1 & 2 & 3}$ como
$1 \mapsto 2 \mapsto 3 \mapsto 1$.
%%}}}

%%{{{ x: verify cycle notation 
\exercise.
Verifique que as duas permutações que escrevemos usando ciclos
realmente correspondem nas permutações anteriores.

\endexercise
%%}}}

%%{{{ beware: beware_notation_with_cycles 
\beware.
\label{beware_notation_with_cycles}
Para usar a notação com ciclos para denotar os membros de algum
$S_n$, precisamos esclarecer o $n$---algo que não é necessário
com a notação completa, onde esta informação é dedutível pela
sua forma.  Por exemplo as duas permutações
$$
\underbrace{
\permf{
1 & 2 & 3 & 4 & 5\\
2 & 1 & 3 & 5 & 4
}}_{\permc{1 & 2}\permc{4 & 5}}
\quad\text{e}\quad
\underbrace{
\permf{
1 & 2 & 3 & 4 & 5 & 6 & 7\\
2 & 1 & 3 & 5 & 4 & 6 & 7
}}_{\permc{1 & 2}\permc{4 & 5}}
$$
compartilham a mesma forma usando a notação com ciclos!
Olhando para as formas em cima, sabemos que a primeira
é uma permutação do $S_5$, e a segunda do $S_7$.
\endgraf
Mais um defeito dessa notação é que não temos como denotar
a identidade numa forma consistente: podemos concordar
denotá-la pelo $\permc {1}$ ou $\permc{}$, mas na prática
optamos para o $\id$ mesmo.
%%}}}

%%{{{ The members of S_3 
\note Os membros de $S_3$.
Já achamos $2$ dos $6$ elementos de $S_3$:
$$
\xalignat2
\id
&=\permf{
1 & 2 & 3\\
1 & 2 & 3
}
&
\phi
&\leteq\permf{
1 & 2 & 3\\
2 & 1 & 3
}
=\permc{1 & 2}
\endxalignat
$$
Sabendo
que a composição de bijeções é bijeção (\ref{fcom_respects_jections}),
tentamos a
$$
\phi^2 = \phi\com\phi
=\permf{
1 & 2 & 3\\
2 & 1 & 3
}\com\permf{
1 & 2 & 3\\
2 & 1 & 3
}
=\permf{
1 & 2 & 3\\
1 & 2 & 3
}
=\id
$$
e voltamos para a própria $\id$!
Uma outra permutação no $S_3$ é a
$$
\psi\leteq\permf{
1 & 2 & 3\\
2 & 3 & 1
}
=\permc{1 & 2 & 3}.
$$
Vamos agora ver quais diferentes permutações ganhamos combinando essas:
$$
\align
\psi^2
= \psi\com\psi
&=\permf{
1 & 2 & 3\\
2 & 3 & 1
}\com\permf{
1 & 2 & 3\\
2 & 3 & 1
}
=\permf{
1 & 2 & 3\\
3 & 1 & 2
}
=\permc{1 & 3 & 2}
\\
\phi\com\psi
&=\permf{
1 & 2 & 3\\
2 & 1 & 3
}\com\permf{
1 & 2 & 3\\
2 & 3 & 1
}
=\permf{
1 & 2 & 3\\
1 & 3 & 2
}
=\permc{2 & 3}\\
\psi\com\phi
&=\permf{
1 & 2 & 3\\
2 & 3 & 1
}\com\permf{
1 & 2 & 3\\
2 & 1 & 3
}
=\permf{
1 & 2 & 3\\
3 & 2 & 1
}
=\permc{1 & 3}
\endalign
$$
E achamos $6$ membros distintos do $S_3$.%
\footnote{Por que são distintos?
Veja~\ref{why_phipsi_neq_psiphi} por exemplo.}
Mas $\card{S_3} = 6$, e logo achamos \emph{todos} os membros de
$S_3 = \set{\id, \phi, \psi, \psi^2, \phi\com\psi,\psi\com\phi}$:
$$
\xalignat3
\id  &= \permc {1}         &\psi         &= \permc {1 & 2 & 3}  & \phi\com\psi &= \permc {2 & 3}\\
\phi &= \permc {1 & 2}     &\psi^2       &= \permc {1 & 3 & 2}  & \psi\com\phi &= \permc {1 & 3}.
\endxalignat
$$
%%}}}

%%{{{ remark: is_it_needed_to_compute_last_value_of_perm 
\remark.
\label{is_it_needed_to_compute_last_value_of_perm}%
Preciso mesmo calcular os últimos números das permutações?
Vamos voltar no momento que estamos calculando o $\phi\fcom\psi$.
Já temos calculado as imagens de $1$ e $2$:
$$
\align
1 &\mapstoarrow {\psi} 2 \mapstoarrow {\phi} 1\\
2 &\mapstoarrow {\psi} 3 \mapstoarrow {\phi} 3
\endalign
$$
Estamos então aqui:
$$
\phi\fcom\psi
=\permf{
1 & 2 & 3\\
1 & 3 & 
}
$$
Agora podemos continuar no mesmo jeito, para calcular a imagem de $3$:
$$
3 \mapstoarrow {\psi} 1 \mapstoarrow {\phi} 2
$$
e assim chegar no
$$
\phi\fcom\psi
=\permf{
1 & 2 & 3\\
1 & 3 & 2
}
$$
Mas, $\phi\fcom\psi$ é uma bijeção (pois $\phi,\psi$ são,
e~\ref{fcom_respects_jections}), e logo podemos concluir neste momento que
$(\phi\fcom\psi)(3) = 2$.
Mesmo assim, sendo humanos, faz sentido achar esse último valor
\emph{com as duas maneiras}.
Assim, caso que elas chegam em resultados diferentes, teriamos um aviso sobre
um erro nos nossos cálculos anteriores!
%%}}}

%%{{{ x: calculate psi 
\exercise.
Calcule a $\psi\com\psi^2$ e justifique que ela é igual à $\psi^2\com\psi$.

\solution
Calculamos:
$$
\align
1 &\mapstoarrow {\psi^2} 3 \mapstoarrow \psi 1\\
2 &\mapstoarrow {\psi^2} 1 \mapstoarrow \psi 2\\
3 &\mapstoarrow {\psi^2} 2 \mapstoarrow \psi 3
\endalign
$$
ou seja, $\psi\com\psi^2 = \id$.
A igualdade é imediata pela associatividade da $\fcom$.
(E ambas são iguais à $\psi^3$.)

\endexercise
%%}}}

%%{{{ x: calculate phi psi^2 and psi^2 phi 
\exercise.
Calcule as $\phi\com\psi^2$ e $\psi^2\com\phi$.

\endexercise
%%}}}

%%{{{ note: abstracting_the_notion_of_group 
\note Abstraindo.
\label{abstracting_the_notion_of_group}%
Temos um conjunto cujos elementos podemos ``combinar'' através duma operação binária.
As seguintes propriedades são satisfeitas nesse caso:
\beginil
\item{(G0)}
O conjunto é fechado sobre a operação.%
\footnote{Aplicando a operação em quaisquer membros do nosso conjunto,
o resultado pertence no conjunto.}
\item{(G1)}
A operação é associatíva.
\item{(G2)}
A operação tem identidade no conjunto.
\item{(G3)}
Cada elemento do conjunto possui inverso no conjunto.
\endil
\endgraf
Conjuntos onde é definida uma operação que satisfaz essas propriedades
aparecem com freqüência, e vamos ver que são suficientes para construir
uma teoria rica baseada neles.
%%}}}

\endsection
%%}}}

%%{{{ What is a group? 
\section O que é um grupo?.

\blah.
Seguindo a abstraição do~\refn{abstracting_the_notion_of_group}, chegamos numa
primeira definição:

%%{{{ df: group_def_wordy 
\definition.
\label{group_def_wordy}%
\tdefined{grupo}%
Um conjunto $G$ com uma operação binária $\ast$ é um \dterm{grupo}
sse:
o $G$ é $\ast$-\emph{fechado};
a $\ast$ é \emph{associativa};
a $\ast$ tem \emph{identidade} no $G$;
cada elemento de $G$ possui $\ast$-\emph{inverso}.
%%}}}

\blah.
Lembrando a~\ref{closed_associative_identity_inverse}, tentamos esclarecer pouco essa definição:

%%{{{ df: group (0) 
\pseudodefinition Grupo (0).
\label{group_def_classic}%
\tdefined{grupo}%
Um conjunto $G$ com uma operação binária $\ast$ no $G$ é um \dterm{grupo} sse
as leis seguintes
$$
\gather
a,b \in G \implies a\ast b \in G                                             \tag{G0} \\
a\ast(b\ast c) = (a\ast b)\ast c                                             \tag{G1} \\
\text{existe $e\in G$ tal que para todo $a\in G$, $e\ast a = a = a\ast e$}   \tag{G2} \\
\text{para todo $a\in G$, existe $y\in G$, tal que $y\ast a = e = a \ast y$} \tag{G3}
\endgather
$$
são satisfeitas.
\mistake
%%}}}

%%{{{ Laws vs. axioms 
\note Leis vs.~axiomas.
\label{laws_vs_axioms}%
\tdefined{lei}%
\tdefined{leis de grupo}%
\ii{axioma!vs.~lei}%
As (G0)--(G3) são conhecidas como as \dterm{leis de grupos},
ou os \dterm{axiomas de grupos}.
Vamos evitar usar a palavra \emph{axioma} com esse sentido,
optando para a palavra \emph{lei} mesmo, pois chamamos
``axioma'' algo que aceitamos como verdade em noso univérso
(mais sobre isso no~\ref{Axiomatic_set_theory}), mas nesse
caso não estamos afirmanos a vericidade das (G0)--(G3).
Faz apenas parte do que significa ``ser grupo''.
Se um conjunto estruturado satisfaz todas as leis,
bem, ele ganha o direito de ser chamado um ``grupo''.
Se não, beleza, ele não é um grupo.
%%}}}

%%{{{ Notational abuse 
\note Abuso notacional.
Lembra-se o abuso notacional que introduzímos
no~\refn{notational_abuse_stuctured_sets}:
usamos $a,b\in\ssetfont G$, $G=\sset G {\bullet}$, etc.
%%}}}

%%{{{ Multiplicative and additive groups 
\note Grupos multiplicativos e aditivos.
Dependendo da situação, podemos adoptar um ``jeito multiplicativo'' para a
notação dum grupo,
ou um ``jeito additivo'', ou ficar realmente com um jeito neutro.
Num \dterm{grupo multiplicativo} usamos $\cdot$ para denotar a operação do grupo,
aproveitamos a convenção de omitir o símbolo totalmente, usando apenas
juxtaposição: $a(bc)$ significa $a\cdot(b\cdot c)$ por exemplo.
A identidade parecerá com $e$ ou $1$, e $a^{-1}$ será o inverso de~$a$.
Num \dterm{grupo aditivo} usamos $+$ para denotar a operação do grupo,
a identidade parecerá com $e$ ou $0$; e $-a$ será o inverso de~$a$.
Naturalmente usamos $\ast$, $\bullet$, etc.~para denotar operação
de grupo, $e$ para sua identidade e $a^{-1}$ ou $a'$ para denotar o inverso de~$a$.
É importante entender que os termos ``grupo multiplicativo'' e ``grupo aditivo''
usados assim não carregam nenhum significado matemático mesmo: apenas mostram
uma preferência notacional.
Mas quando um conjunto já tem adição e/ou multiplicação definida
(como por exemplo os inteiros), então usamos frases como
``o grupo aditivo dos inteiros'' para referir no $\sset \ints +$.
%%}}}

\blah.
Para entender melhor as quatro leis de grupo, as escrevemos novamente, essa vez
sem deixar nenhum quantificador como implicito, e começando com um
\emph{conjunto estruturado} com uma operação binária:

%%{{{ pseudodf: group (1) 
\pseudodefinition Grupo (1).
\label{group_def_struct_1}%
Um conjunto estruturado $\ssetfont G = \sset G {\ast}$ é um \dterm{grupo} sse
$$
\alignat2
\paren{\forall a,b\in G}                      &\bracket{a\ast b \in G}                   &\tag{G0} \\
\paren{\forall a,b,c\in G}                    &\bracket{a\ast(b\ast c) = (a\ast b)\ast c}&\tag{G1} \\
\paren{\exists e\in G} \paren{\forall a \in G}&\bracket{e\ast a = a = a\ast e}           &\tag{G2} \\
\paren{\forall a\in G} \paren{\exists y\in G} &\bracket{y\ast a = e = a \ast y}          &\tag{G3} 
\endalignat
$$
Chamamos o elemento garantido pela~(G2) a \dterm{identidade} do grupo,
chamamos o~$y$ da~(G3) o \dterm{inverso} de~$a$.
\mistake
%%}}}

%%{{{ x: check_group_def_classic_and_group_def_struct_1 
\exercise.
Tá tudo certo com as definições~\refn{group_def_classic}
e~\refn{group_def_struct_1}?

\hint
Quem é esse $e$ que aparece no (G3)?

\hint
Como assim \emph{o} inverso?

\endexercise
%%}}}

%%{{{ Galois, Abel, Cayley.
\note Galois, Abel, Cayley.
Como vimos, na mesma época o \Galois{}Galois e o \Abel{}Abel chegaram na
idéia abstrata de grupo.  Galois mesmo escolheu a palavra ``group'' para
esse conceito.
A definição ``moderna'' de grupo como conjunto com operação que satisfaz
as leis~(G0)--(G3) é de~\Cayley{}Cayley.
Como Abel focou em grupos cuja operação é comutativa, chamamos esse tipo
de grupos abelianos:
%%}}}

%%{{{ df: abelian group 
\definition Grupo abeliano.
\label{abelian_group}%
\tdefined{grupo!abeliano}%
Um grupo $\sset G {\ast}$ é \dterm{abeliano}
(também: \dterm{comutativo})
sse sua operação $\ast$ é comutativa:
$$
\alignat2
\paren{\forall a,b\in G}    &\bracket{a\ast b = b\ast a}.                   &\tag{GA}
\endalignat
$$
%%}}}

%%{{{ Groups and abelian groups schematically 
\note.
Esquematicamente:
$$
\left.
\aligned
&\left.
\aligned
\text{(G0)}&\\
\text{(G1)}&\\
\text{(G2)}&\\
\text{(G3)}&\\
\endaligned
\ \ 
\right\}
\text{grupo}\\
&\ \text{(GA)}
\endaligned
\ \ 
\right\}
\text{grupo abeliano}
$$
%%}}}

%%{{{ eg: S3_is_a_non_abelian_group 
\example.
\label{S3_is_a_non_abelian_group}%
Verifique que $S_3$ é um grupo.
Ele é abeliano?

\solution
Precisamos verificar as leis de grupo.
\endgraf\noindent
{(G0).}
Para provar que $S_3$ é fechado pela $\fcom$, precisamos verificar
que para todo $a,b\in S_3$, $a \fcom b \in S_3$.
Pela definição do $S_3$, isso segue pelo~\ref{fcom_respects_jections}~(3).
\endgraf\noindent
{(G1).}
Ja provamos a associatividade da $\fcom$ na~\ref{associativity_of_fcom}.
\endgraf\noindent
{(G2).}
Facilmente verificamos que a $\idof {\set{1,2,3}}$ é a identidade do
$\sset {S_3} {\fcom}$, pela sua definição.
\endgraf\noindent
{(G3).}
Cada bijeção tem uma função-inversa, que satisfaz as equações dessa lei
pela definição de função-inversa.
(Veja~\ref{finverse} e~\ref{finv_is_bij}.
\endgraf\noindent
{(GA).}
Basta mostrar pelo menos um contraexemplo, ou seja, duas permutações
$a,b$ do $S_3$ tais que $a\fcom b \neq b\fcom a$.
Agora preciso saber o que significa igualdade entre \emph{funções}
(\ref{f_eq_g}).
Escolho os $\phi,\psi$.
Já calculamos as $\phi\fcom\psi$ e $\psi\fcom\phi$ e são diferentes.
\endgraf
Logo, $S_3$ é um grupo não abeliano.
\endexample
%%}}}

%%{{{ x: why_phipsi_neq_psiphi 
\exercise.
\label{why_phipsi_neq_psiphi}%
Por quê $\phi\fcom\psi \neq \psi\fcom\phi$?

\solution
Pois elas discordam em pelo menos um valor: tome o $1$.
Agora
$$
(\phi\fcom\psi)(1)
= \phi(\psi(1))
= \phi(2)
= 1
\neq
3
= \psi(2)
= \psi(\phi(1))
= (\psi\fcom\phi)(1).
$$
Logo $\phi\fcom\psi \neq \psi\fcom\phi$.

\endexercise
%%}}}

%%{{{ And why 1 neq 3? 
\note E por que $1\neq 3$?.
Na resolução do~\ref{why_phipsi_neq_psiphi} nosso argumento reduziu
o que queriamos provar à afirmação $1 \neq 3$.
\emph{E por que $1 \neq 3$?}
Bem, precisamos saber o que significa igualdade no $\nats$!
Mas podemos já considerar o $1 \neq 3$ como um fato conhecido sobre os números
naturais.  Depois, no~\ref{Axiomatic_set_theory}, vamos \emph{fundamentar}
o~$\nats$ na teoria de conjuntos, e logo vamos ter como realmente provar essa
afirmação para nosso $\nats$, por exemplo.
%%}}}

%%{{{ x: find_all_inverses_on_S3 
\exercise.
\label{find_all_inverses_on_S3}%
Ache o inverso de cada elemento de $S_3$.%
\footnote{Se tu já fez isso para resolver o~\ref{S3_is_a_non_abelian_group},
não foi necessário.  Por quê?  Veja a resolução do~\refn{S3_is_a_non_abelian_group} mesmo.}

\endexercise
%%}}}

%%{{{ df: group (2) 
\definition Grupo (2).
\label{group_def_struct_2}%
\tdefined{grupo}%
Um conjunto estruturado $\ssetfont G = \sset G {e, \ast}$ é um grupo sse
$$
\alignat2
\paren{\forall a,b\in G}                      &\bracket{a\ast b \in G}                   &\tag{G0} \\
\paren{\forall a,b,c\in G}                    &\bracket{a\ast(b\ast c) = (a\ast b)\ast c}&\tag{G1} \\
\paren{\forall a \in G}                       &\bracket{e\ast a = a = a\ast e}           &\tag{G2} \\
\paren{\forall a\in G} \paren{\exists y\in G} &\bracket{y\ast a = e = a \ast y}          &\tag{G3}
\endalignat
$$
%%}}}

%%{{{ x: check_group_def_struct_2 
\exercise.
Tá tudo certo com a~\ref{group_def_struct_2}?

\solution
Sim!
Pois, veja~\ref{a_vs_the_identity_of_a_group}.

\endexercise
%%}}}

%%{{{ remark: a_vs_the_identity_of_a_group 
\remark.
\label{a_vs_the_identity_of_a_group}%
O $e$ que aparece na~(G3) não é ``\emph{a} identidade do $\cal G$''.
É sim \emph{a} constante que aparece na estrutura do $\sset G {e,\ast}$,
que---graças à~(G2)---é \emph{uma} identidade do $\cal G$.
No~\ref{uniqueness_of_identity_in_group} vamos provar que cada grupo tem identidade única,
e apartir dessa prova, vamos ganhar o direito de usar o artigo definido ``a''.
%%}}}

%%{{{ x: group (4) 
\exercise Grupo (4).
Defina \dterm{grupo} como um conjunto estruturado
$\sset G {e, {}^{-1}, \ast}$.
Comece escrevendo as aridades da assinatura.

\solution
Um conjunto estruturado $\ssetfont G = \sset G {e, {}^{\prime}, \ast}$ é um \dterm{grupo} sse
$$
\alignat2
\paren{\forall a,b\in G}    &\bracket{a\ast b \in G}                    &\tag{G0} \\
\paren{\forall a,b,c\in G}  &\bracket{a\ast(b\ast c) = (a\ast b)\ast c} &\tag{G1} \\
\paren{\forall a \in G}     &\bracket{e\ast a = a = a\ast e}            &\tag{G2} \\
\paren{\forall a\in G}      &\bracket{a^{-1}\ast a = e = a \ast a^{-1}} &\tag{G3} 
\endalignat
$$

\endexercise
%%}}}

%%{{{ df: order of group 
\definition Ordem de grupo.
\label{order_of_group}%
\tdefined{ordem!de grupo}%
\sdefined {\gord {\holed G}} {a ordem do grupo $\holed G$}%
O número de elementos de um grupo $G$ é sua \dterm{ordem}.
Denotamos a ordem de $G$ com:
$\gord G$, $\tord G$, ou até $\bord G$ quando não existe ambigüidade.
Se o carrier set do grupo é infinito, escrevemos
$\gord G = \infty$.
%%}}}

%%{{{ x: group_of_any_finite_order 
\exercise.
\label{group_of_any_finite_order}
Ja conhecemos um grupo finito bem, o $S_3$, com $\gord {S_3} = 6$.
No~\ref{Sn_is_a_group} provarás que para todo $n\in\nats$, o $S_n$ é um grupo.
Seja $m\in\nats$.  Tem como achar um grupo com ordem $m$?
Observe que como sabemos que $\gord {S_n} = n!$, podemos já achar um grupo com ordem $m$ para qualqeur $m$ que fosse um fatorial.
Por exemplo, se $m=120$ já temos o grupo $S_5$, pois $\gord {S_5} = 5! = 120$.
Mas para um $m$ arbitrário, existe grupo com ordem $m$?

\hint
\ref{Congruences}.

\endexercise
%%}}}

\endsection
%%}}}

%%{{{ Examples and nonexamples 
\section Exemplos e nãœxemplos.

%%{{{ eg: with numbers 
\example Com números.
Todos os seguintes conjuntos estruturados são grupos:
\beginol
\li$\sset A {0, +}$, onde $A \asseq \ints, \rats, \reals, \complex$;
\li$\sset B {1, \ntimes}$, onde $B \asseq \rats_{\neq0}, \reals_{\neq0}$;
\li$\sset C {1, \ntimes}$, onde $C \asseq \set{1,-1}, \set{1,i,-1,-i} \subseteq\complex$.
\endol
\endexample
%%}}}

%%{{{ noneg: with numbers 
\nonexample Com números.
E \emph{nenhum} dos seguintes é um grupo:
$\sset \nats {0, +}$;
$\sset {\ints_{\neq 0}} {1, \ntimes}$;
$\sset \reals {1, +}$.
\endnonexample
%%}}}

%%{{{ x: why? 
\exercise.
Por quê?

\endexercise
%%}}}

%%{{{ noneg: strings_with_concat_is_not_a_group 
\nonexample Strings.
\label{strings_with_concat_is_not_a_group}%
Sejam $\Sigma\neq\emptyset$ um alfabeto finíto,
e $S$ o conjunto de todos os strings finitos formados por símbolos do $\Sigma$.
O $\sset S +$ onde $+$ é a \emph{concatenação} de strings \emph{não} é um grupo.
\endnonexample
%%}}}

%%{{{ x: verify_all_group_laws_for_strings 
\exercise.
Para cada uma das leis~(G0)--(G3) e~(GA), decida se é
satisfeita pelo $\sset S +$ do~\ref{strings_with_concat_is_not_a_group}.
Se é, prove; se não é, refute!

\endexercise
%%}}}

%%{{{ x: shared_carrierset_group_notgroup 
\exercise.
\label{shared_carrierset_group_notgroup}%
Sejam $B = \setst {2^m} {m \in \ints}$ e $B_0 = B \union \set{0}$.
Considere os conjuntos estruturados
$$
\xalignat4
&\sset B {+}
&&\sset B {\ntimes}
&&\sset {B_0} {+}
&&\sset {B_0} {\ntimes}.
\endxalignat
$$
Para cada um deles decida se satisfaz cada uma das leis (G0)--(GA).

\endexercise
%%}}}

%%{{{ x: more number-based groups 
\exercise.
Mostre mais grupos formados de números dos
$\nats$, $\ints$, $\rats$, $\reals$, $\complex$,
e uma operação não-padrão da sua escolha.

\endexercise
%%}}}

%%{{{ eg: matrix_group_examples 
\example Matrizes.
\label{matrix_group_examples}%
Considere os conjuntos seguintes:
$$
\align
A &= \setst{ \matrixp{a & b\\c & d}\in\reals^{2\times2}} {a,b,c,d \in \reals }\\
M &= \setst{ \matrixp{a & b\\c & d}\in\reals^{2\times2}} {ad - bc \neq 0 }.
\endalign
$$
\TODOthis{Terminar o exemplo e mostrar mais.}
\endexample
%%}}}

%%{{{ eg: modular_arithmetic_group_eg
\example Aritmética modular.
\label{modular_arithmetic_group_eg}%
O $\finord n \defeq \set{0,\dotsc,n-1}$ com a operação $+_n$ da adição
módulo $n$.
Qual é o inverso de um elemento $a$ nesse caso?
É o $n-a$, pois $a +_n (n-a) = 0 = (n-a) +_n a$.
\endexample
%%}}}

%%{{{ df: symmetric_group 
\definition.
\label{symmetric_group}%
\tdefined {grupo}[simétrico]%
\sdefined {S_{\holed n}} {O grupo simétrico $S_{\holed n}$}%
Usamos $S_n$ para denotar o conjunto de todas as permutações
dum conjunto de tamanho $n\in\nats$.
Para definir mesmo o $S_n$ escolhemos o conjunto canônico:
$$
S_n \pseudodefeq (\set{1,\dotsc,n}\bijto\set{1,\dotsc,n}).
$$
Para qualquer $n\in\nats$, chamamos o $\sset {S_n} {\fcom}$
o \dterm{grupo simétrico} de tamanho $n$.
%%}}}

%%{{{ x: Sn_is_a_group 
\exercise.
\label{Sn_is_a_group}%
Justifique a~\ref{symmetric_group}: prove que o grupo simétrico $S_n$
realmente é um grupo.
Ele é abeliano?

\endexercise
%%}}}

%%{{{ x: pset_with_setops_group 
\exercise Conjuntos.
\label{pset_with_setops_group}%
Seja $A$ conjunto.
Com quais das operações $\union$, $\inter$, $\symdiff$, e $\setminus$,
o $\pset A$ é um grupo?

\endexercise
%%}}}

%%{{{ x: real_functions_pointwise_plus_group 
\exercise Funções reais: adição pointwise.
\label{real_functions_pointwise_plus_group}%
O $(\reals \to \reals)$ com operação a pointwise $+$, é um grupo?%
\footnote{Qual operação é a pointwise $+$?  Veja a~\ref{pointwise_operation}.}
Ele é abeliano?

\endexercise
%%}}}

%%{{{ x: real_functions_pointwise_times_nongroup 
\exercise Funções reais: multiplicação pointwise.
\label{real_functions_pointwise_times_nongroup}%
O $(\reals \to \reals) \setminus \set{ \lam x 0 }$ com operação a
pointwise~$\ntimes$, é um grupo?
Ele é abeliano?

\endexercise
%%}}}

\blah.
Chegam os exemplos por enquanto.
Vamos começar ver a \emph{teoria} de grupos, investigando propriedades que
todos os grupos necessariamente têm.
Ou seja, procuramos as \emph{conseqüências das leis} (G0)--(G3).

\endsection
%%}}}

%%{{{ First consequences 
\section Primeiras conseqüências.

%%{{{ lm: uniqueness_of_identity_in_group 
\lemma Unicidade da identidade.
\label{uniqueness_of_identity_in_group}%
\ii{unicidade!da identidade}%
Em todo grupo $G$ existe único elemento $e$ que satisfaz a (G2).
\sketch.
Seja $G$ grupo.
Sabemos que existe pelo menos uma identidade no $G$ pela (G2),
então precisamos mostrar que existe no máximo uma (unicidade).
Vamos supor que $e_1, e_2$ são identidades do $G$, e usando
as leis~(G0)--(G2) mostrar que $e_1 = e_2$.
\qes
\proof.
Seja $G$ grupo.
Sabemos que $G$ tem pelo menos uma identidade graças à (G2),
então o que precisamos mostrar é sua unicidade mesmo.
Suponha que $e_1,e_2\in G$ tais que $e_1,e_2$ são identidades do $G$;
em outras palavras:
$$
\align
\text{para todo $a\in G$},\quad & e_1\ast a \eqlabel L a \eqlabel R a\ast e_1   \tag{1}\\
\text{para todo $b\in G$},\quad & e_2\ast b \eqlabel L b \eqlabel R b\ast e_2.  \tag{2}
\endalign
$$
Agora exatamente a mesma prova pode ser escrita em dois caminho meio
diferentes:
\endgraf
\proofstyle{Caminho 1:}
Temos
$$
\alignat2
e_1
&= e_1 \ast e_2  \qqby{pela (2R), com $b\asseq e_1$}\\
&= e_2           \qqby{pela (1L), com $a\asseq e_2$}
\endalignat
$$
e provamos o que queremos: $e_1 = e_2$, ou seja,
em cada grupo existe única identidade.
\endgraf
\proofstyle{Caminho 2.}
Temos
$$
\alignat2
e_1 \ast e_2 = e_1  \qqby{pois $e_2$ é uma identidade (2R)}\\
e_1 \ast e_2 = e_2  \qqby{pois $e_1$ é uma identidade (1L)}
\endalignat
$$
e concluimos o desejado $e_1 = e_2$.
\qed
%%}}}

\blah.
Uma prova errada desse lema aparece
no~\ref{bust_proof_of_uniqueness_of_identity_in_group},
onde peço identificar seus erros.

%%{{{ Q: What have we just won? 
\question.
O que acabamos de ganhar?
%%}}}

%%{{{ A: The right to use the definite article 
\note Resposta.
Ganhamos o direito de usar o artigo definido: para cada grupo~$\cal G$ falar
\emph{da}~identidade do~$\cal G$, em vez \emph{duma}~identidade do~$\cal G$.
Observe que dado algum $a\in \cal G$ ainda não podemos falar sobre
\emph{o}~inverso de~$a$, mas apenas sobre \emph{um}~inverso de~$a$,
pois por enquanto a~(G3) garanta que pelo menos um inverso existe.
Vamos resolver isso agora.
%%}}}

%%{{{ beware: bound_variables_in_proof_of_uniqueness_of_identity_in_group 
\beware.
\label{bound_variables_in_proof_of_uniqueness_of_identity_in_group}%
\ii{variável!ligada}%
Os $a$ e $b$ que aparecem nas~(1)--(2) na prova do~\ref{uniqueness_of_identity_in_group}
são \emph{variaveis ligadas} aos correspondentes <<para todo $\holed{\dots} \in G$>>
e logo, ``nascem'' com essa frase e ``morrem'' no fim da mesma linha!%
\footnote{Veja~\ref{Bound_and_free_variables} também.}
Daí, não faz sentido afirmar logo após das~(1)--(2) algo do tipo $e = a \ast e$, pois o $a$ não foi declarado!
Podemos escrever as duas afirmações sem usar o nome $a$:
\emph{para cada elemento do $G$, operando com o~$e_1$ ao qualquer lado (direito ou esquerdo), o resultado é o proprio elemento}.
E para enfatisar ainda mais a independência do~$a$ que aparece na~(1) com o~$b$ que aparece na~(2) escolhemos variáveis diferentes.
Mas isso é \emph{desnecessário}, em geral vamos reusar variáveis ligadas quando não gera confusão---e aqui não geraria nenhuma.
%%}}}

%%{{{ x: rewrite_proof_with_same_bound_var 
\exercise.
\label{rewrite_proof_of_uniqueness_of_identity_in_group_with_same_bound_var}%
O que muda na prova do~\ref{uniqueness_of_identity_in_group} se usar a mesma
variável ligada nas afirmações~(1) e~(2)?

\solution
É apenas escrever
$$
\align
\text{para todo $a\in G$},\quad & e_1\ast a \eqlabel L a \eqlabel R a\ast e_1 \tag{1}\\
\text{para todo $a\in G$},\quad & e_2\ast a \eqlabel L a \eqlabel R a\ast e_2 \tag{2}
\endalign
$$
e depois
$$
\alignat2
e_1
&= e_1 \ast e_2     \qqby{pela (2R), com $a\asseq e_1$}\\
&= e_2              \qqby{pela (1L), com $a\asseq e_2$}.
\endalignat
$$
Observe que os $a$ que aparecem nas instanciações $a \asseq \dots$ são completemante independentes.
Ou seja, nada muda mesmo!

\endexercise
%%}}}

%%{{{ lm: uniqueness_of_inverses_in_group 
\lemma Unicidade dos inversos.
\label{uniqueness_of_inverses_in_group}%
\ii{unicidade!dos inversos}%
Em todo grupo $G$, cada $a\in G$ tem
exatamente um inverso $\ginv a$ que satisfaz a (G3).
\sketch.
Supondo que existe um certo $a \in G$ que possui inversos
$a_1,a_2\in G$, mostramos que necessariamente $a_1 = a_2$.
Ganhamos isso como corolário do~\ref{cancellation_laws_in_group}.
(Como?)
\qes
\proof.
Seja $\sset G {e, \ast}$ grupo, e suponha que existem $a,a_1,a_2\in G$
tais que $a_1,a_2$ são inversos de $a$, ou seja,
$$
\alignat2
a_1 \ast a &\eqlabel L e \eqlabel R a \ast a_1  \qqby{$a_1$ é um inverso de $a$} \tag{1}\\
a_2 \ast a &\eqlabel L e \eqlabel R a \ast a_2. \qqby{$a_2$ é um inverso de $a$} \tag{2}
\endalignat
$$
Vamos mostrar que $a_1 = a_2$.
Temos:
$$
\alignat2
a_1 \ast a &= a_2 \ast a \qqby {pelas (1L), (2L)}\\
a_1        &= a_2        \qqby {pelo~\refn{cancellation_laws_in_group}~(GCR)}
\endalignat
$$
e \emph{ficamos devendo} provar a (GCR) do~\ref{cancellation_laws_in_group}.
\qed
%%}}}

%%{{{ beware: Dependencies 
\beware Dependências.
Até realmente provar as leis de cancelação~(\refn{cancellation_laws_in_group}) não temos
a unicidade dos inversos~(\refn{uniqueness_of_inverses_in_group}).
Dado um elemento $a$ dum grupo $G$ não podemos aind falar \emph{do}
inverso do $a$, nem usar a notação $\ginv a$ (seria mal-definida), etc.
Crucialmente, não podemos usar nada disso em nossa prova do~\refn{cancellation_laws_in_group};
caso contrario criamos uma loope de dependências.
``Forward dependencies'' são perigosos exatamente por causa disso, e nos evitamos mesmo.%
\footnote{Aqui escolhi essa abordagem para enfatisar a importância de ficar
alertos para identificar chances de afirmar e provar lêmata separadamente,
usando-los em nossa prova e para provar outros teoremas depois a vontade.
Fazemos isso exatamente no mesmo jeito que um bom programador bom,
percebe padrões nas suas funções e separa certas partes para outras funções
chamando-las na sua função e nos seus programs próximos a vontade.}
%%}}}

%%{{{ lm: cancellation_laws_in_group 
\lemma Leis de cancelação.
\label{cancellation_laws_in_group}%
Seja $G$ grupo.
Então as leis de cancelação pela esquerda e pela direita
$$
\alignat2
\paren{\forall a,x,y\in G}  &\bracket{a\ast x = a\ast y \implies x=y}  &\tag{GCL} \\
\paren{\forall a,x,y\in G}  &\bracket{x\ast a = y\ast a \implies x=y}  &\tag{GCR} \\
\endalignat
$$
são válidas em $G$.
\sketch.
Sejam $a,x,y\in G$ tais que
$$
a \ast x = a \ast y.   \tag{1}
$$
Queremos provar $x=y$.
Tome a~(1) então, e usando umas das leis de grupo---comece com a~(G3)---chegue no desejado $x=y$, provando assim a~(GCL).
A~(GCR) é similar.
\qes
\proof.
Sejam $a,x,y\in G$ tais que
$$
a \ast x = a \ast y.   \tag{1}
$$
Pela (G2) o $a$ possui inverso no $G$;
daí, seja $a_0$ \emph{um} inverso de $a$, ou seja,
$$
a_0 \ast a \eqlabel L e  \eqlabel R a \ast a_0.  \tag{2}
$$
Agora temos:
$$
\alignat2
a_0 \ast (a \ast x) &= a_0 \ast (a \ast y)  \qqby{pela (1)}\\
(a_0 \ast a) \ast x &= (a_0 \ast a) \ast y  \qqby{pela (G1): $\ast$ é associativa}\\
e \ast x            &= e \ast y             \qqby{pela (2L)}\\
x                   &= y                    \qqby{pela definição do $a_0$ como um inverso de $a$}
\endalignat
$$
Provamos assim a~(GCL).
A~(GCR) é completamente simétrica (e vamos precisar a~(2R) em vez da~(2L)).
\qed
%%}}}

%%{{{ x: converses_of_cancellation_laws 
\exercise.
\label{converses_of_cancellation_laws}%
Os conversos das leis de cancelação~(GCL)~\&~(GCR) são válidos?

\hint
Sim; mas por quê?

\hint
Não precisamos nem saber que $G$ é um grupo nem nada sobre $\ast$, etc.

\solution
Se $x=y$ isso quis dizer que podemos substituir a vontade em cada
\emph{expressão} que envolve $x$ e $y$, uns $x$'s por $y$'s, e vice versa.
Nesse caso, começa com o
$$
a \ast x
$$
e troca a única instância de $x$ nessa expressão por $y$, e já chegamos no
$$
a \ast y
$$
ou seja, $a \ast x = a \ast y$.

\endexercise
%%}}}

%%{{{ refute the arbitrary cancellation law 
\exercise.
Refuta: para todo grupo $\sset G {e, \ast}$ e $a,x,y\in G$
$$
a\ast x = y\ast a \implies x=y
$$

\endexercise
%%}}}

%%{{{ ginv_is_a_function 
\note Inverso.
\label{ginv_is_a_function}%
Provando finalmente a unicidade dos inversos
(\refn{uniqueness_of_inverses_in_group}) ganhamos em cada grupo $G$ uma
\emph{função} (unária) de inverso
$$
\ginvt : G \to G.
$$
Sua \emph{totalidade} era já garantida pela~(G3);
e agora acabamos de ganhar sua \emph{determinicidade} com
o~\ref{uniqueness_of_inverses_in_group}.
Ou seja: \emph{função!}
Podemos finalmente definir uma notação para denotar o inverso de qualquer $a\in G$.
Bora!
%%}}}

%%{{{ df: ginv 
\definition.
\label{ginv}%
\sdefined {\ginv {\holed a}} {o inverso de $\holed a$ num grupo}
Seja $\sset G \ast$ grupo.  Para qualquer $a\in G$, definimos o
$$
\ginv a \defeq \text{o único inverso de $a$ no $G$}.
$$
%%}}}

\blah.
Tendo ganhado unicidade da identidade e dos inversos,
vamos responder agora em duas perguntas.

%%{{{ Q1: When is y the inv(a)? 
\question 1.
Num grupo $G$, dado $a\in G$,
o que precisamos mostrar para provar que um certo $y\in G$ é o inverso de $a$?
%%}}}

%%{{{ wrong answer 
\note Resposta errada.
Basta mostrar que $a \ast y = e$
(ou, alternativamente, que $y \ast a = e$)
pois, \emph{graças à unicidade dos inversos},
apenas um membro do grupo pode satisfazer essa equação,
e logo necessariamente $y = \ginv a$.
%%}}}

%%{{{ Q2: When is u the identity? 
\question 2.
Num grupo $G$, o que precisamos mostrar para provar que um certo $u\in G$ é a identidade do grupo?
%%}}}

%%{{{ wrong answer 
\note Resposta errada.
Basta achar um $a\in G$ tal que $a \ast u = a$
(ou, alternativamente, tal que $u \ast a = a$),
pois, \emph{graças à unicidade da identidade},
apenas um membro do grupo pode satisfazer essa equação,
e logo necessariamente $u = e$.
%%}}}

%%{{{ Where's the mistake? 
\note Cadê o erro?.
O \emph{raciocínio} nas duas respostas é errado numa maneira parecida:
\endgraf
Na (1), pode ser que $y$ satisfaz a $a\ast y = e$ sem $y$ ser o inverso do $a$.
\emph{E isso não violaria a unicidade do inverso $\ginv a$},
pois pela definição de \emph{inverso do $a$}, ambas equações $a \ast y = e = y \ast a$
precisam ser satisfeitas, e talvez $y \ast a \neq e$.
\endgraf
Na (2), pode ser que $u$ satisfaz $a \ast u = a$ para algum membro $a \in G$ sem $u$ ser a identidade do grupo.
\emph{E isso não violaria a unicidade da identidade $e$},
pois pela definição de \emph{identidade do $G$}, o $u$ precisa satisfazer ambas as $a \ast u = e = u \ast e$ e mesmo se satisfaria ambas isso não seria sufiziente: ele tem que as satisfazer não apenas \emph{para algum} $a\in G$ que deu certo, mas \emph{para todo} $a\in G$!
Ou seja: o fato que achamos \emph{algum} $a \in G$ tal que $a \ast u = a (= u \ast a)$
não quis dizer que esse $u$ merece ser chamado \emph{a identidade do $G$} ainda,
pois talvez tem membros $c \in G$ tais que $c \ast u \neq c$ ou $u \ast c \neq c$.
%%}}}

%%{{{ warning: wrong_reasoning_nimplies_wrong_claim_group_eg 
\warning.
\label{wrong_reasoning_nimplies_wrong_claim_group_eg}%
Os raciocínios em cima sendo errados não quis dizer que as afirmações também são!
Na verdade, nos dois casos podemos realmente ganhar o que queremos:
identidades e inversos \emph{mais baratos}, sem pagar todo o preço das definições.
Ambos resultados seguem como corolários diretos do~\ref{solution_of_group_equations}
que vamos provar daqui a pouco.
%%}}}

\blah.
Por enquanto, vamos continuar pesquisando o que mais podemos concluir assim
que tivermos um grupo~$G$, e voltaremos logo nessas duas questões.

%%{{{ Q: Can we conclude something about thses ?
\question.
Se $a,b$ são membros de algum grupo $\sset G \ast$, podemos concluir algo sobre os\dots:?
$$
\align
\ginvp{\ginv a}     &\askeq \dots?\\
\ginvp{a \ast b}    &\askeq \dots?
\endalign
$$
%%}}}
\spoiler.

%%{{{ lm: inverse_of_inverse_in_group 
\lemma Inverso de inverso.
\label{inverse_of_inverse_in_group}%
Em todo grupo $G$, $\ginvp {\ginv a} = a$ para todo $a\in G$.
\sketch.
Usamos as definições dos inversos involvidos para ganhar duas equações.
Com elas, chegamos na afirmação desejada.
\qes
\proof.
Temos
$$
\alignat2
\ginvp{\ginv a} \ast \ginv a &= e   \qqby{def.~$\ginvp{\ginv a}$}\\
a               \ast \ginv a &= e.  \qqby{def.~$\ginv a$}
\intertext{Logo}
\ginvp{\ginv a} \ast \ginv a &= a \ast \ginv a
\endalignat
$$
e cancelando agora pela direita (GCR), chegamos na desejada
$\ginvp{\ginv a} = a$.
\qed
%%}}}

%%{{{ x: cd 
\exercise.
Desenha um diagrama cuja comutatividade é a lei que tu acabou de provar.

\solution
$$
\cdopt{sep=2cm}
A   \ar[r, "\ginvt"]\ar[dr, "\id"'] \| A \ar[d, "\ginvt"] \\
                                    \| A
\endcd
$$

\endexercise
%%}}}

%%{{{ remark: where did the inversion come from 
\remark.
É comum adivinhar erroneamente que em geral
$\ginvp{a \ast b} = \ginv a \ast \ginv b$.
O erro é feito pois, acostumados com um certo grupo \emph{bem especial} como o
$\sset {\reals_{\neq0}} {\ntimes}$ onde essa lei realmente é valida,
generalizamos para o caso geral de grupos, sem perceber algo estranho e
esquisito que acontece nessa equação.  Repensanso em nosso exemplo-guia de
grupos, o~$S_3$, o que significa~$a \ast b$?
<<Faça a~$b$, depois a~$a$.>>
E o que signfica $\ginvp{a \ast b}$ então?
<<Desfaça a~$\paren{a \ast b}$.>>
E se aplicar uma transformação~$b$, e depois mais uma~$a$, qual seria o jeito
para desfazer tudo isso e voltar na configuração inicial?
<<Desfaça a~$a$, e depois desfaça a $b$.>>
Ou seja,~$\ginv b \ast \ginv a$.
Isso é bem natural sim: para desfazer uma seqüência de ações, começamos
desfazenso a última, depois a penúltima, etc., até finalmente desfazer
a primeira.
Sendo o inverso então, faz sentido que a ordem é a inversa também!
E nos reais, por quê não foi a inversa?
Foi sim!
É apenas que o~$\sset {\reals_{\neq0}} {\ntimes}$ é um grupo abeliano;
em outras palavras a ``ordem que acontecem os membros'' não importa.
Mas tudo isso é apenas uma \emph{intuição correta} para adivinhar essa lei.
Precisamos a provar.
%%}}}

%%{{{ lm: inverse_of_product_in_group 
\lemma Inverso de produto.
\label{inverse_of_product_in_group}%
Em todo grupo $G$, $\ginvp{a\ast b} = \ginv b \ast \ginv a$
para todo $a,b\in G$.
\sketch.
Queremos mostrar que $\ginv b \ast \ginv a$ é o inverso do $a \ast b$.
Mas o que <<ser o inverso do $a \ast b$>> significa?
Precisamos verificar que o $\ginv b \ast \ginv a$ satisfaz a definição:
$$
\paren{\ginv b \ast \ginv a} \ast \paren{a \ast b} = e = \paren{a \ast b} \ast \paren{\ginv b \ast \ginv a}.
$$
Agora só basta fazer esse cálculo mesmo.
\qes
\proof.
Calculamos
$$
\alignat2
\paren{\ginv b \ast \ginv a} \ast \paren{a \ast b}
&= \paren{\paren{\ginv b \ast \ginv a} \ast a} \ast b   \qqby{assoc.}\\
&= \paren{\ginv b \ast \paren{\ginv a \ast a}} \ast b   \qqby{assoc.}\\
&= \paren{\ginv b \ast e} \ast b                        \qqby{def.~$\ginv a$}\\
&= \paren{\ginv b} \ast b                               \qqby{def.~$e$}\\
&= e                                                    \qqby{def.~$\ginv b$}
\endalignat
$$
A $\paren{a \ast b} \ast \paren{\ginv b \ast \ginv a} = e$ é similar.
\qed
%%}}}

%%{{{ x: cd 
\exercise.
Desenha um diagrama cuja comutatividade é a lei que tu acabou de provar.

\hint
Podes começar com os conjuntos seguintes:
$$
\cdopt{column sep=1cm, row sep=2cm}
            \| G\times G \|           \| \| \| G \\
G\times G   \|           \| G\times G \| \| \|   \\
            \| G\times G \|           \| \| \| G
\endcd
$$
Só basta botar as setas (e seus nómes).

\hint
Já botei as setas pra ti.
Basta botas os nomes.
$$
\cdopt{column sep=6mm, row sep=2cm}
                        \| |[alias=N]| G\times G \|                      \| \| \| |[alias=NE]| G \\
|[alias=W]| G\times G   \|                       \||[alias=E]| G\times G \| \| \|                \\
                        \| |[alias=S]| G\times G \|                      \| \| \| |[alias=SE]| G
\ar[from=N,to=W]
\ar[from=N,to=E]
\ar[from=W,to=S]
\ar[from=E,to=S]
\ar[from=N,to=NE]
\ar[from=NE,to=SE]
\ar[from=S,to=SE]
\endcd
$$

\solution
Uma tal diagrama é o seguinte:
$$
\cdopt{column sep=6mm, row sep=2cm}
                        \| |[alias=N]| G\times G \|                      \| \| \| |[alias=NE]| G \\
|[alias=W]| G\times G   \|                       \||[alias=E]| G\times G \| \| \|                \\
                        \| |[alias=S]| G\times G \|                      \| \| \| |[alias=SE]| G
\ar[from=N,to=W,"\swap"']
\ar[from=N,to=E,"\ginvt\times\ginvt"]
\ar[from=W,to=S,"\ginvt\times\ginvt"']
\ar[from=E,to=S,"\swap"]
\ar[from=N,to=NE,"\ast"]
\ar[from=NE,to=SE,"\ginvt"]
\ar[from=S,to=SE,"\ast"]
\endcd
$$
onde $\swap(x,y) = \tup{y,x}$.

\endexercise
%%}}}

%%{{{ criterion: Cancellation-based group definition 
\criterion Definição de grupo com cancelação.
\label{cancellation_based_group_def}%
Seja $\ssetfont G = \sset G \ast$ um conjunto \emph{finito} estruturado
que satisfaz:
$$
\align
\paren{\forall a,b\in G}    &\bracket{a\ast b \in G}                    \tag{G0} \\
\paren{\forall a,b,c\in G}  &\bracket{a\ast(b\ast c) = (a\ast b)\ast c} \tag{G1} \\
\paren{\forall a,x,y\in G}  &\bracket{a\ast x = a\ast y \implies x=y}   \tag{GCL}\\
\paren{\forall a,x,y\in G}  &\bracket{x\ast a = y\ast a \implies x=y}   \tag{GCR}
\endalign
$$
Então $\ssetfont G$ é um grupo.
%%}}}

%%{{{ x: we_need_both_cancellation_laws_to_have_a_group 
\exercise.
\label{we_need_both_cancellation_laws_to_have_a_group}
Podemos apagar um dos (GCL), (GCR) das hipoteses do~\ref{cancellation_based_group_def}?

\endexercise
%%}}}

%%{{{ x: cancellation_based_group_def_only_valid_for_finite_G 
\exercise.
Mostre que não podemos apagar o ``finito'' das nossas hipoteses.

\hint
Não.  Procure contraexemplo!

\endexercise
%%}}}

%%{{{ lm: solution_of_group_equations 
\lemma Resolução de equações.
\label{solution_of_group_equations}%
Seja $G$ grupo.
Para quaisquer $a,b,x,y\in G$,
cada uma das equações abaixo tem resolução única para $x$ e $y$:
$$
\xalignat2
a\ast x &= b
&
y\ast a &= b
\endxalignat
$$
\sketch.
Aplicando o inverso de $a$ em cada equação pelo lado certo,
achamos que as soluções necessariamente são
$x = \ginv a \ast b$ e $y = b \ast \ginv a$.
\qes
\proof.
Aplicando $(a^{-1}\ast)$ nos dois lados da primeira
e $(\ast a^{-1})$ nos dois lados da segunda temos:
$$
\xalignat2
  a\ast x = b &\impliesbecause{$\ginv a\ast$} \ginv a\ast \paren{a\ast x} = a^{-1} \ast b 
& y\ast a = b &\impliesbecause{$\ast\ginv a$} \paren{x\ast a}\ast \ginv a = b \ast \ginv a
\\
&\implies\paren{\ginv a\ast a}\ast x = \ginv a \ast b
&&\implies y\ast \paren{a\ast \ginv a} = b \ast \ginv a
\\
&\implies e \ast x = \ginv a \ast b
&&\implies y\ast e = b \ast \ginv a
\\
&\implies x = \ginv a \ast b
&&\implies y = b \ast \ginv a
\endxalignat
$$
\qed
%%}}}

%%{{{ remark: ab_eq_c_each_determined_by_other_two 
\remark.
\label{ab_eq_c_each_determined_by_other_two}%
Isso quis dizer que dada uma equação $a \ast b = c$,
cada um dos $a,b,c$ é determinado pelos outros dois!
Assim, podemos \emph{definir} por exemplo o objeto $x$
como \emph{a única solução da} $a\ast x = b$, etc.
%%}}}

%%{{{ cor: cheaper_ginv 
\corollary Inverso mais barato.
\label{cheaper_ginv}%
Seja $G$ grupo e $a,y \in G$
tais que $a\ast y = e$ ou $y \ast a = e$.
Logo $y = \ginv a$.
%%}}}

%%{{{ cor: cheaper_gid 
\corollary Identidade mais barata.
\label{cheaper_gid}%
Seja $G$ grupo $u\in G$.
Se para algum $a\in G$, $au = a$ ou $ua = a$, então $u$ é a identidade do grupo:
$u = e$.
%%}}}

%%{{{ x: onesided_group_def 
\exercise Unilateral definição de grupo.
\label{onesided_group_def}%
Seja $\ssetfont G = \sset G {e, \ast}$ um conjunto estruturado que satisfaz:
$$
\align
\paren{\forall a,b\in G}                        &\bracket{a\ast b \in G}                    \tag{G0}  \\
\paren{\forall a,b,c\in G}                      &\bracket{a\ast(b\ast c) = (a\ast b)\ast c} \tag{G1}  \\
\paren{\forall a \in G}                         &\bracket{a\ast e = a}                      \tag{G2R} \\
\paren{\forall a\in G} \paren{\exists y\in G}   &\bracket{a \ast y = e}                     \tag{G3R} \\
\intertext{Mostre que $\ssetfont G$ é um grupo.
Comece verificando que mesmo se conseguir provar as}
\paren{\forall a \in G}                         &\bracket{e\ast a = a}                      \tag{G2L} \\
\paren{\forall a\in G} \paren{\exists a'\in G}  &\bracket{a' \ast a = e}                    \tag{G3L} 
\endalign
$$
isso não nos permite deduzir trivialmente as (G2) e (G3)!  Explique o por quê.

\hint
Para evitar as chances de ``roubar'' sem querer, use uma notação adaptada às novas leis:
chame $e_r$ a identidade-direita grantida pela (G2R);
$a_{R}$ para os inversos-direitos grantidos pela (G3R); e
$a_{L}$ para os inversos-esquerdos garantidos pela (G3L).

\hint
Precisas mostrar as (G2)--(G3).
Ou seja, verificar que o $e_r$ é uma identidade-esquerda,
$$
\align
\paren{\forall a\in G} & \bracket{e_r \ast a = a}           \tag{G2'}\\
\intertext{e que para todo $a\in G$,
seu inverso-direito $a_{R}$ é um inverso-esquerdo também:}
\paren{\forall a\in G} &\bracket{a_{R} \ast a = e}.          \tag{G3'}
\endalign
$$

\hint
Para provar as (G2') e (G3'),
tome um $a\in G$ e comece com:
$$
\xalignat2
e \ast a &= (a \ast a_R) \ast a     & e \ast a &= (e \ast a) \ast e\\
         &= \dotsb                  &          &= e \ast (a \ast e)\\
         &=                         &          &= e \ast (a \ast (\;?\;))\\
         &=                         &          &= \dotsc
\endxalignat
$$

\hint
Espero que já conseguiu a (G2').
Para a (G3'), qual produto podes botar no lugar do $(\;?\;)$ da dica anterior?
Sabendo que (G2G) é válida, faz sentido o substituir com um termo
$\paren{x \ast x_R}$ para algum $x\in G$, mas qual seria esse $x$?
\emph{Não precisamos adivinhar ainda!}
Continue assim com um $x$ não-especificado por enquanto,
e logo tu vai chegar numa expressão que vai te ajudar escolher teu $x$ para
continuar.

\endexercise
%%}}}

%%{{{ x: splitsided_group_notdef 
\exercise.
\label{splitsided_grou_notdef}%
Podemos substituir a (G3R) do~\ref{onesided_group_def} por
$$
\align
\paren{\forall a\in G} \paren{\exists a'\in G}  &\bracket{a' \ast a = e}                    \tag{G3L} \\
\endalign
$$
e ainda concluir que $\ssetfont G$ é grupo?

\hint
Não!

\hint
Procure um contraexemplo.
(Claramente, a operação não pode ser comutativa.)

\endexercise
%%}}}

\endsection
%%}}}

%%{{{ Cayley tables 
\section Tabelas Cayley.

%%{{{ Q: In how many ways can we define an operation to create a finite group? 
\question.
De quantas maneiras podemos definir uma operação binária $\ast$ num conjunto
finito $G$, tal que $\sset G \ast$ é um grupo?
%%}}}
\spoiler.

%%{{{ What determines an operation 
\note O que determina uma operação.
O que significa \emph{definir uma operação} (binária)?
Seguindo nossa definição de igualdade (extensional) entre funções,
precisamos deixar claro para qualquer $\tup{x,y}\in G\times G$,
seu único valor $x \ast y \in G$.
Vamos brincar com os casos mais simples.
Se $\card G = 0$, não tem como virar um grupo, graças ao~(G2).
Se $\card G = 1$, só tem uma operação possível, pois não existe
nenhuma opção para o valor $e \ast e$: necessariamente $e \ast e = e$.
E essa opção realmente vira-se o $\sset G \ast$ um grupo (trivial).
%%}}}

%%{{{ 234_groups 
\note Os casos 2,3,4.
Vamos dizer que temos um conjunto $G$ com $\card G = 4$.
Não sabemos nada sobre seus membros, podem ser números, letras, pessoas,
funções, conjuntos, sapos, sei-lá.
$$
G = \set{ \bullet, \bullet, \bullet, \bullet }
$$
Então faz sentido começar nossa investigação dando uns nomes para
esses membros, por exemplo $a,b,c,d$, onde não vamos supor nada mais
sobre eles exceto que são distintos dois a dois.
$$
\text{<<Sejam $G\eqass\set{a,b,c,d}$.>>}
$$
Sera que podemos fazer algo melhor?
Querendo tornar o $G$ em grupo, sabemos que ele vai ter exatamente uma
identidade, então melhor denotar com $e$ sua a identidade,
e escolher nomes para os outros três membros do $G$:
$$
G = \set {e, a, b, c}.
$$
Similarmente, caso que $\card G = 2$ ou $3$, teremos
$G = \set {e, a}$
ou
$G = \set {e, a, b}$
respectivamente.
%%}}}

%%{{{ Cayley tables 
\note Tabelas Cayley.
\tdefined{tabela!Cayley}%
Temos então que ver o que podemos botar nos $\faded?$ para completar
as \dterm{tabelas Cayley}\Cayley[tabela]{} em baixo:
$$
\xalignat3
&\matrix
\ast& e & a \\
e   & \faded? & \faded? \\
a   & \faded? & \faded?
\endmatrix
&
&\matrix
\ast& e & a & b \\
e   & \faded? & \faded? & \faded? \\
a   & \faded? & \faded? & \faded? \\
b   & \faded? & \faded? & \faded?
\endmatrix
&
&\matrix
\ast& e & a & b & c \\
e   & \faded? & \faded? & \faded? & \faded? \\
a   & \faded? & \faded? & \faded? & \faded? \\
b   & \faded? & \faded? & \faded? & \faded? \\
c   & \faded? & \faded? & \faded? & \faded?
\endmatrix
\endxalignat
$$
Mas, não todos os $\faded?$ realmente representam uma escolha,
pois certos deles são determinados; e cada vez que fazemos uma escolha
para um deles, possivelmente nossas opções próximas deminuiam.
Para começar, como $e\ast x = x = x \ast e$ para qualquer $x$ do grupo,
a primeira linha e a primeira coluna da tabela já são determinadas:%
\footnote{De fato, foi por isso que Cayley realmente nem escreveu a coluna e a linha ``exterior'', tomando a convenção que o elemento que aparece primeiro é a sua identidade.
Para um grupo de três elementos então, ele começaria assim:
$$
\matrix
a & b       & c       \\
b & \faded? & \faded? \\
c & \faded? & \faded?
\endmatrix
\qquad
\text{que, seguindo nossa convenção com `$e$', escreveríamos}
\qquad
\matrix
e & a       & b       \\
a & \faded? & \faded? \\
b & \faded? & \faded?
\endmatrix
$$
}
$$
\xalignat3
&\matrix
\ast& e & a \\
e   & e & a \\
a   & a & \faded?
\endmatrix
&
&\matrix
\ast& e & a & b \\
e   & e & a & b \\
a   & a & \faded? & \faded? \\
b   & b & \faded? & \faded?
\endmatrix
&
&\matrix
\ast& e & a & b & c \\
e   & e & a & b & c \\
a   & a & \faded? & \faded? & \faded? \\
b   & b & \faded? & \faded? & \faded? \\
c   & c & \faded? & \faded? & \faded?
\endmatrix
\endxalignat
$$
%%}}}

%%{{{ Q: how can we replace the blanks? 
\question.
O que tu podes botar nos $\faded?$ para chegar em grupo?
O que muda se tu queres construir um grupo abeliano?
%%}}}
\spoiler.

%%{{{ 2 members 
\note 2 membros.
Vamos começar no caso mais simples, onde temos apenas um $\faded?$ para preencher.
Em teoria temos duas opções: $e,a$.  Mas precisamos verificar se o conjunto
realmente torna-se um grupo ou não.  Escolha $a$:
$$
\matrix
\ast& e & a \\
e   & e & a \\
a   & a & \alert{a}
\endmatrix
$$
Qual seria o inverso do $a$?
Nenhum!
Assim o~(G3) será violado, ou seja, não podemos escolher o $a$.
Se escolher nossa única outra opção ($e$) temos:
$$
\matrix
\ast& e & a \\
e   & e & a \\
a   & a & \alert{e}
\endmatrix
$$
Que realmente é um grupo.
%%}}}

%%{{{ x 
\exercise.
Verifique!

\endexercise

%%}}}

%%{{{ x: find_all_groups_of_order_3 
\exercise.
\label{find_all_groups_of_order_3}%
Ache todas as operações possíveis que tornam um conjunto com $3$ elementos um grupo.

\hint
Só tem uma maneira.  Qual?  Por quê?

\endexercise
%%}}}

%%{{{ Playing ``grupoku'' 
\note Jogando ``Grupoku''.
\tdefined{Grupoku}%
Investigar todas as possíveis escolhas para os~$\faded?$ parece como um jogo
de Sudoku, só que nossa restrição não é com a soma dos números de cada linha e
cada coluna como no Sudoku---nem poderia ser isso: nossos membros possivelmente
nem são números---mas as leis~(G0)--(G3) que tem que ser satisfeitas.
E caso que queremos criar um grupo abeliano, a~(GA) também.
%%}}}

%%{{{ Find the rules of grupoku 
\exercise.
\label{find_the_rules_of_grupoku}%
Que restrições pode afirmar que temos nesse jogo de ``Grupoku'', graças todos os resultados
que temos provado até agora sobre grupos?
E se queremos um grupo abeliano, muda o quê?

\endexercise
%%}}}

%%{{{ x: find_all_groups_of_order_4 
\exercise.
\label{find_all_groups_of_order_4}%
Ache todas as operações possíveis que tornam um conjunto com $4$ elementos
um grupo.

\hint
\emph{Essencialmente} são apenas $2$.
Se achar mais, verifique que renomeando seus membros umas viram iguais, e só
tem dois que realmente não tem como as identificar, mesmo renomeanos seus
membros.
Tudo isso vai fazer bem mais sentido daqui umas secções onde vamos estudar
o conceito de isomorfia~(\refn{Group_morphisms}).

\endexercise
%%}}}

%%{{{ x: find_all_groups_of_order_0_and_1 
\exercise.
\label{find_all_groups_of_order_0_and_1}%
Tem grupos com ordem 1?  Com 0?

\solution
Com ordem $1$ sim.  Cada um tem a mesma forma: seu único elemento é sua identidade.
Com ordem $0$, não: a lei (G2) \emph{manda a existência} de um certo membro do grupo (a sua identidade).

\endexercise
%%}}}

\endsection
%%}}}

%%{{{ Powers and orders 
\section Potências e ordens.

%%{{{ df: exp_in_group 
\definition.
\label{exp_in_group}%
\sdefined {{\holed a}^{\holed m}} {$a \ast \dotsb \ast a$ ($m$ vezes)}
Seja $a$ elemento dum grupo $\sset G {e, \ast}$.
Definimos suas potências $a^n$ onde $n\in\nats$ recursivamente:
$$
\align
a^0     &\defeq e\\
a^{n+1} &\defeq a \ast a^n
\endalign
$$
%%}}}

%%{{{ x: exp_in_group_altdef 
\exercise.
Demostre que a definição alternativa de exponenciação ao natural
$$
\align
a^0     &= e\\
a^{n+1} &= a^n \ast a
\endalign
$$
é equivalente.

\hint%%{{{
Como o proprio ``operador'' de exponenciar não aparece na notação comum $a^b$,
melhor escrever temporariamente as duas definições como:
$$
\xalignat2
a \uparrow_1 0 &= e &
a \uparrow_2 0 &= e \\
a \uparrow_1 (n+1) &= a \ast (a \uparrow_1 n) &
a \uparrow_2 (n+1) &= (a \uparrow_2 n) \ast a
\endxalignat
$$
Agora tem uma notação melhor para provar o que queremos: ${\uparrow_1} = {\uparrow_2}$.
O que significa que duas operações (funções) são iguais?
%%}}}

\hint%%{{{
Precisamos mostrar que:
$$
\text{para todo $a\in G$ e todo $n\in\nats$,}\quad
a \uparrow_1 n = a \uparrow_2 n
$$
ou, simbolicamente:
$$
\align
\paren{\forall a \in G}
&\lforall {n\in\nats} {a \uparrow_1 n = a \uparrow_2 n}.
\intertext{\emph{Seja $a\in G$.}  Agora queremos provar:}
&\lforall {n\in\nats} {a \uparrow_1 n = a \uparrow_2 n}.
\endalign
$$
Como provar isso?
%%}}}

\hint%%{{{
As definições envolvidas são recursivas.
%%}}}

\hint%%{{{
Ou seja: indução.
%%}}}

\hint%%{{{
\casestyle{Base:} provar que $a \uparrow_1 0 = a \uparrow_2 0$.
%%}}}

\hint%%{{{
Seja $k\in\nats$ tal que $a \uparrow_1 k = a \uparrow_2 k$ (hipótese indutiva).
Queremos mostrar que
$$
a \uparrow_1 (k+1) = a \uparrow_2 (k+1).
$$
%%}}}

\hint%%{{{
Seguinto as dicas anteriores, provavelmente tu chegou aqui:
$$
\alignat2
a \uparrow_1 (k+1)
&= a \ast (a \uparrow_1 k)   \qqby{def.~$\uparrow_1$}\\
&= a \ast (a \uparrow_2 k)   \qqby{HI}
\intertext{\dots e agora?
\emph{Se} $\ast$ fosse comutativa (ou seja, se o grupo fosse abeliano),
a gente \emph{poderia} continuar assim:}
&= (a \uparrow_2 k) \ast a   \qqby{comutatividade~(GA)}\\
&= a \uparrow_2 (k+1).       \qqby{def.~$\uparrow_2$}
\intertext{\emph{So que não!}
Sobre o $G$ sabemos apenas que é um grupo, então não podemos contar na
comutatividade da sua operação $\ast$.
Voltando no passo que tivemos colado}
a \uparrow_1 (k+1)
&= a \ast (a \uparrow_1 k)   \qqby{def.~$\uparrow_1$}\\
&= a \ast (a \uparrow_2 k)   \qqby{HI}
\endalignat
$$
percebemos que precisamos ``abrir mais'' a expressão $(a \uparrow_2 k)$,
aplicando a definição de $\uparrow_2$, mas não podemos, pois não sabemos se $k=0$ ou não.
Neste momento então percebemos que saber a vericidade dessa equação apenas para o valor $n=k$ não é suficiente.
Tu vai precisar o $n=k-1$ também.
\endgraf
Ou seja, tu vai precisar \emph{duas} bases ($n=0,1$),
e supor que tem um $k \geq 2$ tal que ambos os $k-1$ e $k-2$ satisfazem a
$a \uparrow_1 n = a \uparrow_2 n$.
Ou seja, tu ganharás as \emph{duas} hipoteses indutivas:
$$
\align
a \uparrow_1 (k-1) &= a \uparrow_2 (k-1) \tag{HI1}\\
a \uparrow_1 (k-2) &= a \uparrow_2 (k-2) \tag{HI2}
\endalign
$$
e só bastará provar que $a \uparrow_1 k = a \uparrow_2 k$.
%%}}}

\solution%%{{{
Seja $a \in G$.
Vamos provar que para todo $n\in\nats$, $a \uparrow_1 n = a \uparrow_2 n$
por indução no $n$.
\endgraf\noindent
\casestyle{Bases $n=0,1$:}
Temos
$$
\alignedat2
a \uparrow_1 0
&= e               \qqby{def.~$\uparrow_1$} \\
&= a \uparrow_2 0  \qqby{def.~$\uparrow_2$}
\endalignedat
\qqqquad
\alignedat2
a \uparrow_1 1
&= a \ast (a \uparrow_1 0)  \qqby{def.~$\uparrow_1$} \\
&= a \ast e                 \qqby{def.~$\uparrow_1$} \\
&= a                        \qqby{def.~$e$} \\
&= e \ast a                 \qqby{def.~$e$} \\
&= (a \uparrow_2 0) \ast a  \qqby{def.~$\uparrow_2$} \\
&= a \uparrow_2 1           \qqby{def.~$\uparrow_2$} \\
\endalignedat
$$
\noindent
\casestyle{Passo indutivo:}
Seja $k\in\nats$, tal que $k\geq 2$ e:
$$
\align
a \uparrow_1 (k-1) &= a \uparrow_2 (k-1) \tag{HI1}\\
a \uparrow_1 (k-2) &= a \uparrow_2 (k-2).\tag{HI2}
\endalign
$$
Precisamos provar que $a \uparrow_1 k = a \uparrow_2 k$.
Calculamos:
$$
\alignat2
a \uparrow_1 k
&= a \ast \paren{a \uparrow_1 (k-1)}                \qqby{def.~$\uparrow_1$} \\
&= a \ast \paren{a \uparrow_2 (k-1)}                \qqby{HI1} \\
&= a \ast \paren{\paren{a \uparrow_2 (k-2)} \ast a} \qqby{def.~$\uparrow_2$} \\
&= a \ast \paren{\paren{a \uparrow_1 (k-2)} \ast a} \qqby{HI2} \\
&= \paren{a \ast \paren{a \uparrow_1 (k-2)}} \ast a \qqby{associatividade~(G1)} \\
&= \paren{a \uparrow_1 (k-1)} \ast a                \qqby{def.~$\uparrow_1$} \\
&= \paren{a \uparrow_2 (k-1)} \ast a                \qqby{HI1} \\
&= a \uparrow_2 k.                                  \qqby{def.~$\uparrow_2$}
\endalignat
$$
Pelo principio da indução segue que para todo $n\in\nats$,
$a \uparrow_1 n = a \uparrow_2 n$.
Como $a$ foi arbitrário membro de $G$, isso termina nossa prova que
${\uparrow_1} = {\uparrow_2}$.
%%}}}

\endexercise
%%}}}

%%{{{ x: sq_of_prod_not_prod_of_sq 
\exercise.
\label{sq_of_prod_not_prod_of_sq}%
Mostre que, \emph{em geral}, $(a \ast b)^2 \neq a^2 \ast b^2$.

\endexercise
%%}}}

\beware.
Neste momento então, para qualquer grupo $G$ e qualquer $a\in G$,
o símbolo $a^w$ é definido \emph{apenas} para $w \asseq -1, 0, 1, 2, \dots$
e nada mais!
Vamos agora estender para os valores $w \asseq -2, -3, -4, \dots$
num jeito razoável.

%%{{{ Q: what do you think a^{-2} should mean? 
\question.
O que você acha que deveria ser denotado por $a^{-2}$?
%%}}}
\spoiler.

%%{{{ A 
\note Resposta.
Bem, tem duas interpretações, ambas razoáveis:
$$
a^{-2} =
\knuthcases{
\paren{a^{-1}}^2  & (o quadrado do inverso do $a$)\cr
\paren{a^2}^{-1}  & (o inverso do quadrado do $a$)
}
$$
Prove que as duas interpretações são equivalentes, ou seja:
%%}}}

%%{{{ x: inv_of_square_eq_square_of_inv 
\exercise.
\label{inv_of_square_eq_square_of_inv}%
Seja $G$ grupo.  Para todo $a\in G$,
$\paren{a^{-1}}^2 = \paren{a^2}^{-1}$.
Ou seja, o diagrama
$$
\cdopt{sep=2cm}
G \ar[r, "\ginv\bhole"] \ar[d, "\bhole^2"] \| G \ar[d, "\bhole^2"]\\
G \ar[r, "\ginv\bhole"]                    \| G
\endcd
$$
comuta.

\hint
Seja $a\in G$.
O que tu precisas mostrar, é que $\paren{\ginv a}^2$ é o inverso do $a^2$.
O que significa <<ser o inverso do $a^2$>>?

\hint
$
\paren{\ginv a}^2 \ast {a^2}
\askeq e
\askeq {a^2} \ast \paren{\ginv a}^2
$.

\solution
Graças ao~\ref{cheaper_ginv}, basta provar que $\paren{\ginv a}^2 \ast a^2 = e$.
Calculamos então:
$$
\alignat2
\paren{a^{-1}}^2 \ast {a^2}
&= (\ginv a \ast \ginv a) \ast {a^2}             \qqby{def.~$\paren{\ginv a}^2$}\\
&= (\ginv a \ast \ginv a) \ast \paren{a\ast a}   \qqby{def.~$a^2$}\\
&= \paren{(\ginv a \ast \ginv a) \ast a}\ast a   \qqby{ass.}\\
&= \paren{\ginv a \ast (\ginv a \ast a)}\ast a   \qqby{ass.}\\
&= \paren{\ginv a \ast e}\ast a                  \qqby{def.~$\ginv a$}\\
&= \ginv a \ast a                                \qqby{def.~$e$}\\
&= e                                             \qqby{def.~$\ginv a$}
\endalignat
$$
Logo $\paren{\ginv a}^2$ é o inverso de $a^2$.

\endexercise
%%}}}

%%{{{ x: inv_of_npow_eq_npow_of_inv 
\exercise.
\label{inv_of_npow_eq_npow_of_inv}%
Generalize o~\ref{inv_of_square_eq_square_of_inv} para $n\in\nats$:
\emph{para todo grupo $G$ e todo $n\in\nats$, se $a\in G$ então
$\paren{a^{-1}}^n = \paren{a^n}^{-1}$}.

\hint
As ``potências'' $\bhole^n$ de elementos de grupo foram definidas \emph{recursivamente}.

\hint
Indução.

\endexercise
%%}}}

%%{{{ df: negexp_in_group 
\definition.
\label{negexp_in_group}%
Seja $a$ elemento dum grupo $\sset G {e, \ast}$.
Definimos para todo $n\in\nats_{>0}$.
$$
a^{-n} \defeq \paren{a^{-1}}^n
$$
%%}}}

%%{{{ x: negexp_in_group_altdef 
\exercise.
Demostre que a definição alternativa de exponenciação ao inteiro negativo
$$
a^{-n} \defeq \paren{a^n}^{-1}
$$
é equivalente; ou seja prove que para todo $n\in\nats$,
$$
\paren{a^{-1}}^n = \paren{a^n}^{-1}.
$$

\endexercise
%%}}}

%%{{{ property: properties_of_exp_in_groups 
\property Potências.
\label{properties_of_exp_in_groups}%
Sejam $G$ grupo, $a \in G$, e $m,n\in\ints$.
Temos:
\item{\rm (1)} $a^m \ast a^n = a^{m+n}$;
\item{\rm (2)} $(a^m)^n = a^{m\ntimes n}$;
\item{\rm (3)} $a^{-n} = \paren{a^n}^{-1} = \paren{a^{-1}}^n$.
\sketch.
Provamos primeiro para $m,n\in\nats$ por indução.
Depois consideramos os casos de inteiros negativos.
\qes
%%}}}

%%{{{ df: order_of_member_in_group 
\definition Ordem de membro em grupo.
\label{order_of_member_in_group}%
\tdefined{ordem!de membro em grupo}%
\sdefined {\gord {\holed a}} {a ordem do elemento $\holed a$ num grupo}%
Seja $\sset G {e,\ast}$ grupo e $a\in G$.
Chamamos \dterm{ordem} de $a$ o menor positivo $n\in\nats$ tal que
$a^n = e$, se tal $n$ existe; caso contrário, dizemos que o $a$ tem ordem infinita.
Usamos a mesma notação como no caso de ordem de grupos:
$\gord a$, $\tord a$, ou $\bord a$, com os mesmos cuidados.
Logo:
$$
\gord a =
\knuthcases{
\min\set{ n\in\nats_{>0} \st a^n = e }, & se $\set{ n\in\nats_{>0} \st a^n = e }\neq\emptyset$\cr
\infty,                                 & caso contrário.
}
$$
%%}}}

%%{{{ eg: orders_of_members_of_S3 
\example.
\label{orders_of_members_of_S3}%
No $S_3$, temos
$$
\xalignat3
\gord {\id} &= 1 & \gord {\phi}         &= 2  &  \gord {\psi}   &= 3\\
            &    & \gord {\phi\com\psi} &= 2  &  \gord {\psi^2} &= 3.\\
            &    & \gord {\psi\com\phi} &= 2  &  
\endxalignat
$$
\endexample
%%}}}

%%{{{ x: verify the numbers of orders_of_members_of_S3 
\exercise.
Verifique os números do~\ref{orders_of_members_of_S3}.

\endexercise
%%}}}

%%{{{ x: non-zero integer power = e guarantees positive one 
\exercise.
Seja $G$ grupo e $a\in G$.
Se existe $m\in\ints_{\neq 0}$ tal que $a^m = e$, então $\gord a < \infty$.

\hint
Precisamos mostrar que existe $n\in\nats_{>0}$ com $a^n = e$.

\hint
Se $m>0$, tome $n\leteq m$.  Se não?

\solution%%{{{
Precisamos mostrar que existe $n\in\nats_{>0}$ com $a^n = e$.
Se $m>0$, o conjunto $N \leteq \set{ n\in\nats_{>0} \st a^n = e}$ não é vazio,
então pelo princípio da boa ordem~\ii{princípio!da boa ordem}(PBO)
possui elemento mínimo e logo
$\gord a = \min N < \infty$
Se $m<0$, observe que $-m > 0$, e calcule:
$$
a^{-m} = \ginvp{a^m} = \ginv e = e,
$$
e novamente
$N\neq\emptyset$ e $\gord a < \infty$.
%%}}}

\endexercise
%%}}}

%%{{{ lm: a_has_exactly_gord_a_powers 
\lemma.
\label{a_has_exactly_gord_a_powers}%
Sejam $G$ grupo e $a\in G$, e suponha $\gord a = n\in\nats$.
Existem exatamente $n$ potências diferentes de $a$.
\sketch.
As potências de $a$ são as:
$$
\dotsc, a^{-2}, a^{-1}, a^0, a^1, a^2, \dotsc, a^{n-1}, a^n, a^{n+1}, a^{n+2}, \dotsc
$$
Queremos provar que, no final das contas, nesta lista aparecem exatemente $n$
distintos membros de $G$.  Ou seja,
$$
\card{\set{ a^k \st k \in \ints }} = n.
$$
Consideramos os
$$
a^0, a^1, \dotsc, a^{n-1}
$$
e usando a definição de $\gord a$ provamos:
\endgraf\noindent
\casestyle{Existência:} os $a^0,\dotsc,a^{n-1}$ são distintos dois a dois.
\endgraf\noindent
Ou seja: para todo $i,j\in \set{0,\dotsc,n-1}$ com $i\neq j$, temos $a^i \neq a^j$.
\endgraf\noindent
\casestyle{Unicidade:} para todo $M\in\ints$ o $a^M$ é um dos $a^0,\dotsc,a^{n-1}$.
\endgraf\noindent
Sabemos diretamente pela sua definição que $a^0 = e$, e que $a^n = e$, pois $\gord a = n$.
Com pouca imaginação chegamos na idéia que a cadeia de membros
$$
\alignat{12}
&\dotsc,\ && a^{-2},\ && a^{-1},\ && a^0,\ && a^1,\ && a^2,\ && \dotsc,\ && a^{n-1},\ && a^n,\ && a^{n+1},\ && a^{n+2},\ && \dotsc\\
\intertext{é feita por uma copia de $a^0, \dotsc, a^{n-1}$ se repetindo infinitamente para as duas direções:}
&\dotsc,\ && a^{n-2},\ && a^{n-1},\ && a^0,\ && a^1,\ && a^2,\ && \dotsc,\ && a^{n-1},\ && a^0,\ && a^1,\ && a^2,\ && \dotsc
\endalignat
$$
Aplicando a divisão de \Euclid[divisão]Euclides~(\ref{euclidean_division})
no~$M$ por~$n$, ganhamos $q,r\in\ints$ tais que:
$$
M = q\ntimes n + r,\qquad 0 \leq r < n.
$$
Só basta calcular o $a^M$ para verificar que realmente é um dos $a^0, \dotsc, a^{n-1}$.
\qes
\proof.
\casestyle{Existência:}
existem $n$ potências de $a$.
\endgraf\noindent
Provamos isso demonstrando que os $a^0,\dotsc,a^{n-1}$ são distintos dois a dois.
Sejam $i,j\in \set{0,\dotsc,n-1}$ com $i\neq j$.
\emph{Sem perda de generalidade}, suponha que $i<j$, ou seja:
$$
0 \leq i < j < n.
$$
Para chegar num absurdo, suponha que $a^i = a^j$.
Assim temos:
$$
\align
\munderbrace{aa\dotsb a} i &= \munderbrace{aa\dotsb aaa\dotsb a} j
\intertext{E como $i<j$, quebramos o lado direito assim:}
\munderbrace{aa\dotsb a} i &= \munderbrace{\moverbrace{aa\dotsb a} i \moverbrace{aa\dotsb a} {j-i}} j
\intertext{Ou seja,}
a^i &= a^i a^{j-i}
\intertext{e multiplicando ambos lados pela esquerda por $\ginvp{a^i}$ chegamos no}
e &= a^{j-i}.
\endalign
$$
Achamos então uma potência de $a$ igual à identidade $e$: $a^{j-i} = e$.
Como $0 < j-i < n$, isso contradiza a definição de $n$ como a ordem de $a$:
$n = \gord a$.
Logo $a^i \neq a^j$, que foi o que queremos provar.
\endgraf\noindent
\casestyle{Unicidade:}
os $a^0,\dotsc,a^{n-1}$ são as \emph{únicas} potências de $a$.
Ou seja, para todo $M\in\ints$ o $a^M$ é um dos $a^0,\dotsc,a^{n-1}$.
\endgraf\noindent
Tome $M\in \ints$.
Aplicando a divisão de \Euclid[divisão]Euclides~(\ref{euclidean_division})
no~$M$ por~$n$, ganhamos $q,r\in\ints$ tais que:
$$
M = q\ntimes n + r,\qquad 0 \leq r < n.
$$
Só basta calcular o $a^M$ para verificar que realmente é um dos $a^0, \dotsc, a^{n-1}$:
$$
a^M
= a^{q\ntimes n + r}
= a^{q\ntimes n} a^r
= \paren{a^n}^q a^r
= e^q a^r
= e a^r
= a^r
$$
e como $0\leq r < n$, provamos o que queriamos provar.
\qed
%%}}}

\question.
Se $a^m = e$ para algum $m\in\ints$, o que podemos concluir sobre o $m$ e a $\gord a$?
Se $\gord a = \infty$, o que podemos concluir sobre todas as potências de $a$?
\spoiler.

%%{{{ lm: a_exp_m_is_e_iff_gord_a_divides_m 
\lemma.
\label{a_exp_m_is_e_iff_gord_a_divides_m}%
Sejam $\sset G \ast$ grupo, $a\in G$, e $m\in\ints$.
Logo
$$
a^m = e \iff \gord a \divides m.
$$
\proof.
\rldir:
Como $\gord a \divides m$,
temos $m = k\gord a$ para algum $k\in\ints$.
Calculamos:
$$
a^m = a^{k\gord a} = a^{\gord a k} = \paren{a^{\gord a}}^k = e^k = e.
$$
\endgraf
\lrdir:
Para provar que $\gord a \divides m$,
aplicamos o~\ref{euclidean_division} da divisão de Euclides\Euclid[lema da divisão],
dividindo o $m$ por $\gord a$, e ganhando assim inteiros $q$ e $r$ tais que
$$
m = \gord a q + r
\qquad
0\leq r < \gord a.
$$
Vamos provar que o resto $r=0$.
Calculamos:
$$
e
= a^m
= a^{\gord a q + r}
= a^{\gord a q} \ast a^r
= \paren{a^{\gord a}}^q \ast a^r
= e^q \ast a^r
= e \ast a^r
= a^r.
$$
Ou seja, $a^r = e$ com $0\leq r < \gord a$, então pela definição de $\gord a$
como o mínimo inteiro positivo $n$ que satisfaz a $a^n = e$,
o $r$ não pode ser positivo.
Logo $r=0$ e $\gord a \divides m$.
\qed
%%}}}

%%{{{ lm: infinite_order_of_a_guarantees_all_powers_distinct 
\lemma.
\label{infinite_order_of_a_guarantees_all_powers_distinct}%
Sejam $G$ grupo e $a\in G$.  Se $\gord a = \infty$,
então as potências de $a$ são distintas dois a dois.
\sketch.
Precisamos provar que para todo $r,s\in\ints$,
$$
a^r = a^s \implies r = s.
$$
Sem perda de generalidade suponhamos $s\leq r$ e cancelamos pela direita
o $a^s$.  Usando a definição de $\gord a$ concluimos que $r-s=0$ e logo $r=s$.
\qes
%%}}}

\endsection
%%}}}

%%{{{ Subgroups 
\section Subgrupos.

%%{{{ df: subgroup 
\definition Subgrupo.
\label{subgroup}%
\tdefined{subgrupo}%
\tdefined{subgrupo!trivial}%
\sdefined {\holed H \subgroup \holed G} {$\holed H$ é um subgrupo de $\holed G$}
Seja $\sset G {e, \ast}$ grupo.
Um subconjunto $H\subseteq G$ é um \dterm{subgrupo} de $G$ sse
$H$ é um grupo com a mesma operação $\ast$.
Escrevemos $H \subgroup G$.
Chamamos os $\set e$ e $G$ \dterm{subgrupos triviais} de $G$.
%%}}}

%%{{{ x: singleton_e_is_a_subgroup 
\exercise.
\label{singleton_e_is_a_subgroup}%
Verifique que para todo grupo $\sset G {e, \ast}$, temos $\set e \subgroup G$.

\endexercise
%%}}}

%%{{{ eg: Numbers 
\example Números reais.
(1) Considere o grupo $\sset \reals {+}$.
Observe que $\rats$ e $\ints$ são subgrupos dele, mas $\nats$ não é.
\endgraf\noindent
(2) Considere o grupo $\sset {\reals_{\neq0}} {\ntimes}$.
Observe que $\set{1, -1}$, $\rats_{\neq0}$, e para qualquer
$\alpha\in\reals_{\neq0}$ o conjunto $\set{ \alpha^k \st k\in\ints }$ são todos
subgrupos dele, mas $\ints$ e $\set{ \alpha^n \st n\in\nats }$ não são.
\endexample
%%}}}

%%{{{ x: rationals 
\exercise.
Considere o grupo $\ssetfont Q \leteq \sset {\rats\setminus\set0} {\ntimes}$
e seus subconjuntos:
$$
\align
Q_1 &\leteq \setst { p/q } { p, q \in \ints,\ \text{$p$ e $q$ ímpares} }\\
Q_2 &\leteq \setst { p/q } { p, q \in \ints,\ \text{$p$ ímpar, $q$ par} }\\
Q_3 &\leteq \setst { p/q } { p, q \in \ints,\ \text{$p$ par, $q$ ímpar} }.
\endalign
$$
Para quais dos $i=1,2,3$ temos $Q_i \subgroup \ssetfont Q$?

\endexercise
%%}}}

%%{{{ property: empty_is_never_a_subgroup 
\property.
\label{empty_is_never_a_subgroup}%
$H\subgroup G \implies H \neq\emptyset$.
\proof.
Como $H$ é um grupo, necessariamente $e\in H$.
\qed
%%}}}

%%{{{ x: nontrivial_subgroups_of_multiplicative_reals 
\exercise.
\label{nontrivial_subgroups_of_multiplicative_reals}%
Ache uns subgrupos não-triviais do
$\sset {\reals\setminus\set0} {\ntimes}$?

\endexercise
%%}}}

%%{{{ remark: ass_for_free_in_subgroup
\remark Associatividade de graça.
\label{ass_for_free_in_subgroup}%
Seja $\sset G \ast$ grupo, e tome um $H\subseteq G$.
Para ver se $H \subgroup G$, seguindo a definição,
precisamos verificar as (G0)--(G3) no $\sset H \ast$.
Mas a lei~(G1) da associatividade não tem como ser violada no $\sset H \ast$.
O que significaria violar essa lei?
$$
\text{Teriamos $a,b,c \in H$, tais que $a\ast (b\ast c) \neq (a \ast b) \ast c$.}
$$
Mas como $H \subseteq G$, nos teriamos o mesmo contraexemplo para a (G1) de $G$,
impossível pois $G$ é um grupo mesmo (e a operação é a mesma).
Logo, jamais precisaramos verificar a~(G1) para um possível subgrupo.
%%}}}

\blah.
E isso não é o único ``desconto'' que temos quando queremos provar
que~$H \subgroup G$.  Vamos ver mais dois critéria agora:

%%{{{ criterion: subgroup_criterion 
\criterion Subgrupo.
\label{subgroup_criterion}%
Se $H$ é um subconjunto não vazio do grupo $\sset G {e,\ast}$, então:
$$
\left.
\alignedat3
\textrm{(0)} &\quad &            &\emptyset \neq H \subseteq G  &\qquad &\textrm{($H$ é não vazio)}\\
\textrm{(1)} &\quad & a, b \in H &\implies a\ast b \in H        &\qquad &\textrm{($H$ é $\ast$-fechado)}\\
\textrm{(2)} &\quad & a \in H    &\implies a^{-1}  \in H        &\qquad &\textrm{($H$ é $^{-1}$-fechado)}
\endalignedat
\right\}
\implies
H \subgroup G
$$
\sketch.
A direção \lrdir\ é trivial.
Para a \rldir\ observamos que a associatividade é garantida pelo fato que $G$ é grupo,
então basta provar que $e\in H$.
Por isso, usamos a hipótese $H\neq\emptyset$ para tomar um $a\in H$
e usando nossas (poucas) hipoteses concluimos que $e\in H$ também.
\qes
\proof.
\lrdir: Trivial.
\endgraf
\rldir:
Precisamos verificar as leis (G0)--(G3) para o $\sset H {\ast}$.
Os (1) e (2) garantam as (G0) e (G3) respectivamente,
e o fato que $G$ é grupo garanta a (G1) também.
Basta verificar a (G2), ou seja, que $e\in H$:
$$
\alignat2
&\text{Tome $a\in H$.}                   \qqby{$H\neq\emptyset$\,}\\
&\text{\ttherefore $a^{-1} \in H$}       \qqby{pela (ii)}\\
&\text{\ttherefore $a\ast a^{-1} \in H$} \qqby{pela (i)}\\
&\text{\ttherefore $e \in H$}            \qqby{def.~$a^{-1}$}
\endalignat
$$
\qed
%%}}}

%%{{{ criterion: finite_subgroup_criterion 
\criterion Subgrupo finito.
\label{finite_subgroup_criterion}%
Se $\sset G {e, \ast}$ é um grupo e $H$ é um subconjunto finito e não vazio do $G$,
fechado sobre a operação $\ast$, então $H$ é um subgrupo de $G$.
Em símbolos:
$$
\left.
\alignedat3
\textrm{(0)} &\quad &            &\emptyset \neq H \finsubseteq G   &\qquad &\textrm{($H$ é \emph{finito} e não vazio)}\\
\textrm{(1)} &\quad & a, b \in H &\implies a\ast b \in H            &\qquad &\textrm{($H$ é $\ast$-fechado)}
\endalignedat
\right\}
\implies
H \subgroup G
$$
\sketch.
Basta demonstar o (2) para aplicar o~\ref{subgroup_criterion}.
Tome um $a\in H$ e considere a seqüência das suas potências
$a, a^2, a^3, \dotsc \in H$.
Como o $H$ é finito vamos ter um elemento repetido,
$a^r = a^s$ com inteiros $r > s > 0$, e usamos isso
para achar qual dos $a, a^2, a^3, \dotsc$ deve ser o $\ginv a$,
provando assim que $\ginv a\in H$.
\qes
\proof.
Basta demonstar o (2) para aplicar o~\ref{subgroup_criterion}.
Tome um $a\in H$ e considere a seqüência das suas potências
$a, a^2, a^3, \dotsc \in H$, graças à hipótese (1).
Como o $H$ é finito vamos ter um elemento repetido,
$a^r = a^s$ para alguns inteiros $r > s > 0$.
Temos
$$
\align
\toverbrace{a\ast a \ast \dotsb \ast a}{$r$ vezes}
&=\toverbrace{a\ast a \ast \dotsb \ast a}{$s$ vezes}
\intertext{e como $r>s$, reescrevemos assim:}
\toverbrace{
\tunderbrace{a\ast a \ast \dotsb \ast a}{$r-s$ vezes}
\ast
\tunderbrace{\cancel{a\ast a \ast \dotsb \ast a}}{$s$ vezes}
}{$r$ vezes}
&=\toverbrace{\cancel{a\ast a \ast \dotsb \ast a}}{$s$ vezes}.
\intertext{Agora usamos a lei de cancelamento-direito:}
\toverbrace{a\ast a \ast \dotsb \ast a}{$r-s$ vezes}
&=e
\intertext{Acabamos de provar que $e\in H$.  Observe que $r-s>0$, então temos:}
\toverbrace{a\ast \tunderbrace{a \ast \dotsb \ast a}{$r-s-1$ vezes}}{$r-s$ vezes}
&=e
\endalign
$$
ou seja, $\ginv a = a^{r-s-1} \in H$ pois $r-s-1\geq 0$ e podemos aplicar
o~\ref{subgroup_criterion} para concluir $H\subgroup G$.
\qed
%%}}}

%%{{{ x: subgroup_is_an_order 
\exercise.
\label{subgroup_is_an_order}%
Mostre que $\subgroup$ é uma relação de ordem:
$$
\gather
G \subgroup G\\
K \subgroup H \mland H \subgroup G \implies K\subgroup G\\
H \subgroup G \mland G \subgroup H \implies H = G.
\endgather
$$

\endexercise
%%}}}

%%{{{ x: Matrices 
\exercise Matrizes.
Verificamos que
$$
G \leteq \setst{ \matrixp{a & b\\c & d}\in\reals^{2\times2}} {ad - bc \neq 0 }
$$
com multiplicação é um grupo no~\ref{matrix_group_examples}.
Considere seus subconjuntos:
$$
\xalignat2
G_{\rats} &\leteq \setst{ \matrixp{a & b\\c & d}\in\rats^{2\times2}} {ad - bc \neq 0 }&\quad H &\leteq \setlst { \matrixp{a & b\\0 & d}      } { a,b,d \in\reals,\ ad \neq 0 }\\
G_{\ints} &\leteq \setst{ \matrixp{a & b\\c & d}\in\ints^{2\times2}} {ad - bc \neq 0 }&\quad K &\leteq \setlst { \matrixp{1 & b\\0 & 1}      } { b\in\reals }                 \\
G_{\nats} &\leteq \setst{ \matrixp{a & b\\c & d}\in\nats^{2\times2}} {ad - bc \neq 0 }&\quad L &\leteq \setlst { \matrixp{a & 0^{\phantom{-1}}\\0 & a^{-1}} } { a\in\reals,\ a\neq 0 }
\endxalignat
$$
Para cada relação de $\subseteq$ válida entre 2 dos 7 conjuntos em cima,
decida se a correspondente relação~$\subgroup$ também é válida.

\endexercise
%%}}}

%%{{{ eg: Modular arithmetic 
\example Aritmética modular.
Considere o grupo $\sset {\finord 6} {+_6}$ onde $+_6$ é a adição modulo $6$.
Os $\set{0,2,4}$ e $\set{0,3}$ são seus únicos subgrupos não-triviais.
\endexample
%%}}}

%%{{{ eg: Permutations (bijections) 
\example Permutações (bijeções).
O $\set{\id, \phi}$ é um subgrupo de $S_3$, onde $\phi = \permc{1 & 2}$.
\endexample
%%}}}

%%{{{ eg: Real functions 
\example Funções reais.
Seja $F = (\reals\to\reals_{\neq0})$ com operação a
multiplicação pointwise~(\ref{pointwise_operation}).
Os subconjuntos seguintes de $F$ são todos subgrupos dele:
$$
\xalignat2
&\setstt {f\in F} {$f$ continua} &
&\setst  {f\in F} {f(0) = 1} \\
&\setstt {f\in F} {$f$ constante} &
&\setstt {f\in F} {$f(r) = 1$ para todo $r\in\rats$}
\endxalignat
$$
\endexample
%%}}}

%%{{{ eg: Sets 
\example Conjuntos.
Sejam $A$ conjunto, e $X\subseteq A$.
Lembra que $\sset {\pset A} {\symdiff}$ é um
grupo~(\ref{pset_with_setops_group}).
Os $\set{\emptyset, X}$, $\set{\emptyset, X, A\setminus X}$,
e $\set{\emptyset, X, A\setminus X, A}$ são todos subgrupos dele,
mas o $\set{\emptyset, X, A}$ não é.
\endexample
%%}}}

%%{{{ x: why not? 
\exercise.
Por que não?

\hint
(G0).

\solution
O $H = \set{\emptyset, X, A}$ em geral não é um subgrupo, pois pode violar a
lei~(G0) no caso que $A\setminus X \notin H$, pois~$A \symdiff X = A \setminus X$.

\endexercise
%%}}}

%%{{{ x: intersection_of_subgroups_is_a_subgroup 
\exercise.
\label{intersection_of_subgroups_is_a_subgroup}%
Seja $G$ grupo, e $H_1,H_2\subgroup G$.
Então $H_1\inter H_2 \subgroup G$.

\hint
Precisa mostrar que $H_1\inter H_2$ é fechado sobre a operação do grupo
e sobre inversos.

\hint
Para mostrar que é fechado sobre a operação do grupo
suponha $a,b\in H_1\inter H_2$ e mostre que
$ab\in H_1\inter H_2$.
Similarmente, para mostrar que é fechado sobre os inversos,
suponha $a\in H_1\inter H_2$ e mostre que
$a^{-1} \in H_1\inter H_2$.

\solution%%{{{
Como $H_1\inter H_2 \subseteq G$,
precisamos mostrar que é fechado sobre a operação:
$$
\alignat2
&a,b \in H_1\inter H_2\\
&\text{\ttherefore $a,b \in H_1$ e $a,b\in H_2$   }      \qqby{def.~$\inter$}\\
&\text{\ttherefore $ab  \in H_1$ e $ab\in H_2$    }      \qqby{$H_1$ e $H_2$ grupos}\\
&\text{\ttherefore $ab  \in H_1\inter H_2$ }      \qqby{def.~$\inter$}
\intertext{e sobre os inversos:}
&a \in H_1\inter H_2\\
&\text{\ttherefore $a      \in H_1$ e $a\in H_2$      } \qqby{def.~$\inter$}\\
&\text{\ttherefore $a^{-1} \in H_1$ e $a^{-1}\in H_2$ } \qqby{$H_1$ e $H_2$ grupos}\\
&\text{\ttherefore $a^{-1} \in H_1\inter H_2$  } \qqby{def.~$\inter$}
\endalignat
$$
%%}}}

\endexercise
%%}}}

%%{{{ x: union_of_subgroups_is_a_subgroup_wrong 
\exercise.
\label{union_of_subgroups_is_a_subgroup_wrong}%
Trocamos o $\inter$ para $\union$
no~\ref{intersection_of_subgroups_is_a_subgroup}.
\item{(i)} Ache o erro na prova seguinte:
\quote{
<<Como $H \leteq H_1\inter H_2 \subseteq G$,
precisamos mostrar que $H$ é fechado sobre a operação:
$$
\alignat2
&a,b \in H_1\union H_2\\
&\text{\ttherefore $a,b \in H_1$ ou $a,b\in H_2$} \qqby{def.~$\union$}\\
&\text{\ttherefore $ab \in H_1$ ou $ab\in H_2$  } \qqby{$H_1$ e $H_2$ grupos}\\
&\text{\ttherefore $ab \in H_1\union H_2$       } \qqby{def.~$\union$}
\intertext{e sobre os inversos:}
&a \in H_1\union H_2\\
&\text{\ttherefore $a \in H_1$ ou $a\in H_2$          } \qqby{def.~$\union$}\\
&\text{\ttherefore $a^{-1} \in H_1$ ou $a^{-1}\in H_2$} \qqby{$H_1$ e $H_2$ grupos}\\
&\text{\ttherefore $a^{-1} \in H_1\union H_2$         } \qqby{def.~$\union$}
\endalignat
$$
Logo, $H_1 \union H_2 \subgroup G$.>>
}
\item{(ii)} Prove que a proposição não é válida.

\endexercise
%%}}}

%%{{{ x: arbitrary_intersection_of_subgroups_is_a_subgroup 
\exercise.
\label{arbitrary_intersection_of_subgroups_is_a_subgroup}%
Generalize a~\ref{intersection_of_subgroups_is_a_subgroup} para
intersecções arbitrárias:
se $G$ é um grupo, e $\scr H$ uma famílica de subconjuntos de $G$,
então $\Inter {\scr H} \subgroup G$.

\endexercise
%%}}}

%%{{{ x: congruence_mod_H_teaser 
\exercise.
\label{congruence_mod_H_teaser}%
Seja $G$ conjunto e $H\subgroup G$.
Defina no $G$ a relação $\rel R$ pela
$$
a \rel R b \defiff ab^{-1} \in H.
$$
Decida se a relação $\rel R$ é uma
relação de ordem parcial, de ordem total, de equivalência, ou nada disso.

\endexercise
%%}}}

\endsection
%%}}}

%%{{{ Orders, generators, and cyclic groups 
\section Ordens, geradores, e grupos cíclicos.

%%{{{ df: subgroup_generated_by_a 
\definition.
\label{subgroup_generated_by_a}%
\tdefined{grupo}[subgrupo!gerado por elemento]%
\sdefined {\generate {\holed a}} {O subgrupo gerado por o elemento $\holed a$}%
Sejam $G$ grupo e $a\in G$.
Chamamos o
$$
\align
\generate{a}
&\defeq
\set{a^m\st m\in\ints}\\
&=\set{\dotsc,a^{-2},a^{-1},a^0,a,a^2,\dotsc}
\endalign
$$
o \dterm{subgrupo de $G$ gerado por $a$}.
%%}}}

%%{{{ x: justify_subgroup_on_subgroup_generated_by_a 
\exercise.
\label{justify_subgroup_on_subgroup_generated_by_a}%
Justifica a palavra ``subgrupo'' na~\ref{subgroup_generated_by_a}.
Ou seja, prove que para qualquer grupo $G$ e qualquer $a\in G$,
$\generate a \subgroup G$.

\hint
Use o~\ref{subgroup_criterion}.

\solution
Graças ao~\ref{subgroup_criterion},
precisamos verificar que $\generate a$ é fechado pela operação e pelos inversos.
\endgraf\noindent
\casestyle{Fechado pela operação:}
Sejam $h_1,h_2\in\generate a$.
Logo
$h_1 = a^{k_1}$\fact1
e 
$h_2 = a^{k_2}$\fact2
para alguns $k_1,k_2\in\ints$.
Precisamos mostrar que $h_1h_2\in\generate a$.
Calculamos:
$$
\alignat2
h_1h_2
&= a^{k_1} a^{k_2}  \qqby{pelas~\byfact1,\byfact2}\\
&= a^{k_1 + k_2}    \qqby{pela~\ref{properties_of_exp_in_groups}~(1)}\\
&\in \generate a.   \qqby{def.~$\generate a$, pois $k_1+k_2\in\ints$}
\endalignat
$$
\casestyle{Fechado pelos inversos:}
Seja $h \in \generate a$,
logo $h = a^k$ para algum $k\in\ints$.
Pela~\ref{properties_of_exp_in_groups}~(3),
$$
\ginv h = \ginvp {a^k} = a^{-k} \in \generate a.
$$

\endexercise
%%}}}

%%{{{ x: Cyclic group of e 
\exercise.
Seja $\sset G {e,\ast}$ grupo.
Calcule o $\generate e$.

\endexercise
%%}}}

%%{{{ Calculate some sets and subgroups of ints 
\exercise.
Nos inteiros com adição, calcule os
$\generate 4$,
$\generate 4 \inter \generate 6$,
$\generate 4 \inter \generate {15}$,
e
$\generate 4 \union \generate 6$.
\endgraf\noindent
Quais deles são subgrupos do $\ints$?

\endexercise
%%}}}

%%{{{ df: cyclic_group 
\definition Grupo cíclico.
\label{cyclic_group}%
\tdefined {Grupo}[cíclico]%
Um grupo $G$ é \dterm{cíclico} sse existe $a\in G$ tal que $\generate a = G$.
%%}}}

\question.
Como tu generalizaria o
<<subgrupo gerado por $a\in G$>>
para
<<subgrupo gerado por $A \subseteq G$>>?
Ou seja, como definiria o $\generate A$ para qualquer $A \subseteq G$?
\spoiler.

%%{{{ A wrong generalization of <a> to <A> 
\note.
\label{wrong_generalization_of_gord_a_to_gord_A}%
Queremos generalizar o conceito de geradores para definir $\generate A$,
onde $A\subseteq G$.
Seguindo ingenuamente a definição de $\generate a$,
uma primeira abordagem seria botar
$$
\generate A = \set{ a^m \st a \in A,\  m \in\ints }
$$
e chamar $\generate A$ o subgrupo de $G$ gerado por $A$.\mistake
%%}}}

%%{{{ x: find_the_problem_in_wrong_generalization_of_gord_a_to_gord_A 
\exercise.
\label{find_the_problem_in_wrong_generalization_of_gord_a_to_gord_A}%
Qual o problema com a definição de $\generate A$ em cima?

\hint
Calcule o $\generate {4,6}$ no $\sset \ints +$.

\hint
Ele é um subgrupo?

\endexercise
%%}}}

%%{{{ x: subgroup_generated_by_two_members 
\exercise.
\label{subgroup_generated_by_two_members}%
Generalize primeiramente para o caso mais simples de <<subgrupo gerado por dois membros $a,b\in G$>>.
Ou seja, defina o $\generate {a,b}$, para quaisquer $a,b\in G$.

\solution
Sejam $\sset G \ast$ um grupo, e $a,b\in G$.
Definimos
$$
\generate {a,b} \defeq \setst {a^m \ast a^n} {m,n \in \ints}.
$$

\endexercise
%%}}}

\blah.
Finalmente chegamos numa idéia facilmente generalizável para o arbitrário
$A \subseteq G$.

%%{{{ df: subgroup_generate_A 
\definition direta.
\label{subgroup_generate_A_direct}%
\tdefined{grupo}[subgrupo!gerado por subconjunto]%
\sdefined {\generate {\holed a}} {O subgrupo gerado por o conjunto $\holed A$}%
Sejam $\sset G \ast$ grupo e $A\subseteq G$.
Chamamos o
$$
\align
\generate{A}
&\defeq
\set{a_0^{m_0}\ast\dotsb\ast a_{k-1}^{m_{k-1}}\st
a_i \in A,\ 
m_i \in \ints,\ 
k\in\nats,\ 
i\in\finord k
}
\endalign
$$
o \dterm{subgrupo de $G$ gerado por $A$}.
Abusando a notação, escrevemos também $\generate{ a_1, a_2,\dotsc, a_n }$
para o $\generate{ \set{a_1, a_2,\dotsc, a_n}}$.
%%}}}

%%{{{ x: Calculate <> and <G> 
\exercise.
Dado grupo $G$,
calcule os $\generate \emptyset$ e $\generate G$.

\endexercise
%%}}}

%%{{{ x: justify_subgroup_on_subgroup_generate_A_direct 
\exercise.
\label{justify_subgroup_on_subgroup_generate_A_direct}%
Prove que $\generate A \subgroup G$ para qualquer grupo $G$ e qualquer $A\subseteq G$.

\endexercise
%%}}}

%%{{{ Top-down and bottom-up: idea 
\note ``Top-down'' e ``bottom-up'' idéia.
Temos então duas caracterizações do~$\generate A$:
\endgraf
\casestyle{Bottom-up:}
Começamos com um conjunto~$T$ onde botamos todos os elementos que desejamos
no subgrupo (os elementos de~$A$ nesse caso),
e enquanto isso não forma um grupo, ficamos nos perguntando \emph{por que não}.
A resposta sempre é que (pelo menos) uma lei de grupo ((G0)--(G3)) está sendo
violada.
Vendo as leis, isso sempre quis dizer que um certo elemento tá faltando:
\beginul
\li (G0) violada: para alguns~$s,t \in T$, o~$s\ast t \notin T$.
\li (G1) violada é impossível: (veja~\ref{ass_for_free_in_subgroup}).
\li (G2) violada: a identidade~$e \notin T$.
\li (G3) violada: para algum~$t \in T$, seu inverso~$\ginv t \notin T$.
\endul
\noindent
Então ficamos \emph{adicionando} os elementos faltantes, e repetindo a mesma
pergunta, até chegar num conjunto que não viola nenhuma das leis, ou seja, um
grupo mesmo.  É o conjunto~$T$ que tem todos os elementos de~$A$ e todos os
necessários de~$G$ para formar um grupo.
\endgraf
\casestyle{Top-down:}
Começamos com a família~$\cal F$ de \emph{todos} os subgrupos de~$G$
que contenham o~$A$.
\emph{Primeiramente verificamos que essa família não é vazia:}
fato, pois já temos um membro dela: o proprio~$G$ já é um grupo que~$G \supseteq A$.
Mas no~$G$ existe possível ``lixo'': membros fora do~$A$ cuja presença não
foi justificada como necessária pelas leis de grupo.
Precisamos filtrar esse lixo, para ficar apenas com os membros ``necessários''.
Quais são esses membros?  São aqueles que pertencem em \emph{todos} os grupos
dessa família~$\cal F$.  Ou seja, a intersecção~$\Inter \cal F$ é o
desejado~$\generate A$.
%%}}}

%%{{{ prop: subgroup_generate_A_topdown 
\proposition Definição top-down.
\label{subgroup_generate_A_topdown}%
Sejam $\sset G \ast$ grupo e $A \subseteq G$.
$$
\generate A = \Inter \setst { H \subgroup G } { H \supseteq A }.
$$
\sketch.
Provamos cada direção separadamente:
Para a \lrdirset, tome um arbitrário membro $\alpha \in \generate A$
e um arbitrário $H \subgroup G$ tal que $H \supseteq A$, e mostre
que $\alpha\in H$.
Para a \rldirset, tome um $\alpha$ que pertence em todos os subgrupos de $G$ que contenham o $A$
e mostre que ele pode ser escrito na forma desejada (da definição de~$\generate A$).
\qes
%%}}}

%%{{{ prop: subgroup_generate_A_bottomup 
\proposition Definição bottom-up.
\label{subgroup_generate_A_bottomup}%
Sejam $\sset G \ast$ grupo e $A \subseteq G$.
Definimos a seqüência de conjuntos:
$$
\align
T_1     &= A\\
T_{n+1} &= T_n
\union \tunderbrace {\setst { s \ast t } {s,t \in T_n}} {(G0)}
\union \tunderbrace {\set e} {(G2)}
\union \tunderbrace {\setst {\ginv t} {t \in T_n}} {(G3)}
\endalign
$$
Temos
$$
\generate A = \Union_{n=1}^{\infty} T_n.
$$
\sketch.
Provamos cada direção separadamente:
Para a \lrdirset, tome um arbitrário membro $\alpha \in \generate A$
e ache um inteiro $w \geq 1$ tal que $\alpha \in T_w$.
Para a \rldirset, basta provar que cada um dos $T_1, T_2, T_3, \dots$
é subconjunto de $\generate A$.  Podes usar indução.
\qes
%%}}}

\endsection
%%}}}

%%{{{ Congruences and cosets 
\section Congruências e cosets.

%%{{{ The relation R was actually congruence mod subgroup 
\note.
A relação de equivalência que definimos no~\ref{congruence_mod_H_teaser}
não é tão desconhecida como talvez apareceu ser.
Vamos a redefinir com seu próprio nome e sua própria notação:
%%}}}

%%{{{ df: congruence mod H 
\definition Congruência.
\label{congruence_mod_H}%
\tdefined{congruência}[módulo subgrupo]%
\tdefined{módulo!subgrupo}%
\sdefined{\holed a \cong {\holed b} \pmodr {\holed H}}{\holed a é congruente com \holed b módulo-direito \holed H, onde $\holed H \subgroup G$}
Seja $G$ grupo e $H\subgroup G$.
Definimos
$$
a \cong b \pmodr H \defiff ab^{-1} \in H.
$$
%%}}}

%%{{{ x: mod_m_as_a_special_case_of_mod_H 
\exercise.
\label{mod_m_as_a_special_case_of_mod_H}%
Explique a notação da~\ref{congruence_mod_H} a comparando com a
congruência módulo algum inteiro da~\ref{congruence}.
Ou seja, mostre como a definição nos inteiros é apenas um caso especial da definição nos grupos.

\endexercise
%%}}}

%%{{{ Investigating the congruence 
\note Investigando a congruência.
\label{investigation_of_cong_mod_H}%
Vamos supor que temos um grupo $G$ e um $H\subgroup G$.
Tomamos $a,b\in G$ e queremos ver se $a\cong b\pmodr H$.
Vamos separar em casos:
\beginil
\item{\casestyle{Caso 1:}} os dois elementos $a$ e $b$ estão no $H$;
\item{\casestyle{Caso 2:}} um dos elementos está dentro do $H$, o outro fora;
\item{\casestyle{Caso 3:}} os dois estão fora do $H$.
\endil
\topinsert
\centerline{
\tikzpicture[>=stealth]%%{{{
\tikzi groupconginvestbase;
\node at (-1,0.5)    (a) {$\bullet$};
\node at (-0.5,-0.5) (b) {$\bullet$};
\node[above right, outer sep=1pt] at (a) {$a$};
\node[above right, outer sep=1pt] at (b) {$b$};
\endtikzpicture
%%}}}
\hfil
\tikzpicture[>=stealth]%%{{{
\tikzi groupconginvestbase;
\node at (-1,0.5)   (a) {$\bullet$};
\node at (1,-0.5)   (b) {$\bullet$};
\node[above right, outer sep=1pt] at (a) {$a$};
\node[above right, outer sep=1pt] at (b) {$b$};
\endtikzpicture
%%}}}
\hfil
\tikzpicture[>=stealth]%%{{{
\tikzi groupconginvestbase;
\node at (0.5,1)    (a) {$\bullet$};
\node at (1,-0.5)   (b) {$\bullet$};
\node[above right, outer sep=1pt] at (a) {$a$};
\node[above right, outer sep=1pt] at (b) {$b$};
\endtikzpicture
%%}}}
}
\botcaption{}
Os 3 casos da Investigação~\refn{investigation_of_cong_mod_H}.
\endcaption
\endinsert
\noindent
Para cada caso, queremos decidir se:
\beginil
\item{(i)} podemos concluir que $a \ncong b \pmodr H$;
\item{(ii)} podemos concluir que $a \ncong b \pmodr H$;
\item{(iii)} nenhum dos (i)--(ii).
\endil
Vamos ver qual dos (i)--(iii) aplica no \casestyle{caso 1}.
Temos
$$
a \cong b \pmodr H
\defiff a\ginv b \in H
$$
Como $b\in H$ e $H$ é um grupo ($H\subgroup G$) concluimos que
$\ginv b \in H$.
Agora, como $a, \ginv b \in H$ ganhamos $a\ginv b\in H$, ou seja,
$a \cong b \pmodr H$:
\topinsert
\centerline{
\tikzpicture[>=stealth]%%{{{
\tikzi groupconginvestcase1base;
\node[color=blue,inner sep=1pt] at (-1.5,-0.5)  (binv)  {$\bullet$};
\draw[->,dashed,color=blue] (b)--(binv);
\node[above,  outer sep=1pt] at (binv) {$\ginv b$};
\endtikzpicture
%%}}}
\hfil
\tikzpicture[>=stealth]%%{{{
\tikzi groupconginvestcase1base;
\node[inner sep=1pt] (binv)  at (-1.5,-0.5)  {$\bullet$};
\node[color=blue,inner sep=1pt] (abinv)  at (-0.75,-0.75) {$\bullet$};
\node[above, outer sep=1pt] at (binv) {$\ginv b$};
\node[below, inner sep=1pt, outer sep=1pt] at (abinv) {$a\ginv b$};
\draw[|-,color=blue]  (a)    edge[out=240,  in=135] (abinv);
\draw[|->,color=blue] (binv) edge[out=60, in=135] (abinv);
\endtikzpicture
%%}}}
}
\botcaption{}
Os passos do \casestyle{caso 1} do~\refn{investigation_of_cong_mod_H}.
\endcaption
\endinsert
%%}}}

%%{{{ x: do case 2 
\exercise.
\label{investigation_of_cong_mod_H_case_2}%
Decida qual dos (i)--(iii) aplica no \casestyle{caso 2}
do~\refn{investigation_of_cong_mod_H}.

\hint
Sem perca de generalidade, podes supor que $a \in H$ e $b \notin H$.
Comece desenhando como fizermos no \casestyle{caso 1}.

\hint
Mostre que $b^{-1}\notin H$.  (Qual seria o problema se $b^{-1} \in H$?)

\hint
Se $b{-1} \in H$ seu inverso também deveria estar no $H$, pois $H$ é um grupo.

\hint
Suponha que $ab^{-1} \in H$ para chegar num absurdo, mostrando assim que necessariamente,
$a \ncong b \pmodr H$.  Cuidado:
$$
xy \in H \nimplies x \in H \mland y \in H.
$$

\hint
Mostre que $a^{-1}\in H$.

\solution%%{{{
Sem perda de generalidade, suponha $a\in H$ e $b\notin H$.
Primeiramente mostramos que $b^{-1} \notin H$:
$$
b^{-1}\in H \implies \paren{b^{-1}}^{-1} \in H \implies b \in H,
$$
logo $b^{-1}\notin H$.
Para chegar num absurdo, vamos supor que $ab^{-1} \in H$.
\topinsert
\centerline{
\tikzpicture[>=stealth]%%{{{
\tikzi groupconginvestcase2base;
\endtikzpicture
%%}}}
\hfil
\tikzpicture[>=stealth]%%{{{
\tikzi groupconginvestcase2base;
\node[inner sep=1pt, color=blue] at (0,1.5) (binv) {$\bullet$};
\draw[->,dashed,color=blue] (b) edge[out=180, in=225] (binv);
\node[above, inner sep=3pt, outer sep=1pt] at (binv) {$\ginv b$};
\endtikzpicture
%%}}}
\hfil
\tikzpicture[>=stealth]%%{{{
\tikzi groupconginvestcase2base;
\node[inner sep=1pt] at (0,1.5) (binv) {$\bullet$};
\node[color=red] at (-0.5,-1) (abinv) {$\bullet$};
\node[above, inner sep=3pt, outer sep=1pt] at (binv) {$\ginv b$};
\node[below, inner sep=1pt, outer sep=1pt] at (abinv) {$a\ginv b$};
\endtikzpicture
%%}}}
}
\medskip
\centerline{
\tikzpicture[>=stealth]%%{{{
\tikzi groupconginvestcase2base;
\node[inner sep=1pt] at (0,1.5) (binv) {$\bullet$};
\node[inner sep=1pt, color=blue] at (-1.25,-1) (ainv) {$\bullet$};
\node[color=red] at (-0.5,-1) (abinv) {$\bullet$};
\draw[->,dashed,color=blue] (a) to (ainv);
\node[below, inner sep=1pt, outer sep=1pt] at (ainv) {$\ginv a$};
\node[above, inner sep=3pt, outer sep=1pt] at (binv) {$\ginv b$};
\node[below, inner sep=1pt, outer sep=1pt] at (abinv) {$a\ginv b$};
\endtikzpicture
%%}}}
\hfil
\tikzpicture[>=stealth]%%{{{
\tikzi groupconginvestcase2base;
\node at (0,1.5) (binv) {$\bullet$};
\node at (-1.25,-1) (ainv) {$\bullet$};
\node[color=red] at (-0.5,-1) (abinv) {$\bullet$};
\node[below, inner sep=1pt, outer sep=1pt] at (ainv) {$\ginv a$};
\node[below, inner sep=1pt, outer sep=1pt] at (abinv) {$a\ginv b$};
\node[above, outer sep=1pt] at (binv) {$\ginv b$};
\draw[|-]  (ainv)  edge[out=60, in=250] (binv);
\draw[|->,color=red] (abinv) edge[out=90, in=250] (binv);
\endtikzpicture
%%}}}
\hfil
\tikzpicture[>=stealth]%%{{{
\tikzi groupconginvestcase2base;
\node at (0,1.5) (binv) {$\bullet$};
\node at (-1.25,-1) (ainv) {$\bullet$};
\node[inner sep=1pt,color=blue] at (1,1) (abinv) {$\bullet$};
\node[below, inner sep=1pt, outer sep=1pt] at (ainv) {$\ginv a$};
\node[below, inner sep=1pt, outer sep=1pt] at (abinv) {$a\ginv b$};
\node[above, outer sep=1pt] at (binv) {$\ginv b$};
\draw[-,dashed,color=blue] (a) edge[out=20,  in=160] (abinv);
\draw[->,dashed,color=blue] (b) edge[out=110, in=160] (abinv);
\endtikzpicture
%%}}}
}
\botcaption{}
Os passos da resolução do~\ref{investigation_of_cong_mod_H_case_2}.
\endcaption
\endinsert
\noindent
Vamos deduzir a contradição $b^{-1}\in H$.
Para conseguir isso, observamos que $a^{-1} \in H$ (pois $a\in H$),
e logo
$$
\overbrace{\underbrace{\vphantom{(}a^{-1}}_{\in H}\underbrace{(ab^{-1})}_{\in H}}^{\dsize b^{-1}} \in H.
$$
\noindent
Achamos assim nossa contradição:
$$
b^{-1} = eb^{-1} = (a^{-1}a)b^{-1} = a^{-1}(ab^{-1}) \in H.
$$
Concluimos então que $ab^{-1} \notin H$, ou seja $a \ncong b \pmodr H$.
%%}}}

\endexercise
%%}}}

%%{{{ x: do case 1 
\exercise.
Decida qual dos (i)--(iii) aplica no \casestyle{caso 3} do~\refn{investigation_of_cong_mod_H}.

\hint
Ache dois exemplos com $a,b\notin H$, tais que num
$a \cong b \pmodr H$ e no outro $a \ncong b \pmodr H$.

\hint
Um exemplo consiste em: um grupo $G$, um subgrupo $H\subgroup G$, e dois elementos $a,b\in G\setminus H$ tais que satisfazem (ou não) a congruência $a\cong b \pmodr H$.

\hint
Procure teus exemplos no grupo dos inteiros com adição $\sset \ints +$.

\hint
Tome $G \leteq \sset \ints +$, $H \leteq 2\ints$, $a \leteq 1$, $b \leteq 3$
e observe que $1 \cong 3 \pmodr {2\ints}$.

\hint
Não é possível achar o outro exemplo com o mesmo subgrupo $2\ints$,
mas pode sim no $3\ints$.

\solution%%{{{
No grupo $G\leteq\sset \ints +$ considere seu subgrupo $H\leteq 5\ints$.
Temos:
$$
\xalignat2
\left.
\aligned
G&\leteq \sset \ints +\\
H&\leteq 5\ints\\
a&\leteq 1  \\
b&\leteq 6  
\endaligned
\right\}
&\implies a \cong b \pmodr H
&
\left.
\aligned
G&\leteq \sset \ints +\\
H&\leteq 5\ints\\
a&\leteq 1  \\
b&\leteq 3  
\endaligned
\right\}
&\implies a \ncong b \pmodr H,
\endxalignat
$$
porque $1 + (-6) = -5 \in 5\ints$ e $1 + (-3) = -2 \notin 5\ints$ respectivamente.
%%}}}

\endexercise
%%}}}

%%{{{ def: cosets 
\definition Cosets.
\label{coset}%
\tdefined{coset}%
\iisee{coclasse}{coset}%
\sdefined{\holed a H} {o left-coset do $H\subgrp G$ através do $a\in G$}%
\sdefined{H\holed a} {o right-coset do $H\subgrp G$ através do $a\in G$}%
Seja $G$ grupo e $H\subgrp G$.  Para $a\in G$ definimos
$$
\xalignat2
aH &\defeq \set{ ah \st h\in H }.
&
Ha &\defeq \set{ ha \st h\in H }
\endxalignat
$$
Chamamos o $aH$ o \dterm{left-coset} do $H$ através de $a$,
e similarmente o $Ha$ seu \dterm{right-coset}.
Também usamos os termos \dterm{coclasse (lateral) à esquerda/direita}.
%%}}}

%%{{{ Q: How many cosets? 
\question.
\emph{Dados um $G$ grupo e $H\subgroup G$, quantas coclasses $Ha$ tem?
Ou seja, qual é a cardinalidade do conjunto $\set { Ha \st a \in G }$?}
%%}}}

%%{{{ thm: the cosets of H are the quotient set G / congmodR 
\theorem.
\ii{partição}%
\ii{relação!de equivalência}%
Seja $G$ grupo e $H\subgroup G$.
A família $\scr A = \set { Ha \st a\in G }$ é uma partição do $G$
e sua correspondente relação de equivalência é a congruência módulo-direito $H$.
\proof.
Vamos denotar por $\eqclassimp a$ a classe de equivalência de $a\in G$.
Provamos que $\eqclassimp a = Ha$.
\endgraf
``$\subseteq$'':
Suponha $x \in \eqclassimp a$.
Logo $x \cong a \pmodr H$, ou seja, $xa^{-1} \in H$.
Pela definição de $Ha$ então temos
$$
Ha\ni
\underbrace{\paren{xa^{-1}}}_{\in H}a
= x\paren{a^{-1}a}
= x.
$$
\endgraf
``$\supseteq$'':
Suponha que $x\in Ha$.
Logo $x=ha$ para algum $h\in H$ e queremos mostrar que $ha\in\eqclassimp a$,
ou seja $ha \cong a \pmodr H$.
Confirmamos:
$$
\paren{ha} a^{-1}
= h \paren{aa^{-1}}
= h
\in H
$$
e pronto.
\qed
%%}}}

%%{{{ x: direct proof that coclasses form a partition 
\exercise.
\ii{partição}%
Prove diretamente (sem passar pela relação de equivalência) que
a família de todas as coclásses dum $H\subgroup G$ é uma partição de $G$:
\beginil
\item{(i)}   para todo $x\in G$, existe coclasse $Ha$ com $x \in Ha$;
\item{(ii)}  para todo $c\in G$, se $c\in Ha\inter Hb$ então $Ha = Hb$;
\item{(iii)} nenhuma coclasse $Ha$ é vazia.
\endilnoskip

\hint%%{{{
Para os (i) e (iii) explique porque $a \in Ha$ para todo $a\in G$ e vice versa.
Para o (ii) tome um elemento comum $c\in G$ tal que
$$
h_a a = c = h_b b.
$$
Agora mostra que $Ha \subseteq Hb$; a outra diração será similar.
%%}}}

\hint%%{{{
Para os (i) e (iii) lembra que $H$ sendo subgrupo de $G$ ele também é grupo.
Qual elemento é garantido de pertencer no $H$ e como podemos usar esse fato aqui?
Para o (ii) tome $ha \in Ha$ e mostre que $ha \in Hb$.
Aplique uma série de substituições $ha = \dotsb$ para chegar num elemento
do $Hb$, ou seja, um produto que começa com elementos de $H$ e termina em $b$.
%%}}}

\hint%%{{{
Para o (ii):
como temos que $h_aa=h_bb$, precisamos ``criar'' a expressão
``$h_aa$'' para a substituir por $h_bb$.
Com que precisamos e podemos multiplicar o $ha$ pela equerda para conseguir isso?
Cuidado: para garantir igualdade só pode multiplicar com $e$,
mas pode substituir o $e$ com muitos produtos iguais ao $e$.
%%}}}

\solution%%{{{
Como $H$ é um grupo, sabemos que $e\in H$, então para qualquer $x\in G$,
temos que $x = ex \in Hx$, ou seja todos os elementos de $G$ pertencem em
alguma coclasse, e vice versa:
toda coclasse $Ha$ tem pelo menos um elemento: o proprio $a$.
Para o (iii) suponha que $Ha\inter Hb\neq \emptyset$ e tome $c\in Ha\inter Hb$.
Logo
$$
h_aa = c = h_bb
\qqquad
\text{para alguns $h_a, h_b\in H$}
$$
Para provar que $Ha = Hb$, mostramos que $Ha\subseteq Hb$ e $Hb\subseteq Ha$.
Suponha então que $x \in Ha$, logo $x = ha$ para algum $h\in H$.
Precisamos mostrar que $x \in Hb$.
Calculamos:
$$
x
= ha
= e(ha)
= \paren{\paren{h_a h^{-1}}^{-1}\paren{h_a h^{-1}}} (ha)
= \paren{\paren{h^{-1}}^{-1} h_a^{-1}} h_a \cancel{\paren{h^{-1} h}} a
= \paren{h h_a^{-1}} \paren{h_a a}
= \paren{h h_a^{-1}} \paren{h_b b}
= \paren{h h_a^{-1} h_b} b
\in Hb.
$$%%}}}

\endexercise
%%}}}

%%{{{ df: index_of_subgroup 
\definition Índice.
\label{index_of_subgroup}%
\tdefined{grupo}[índice de subgrupo]%
\sdefined {\groupind {\holed G} {\holed H}} {o índice do subgrupo \holed H no grupo \holed G}%
Sejam $G$ grupo e $H\subgroup G$.
O \dterm{índice} de $H$ no $G$ é o número de right-cosets de $H$ no $G$.
Denotamos-lo com os símbolos $\groupind G H$ ou $\altgroupind G H$.
%%}}}

\TODO Mention that there is a correspondence between left and right cosets.

%%{{{ thm: Lagrange 
\theorem Lagrange.
\Lagrange[teorema]%
\ii{teorema}[Lagrange]%
Seja $G$ grupo finito e $H\subgroup G$.
Então $\groupord H \divides \groupord G$.
\sketch.
Sabemos que o $G$ pode ser particionado pelos right cosets de $H$,
e que cada um deles tem a mesma cardinalidade com o proprio $H$.
\qes
%%}}}

%%{{{ Corollaries as exercises 
\note Corollários.
Graças ao teorema de Lagrange ganhamos muitos
corollários diretamente, como tu vai verificar
agora resolvendo os exercícios seguintes:
%%}}}

%%{{{ x: subgroups_of_group_with_prime_order 
\exercise.
\label{subgroups_of_group_with_prime_order}%
Seja $G$ grupo com $\gord G = p$, onde $p$ primo.
Quais são todos os subgrupos de $G$?

\hint
Se $H\subgroup G$, pelo teorema de Lagrange temos que
$\gord H \divides \gord G$.  Quais são os divisores de $\gord G$?

\solution
Se $H\subgroup G$, pelo teorema de Lagrange temos que
$\gord H \divides \gord G = p$, logo $\gord H = 1$ ou $p$.
No primeiro caso $H = \set e$, no segundo, $H = G$.
Ou seja:
\emph{um grupo com ordem primo não tem subgrupos não-triviais}.

\endexercise
%%}}}

%%{{{ x: order_of_generate_a_divides_order_of_G 
\exercise.
\label{order_of_generate_a_divides_order_of_G}%
Seja $G$ grupo finito e $a\in G$.  Então $\gord a \divides \gord G$.

\hint
Se achar um subgrupo $H\subgroup G$ com $\gord H = \gord a$,
acabou (graças ao Lagrange).

\hint
Lembra que $\generate a$ é um subgrupo do $G$\dots?

\hint
\dots e que sua ordem é $\gord {\generate a} = \gord a$?

\solution
Sabemos que $\generate a$ é um subgrupo de $G$, com ordem
$\gord {\generate a} = \gord a$,
e pelo teorema de Lagrange, como $\generate a \subgroup G$
e $G$ é finito temos
$$
\gord a = \gord {\generate a} \divides \gord G.
$$

\endexercise
%%}}}

%%{{{ x: a_to_the_order_of_G_is_e 
\exercise.
\label{a_to_the_order_of_G_is_e}%
Seja $G$ grupo finito e $a\in G$.
Então $a^{\gord G} = e$.

\hint
Resolveu já o~\ref{order_of_generate_a_divides_order_of_G}?

\solution
Graças o~\ref{order_of_generate_a_divides_order_of_G} temos que
$\gord a \divides gord G$, ou seja, $\gord G = k\gord a$ para algum
$k\in\ints$.
Agora calculamos:
$$
a^{\gord G}
= a^{k\gord a}
= a^{\gord a k}
= \paren{a^{\gord a}}^k
= e^k
= e.
$$

\endexercise
%%}}}

\blah.
Vamos agora generalizar a notação que usamos nos cosets de
``multiplicação de subgrupo por elemento'' para
``multiplicação de subgrupo por subgrupo''.

%%{{{ df: HK 
\definition.
\label{HK_of_groups}%
Seja $G$ grupo e $H,K\subgroup G$.  Definimos
$$
HK \defeq \set { hk \st h\in H,\ k\in K }.
$$
%%}}}

%%{{{ Q: are HK and KH subgroups of G? 
\question.
$HK = KH$?
$HK \subgroup G$?
$KH \subgroup G$?
%%}}}

%%{{{ eg: calculate_HK_KH_S3 
\example.
No grupo $S_3$, sejam seus subgrupos
$$
\xalignat2
H &\leteq \set{ \id, \phi }
&
K &\leteq \set{ \id, \psi\phi }.
\endxalignat
$$
Calcule os $HK$ e $KH$ e decida se $HK=KH$ e se $HK$ e $KH$ são subgrupos de $S_3$.

\solution%%{{{
Pela definição
$$
\xalignat2
HK &= \set{ hk \st h\in H,\ k\in K}
& 
KH &= \set{ kh \st h\in H,\ k\in K}
\\
&= \set{
\id\compose\id,
\id\compose(\psi\compose\phi),
\phi\compose\id,
\phi\compose(\psi\compose\phi)
}
&
&= \set{
\id\compose\id,
\id\compose\phi,
(\psi\compose\phi)\compose\id,
(\psi\compose\phi)\compose\phi)
}
\\
&= \set{ \id, \psi\phi, \phi, \phi\psi\phi }
&
&= \set{ \id, \phi, \psi\phi, \psi\phi^2 }
\\
&= \set{ \id, \psi\phi, \phi, \psi^2}
&
&= \set{ \id, \phi, \psi\phi, \psi }\\
\intertext{Observamos que $HK\neq KH$ e nenhum deles é subgrupo de $S_3$, pois}
HK&\notni {(\psi\phi)}{\phi} = \psi(\phi\phi) = \psi\phi^2 = \psi
&KH&\notni {\psi}{\psi} = \psi^2.
\endxalignat
$$
%%}}}
\endexample
%%}}}

\blah.
Então descobrimos que, em geral, nem $HK=KH$, nem $HK\subgroup G$,
nem $KH\subgroup G$ são garantidos.
Pode acontecer que $HK \subgroup G$ mas $KH \not\subgroup G$?
E o que a igualdade $HK=KH$ tem a ver com a ``subgrupidade'' dos $HK$ e $KH$?
Vamos responder em todas essas perguntas com o teorema seguinte:

%%{{{ thm: HK = KH <=> HK subgrp G 
\theorem.
\label{HK_equals_KH_iff_HK_subgroup}
Seja $G$ grupo e subgrupos $H,K \subgrp G$.  Então:
$$
HK = KH
\iff
HK \subgrp G
$$
%%{{{ sketch and proof 
\sketch.
Para a direção \lrdir, precisamos mostrar que $HK$ é fechado sobre a operação e fechado sobre os inversos.
Tomamos aleatorios $h_1k_1,h_2k_2\in HK$ e aplicando as propriedades de grupo e nossa hipótese,
mostramos que $(h_1k_1)(h_2k_2) \in HK$.  Similarmente para os inversos: consideramos um
arbitrário elemento $hk\in HK$ e mostramos que seu inverso $\ginvp{hk} \in HK$.  Aqui, alem da
hipótese precisamos o~\ref{inverse_of_product_in_group}.
Para a direção \rldir, mostramos as ``$\subseteq$'' e ``$\supseteq$'' separadamente,
usando idéias parecidas.
\qes
\proof.
\lrdir:
Precisamos mostrar que $HK$ é fechado sobre a operação e fechado sobre os inversos.
Tomamos aleatorios $h_1k_1,h_2k_2\in HK$ e calculamos:
$$
\alignat2
(h_1k_1)(h_2k_2)
&= h_1(k_1h_2)k_2 \qqby{ass.}\\
&= h_1(h_3k_3)k_2\quad\text{para alguns $h_3\in H$ e $k_3 \in K$} \qqby{$k_1h_2\in KH = HK$}\\
&= (h_1h_3)(k_3k_2)\qqby{ass.}\\
&\in HK
\endalignat
$$
Para os inversos temos:
$$
\ginvp {h_1k_1} = \ginv{k_1} \ginv{h_1} \in KH = HK.
$$
\endgraf
\rldir:
Mostramos as ``$\subseteq$'' e ``$\supseteq$'' separadamente.
``$\subseteq$'':
Tome $x \in HK$, logo $x^{-1} \in HK$ e $x^{-1} = hk$ \emph{para alguns}
$h\in H$ e $k\in K$.  Como $H$ e $K$ são sub\emph{grupos} de $G$ seus inversos
também estão em $H$ e $K$ respectivamente.
Mas
$$
x = \ginvp{\ginv x} = \ginvp{hk} \ginv k \ginv h \in KH.
$$
``$\supseteq$'':
Tome $x \in KH$, logo $x = kh$ \emph{para alguns} $k\in K$, $h\in H$.
Logo $\ginv k \in K$ e $\ginv h \in H$.
Como $HK\subgrp G$, basta apenas provar que $\ginv x \in HK$ pois isso
garantará que $x \in HK$ também.
Realmente, $\ginv x = \ginvp{kh} = \ginv h \ginv k \in HK$.
\qed
%%}}}
%%}}}

%%{{{ x: cosets_of_subgroup_of_index_2 
\exercise.
\label{cosets_of_subgroup_of_index_2}
Se $H\subgroup G$ de indice $2$, então $H \normal G$.

\solution
Suponha $H\subgroup G$ com 
Precisamos mostrar que $aH = Ha$ para todo $a\in G$.
Como o índice de $H$ é 2, só tem 2 cosets, logo, fora do proprio $H$, seu
complemento $G\setminus H$ tem que ser um coset.
Agora para qualquer $aH$ com $a \notin H$, temos
$$
aH = G\setminus H = Ha.
$$

\endexercise
%%}}}

\endsection
%%}}}

%%{{{ Number theory revisited 
\section Teoria de números revisitada.

%%{{{ x: the multiplicative group Zp 
\exercise.
Seja $p$ primo e defina
$\cal Z_p = \sset {\finord p \setminus \set0} {\ntimes}$
onde $\ntimes$ é a multiplicação módulo~$p$.
Mostre que $\cal Z_p$ é um grupo e ache sua ordem.

\endexercise
%%}}}

%%{{{ x: the multiplicative group Zn 
\exercise.
Seja $n\in\nats$ com $n>1$ e defina
$\cal Z_n = \sset {\set{a \in \finord n \st \gcd a n = 1}} {\ntimes}$
onde $\ntimes$ é a multiplicação módulo~$n$.
Mostre que $\cal Z_n$ é um grupo e ache sua ordem.

\endexercise
%%}}}

\endsection
%%}}}

%%{{{ Normal subgroups 
\section Subgrupos normais.

%%{{{ Notation 
\note Notação.
Observe que definimos os $aH$, $Ha$, e $HK$, num grupo $G$
para \emph{subgrupos} $H,K\subgroup G$.
Mas não usamos nenhuma propriedade de subgrupo mesmo.
Podemos realmente estender essa notação para arbitrários
\emph{subconjuntos} de $G$, e, por que não,
até usar notação como a seguinte abominágem:
$$
g_1ABg_2B\ginv{g_3}Ag_1CB
\defeq
\set{
g_1abg_2b'\ginv{g_3}a'g_1cb''
\st
a, a' \in A,\ 
b, b', b'' \in B,\ 
c \in C
}
$$
dados $g_1,g_2,g_3\in G$ e $A,B,C\subseteq G$.
Observe primeiramente que \emph{precisamos} usar variáveis diferentes
para cada instância de elemento de $A$, etc.
Observe também que todos esses objetos que escrevemos juxtaposicionando
elementos e subconjuntos de $G$ são subconjuntos de $G$ se usamos
pelo menos um subconjunto de $G$ na expressão:
$$
\xalignat2
g_1abg_5ab' &\in G
&
g_1aBg_5ab' &\subseteq G.
\endxalignat
$$
Finalmente, \emph{confira} que graças à associatividade da operação do grupo $G$,
não precisamos botar parenteses:
$$
g_1ABg_2B\ginv{g_3}Ag_1CB
=
g_1A(Bg_2B)\ginv{g_3}(Ag_1CB)
=
(g_1A)(Bg_2)(B\ginv{g_3})(Ag_1)(CB)
= \dotsb
$$
etc.
%%}}}

%%{{{ x: calculate_left_and_right_cosets_of_H_K_N 
\exercise.
\label{calculate_left_and_right_cosets_of_H_K_N}%
Calcule todos os left e right cosets dos
$$
\xalignat3
H &\leteq \set{ \id, \phi }&
K &\leteq \set{ \id, \psi\phi }&
N &\leteq \set{ \id, \phi, \psi^2 }
\endxalignat
$$
Quantos right-cosets diferentes cada um deles tem?
Quantos left?  Explique sua resposta.
Os conjuntos de todos os right-cosets de $H$ é igual
com o conjunto de todos os left-cosets de $H$?
Similarmente para os $K$ e $N$.

\endexercise
%%}}}

%%{{{ df: conjugate_of_group_element 
\definition Conjugado.
\label{conjugate_of_group_element}%
\tdefined{conjugado}{de elemento de grupo}%
Seja $G$ grupo e $a\in G$.
Para qualquer $g\in G$, o $ga\ginv g$ é chamado
um \dterm{conjugado} de $a$.
%%}}}

%%{{{ df: normal_subgroup 
\definition Subgrupo normal.
\label{normal_subgroup}%
\tdefined{subgrupo}[normal]%
\sdefined{\holed N \normal \holed G}{\holed N é um subgrupo normal de \holed G}%
\iisee{normal}{subgrupo normal}%
Seja $G$ grupo e $N\subgroup G$.
O $N$ é um \dterm{subgrupo normal} de $G$
sse $N$ é fechado sobre os conjugados.
Em símbolos,
$$
N \normal G
\defiff
\paren{\forall n\in N}
\lforall{g\in G}
{ gn\ginv g \in N }.
$$
%%}}}

\blah.
Seguem umas definições alternativas.

%%{{{ lemma: normal_subgroup_altdef_gN_eq_Ng 
\lemma.
\label{normal_subgroup_altdef_gN_eq_Ng}%
\ii{subgrupo!normal}%
Sejam $G$ grupo, e $N\subgroup G$.
$$
N\normal G
\iff
\lforall{g\in G}
{gN = Ng}.
$$
%%}}}

%%{{{ lemma: normal_subgroup_altdef_gNginvg_eq_N 
\lemma.
\label{normal_subgroup_altdef_gNginvg_eq_N}%
\ii{subgrupo!normal}%
Sejam $G$ grupo, e $N\subgroup G$.
$$
N\normal G
\iff
\lforall{g\in G}
{gN\ginv g = N}.
$$
%%}}}

%%{{{ beware: g n g^{-1} \neq n 
\beware.
Se tivemos $N \normal G$ temos sim que $gn\ginv g \in N$,
mas isso \emph{não garanta} que $gn\ginv g = n$ não!
Sabemos que para todo $n\in N$, temos $gn\ginv g = n'$ \emph{para algum}
$n' \in N$, mas nada nos permite concluir que esse $n'$ é nosso $n$.
%%}}}

%%{{{ x: gnginvg_neq_n 
\exercise.
\label{gnginvg_neq_n}%
Ache um exemplo que mostra que não necessariamente
$gn\ginv g = n$ mesmo $n\in N \normal G$.

\endexercise
%%}}}

\endsection
%%}}}

%%{{{ Symmetries 
\section Simetrias.

\note As simetrias dum triângulo equilátero.
\label{Dih3}%

\note Um primeiro exemplo de isomorfismo.
\label{Dih3_isom_S3}%

\note As simetrias dum quadrado.
\label{Dih4}%

\note Diagramas Hasse.
\label{hasse_diagrams_first_encounter}%


\definition Os grupos dihedrais.
\label{Dihn}%
\tdefined{grupo}[dihedral]%
\iisee{dihedral}{grupo dihedral}%

\exercise.
Qual a $\gord{\dih n}$?

\endexercise

\exercise Simetrias de rectângulo.
Ache todas as simetrias de rectângulo e estude seu grupo.

\endexercise

\endsection
%%}}}

%%{{{ Morphisms 
\section Morfismos.
\label{Group_morphisms}%

\note Abusos notacionais.

%%{{{ Different structures for groups 
\note Diferentes estruturas para grupos.
$$
\align
\phi(xy)      &= \phi(x) \phi(y) \tag{i}   \\
\phi(e_A)     &= e_B             \tag{ii}  \\
\phi(\ginv x) &= \ginvp{\phi(x)} \tag{iii} 
\endalign
$$
Então, dependendo na estrutura que escolhemos para nossa definição
de grupo, precisamos definir ``morfismo'' em forma diferente:
No caso de estrutura $\sset A {\ast_A}$ o morfismo deve satisfazer
o~(i); no caso de $\sset A {e_A, \ast_A}$, os~(i)--(ii), e no caso
de $\sset A {e_A, \ginv{\phantom{\cdot}}, \ast_A}$ todos os~(i)--(iii).
Parece então que chegamos no primeiro ponto onde a estrutura
escolhida na definição de grupo será crucial.
Felizmente, como nos vamos provar logo após, as leis de grupo
são suficientes para garantir que qualquer função $\phi$ que
satisfaz apenas o~(i)~em cima, obrigatoriamente satisfaz
os~(ii)~e~(iii) também!
Então vamos botar em nossa definição apenas o~(i) mesmo:
%%}}}

%%{{{ df: group_homomorphism 
\definition Homomorfismo.
\label{group_homomorphism}%
\tdefined{homomorfismo}%
Sejam $A$ e $B$ grupos.
A função $\phi : A \to B$ é um \dterm{homomorfismo} de $A$ para $B$
sse
$$
\text{para todo $x,y\in A$},
\qquad
\phi(xy) = \phi(x) \phi(y).
$$
%%}}}

%%{{{ lemma: in_groups_respecting_products_means_morphism 
\lemma.
\label{in_groups_respecting_products_means_morphism}%
Sejam grupos $\cal A = \sset A {\ast_A}$ e $\cal B = \sset B {\ast_B}$,
e homomorfismo $\phi : \cal A \to \cal B$.
Então:
$$
\align
\phi(e_A)     &= e_B             \tag{i}  \\
\phi(\ginv x) &= \ginvp{\phi(x)} \tag{ii} 
\endalign
$$
\sketch.
Para o~(i), calculamos $\phi(e_A) = \phi(e_A) \phi(e_A)$
para concluir que $e_B = \phi(e_A)$;
para o~(ii), mostramos que o $\phi(\ginv x)$ satisfaz a
propriedade caraterística de ser inverso de $\phi(x)$:
$$
\phi(\ginv x) \phi(x) = e_B.
$$
\qes
%%}}}

%%{{{ df: group_morphisms 
\definition -morfismos.
\label{group_morphisms}
Sejam grupos $\cal A$ e $\cal B$, e $\phi : \cal A \to \cal B$ um homomorfismo.
Usamos os termos:
$$
\align
\text{$\phi$ monomorfismo} &\defiff \text{$\phi$ injetora}\\
\text{$\phi$ epimorfismo}  &\defiff \text{$\phi$ sobrejetora}\\
\text{$\phi$ isomorfismo}  &\defiff \text{$\phi$ bijetora}\\
\text{$\phi$ endomorfismo} &\defiff \text{$\dom\phi = \cod\phi$}\\
\text{$\phi$ automorfismo} &\defiff \text{$\phi$ endomorfismo e isomorfismo}\\
\endalign
$$
{Rascunho:}
$$
\align
\text{mono} &= \text{injetora}\\
\text{epi}  &= \text{sobre}\\
\text{iso}  &= \text{mono $+$ epi}\\
\text{endo} &= \text{$\phi : \cal A \to \cal A$}\\
\text{auto} &= \text{endo $+$ iso}
\endalign
$$
%%}}}

%%{{{ df: isomorphic_groups 
\definition Grupos isomórficos.
\label{isomorphic_groups}%
\tdefined{grupo}[isomorphicos]%
Sejam grupos $G$ e $G'$.
Chamamos o $G$ \dterm{isomórfico} com o $G'$ sse
existe isomorfismo $\phi : G \to G'$.
Escrevemos $G \iso G'$.
Escrevemos também $\phi : G \iso G'$ para ``$\phi$ é um isomorfismo de $G$ para $G'$''.
%%}}}

%%{{{ x: isomorphic_is_an_equivalence_relation 
\exercise.
\label{isomorphic_is_an_equivalence_relation}%
Mostre que $\isomorphic$ é uma relação de equivalência.

\endexercise
%%}}}

\endsection
%%}}}

%%{{{ Kernel and Image 
\section Kernel e Image.

\definition Kernel.
\label{kernel}%
\tdefined{kernel de homomorfismo}%
\sdefined {\ker{\holed \phi}} {o kernel do homomorfismo $\holed \phi$}
Sejam $G, G'$ grupos e $\phi$ um homomorfismo de $G$ para $G'$.
Definimos
$$
\align
\ker\phi
&\defeq
\set { g \in G \st \phi(g) = e_{G'} }\\
&=
\pre{\phi}{\set{e_{G'}}}
\endalign
$$
e o chamamos o \dterm{kernel} do $\phi$.

\exercise.
Sejam $G$ e $G'$ grupos e $\phi$ um homomorfismo de $G$ para $G'$.
Prove que $\ker\phi\subgroup G$.

\endexercise

\exercise.
Prove ainda mais: $\ker\phi\normal G$.

\endexercise

\endsection
%%}}}

%%{{{ Problems 
\problems.

%%{{{ Direct products 
\blah.
Lembre que já usamos $\times$ entre \emph{conjuntos} $A,B$ para formar seu
produto cartesiano $A \times B$; e também entre \emph{funções} $f : A \to B$,
$g : C \to D$ para formar seu produto
$f \times g : (A\times C) \to (B \times D)$.
Vamos agora sobrecarregar ainda mais esse $\times$:

\definition Produtos diretos.
Sejam $\cal G_1 = \sset {G_1} {\ast_1}$ e $\cal G_2 = \sset {G_2} {\ast_2}$ grupos.
Definimos o grupo
$$
\cal G_1 \times \cal G_2 = \sset {G_1 \times G_2} {{\ast_1} \times {\ast_2}}.
$$

\problem.
Prove que realmente é um grupo.

\endproblem
%%}}}

%%{{{ bust_proof_of_uniqueness_of_identity_in_group 
\problem.
\label{bust_proof_of_uniqueness_of_identity_in_group}%
Considere essa suposta prova da unicidade da identidade~(\ref{uniqueness_of_identity_in_group}):
\endgraf
<<Seja $G$ grupo e suponha que temos identidades $e_1,e_2 \in G$.
Seja $a\in G$.
Como $e_1$ é identidade, temos $a \ast \ginv a = e_1$\fact1.
Como $e_2$ é identidade, também temos $a \ast \ginv a = e_2$\fact2.
Pelas \byfact1~e~\byfact2, como os lados esquerdos são iguais, o lados direitos também são.
Ou seja, $e_1 = e_2$ que foi o que queremos provar.>>
\endgraf
Identifique todos os erros nessa tentativa de prova.

\endproblem
%}}}

%%{{{ prob: aut_G_is_a_group 
\problem.
\label{aut_G_is_a_group}%
Mostre que dado um grupo $G$, o conjunto de todos os seus automorfismos
$$
\aut G
\defeq
\set{
\phi : G\bijto G
\st
\text{$\phi$ é um automorfismo}
}
$$
com operação $\compose$ é um grupo.

\endproblem
%%}}}

%%{{{ prob: inter_of_subgroup_and_normal_is_normal_in_subgroup 
\problem.
\label{inter_of_subgroup_and_normal_is_normal_in_subgroup}%
Se $H\subgroup G$ e $N\normal G$, então $H\inter N \normal H$.

\solution
Temos $H \inter N \subgroup N \subgroup G$ como intersecção de subgrupos
(veja exercícios~\refn{intersection_of_subgroups_is_a_subgroup}
e~\refn{subgroup_is_an_order}).
Basta mostrar que $H \inter N \normal H$,
ou seja, que $H\inter N$ é fechado pelos conjugados.
Tome $x \in H\inter N$ e $h\in H$.
Temos:
$$
\alignat2
h x \ginv h &\in H  \qqby{$H \subgroup G$}\\
h x \ginv h &\in N  \qqby{$N \normal G$}
\endalignat
$$
Logo $hx\ginv h \in H\inter N$.

\endproblem
%%}}}

\endproblems
%%}}}

%%{{{ Further reading 
\further.

Para praticar com propriedades de operações vale a pena
resolver os primeiros 15 problemas do~\cite{halmoslapb}.

Livros introdutórios em álgebra abstrata tratam em geral
a teoria de grupos mais profundamente que podemos tratar aqui.
\cite{pinteralgebra} é um desses livros, bastante acessível,
com exemplos de diversas áreas, mostrando várias aplicações.
Uma \emph{excelente} introdução em vários tópicos de álgebra é
o~\cite{hersteintopics}, famoso para sua exposição e didática.

A álgebra abstrata foi introduzida nos currículos de graduação
com o clássico~\cite{babybm}.  Os mesmos autores, no~\cite{papamb},
apresentam álgebra mais profundamente e com um cheiro
categórico (veja~\ref{Category_theory}).

Depois de se acostumar com as idéias algébricas em geral,
dois livros focados especialmente em teoria de grupos
são os~\cite{rosegroups} e~\cite{rotmangroups}.

\endfurther
%%}}}

\endchapter
%%}}}

%%{{{ chapter: Abstract algebra 
\chapter Álgebra abstrata.

%%{{{ History 
\history.

\TODO \Galois{}Galois, \Abel{}Abel.

\endhistory
%%}}}

%%{{{ Algebraic structures 
\section Estruturas algébricas.

\endsection
%%}}}

%%{{{ Rings 
\section Aneis.

% mention first definition by \Fraenkel{}Fraenkel

\endsection
%%}}}

%%{{{ Integral domains 
\section Domínios integrais.

\endsection
%%}}}

%%{{{ Fields 
\section Corpos.

\endsection
%%}}}

%%{{{ Monoids 
\section Monoides.

\endsection
%%}}}

%%{{{ Problems 
\problems.

\endproblems
%%}}}

%%{{{ Further reading 
\further.

\cite{pinteralgebra},
\cite{hersteintopics},
\cite{babybm},
\cite{papamb}.

\endfurther
%%}}}

\endchapter
%%}}}

%%{{{ chapter: The real numbers 
\chapter Os números reais.

%%{{{ Axioms 
\section Axiomas.

\endsection
%%}}}

%%{{{ Limits 
\section Limites.

\endsection
%%}}}

%%{{{ Series 
\section Séries.

\endsection
%%}}}

%%{{{ Problems 
\problems.

\endproblems
%%}}}

%%{{{ Further reading 
\further.

\cite{spivakcalculus},
\cite{hardypuremath},
\cite{apostol1}~\&~\cite{apostol2}.

\endfurther
%%}}}

\endchapter
%%}}}

%%{{{ chapter: Metric spaces 
\chapter Espaços métricos.

%%{{{ Problems 
\problems.

\endproblems
%%}}}

%%{{{ Further reading 
\further.

\cite{simmonstopology},
\cite{kolmogorovfomin},
\cite{carothersreal}.

``Baby Rudin''~\cite{babyrudin}.

\endfurther
%%}}}

\endchapter
%%}}}

%%{{{ chapter: Cantor's paradise 
\chapter O paraíso de Cantor.
\label{Cantors_paradise}%

%%{{{ A bit of historical context 
\section Um pouco de contexto histórico.

\TODO
\Cantor{}Cantor;
\Liouville{}Liouville;
\Hermite{}Hermite;
\Fourier{}Fourier;
\vonLindemann{}von~Lindemann;
\Weierstrass{}Weierstrass.

\endsection
%%}}}

%%{{{ What is counting? 
\section O que é contar?.

\definition.
\sdefined {\finord {\holed n}} {o conjunto $\set {0,\dotsc,n-1}$}
\label{finord}
Como usamos bastante o conjunto $\set{0,1,\dotsc,n-1}$,
vale a pena introduzir uma notação para o denotar:
$$
\finord n \pseudodefeq \set{0,\dotsc,n-1}.
$$

\exercise.
Defina o $\finord n$ usando a notação ``set-builder''.

\hint
Começa com o $\nats$ e filtre seus elementos usando suas relações de ordem.

\solution
Definimos
$$
\finord n \defeq \setst {i \in \nats} { 0 \leq i < n }.
$$

\endexercise

\exercise.
Defina o operador $\finord{\hole} : \nats \to \pset\nats$ recursivamente.

\hint
Cuidado com as operações e os ``tipos'' dos seus argumentos.

\solution
Qualquer uma das definições abaixo serve:
$$
\xalignat 2
\finord 0 &\defeq \emptyset                                     & \finord 0    &\defeq \emptyset\\
\finord n &\defeq \finord {n-1} \union \set{n-1}\qquad(n > 0)   & \finord {Sn} &\defeq \finord n \union \set n
\endxalignat
$$

\endexercise

\endsection
%%}}}

%%{{{ Equinumerosity 
\section Equinumerosidade.

\definition Eqüinúmeros.
\tdefined {eqüinúmeros}
\sdefined {\holed A \eqc \holed B} {os $A$ e $B$ são equinúmeros}
Chamamos os conjuntos $A$ e $B$ \dterm{eqüinúmeros} sse
existe bijeção $f : A \bijto B$.
Escrevemos:
$$
A \eqc B \defiff \lexists f {f : A \bijto B}.
$$

\exercise.
Prove que a relação $\eqc$ é uma relação de equivalência.

\endexercise

Continuamos em definir mais relações para comparar os tamanhos de conjuntos:

\definition.
\sdefined {\holed A \leqc \holed B} {o $A$ é menor-ou-igual em cardinalidade que o $B$}
\sdefined {\holed A \ltc \holed B}  {o $A$ é menor em cardinalidade que o $B$}
Sejam $A$ e $B$ conjuntos.
Definimos
$$
\align
A \leqc B &\defiff \lexists {B_0\subseteq B} {A \eqc B_0}\\
A \ltc B  &\defiff A\leqc B \mland A\neqc B
\endalign
$$
Seguindo nossa prática comum, usamos também $A\gtc B$ como sinónimo de $B\ltc A$,
$A \not\geqc B$ para significar que não é o caso que $B \leqc A$, etc.

\exercise.
Prove ou disprove a afirmação que podemos usar a seguinde definição como alternativa:
$$
A \ltc B \askiff \lexists {B_0\subsetneq B} {A \eqc B_0}.
$$

\hint
$\nats\ltc\ints$?

\endexercise

\proposition.
A $\eqc$ é:
$$
\align
\text{reflexiva:}\quad& A \eqc A;\\
\text{transitiva:}\quad& A \eqc B \mland B \eqc C \implies A \eqc C;\\
\text{simétrica:}\quad& A \eqc B \implies B \eqc A.
\endalign
$$
\sketch.
Usamos as bijeções seguintes: identidade, composição, inversa.
\qes

\proposition.
A $\leqc$ é:
$$
\align
\text{reflexiva:}              \quad &A \leqc A;\\
\text{transitiva:}             \quad &A \leqc B \mland B \leqc C \implies A \leqc C;\\
\text{não antisimétrica:}      \quad &A \leqc B \mland B \leqc A \nimplies A = B;\\
\text{``equiantissimétrica'':} \quad &A \leqc B \mland B \leqc A \implies A \eqc B.
\endalign
$$
\sketch.
Para as duas primeiras usamos a identidade e a composição respectivamente.
Para a próxima tomando $A\asseq \nats$ e $B\asseq \ints$ serve.
A última é realmente difícil para provar:
é um corolário direto
do teorema Schröder--Bernstein (\ref{schroder_bernstein}).
\qes

\exercise.
Quais das operações $\Union$, $\Inter$, $\powerset$, $\to$, e $\times$ respeitam as cardinalidades?

\endexercise

\endsection
%%}}}

%%{{{ What is cardinality? 
\section O que é cardinalidade?.

\note A dupla abstração de Cantor.

\endsection
%%}}}

%%{{{ Guessing games 
\section Jogos de adivinhar.

\exercise.
Para cada um dos conjuntos seguintes, decida se é contável ou não.
$$
\align
I &= \setst {x\in\reals}   {\text{$x$ é irracional}}\\
A &= \setst {x\in\reals}   {\text{$x$ é algébrico}}\\
T &= \setst {x\in\reals}   {\text{$x$ é transcendental}}\\
C &= \setst {z\in\complex} {\modulus z = 1}\\
L &= \set {\text{os programas sintaticamente corretos numa linguagem de programação}}\\
G &= \graph(f), \quad \text{onde $f : \reals\to\reals$ definida pela $f(x) = x^2$}\\
D &= (\nats \to \set{0,2})\\
S &= (\nats \to \nats)\\
P &= (\nats \bijto \nats)\\
Z &= \setst {f\in(\nats\to\nats)} {(\exists n_0\in\nats)(\forall n>n_0)\left[\,f(n)=0\,\right]} 
\endalign
$$

\endexercise

\remark.
Para corresponder numa bijeção mesmo, o jogo tem que ser modificado, para
proibir repetições do mesmo palpite.
Observamos que se o jogador tem uma estratégia para ganhar num jogo que 
permite repetições de palpites, ele já pode a adaptar, para ganhar no 
jogo com a restrição: ele apenas segue a estratégia do jogo ``livre''
e quando aparecem palpites que ele já adivinhou, ele pula para o próximo,
até chegar num palpite que ele não tentou ainda, para o tentar. 
Por exemplo, para enumerar os racionais sabendo uma enumeração dos pares
de inteiros, copiamos a estrategia do $\ints^2$, pulando pares que ou não 
correspondem em racionais (como o $(0,0)$ por exemplo), ou que são iguais 
com palpites anteriores (como o $(2,4)$ que pulamos por causa do (1,2)).

\codeit EnumPairs.
\label{program_enumpairs}
Implemente uma estratégia para ganhar no jogo sem repetições com
conjunto-segredos o $\nats^2$.  Ou seja, escreva um programa que imprime (pra
sempre) os palpites do jogador na sua ordem. 
\endcodeit

\codeit EnumRatsReps.
\label{program_enumrats}
Modifique o EnumPairs para o caso com conjunto-segredos
o conjunto de racionais não-negativos, mas permitindo repetições. 
Represente cada palpite como fracção,
imprimindo por exemplo o racional $\frac 1 3$ como o string
``{\tt 1/3}''.
\endcodeit

\codeit EnumRats.
\label{program_enumratsnoreps}
Modifique o EnumRats para o caso que o jogo proibe
repetições (mas para o mesmo conjunto $\rats_{\geq0}$). 
\endcodeit

\exercise Jogador amnésico.
Ache uma estratégia para ganhar no jogo com conjunto-segredos o 
conjunto dos racionais não-negativos, se o jogador tem memória que o 
permite lembrar apenas seu último palpite!

\endexercise

\exercise.
\label{nextPair}
Defina uma função 
$\namedop{nextPair} : \nats^2 \to \nats^2$
tal que para cada entrada $(n,m)$ ela retorna o próximo palpite do jogador 
que acabou de tentar o $(n,m)$, no jogo com conjunto-segredos o $\nats^2$. 
Considera que sua estratégia começa com o palpite $(0,0)$. 
Assim, a enumeração representada por a estratégia do jogador seria a: 
$$
    (0,0), f(0,0), f^2(0,0), f^3(0,0), \dotsc,
$$
ou seja, a seqüência $\set{ f^n (0,0) }_{n}$.

\endexercise

\codeit NextPair.
Implemente função do~\ref{nextPair}.
\endcodeit

\codeit RatApprox.
\label{program_ratapprox}
Usando uma implementação de enumeração $\set{q_n}_n$ do $\rats$
(com ou sem repetições), implemente uma função
$a : \reals\times\reals \to \nats$ que, dados $x\in\reals$ e $\varepsilon>0$
retorna o primeiro $n\in\nats$ com a propriedade $\abs{q_n - x} < \varepsilon$:
$$
a(x,\varepsilon) = \min\setst{n\in\nats}{\abs{q_n-x}<\varepsilon}.
$$
\endgraf
Se tua linguagem de programação suporta funções de ordem superior,
considere que seu primeiro argumento deve ser a própria enumeração $q$:
$$
\align
\namedfun{ratApprox} &\eqtype (\nats\to\rats) \to \reals \to \reals \to \nats\\
\namedfun{ratApprox}\ q\ x\ \varepsilon &= \min\setst{n\in\nats}{\abs{q_n-x}<\varepsilon}
\endalign 
$$
Alternativamente, pode representar uma enumeração de racionais
como uma lista (infinita) de racionais.  Considere retornar o
primeiro racional suficientemente próximo alem de apenas seu
indice.  Improvise e teste sua função, vendo quanto ``demora''
uma enumeração para chegar suficientemente perto de um número
pre-determinado, sendo racional ou não.
Por exemplo, use $x=\sqrt 2$ ou $e$ ou $\pi$,
e
$\varepsilon=1, 1/2, 1/4, \dotsc$.
Assim, para qualquer real $x$, tu pode
\emph{construir}---mesmo não muito ``eficientemente''---uma
seqüência de racionais que converge em $x$, apenas aplicando a função
$\lambda \varepsilon. \namedfun{ratApprox}\ q\ x\ \varepsilon$
em argumentos que formam qualquer seqüência que convirja no zero!
\endcodeit

\endsection
%%}}}

%%{{{ Finite and infinite; countable and uncountable 
\section Finitos e infinitos; contáveis e incontáveis.

\definition.
\label{finite_set}
\label{infinite_set}
\tdefined{finito}
\tdefined{infinito}
O conjunto $A$ é \dterm{finito} sse existe $n\in\nats$ tal que $\finord n \eqc A$.
O conjunto $A$ é \dterm{infinito} sse $A$ não é finito.

\definition.
\label{countable_set}
\label{uncountable_set}
\tdefined{contável}[conjunto]
\tdefined{incontável}[conjunto]
O conjunto $A$ é \dterm{contável} sse $A$ é finito ou $\nats\eqc A$.
Usamos os termos \dterm{enumerável} e \dterm{denumerável} como sinónimos
de contável.
O conjunto $A$ é \dterm{incontável} sse $A$ não é contável.

\definition.
\label{enumeração}
\tdefined{enumeração}
Uma \dterm{enumeração} dum conjunto $A$ é qualquer surjecção
$\pi : \nats\surjto A$.
Assim temos:
$$
A = \img \pi {\nats} = \set{\pi(0), \pi(1), \pi(2), \dotsc}
$$

\proposition.
Seja $A$ conjunto.
O.s.s.e.:
\item{\rm (1)}   $A$ é contável
\item{\rm (2)}  $A \leqc \nats$
\item{\rm (3)} $A=\emptyset$ ou $A$ possui enumeração.
\sketch.
As direcões $(1)\Rightarrow(2)\Rightarrow(3)$ são conseqüências fáceis
das definições.  Para ``fechar o round-robin'' ($(3)\Rightarrow(1)$),
precisamos definir uma \emph{bijecção} $f : \nats\bijto A$, dados uma
\emph{surjecção} $\pi:\nats\surjto A$.  Usamos recursão e o princípio
da boa ordem.
\qes

\endsection
%%}}}

%%{{{ Cantor's first diagonal argument 
\section O primeiro argumento diagonal de Cantor.

\endsection
%%}}}

%%{{{ Cantor's segundo diagonal argument 
\section O segundo argumento diagonal de Cantor.

\exercise.
Por que não podemos usar o mesmo argumento para concluir que o $\setst{q \in \rats}{0\leq q\leq 1}$ também é incontável?

\hint
Como tu vai provar que o número construido pela método de Cantor realmente é um elemento de $\rats\inter[0,1]$?

\endexercise

\endsection
%%}}}

%%{{{ Some important applications 
\section Umas aplicações importantes da teoria de Cantor.

\note Os números racionais.

\note Os números irracionais.

\note Os números algebricos.

\note Os números transcendentais.

\endsection
%%}}}

%%{{{ Cantor set 
\section O conjunto de Cantor.

\endsection
%%}}}

%%{{{ Looking for bijections 
\section Procurando bijeções.

\endsection
%%}}}

%%{{{ The Schröder--Bernstein theorem 
\section O teorema Schröder--Bernstein.

\note Abordagem amorosa.

\theorem Schröder--Bernstein.
\Schroder{}\Bernstein{}%
\label{schroder_bernstein}%
\ii{teorema}[Schröder--Bernstein]%
Sejam conjuntos $A$ e $B$ e funções injetoras
$f : A \injto B$ e $g : B \injto A$.
Então existe bijeção $\phi : A \bijto B$.

\endsection
%%}}}

%%{{{ Looking for injections 
\section Procurando injeções.

\endsection
%%}}}

%%{{{ Encodings 
\section Codificações.

\endsection
%%}}}

%%{{{ The different infinities so far 
\section As diferentes infinidades até agora.

\endsection
%%}}}

%%{{{ Cantor's theorem and its consequences 
\section O teorema de Cantor e suas conseqüências.

\theorem Cantor.
\Cantor[teorema]%
\ii{teorema}[Cantor]%
\label{cantor_theorem}%
Seja $A$ conjunto.
Então $A\ltc \pset A$.

\corollary.
Existe uma infinidade (contável) de cardinalidades infinitas:
$$
\nats
\ltc \pset\nats
\ltc \pset\pset\nats
\ltc \pset\pset\pset\nats
\ltc \pset\pset\pset\pset\nats
\ltc \dotsc
$$

\definition.
\sdefined{\aleph_0}{a cardinalidade do $\nats$}%
\sdefined{\continuum}{a cardinalidade do $\pset\nats$ e do $\reals$}%
\tdefined{aleph 0}%
\tdefined{continuum}%
Denotamos a cardinalidade de $\nats$ por $\aleph_0$ (\dterm{aleph 0}),
e a cardinalidade de $\pset\nats\eqc\reals$ por $\continuum$.
Chamamos o $\continuum$ o \dterm{continuum}.

\corollary.
Não existe cardinalidade máxima: para qualquer conjunto $M$, o conjunto $\pset M$ tem cardinalidade maior.

\note.
Considere os conjuntos
$$
\emptyset
\ltc \pset\emptyset
\ltc \pset\pset\emptyset
\ltc \pset\pset\pset\emptyset
\ltc \pset\pset\pset\pset\emptyset
\ltc \dotsb
$$
Observe que cada conjunto nessa seqüência (infinita) de conjuntos é finito.
Mas, quais são suas cardinalidades?
Calculamos:
$$
\align
\card{\emptyset}                     &= 0\\
\card{\pset\emptyset}                &= 1\\
\card{\pset\pset\emptyset}           &= 2\\
\card{\pset\pset\pset\emptyset}      &= 4\\
\card{\pset\pset\pset\pset\emptyset} &= 8\\
                                     &\eqvdots\\
\endalign
$$
Observamos que a seqüência dessas cardinalidades ``tem burácos''.
Por exemplo, nenhum desses conjuntos tem cardinalidade $3$,
mesmo que realmente tem conjuntos com essa cardinalidade
(por exemplo o $\finord 3=\set{0,1,2}$).
Ou seja, existe conjunto $C$ com
$$
\pset\pset\emptyset\ltc C\ltc \pset\pset\pset\emptyset.
$$
Similarmente achamos conjuntos ``estritamente entre'' os conjuntos que aparecem
depois nessa seqüência.

Será que tem conjuntos ``estritamente entre'' alguns dos conjuntos infinitos
da seqüência anterior?

Será que tem algum conjunto incomparável com todos eles?

Será que tem algum conjunto com cardinalidade maior que qualquer uma das
cardinalidades deles?

\endsection
%%}}}

%%{{{ Cardinal comparability 
\section Comparabilidade de cardinais.

\endsection
%%}}}

%%{{{ The continuum hypothesis 
\section Hipótese do continuum.

\endsection
%%}}}

%%{{{ Ordinal vs. cardinal numbers 
\section Números ordinais vs.~cardinais.
\endsection
%%}}}

%%{{{ A taste of measure theory 
\section Um toque de teoria de medida.

\endsection
%%}}}

%%{{{ Problems 
\problems.

\TODO Cantor's letter to Dedekind and Dedekind's reply.

%%{{{ prob: sime_simo_eqclass_card 
\problem.
\label{sime_simo_eqclass_card}%
\def\sime{\rel{\stackrel{{}_{\mathrm e}}=}}%
\def\simo{\rel{\stackrel{{}_{\mathrm o}}=}}%
Seja~$f:\nats\to\nats$ e as $\sime$ e $\simo$ como no~\ref{simz_sime_simo_simi}:
$$
\align
f\sime g&\defiff f(2n)   = g(2n)  \ \text{para todo $n\in\nats$}\\
f\simo g&\defiff f(2k+1) = g(2k+1)\ \text{para todo $k\in\nats$}\\
\endalign
$$
Qual é a cardinalidade do $\eqclass f {\sime} \inter \eqclass f {\simo}$?

\endproblem
%%}}}

\endproblems
%%}}}

%%{{{ Further reading 
\further.

Sobre a teoria de conjuntos de Cantor:
\cite{kleeneIM},
\cite{ynmnst}.

Sobre teoria da medida:
\cite{bartlemeasure},
\cite{halmosmeasure},
\cite{taylorintegration}.

\endfurther
%%}}}

\endchapter
%%}}}

%%{{{ chapter: General topology 
\chapter Topologia geral.

%%{{{ Problems 
\problems.

\endproblems
%%}}}

%%{{{ Further reading 
\further.

\cite{janichtopology},
\cite{simmonstopology},
\cite{munkrestopology},
\cite{willardtopology}.

\cite{vickerstopology}.

\endfurther
%%}}}

\endchapter
%%}}}

%%{{{ chapter: Graph theory 
\chapter Teoria de grafos.

%%{{{ Problems 
\problems.

\endproblems
%%}}}

%%{{{ Further reading 
\further.

Umas introduções extensas são
o clássico~\cite{bondymurty1976}
e o mais novo~\cite{ChartrandZhang}.
Depois continua com~\cite{diestelgraph}
e~\cite{bondymurty2011}.
Finalmente, um nível ainda mais avançado, \cite{bollobasmodern}.

\endfurther
%%}}}

\endchapter
%%}}}

%%{{{ chapter: Structural recursion and induction 
\chapter Recursão e indução estrutural.

%%{{{ Problems 
\problems.

\endproblems
%%}}}

%%{{{ Further reading 
\further.

\cite{aczelinductive},
\cite{ynminduction}.

\endfurther
%%}}}

\endchapter
%%}}}

%%{{{ chapter: Lambda calculus 
\chapter Lambda calculus.

%%{{{ The untyped lambda calculus 
\section O $\lambda$-calculus não-tipado.

\endsection
%%}}}

%%{{{ Faithfully representing mathematics 
\section Representando matemática fielmente.

\endsection
%%}}}

%%{{{ Programming 
\section Programmando.

\endsection
%%}}}

%%{{{ Recursion and fixpoints 
\section Recursão e fixpoints.

\endsection
%%}}}

%%{{{ Functional programming revisited 
\section Programação funcional revisitado.

\endsection
%%}}}

%%{{{ Problems 
\problems.

%%{{{ prob: what_does_this_lambda_term_with_minus_do 
\problem.
\label{what_does_this_lambda_term_with_minus_do}
Suponha que já temos definido um $\lambda$-term $\Lmac{minus}$,
que comporta corretamente, no sentido que:
$$
\align
\Lmac{minus}\ {\Lnum n}\ {\Lnum m}
&\Lfinto \knuthcases{
  \Lnum {n - m},   &se $n \geq m$\cr
  \Lnum {0},       &se $n < m$.
}
\endalign
$$
Explique o comportamento do termo
$$
\Lmac f  \asseq  \Llam n {n\ (\Lmac{minus}\ {\Lnum 1})\ {\Lnum 0}}
$$
quando for aplicado para um numeral de \Church[numeral]{}Church $\Lnum k$:
qual é a função $f : \nats\to\nats$ que o termo $\Lmac f$ computa?

\hint
Cuidado com \Curry[currificação]{}Curry.

\endproblem
%%}}}

\endproblems
%%}}}

%%{{{ Further reading 
\further.

\cite{nederpeltgeuvers},
\cite{lecturesch}.
\cite{hindleyseldinlambdacl},
\cite{krivinelambda}.
\cite{barendregtlambda}.

\endfurther
%%}}}

\endchapter
%%}}}

%%{{{ chapter: Combinatory logic 
\chapter Lógica combinatória.

%%{{{ Our first combinators 
\section Nossos primeiros combinadores.

%%{{{ df: first_combinators 
\definition.
\label{first_combinators}%
$$
\xalignat4
    \cI\,x    &\Cto x     &   \cB\,x\,y\,z &\Cto x\,(y\,z) &   \cS\,x\,y\,z &\Cto x\,z\,(y\,z)&\cR\,x\,y\,z &\Cto y\,z\,x\\\\
    \cK\,x\,y &\Cto x     &   \cBp\,x\,y\,z&\Cto y\,(x\,z) &   \cW\,x\,y    &\Cto x\,y\,y     &\cV\,x\,y\,z &\Cto z\,x\,y\\\\
    \cM\,x    &\Cto x\,x  &   \cC\,x\,y\,z &\Cto x\,z\,y   &                                
\endxalignat
$$
%%}}}

%%{{{ x: some_equiv_to_I 
\exercise.
\label{some_equiv_to_I}%
Mostre que o combinador
$\C S\ \C K\ (\C W\ (\C I\ \C B))$ comporta como o $\C I$.

\endexercise
%%}}}

%%{{{ x: another_equiv_to_I 
\exercise.
\label{another_equiv_to_I}%
Mostre que o combinator
$\C C\ (\C W\ \C K)\ \C K\ \C W$ comporta como o $\C I$.

\endexercise
%%}}}

%%{{{ x: CL_define_Bp 
\exercise.
\label{CL_define_Bp}%
Defina o $\cBp$ dos $\cI$, $\cM$, $\cB$, $\cC$.

\endexercise
%%}}}

%%{{{ x: CL_define_R 
\exercise.
\label{CL_define_R}
Defina o $\cR$ dos $\cI$, $\cK$, $\cM$, $\cB$, $\cBp$, $\cC$, $\cS$, $\cW$.

\endexercise
%%}}}

%%{{{ x: CL_define_V 
\exercise.
\label{CL_define_V}
Defina o $\cV$ dos $\cI$, $\cK$, $\cM$, $\cB$, $\cBp$, $\cC$, $\cS$, $\cW$, $\cR$.

\endexercise
%%}}}

\endsection
%%}}}

%%{{{ Problems 
\problems.

\endproblems
%%}}}

%%{{{ Further reading 
\further.

\cite{mockingbird},
\cite{bimbocl},
\cite{hindleyseldinlambdacl}.

\endfurther
%%}}}

\endchapter
%%}}}

%%{{{ chapter: Mathematical logic 
\chapter Lógica matemática.

%%{{{ Semantics: worlds and models 
\section Semântica: mundos e modelos.

\note Mundos em $\zolang$: valuação e sua extenção recursiva.

\note Tautologias na $\zolang$.

\note Mundos em $\folang$: estruturas, e valuação de variáveis.

\note Tautologias na $\folang$.

\note Emulando constantes por funções.

\note Emulando funções por relações.

\note Modelos.

\note Como provar que duas formulas da $\folang$ (não) são equivalentes.

\note Independência de axiomas: o 5o axioma de Euclides.

\endsection
%%}}}

%%{{{ Complete sets of connectives 
\section Conjuntos de conectivos completos.

\endsection
%%}}}

%%{{{ Classical and intuitionistic logic 
\section Lógica clássica e intuicionista.

\endsection
%%}}}

%%{{{ Problems 
\problems.

\endproblems
%%}}}

%%{{{ Further reading 
\further.

Sobre lógica matemática:
\cite{kleeneIM},
\cite{curryfoundations};
\cite{kleenelogic},
\cite{smullyanfol};
\cite{corilascar1}~\&~\cite{corilascar2},
\cite{shoenfieldlogic},
\cite{bellmachover}.

\endfurther
%%}}}

\endchapter
%%}}}

%%{{{ chapter: Axiomatic set theory 
\chapter Teoria axiomática de conjuntos.
\label{Axiomatic_set_theory}%

{\def\Nats{{\mathord{\mathbf N}}}
\def\Zero{{\mathsf 0}}
\def\Succ{{\mathsf S}}%

%%{{{ Problems in Cantor's paradise: Russell's paradox 
\section Problemas no paraíso de Cantor: o paradoxo de Russell.

%%{{{ Russell's paradox 
\note O paradoxo de Russell (1902).
\label{russells_paradox}%
Russell\Russell[paradoxo]~observou que o conjunto
$$
\Univ \defeq \set{ x \st \text{$x$ é conjunto} }
$$
tem uma peculiaridade, uma propriedade estranha:
\emph{ele pertence nele mesmo}, ou seja, $\Univ\in \Univ$.
Os conjuntos que encontramos em matemática normalmente não têm essa
propriedade: $\nats\notin\nats$, pois o $\nats$ não é um número natural!
Similarmente $\set{0,1,\set{1,2}}\notin\set{0,1,\set{1,2}}$, pois
$\set{0,1,\set{1,2}} \neq 0$,
$\set{0,1,\set{1,2}} \neq 1$, e
$\set{0,1,\set{1,2}} \neq \set{1,2}$.
Também $\emptyset \notin \emptyset$ pois nada pertence ao $\emptyset$.
Tudo bem, nenhum problema com isso, mas faz sentido definir o conjunto de todos os
conjuntos ``normais'', ou seja, aqueles que não têm essa propriedade estranha
de pertencer neles mesmo.  Russell definiu entao o conjunto seguinte:
$$
R \defeq \set{ x \st \text{$x$ é conjunto e $x\notin x$} }.
$$
E se perguntou: \emph{o conjunto $R$ é ``normal'' ou tem essa propriedade estranha?}
Consideramos os dois casos:
Se $R\in R$ então pela definição de $R$ temos que $R\notin R$; então esse caso é impossível.
Se $R\notin R$ então $R$ não pertence nele mesmo e pela definição de $R$ temos $R\in R$; e assim esse caso também é impossível!
Curtamente concluimos que:
$$
R \in R \iff R\notin R
$$
e naturalmente queremos escrever um grande \emph{``absurdo''} neste momento,
mas\dots{}
De onde chegamos nesse absurdo?
Todas as vezes que chegamos num absurdo até agora, foi tentando provar algo:
\emph{supondo uma hipótese $H$}, chegamos num absurdo, então concluimos que
sua negação $\lnot H$ é verdade, ou vice-versa, usando o ``reductio ad absurdum'',
querendo provar que a $H$ é verdade supomos sua negação $\lnot H$,
achamos um absurdo e concluimos que nossa suposição não pode ser correta,
logo $H$.
Mas aqui não começamos supondo algo aleatoriamente.
Qual vai ser nossa conclusão agora?
Parece que chegamos num absurdo apenas com lógica sem supor nada ``extra''.
Será que lógica ou matemática é quebrada?
%%}}}

%%{{{ General Comprehension Principle 
\principle Comprehensão geral.
\label{general_comprehension_principle}%
Seja $P(\dhole)$ uma condição definitiva.
Existe um conjunto
$$
\set { x \st P(x) }
$$
cujos membros são exatamente todos os objetos $x$
que satisfazem a condição: $P(x)$.
%%}}}

%%{{{ cor: The general comprehension principle is invalid 
\corollary Russell.
O princípio da comprehensão geral não é válido.
\proof.
Supondo que é, chegamos no absurdo que achamos no~\refn{russells_paradox}.
\qed
%%}}}

\endsection
%%}}}

%%{{{ Russell's and Zermelo's solutions 
\section As soluções de Russell e de Zermelo.

\note Teoria de tipos (Russell).
\Russell[teoria de tipos]{}

\note Teoria axiomática de conjuntos (Zermelo).
\Zermelo[teoria axiomática de conjuntos]{}

\note Créditos.
A teoria axiomática de conjuntos que estudamos nesse capítulo é conhecida
como ``Zermelo\Zermelo{}--\Fraenkel{}Fraenkel set theory''.
Mesmo assim, mais foram envolvidos na sua evolução, sua definição, e seu
amadurecimento, como os \Mirimanoff{}Mirimanoff, \Skolem{}Skolem,
e \vonNeumann{}von~Neumann, entre outros.

\endsection
%%}}}

%%{{{ Translations to and from the FOL of set theory 
\section Traduções de e para a FOL de conjuntos.
\label{FOL_translations_for_sets}%

%%{{{ The FOL of set theory 
\note A FOL da teoria de conjuntos.
Nessa linguagem de primeira ordem temos apenas um predicado não-constante:
o $\in$, de aridade 2, cuja interpretação canônica é ``\thole\ pertence ao \thole''.
Nosso universo aqui consiste (apenas) em conjuntos.
%%}}}

%%{{{ ex: FOL of set theory translations 
\exercise.
\label{FOL_translations_for_sets_exercise}%
Traduza as frases seguintes para a FOL da teoria de conjuntos.
\beginol
\li Existe conjunto sem membros.
\li O conjunto $x$ não tem membros.
\li O conjunto $y$ tem membros.
\li Existe conjunto com membros.
\li O $x$ é um singleton.
\li Existe conjunto com exatamente um membro.
\li Existe conjunto com pelo menos dois membros.
\li Os $x$ e $y$ têm exatamente um membro em comum.
\li Todos os conjuntos tem o $x$ como membro.
\li Existe conjunto que pertence nele mesmo.
\li O $y$ consiste em todos os subconjuntos de $x$ com exatamente 2 elementos.
\li Existe conjunto com exatamente dois membros.
\li Para todos conjuntos $a$ e $b$ sua intersecção é conjunto.
\li A união de $a$ e $b$ é um conjunto.
\li O $x$ não pertence em nenhum conjunto.
\li Existem conjuntos tais que cada um pertence no outro.
\li Existe conjunto que não é igual com ele mesmo.
\endol

\endexercise
%%}}}

%%{{{ Merely saying something doesn't make it true 
\note Apenas escrever algo não o torna verdade.
O fato que podemos expressar uma afirmação numa linguagem não quer dizer
que essa afirmação é válida.  Isso não é nada profundo: em português
também podemos escrever a frase ``a lua não é feita de queijo'', mas isso
não quer dizer que realmente não é---todos sabemos que é, certo?
Infelizmente existe um hábito de confudir as duas noções e usar
como ``prova de validade'' o fato que apenas algo foi escrito, ou dito.%
\footnote{Veja por exemplo argumentações de várias igrejas de várias religiões.}
A afirmação da formula que tu achou para a última frase
do~\ref{FOL_translations_for_sets_exercise} por exemplo é falsa em nosso
mundo de conjuntos e ainda mais: é falsa em cada mundo possível, com qualquer
interpretação de $\in$!
%%}}}

%%{{{ A useful pattern 
\note Um padrão útil.
Muitas vezes queremos dizer que
existe um certo conjunto \emph{determinado por uma propriedade
caraterística}, ou seja, um conjunto $s$ que consiste em exatamente
todos os objetos que satisfazem um certo critério.
\endgraf
\emph{Como podemos dizer isso na FOL da teoria de conjuntos?}
Fácil!  Assim:
$$
\phantom{
\forall a
\forall b
\forall c
\dotsb
}
{\color{alert}
\exists s
\forall x
\bigparen{
x\in s \liff
{}
{\color{normal}
\tunderbrace{
\text{\vphantom|\xlthole}
}{critério}
}}}.
$$
Na maioria das vezes queremos afirmar a existência dum certo conjunto
dados conjuntos $a,b,c,\dotsc$.
Nesse caso usamos apenas o
$$
\forall a
\forall b
\forall c
\dotsb
{\color{alert}
\exists s
\forall x
\bigparen{
x\in s \liff
{}
{\color{normal}
\text{\xlthole}
}}}.
$$
Esse padrão vai aparecer em muitos dos axiomas abaixo.
%%}}}

\endsection
%%}}}

%%{{{ The first axioms of Zermelo 
\section Os primeiros axiomas de Zermelo.

%%{{{ ax: Extensionality 
\axiom Extensionalidade.
\tdefined{axioma}[Extensionality]%
\label{extensionality}%
Todo conjunto é determinado por seus membros.
$$
\forall a \forall b
\paren{
a = b
\liff
\forall x
\paren{
x \in a
\liff
x \in b
}
}
\axtaglabel{ZF1}{extensionality}
$$
%%}}}

\note.
Qual o efeito disso em nosso mundo?
Ainda nem podemos garantir a existência de nada,
mas pelo menos sabemos dizer se duas coisas são iguais ou não.
Vamos logo garantir a existência dum conjunto familiar:

%%{{{ ax: Emptyset 
\axiom Emptyset.
\tdefined{axioma}[Emptyset]%
\label{emptyset}%
Existe conjunto sem membros.
$$
\exists s \forall x
\paren{
x \notin s
}
\axtaglabel{ZF2}{emptyset}
$$
%%}}}

\blah.
E já nosso mundo mudou completamente:
ganhamos nossa primeira peça, uma coisa para brincar:
\emph{um conjunto sem membros!}
E já estamos em posição de provar nosso primeiro teorema,
seguido pela nossa primeira definição:

%%{{{ thm: uniqueness_of_emptyset 
\theorem Unicidade do conjunto vazio.
\ii{unicidade!do $\emptyset$}%
\label{uniqueness_of_emptyset}%
O conjunto sem membros garantido pelo axioma~\axref{emptyset} é único.
\proof.
Suponha que $e, o$ são conjuntos ambós satisfazendo a propriedade:
$$
\forall x\paren{x \notin e}
\qqqquad
\forall x\paren{x \notin o}.
$$
Então a equivaléncia
$$
x\in e \iff x \in o
$$
é válida para todo $x$, pois ambos lados são falsos.
Logo, pelo axioma~\axref{extensionality}, $e=o$.
\qed
%%}}}

%%{{{ df: emptyset_def 
\definition Conjunto vazio.
\label{emptyset_def}%
\tdefined{conjunto}[vazio]%
\sdefined {\emptyset} {o conjunto vazio}%
Denotamos com $\emptyset$ o \dterm{conjunto vazio} com a propriedade caraterística
$$
\forall x\paren{x\notin \emptyset}.
$$
%%}}}

\blah.
\noindent\dots e agora parece que não podemos fazer muita coisa mais.
Precisamos novos axiomas:

%%{{{ ax: Pairset 
\axiom Pairset.
\tdefined{axioma}[Pairset]%
\label{pairset}%
Dado um par de conjuntos, existe conjunto que consiste
em exatamente os conjuntos do par.
$$
\forall a
\forall b
\exists s
\forall x
\bigparen{
x\in s
\liff
\paren{
x = a
\lor
x = b
}
}
\axtaglabel{ZF3}{pairset}
$$
%%}}}

%%{{{ df: doubleton 
\definition Doubleton.
\tdefined {doubleton}%
\sdefined {\set{\holed a, \holed b}} {o conjunto doubleton de $\holed a$ e $\holed b$}%
Dados $a$ e $b$ quaisquer, o conjunto que consiste nos $a$ e $b$
é chamado o \dterm{doubleton} de $a$ e $b$, e denotado por $\set {a, b}$.
Definimos assim o operador $\set{\dhole, \dhole}$.
%%}}}

%%{{{ Effects 
\note Efeitos.
Como isso muda nosso mundo?
Quais novas peças ganhamos em nosso xadrez?
Para ganhar a existência de algo usando o Pairset
precisamos dá-lo dois objetos,
pois começa com dois quantificadores universais ($\forall$) antes de chegar no seu primeiro existencial ($\exists$):
``$\forall a\forall b \exists \dots$''.
Quais objetos vamos escolher para dá-lo?
Nosso mundo está tão pobre que é fácil responder nessa pergunta:
\emph{vamos usar como $a$ e como $b$ a única peça que temos: o $\emptyset$}.
Ganhamos então que:
$$
\exists s \forall x\bigparen { x \in s \liff \paren{ x = \emptyset \lor x = \emptyset }}
$$
ou seja, $\exists s \forall x \bigparen { x \in s \liff x = \emptyset }$,
ou seja, existe o conjunto
$$
\set { x \st x = \emptyset }
$$
que acostumamos a denotá-lo por $\set \emptyset$.
Uma nova peça!  E teoremas?
%%}}}

%%{{{ thm: singleton_thm 
\theorem Singleton.
\label{singleton_thm}%
Dado conjunto $a$, existe um único conjunto cujo membro único é o $a$.
Formalmente,
$$
\forall a
\exists s
\forall x
\bigparen{
x\in s
\liff
x = a
}.
$$
\proof.
Bote $a\asseq a$ e $b\asseq a$ no Pairset~(\axref{pairset}):
$$
\phantom{\forall a}
\exists s
\forall x
\bigparen{
x \in s
\liff
x = a
}.
$$
O conjunto cuja existência está sendo afirmada tem como membro único o $a$,
e graças ao~\axref{extensionality}, ele é o único conjunto com essa propriedade.
\qed
%%}}}

%%{{{ df: singleton 
\definition.
\label{singleton}%
\sdefined {\set{\holed a}} {o conjunto singleton de $\holed a$, com membro único o $\holed a$}%
Dado qualquer conjunto $a$, o conjunto cujo único membro é o $a$
é chamado o \dterm{singleton} de $a$, e denotado por $\set a$.
Definimos assim o operador $\set{\dhole}$.
%%}}}

%%{{{ x: infinitely_many_sets 
\exercise.
Ache uma infinidade de peças novas usando apenas os axiomas \axref{extensionality}, \axref{emptyset}, e \axref{pairset}.

\hint
Agora temos duas opções para cada $\forall$ do Pairset: $\emptyset$, $\set\powerset$.

\endexercise
%%}}}

\blah.
Mesmo que nosso mundo mudou drasticamente---sim, ganhamos uma infinidade de
objetos---ele ainda tá bem limitado.

%%{{{ x: infinitely_many_singletons 
\exercise.
Mostre como construir uma infinidade de singletons e uma infinidade de doubletons.

\hint
\axref{emptyset}~\&~\ref{singleton_thm}.

\hint
Comece com o $\emptyset$ e aplique iterativamente o operador $\set{\dhole}$.

\solution
Graças ao Emptyset~(\axref{emptyset}) temos o $\emptyset$.
Aplicamos iterativamente o operador $\set{\dhole}$ e assim construimos
a seqüência infinita de singletons
$$
\emptyset,
\set{\emptyset},
\set{\set{\emptyset}},
\set{\set{\set{\emptyset}}},
\set{\set{\set{\set{\emptyset}}}},
\dotsc
$$
Para os doubletons, uma abordagem seria aplicar o $\set{\emptyset,\dhole}$
em todos os membros da seqüência em cima começando com o segundo.
Construimos assim a seqüência seguinte de doubletons:
$$
\set{\emptyset, \set{\emptyset}},
\set{\emptyset, \set{\set{\emptyset}}},
\set{\emptyset, \set{\set{\set{\emptyset}}}},
\set{\emptyset, \set{\set{\set{\set{\emptyset}}}}},
\dotsc
$$

\endexercise
%%}}}

%%{{{ x: only_sets_with_up_to_two_elements 
\exercise.
\label{only_sets_with_up_to_two_elements}%
Podemos garantir a existência de conjuntos com qualquer cardinalidade finita que desejamos?

\hint
No exercício anterior construimos uns conjuntos com cardinalidade até 2.
Tente construir conjunto com cardinalidade 3.

\hint
Não tem como.  Por quê?

\solution
A construção dum conjunto com cardinalidade finita $n$ só pode ter sido garantida
ou pelo Emptyset, se $n=0$ (nesse caso o conjunto construido é o próprio $\emptyset$),
ou pelo Pairset, se $n>0$.  Mas o Pairset so construe
conjuntos de cardinalidade $1$ ou $2$, dependendo se o aplicamos em conjuntos
iguais ou não (respectivamente).
Para concluir: nossos axiomas não são suficientemente poderosos para garantir
a existência de conjuntos com cardinalidades maiores que $2$.

\endexercise
%%}}}

\note.
O próximo axioma não vai nos permitir---por enquanto---definir novos conjuntos.
Mas é a versão ``bug-free'' do princípio da comprehensão geral.
Com isso, o paradoxo de Russell se torna teorema!

%%{{{ ax: Separation 
\axiom Separation (schema).
\tdefined{axioma}[Separation (schema)]%
\label{separation}%
{\rm Para cada propriedade $\phi(\dhole)$, o seguinte:}
para todo conjunto, a colecção de todos os seus membros que têm a propriedade $\phi$ é um conjunto.
$$
\forall w
\exists s
\forall x
\bigparen{
x\in s
\liff
\paren{
x\in w 
\land
\phi(x)
}
}
\axtaglabel{ZF4}{separation}
$$
%%}}}

%%{{{ x: how many axioms? 
\exercise.
Quantos axiomas temos listado até este momento?

\hint
Não são 4.

\hint
Não é um número finito de axiomas!

\solution
Veja a discução no~\refn{axioms_vs_axiomatic_schemata}.

\endexercise
%%}}}

%%{{{ Axioms vs. axiomatic schemata 
\note Axiomas vs{.}~esquemas axiomáticos.
\label{axioms_vs_axiomatic_schemata}%
Usamos o termo ``esquema axiomático'', pois para cada fórmula $\phi(\dhole)$,
ganhamos um novo axioma pelo~\axref{separation}.
Para enfatizar isso podemos até o citar como~\axref{separation}$_{\phi}$.
Antes de o usar então precisamos primeiramente escolher nossa fórmula $\phi$, assim passando do \emph{esquema}~\axref{separation}
para o \emph{axioma}~\axref{separation}$_{\phi}$.
Agora como o axioma começa com ``$\forall w \exists \dots$'',
precisamos escolher em qual conjunto $w$ nos vamos o aplicar,
para ganhar finalmente um novo conjunto.
%%}}}

%%{{{ df: set_builder_separation 
\definition.
Denotamos com
$$
\setst {x \in W} {\phi(x)}
$$
o conjunto único (pelo~\axref{extensionality}) garantido pelo~\axref{separation} quando o aplicamos com uma fórmula $\phi(x)$ para um conjunto $W$.
%%}}}

%%{{{ x: separation_yields_no_new_sets_yet 
\exercise.
\label{separation_yields_no_new_sets_yet}%
Mostre que os conjuntos garantidos pelos~\axref{extensionality}--\axref{pairset}
são os mesmos com os conjuntos garantidos pelos~\axref{extensionality}--\axref{separation}.

\hint
Usando o~\axref{separation}, criamos subconjunto de um conjunto dado.

\hint
Quais são todas as cardinalidades possíveis para o conjunto em qual
usamos o~\axref{separation}?

\endexercise
%%}}}

%%{{{ How to use the Separation 
\note Como usar o~\axref{separation}.
Queremos mostrar que uma certa classe $C$ de objetos realmente é um conjunto.
Para conseguir isso com o~\axref{separation},
precisamos construir (pelos axiomas!) um \emph{conjunto}
$W$ que contenha todos os objetos da nossa classe $C$.
Em geral o $W$ vai ter mais elementos, um certo ``lixo'', que precisamos nos livrar.
E é exatamente com o~\axref{separation} que jogamos fora esse lixo,
usando uma apropriada fórmula $\phi$ como ``tesoura'' para cortar o $W$ e ficar só com o $C$,
garantido agora de ser conjunto.
Vamos ver uns exemplos desse uso,
enriquecendo nosso mundo com uns operadores conhecidos.
%%}}}

%%{{{ eg: define inters 
\example.
Defina o operador $\dhole\inter\dhole$.

\solution
Dados conjuntos $a$ e $b$,
precisamos achar um conjunto $W$ que contenha todos os membros
da intersecção desejada.  Assim vamos conseguir definir o $a \inter b$
usando o~\axref{separation} com filtro a fórmula
$$
\phi(x)\asseq {x \in a \land x \in b}
$$
Observe que todos os elementos da $a \inter b$ são elementos tanto de $a$,
quanto de $b$.
Temos então duas opções.  Vamos escolher a primeira e construir o
$$
\setst {x \in a} {x \in a \land x \in b}
$$
Observamos que com essa escolha nem precisamos a parte ``$x \in a$'' em nosso filtro.
Chegamos assim em duas soluções para nosso problema:
$$
\setst {x \in a} {x \in b}
\qqqquad
\setst {x \in b} {x \in a}
$$

\endexample
%%}}}

%%{{{ df: inters 
\definition Intersecção binária.
\label{inters_constructed}%
Sejam conjuntos $a,b$.
Usando o~\axref{separation} definimos
$$
a\inter b \defeq \setst {x\in a} {x\in b}.
$$
%%}}}

%%{{{ x: define setminus 
\exercise.
Defina o operador $\dhole\setminus\dhole$.

\hint
Começa considerando dados conjuntos $a$ e $b$.
Procure um \emph{conjunto} que contenha todos os membros
da classe $a\setminus a$ que tu queres construir como conjunto.
Cuidado: nesse caso não temos as duas opções que tivemos
na~\ref{inters_constructed}.

\solution
Usando o~\axref{separation} definimos:
$$
a\setminus b\defeq \setst {x \in a} {x \notin b}.
$$

\endexercise
%%}}}

%%{{{ x: cannot_define_union_yet 
\exercise.
\label{cannot_define_union_yet}%
Tente definir os operadores $\dhole\union\dhole$ e $\dhole\symdiff\dhole$.

\hint
Dados conjuntos $a$ e $b$,
precisas achar um \emph{conjunto} $W$
que contenha todos os membros de $a\union b$,
para depois filtarar apenas os certos usando
como filtro a fórmula
$$
\phi(x) \asseq
x \in a \lor x \in b
$$
para a operação $\dhole\union\dhole$;
e similarmente para a $\dhole\symdiff\dhole$ só que
para essa o filtro vai ser a fórmula
$$
\phi(x)\asseq
\paren{x \in a \land x \notin b}
\lor
\paren{x \notin a \land x \in b}.
$$

\hint
Não tem como!

\solution
Não tem como!
Os axiomas que temos por enquanto não são suficientemente poderosos para
definir nenhuma dessas operações!

\endexercise
%%}}}

%%{{{ thm: russells_paradox_to_theorem 
\theorem.
\label{russells_paradox_to_theorem}%
Dado qualquer conjunto existe algo que não pertence nele.
Ainda mais, pelo menos um dos seus subconjuntos não pertence nele.
\sketch.
Tome um conjunto $A$.
Usamos a mesma idéia do paradoxo de Russell\Russell[de paradoxo para teorema],
só que essa vez não consideramos \emph{todos} os conjuntos que não
pertencem neles mesmo, mas apenas aqueles que pertencem ao $A$.
Concluimos que $\russell A \notin A$ pois o caso $\russell A \in A$ chega 
no mesmo absurdo de Russell.
\qes
%%}}}

%%{{{ Univ is not a set 
\corollary.
\label{Univ_is_not_a_set}%
O universo $\Univ$ não é um conjunto.
\proof.
Se $\Univ$ fosse um conjunto, aplicando o teorema teriamos um conjunto $\russell \Univ$
com $\russell\Univ \notin \Univ$, absurdo pela definição do $\Univ$.
\qed
%%}}}

%%{{{ ax: Powerset 
\axiom Powerset.
\tdefined{axioma}[Powerset]%
\label{powerset}%
Para cada conjunto a colecção de todos os seus subconjuntos é um conjunto.
$$
\forall a
\exists s
\forall x
\paren{
x\in s
\liff
x \subseteq a
}
\axtaglabel{ZF5}{powerset}
$$
%%}}}

%%{{{ df: pset 
\definition.
\tdefined{powerset}%
\sdefined {\pset {\holed a}} {o conjunto de partes (powerset) de $\holed a$}%
\iisee{conjunto!de partes}{powerset}%
\iisee{conjunto!potência}{powerset}%
Dado conjunto $a$, escrevemos
$\pset a$
para o conjunto garantido pelo Powerset~(\axref{powerset}),
que é único graças ao~Extensionality(\axref{extensionality}).
Definimos assim o operador $\pset\dhole$.
%%}}}

%%{{{ x: powersingleton 
\exercise.
\label{powersingleton}%
Seja $a$ conjunto.  Mostre que a classe
$$
\class { \set x } { x \in a }
$$
de todos os singletons de elementos de $a$ é conjunto.

\solution
Queremos mostrar que dado conjunto $a$, a classe
$$
\class { \set x } { x \in a }
$$
é conjunto.
Basta só achar um conjunto $W$ tal que todos os $\set x \in W$.
Botamos apenas o $W\asseq\pset a$ que sabemos que é conjunto
pelo~Powerset~(\axref{powerset}), assim ganhando o conjunto
$$
\setst { z\in \pset a } { \lexists {x \in a} {z = \set x} }.
$$
pelo~Separation~(\axref{separation}).

\endexercise
%%}}}

%%{{{ x: arbitrarily_large_finite_sets 
\exercise.
\label{arbitrarily_large_finite_sets}
Usando
os~\axref{extensionality}+\axref{emptyset}+\axref{pairset}+\axref{powerset},
podemos construir conjunto com cardinalidade finita, arbitrariamente grande?

\solution
Sim.
Seja $n\in\nats$.
Precisamos construir um conjunto $A$ com $\card A \geq n$.
Se $n=0$, graças ao Emptyset~(\axref{emptyset}) tomamos $A\asseq\emptyset$.
Se $n>0$, iteramos o operador $\pset\dhole$ até chegar num conjunto
cuja cardinalidade supera o $n$.

\endexercise
%%}}}

%%{{{ x: still_missing_some_finite_cardinalities 
\exercise.
\label{still_missing_some_finite_cardinalities}
Usando os~\axref{extensionality}+\axref{emptyset}+\axref{pairset}+\axref{powerset}, podemos construir conjunto com cardinalidade finita qualquer?

\hint
Não.  Por quê?

\hint
Como vimos no~\ref{only_sets_with_up_to_two_elements},
usando os \axref{extensionality}+\axref{emptyset}+\axref{pairset}
podemos consturir apenas conjuntos com cardinalidades $0$, $1$, e $2$.
Se aplicar o Powerset~(\axref{powerset}) num conjunto $A$ com cardinalidade
finita $n$, qual será a cardinalidade do $\pset A$?

\solution
Não.
Por exemplo, não temos como construir conjunto com cardinalidade $3$,
pois uma tal construção deveria ``terminar'' com uma aplicação do Powerset
(o Emptyset construe conjuntos com cardinalidade $0$, e o Pairset com $1$ ou $2$).
Mas aplicando o Powerset para conjunto com cardinalidade finita $n$,
construimos conjunto com cardinalidade $2^n$, e $3$ não é uma potência de $2$.

\endexercise
%%}}}

%%{{{ Games 
\note Jogos.
Como descobrimos no~\ref{still_missing_some_finite_cardinalities}
não existe estrategia\ii{estratégia}[vencedora]\ vencedora num jogo
onde nosso oponente joga primeiro escolhendo um número $n\in\nats$,
e nosso objetivo é construir pelos axiomas um conjunto
com cardinalidade $n$.
Se ele escolher $n=0$ ou uma potência de $2$, temos como ganhar,
mas caso contrário, não.
\endgraf
No outro lado, num jogo onde  nosso objetivo é construir pelos axiomas um
conjunto com cardinalidade \emph{pelo menos} $n$, como vimos
no~\ref{arbitrarily_large_finite_sets} temos uma estratégia vencedora sim:
comece com uma aplicação do Emptyset~(\axref{emptyset})
para ganhar o $\emptyset$ e aplique iterativamente o
Powerset~(\axref{powerset}) construindo assim conjuntos de
cardinalidades $0$ (do próprio $\emptyset$),
$1$, $2$, $2^2$, $2^{2^2}$, etc., até chegar num conjunto com cardinalidade
maior-ou-igual ao $n$ escolhido por nosso oponente.
%%}}}

%%{{{ x: how many iterations of powerset do we need to win? 
\exercise.
Quantas iterações precisamos para conseguir conjunto com cardinalidade $n\in\nats$?

\endexercise
%%}}}

%%{{{ x: all_finite_cardinalities 
\exercise.
\label{all_finite_cardinalities}
Usando os~\axref{extensionality}+\axref{emptyset}+\axref{separation}+\axref{powerset}, podemos construir conjunto com cardinalidade finita qualquer?

\hint
Dado $n\in\nats$ construa primeiramente um conjunto com cardinalidade
maior-ou-igual, e aplica o Separation~(\axref{separation}) para
ficar com apenas $n$ elementos.
Formalmente, use indução!

\endexercise
%%}}}

%%{{{ ax: Unionset 
\axiom Unionset.
\tdefined{axioma}[Unionset]%
\label{unionset}%
Para cada conjunto, sua união (a colecção de todos os membros dos seus membros) é um conjunto.
$$
\forall a
\exists s
\forall x
\bigparen{
x\in s
\liff
\exists w
\paren{
x \in w
\land
w \in a
}
}
\axtaglabel{ZF6}{unionset}
$$
%%}}}

%%{{{ df: unionset 
\definition.
\tdefined{unionset}%
\sdefined {\Union {\holed a}} {a união de $\holed a$ (operação unária)}
Dado conjunto $a$, escrevemos
$\Union a$
para o conjunto garantido pelo Unionset~(\axref{unionset}),
que é único graças ao~Extensionality~(\axref{extensionality}).
Definimos assim o operador da \dterm{união arbitrária} $\Union\dhole$.
%%}}}

%%{{{ x: union_and_symdiff_constructed 
\exercise.
\label{union_and_symdiff_constructed}%
Defina os operadores binários $\union$ e $\symdiff$.

\hint
(Sobre o operador $\union$.)
Combine os operadores $\Union\dhole$ e $\set{\dhole,\dhole}$!

\hint
(Sobre o operador $\symdiff$.)
Suponha $a,b$ conjuntos.  Temos
$a\symdiff b\subseteq a\union b$.

\solution
Sejam conjuntos $a,b$.
Definimos:
$$
\align
a \union b   &\defeq \Union\set{a,b}\\
a \symdiff b &\defeq \setst {x \in a\union b} {x \notin a\inter b}
\endalign
$$

\endexercise
%%}}}

%%{{{ x: complement_impossible 
\exercise.
\label{complement_impossible}%
Como podemos definir o operador unitário $\complement{\cdot}$ de \emph{complemento},
tal que $\complement a$ é o conjunto de todos os objetos que não pertencem ao $a$?

\hint
De jeito nenhum!  Por quê?

\solution
Não podemos.
Dado conjunto $a$, se seu complemento $\complement a$ também fosse conjunto,
poderiamos aplicar o $\union$ para construir o $a\union\complement a$.
Mas pela definição dos $\union$ e $\complement a$, temos agora
$$
x\in a\union \complement a
\iff
x\in a \lor x \notin a
$$
ou seja, todos os objetos $x$ satisfazem a condição na direita!
Em outras palavras $a \union \complement a$ seria o próprio universo $\Univ$
que sabemos que não é um conjunto.

\endexercise
%%}}}

%%{{{ x: Inter_constructed 
\exercise.
\label{Inter_constructed}%
Defina o operador unitário $\Inter$,
aplicável em qualquer conjunto não vazio.
Precisamos o Unionset~(\axref{unionset})?

\solution
Dado conjunto $A\neq\emptyset$, definimos
$$
\Inter A
\defeq
\setst {x \in \Union A} {\lforall {a \in A} {x \in a}}.
$$

\endexercise
%%}}}

%%{{{ thm: finite_set_constructor 
\theorem.
\label{finite_set_constructor}%
Dados $a_1, a_2, \dotsc, a_n$ (onde $n\in\nats$) existe conjunto único
cujos membros são exatamente os $a_1$, $a_2$, \dots, $a_n$.
\sketch.
Provamos por indução no $n$.
Já temos provado o resultado para os valores $n=0,1,2.$
Usamos o operador $\union$ para o passo indutivo.
\qes
%%}}}

\endsection
%%}}}

%%{{{ Construction trees 
\section Arvores de construção.

%%{{{ Tree explanation 
\note.
Queremos provar que $\set{ \emptyset, \set{\emptyset, \set{\emptyset}}}$
é um conjunto.
Escrevemos:
{Pelo Emptyset~(\axref{emptyset}), $\emptyset$ é um conjunto.
Pelo Pairset~(\axref{pairset}) aplicado com $a,b\asseq\emptyset$ temos
que $\set{\emptyset,\emptyset}$ também é conjunto.
Mas, pelo Extensionality~(\axref{extensionality}),
$\set{\emptyset,\emptyset}=\set{\emptyset}$.
Agora, novamente pelo Pairset~(\axref{pairset}) essa vez
aplicado com $a\asseq\emptyset$ e $b\asseq\set{\emptyset}$
temos que $\set{\emptyset, \set{\emptyset}}$ é um conjunto.
Aplicando uma última vez o Pairset~(\axref{pairset}) com
$a\asseq\emptyset$ e $b\asseq \set{\emptyset, \set{\emptyset}}$,
conseguimos construir o conjunto
$\set{ \emptyset, \set{\emptyset, \set{\emptyset}}}$.
}
Podemos representar essa construção em forma de árvore:
$$
\def\fCenter{}
\centerAlignProof
\AxiomC{}
\RightLabel{\axref{emptyset}}
\UnaryInf$\fCenter\emptyset$
\AxiomC{}
\RightLabel{\axref{emptyset}}
\UnaryInf$\fCenter\emptyset$
\AxiomC{}
\RightLabel{\axref{emptyset}}
\UnaryInf$\fCenter\emptyset$
\AxiomC{}
\RightLabel{\axref{emptyset}}
\UnaryInf$\fCenter\emptyset$
\doubleLine
\RightLabel{\axref{pairset}}
\BinaryInf$\fCenter\set{\emptyset}$
\RightLabel{\axref{pairset}}
\BinaryInf$\fCenter\set{\emptyset, \set{\emptyset}}$
\RightLabel{\axref{pairset}}
\BinaryInf$\fCenter\set{\emptyset, \set{\emptyset, \set{\emptyset}}}$
\DisplayProof{}
$$
onde a dupla linha indica que pulamos alguns passos implícitios,
como o uso de Extensionality~(\axref{extensionality}) nesse caso.
Usamos agora essa construção para construir um conjunto de cardinalidade $3$:
$$
\centerAlignProof
\AxiomC{}
\RightLabel{\axref{emptyset}}
\UnaryInfC{$\emptyset$}
\AxiomC{}
\RightLabel{\axref{emptyset}}
\UnaryInfC{$\emptyset$}
\AxiomC{}
\RightLabel{\axref{emptyset}}
\UnaryInfC{$\emptyset$}
\AxiomC{}
\RightLabel{\axref{emptyset}}
\UnaryInfC{$\emptyset$}
\doubleLine
\RightLabel{\axref{pairset}}
\BinaryInfC{$\set{\emptyset}$}
\RightLabel{\axref{pairset}}
\BinaryInfC{$\set{\emptyset, \set{\emptyset}}$}
\RightLabel{\axref{pairset}}
\BinaryInfC{$\set{\emptyset, \set{\emptyset, \set{\emptyset}}}$}
\RightLabel{\axref{powerset}}
\UnaryInfC{$\set{\emptyset, \set{\emptyset}, \set{\set{\emptyset, \set{\emptyset}}}, \set{\emptyset, \set{\emptyset, \set{\emptyset}}}}$}
\RightLabel{\axref{separation},\ $\phi(x)\asseq \exists w(w\in x)$}
\UnaryInfC{$\set{\set{\emptyset}, \set{\set{\emptyset, \set{\emptyset}}}, \set{\emptyset, \set{\emptyset, \set{\emptyset}}}}$}
\DisplayProof{}
$$
%%}}}

%%{{{ eg: tree_construction_with_open_leaves 
\example.
\label{tree_construction_with_open_leaves}%
Seja $a,b$ conjuntos.
Mostre que $\set{b, \set{\emptyset, \set{a}}}$ é conjunto.

\solution
$$
\centerAlignProof
\AxiomC{$b$}
\AxiomC{}
\RightLabel{Empty}
\UnaryInfC{$\emptyset$}
\AxiomC{$a$}
\doubleLine
\RightLabel{Singleton}
\UnaryInfC{$\set{a}$}
\RightLabel{Pair}
\BinaryInfC{$\set{\emptyset, \set{a}}$}
\RightLabel{Pair}
\BinaryInfC{$\set{b, \set{\emptyset, \set{a}}}$}
\DisplayProof{}
$$
Onde deixamos as ``folhas'' da árvore $a,b$ ``sem fechar'', pois
correspondem realmente em nossas hipotéses: os conjuntos $a,b$ são dados!
Observe também que ``Singleton'' se-refere no~\ref{singleton_thm}.

\endexample
%%}}}

%%{{{ x: tree_construction_practice 
\exercise.
\label{tree_construction_practice}%
Sejam $a,b,c,d$ conjuntos.
Mostre pelos axiomas que os seguintes também são:
$$
\align
A &= \set{a,b,c,d}\\
B &= \set{a,b, \set{c,d}}\\
C &= \set{x \st x \subseteq a\union b\union c\union d \mland \text{$x$ tem exatamente 2 membros}}
\endalign
$$

\solution
Como $a,b$ são conjuntos, pelo Pairset o $\set{a,b}$ também é.
Similarmente o $\set{c,d}$ é conjunto, e aplicando mais uma vez o Pairset neles
temos que o $\set{\set{a,b},\set{c,d}}$ é conjunto.
Agora aplicando o Union nele o ganhamos o $A$.

\bigskip

Aqui uma construção do $B$ pelos axiomas, em forma de árvore:
$$
\centerAlignProof
\AxiomC{$a$}
\AxiomC{$b$}
\RightLabel{ZF3}
\BinaryInfC{$\set{a,b}$}
\AxiomC{$c$}
\AxiomC{$d$}
\RightLabel{ZF3}
\BinaryInfC{$\set{c,d}$}
\RightLabel{ZF3}
\doubleLine
\UnaryInfC{$\set{\set{c,d}}$}
\RightLabel{ZF3}
\BinaryInfC{$\set{\set{a,b},\set{\set{c,d}}}$}
\RightLabel{ZF6}
\UnaryInfC{$\set{a,b,\set{c,d}}$}
\DisplayProof{}
$$

\bigskip

\noindent Para o $C$, usamos o Separation~(\axref{separation})
no $\pset \paren{\Union A}$, que é conjunto graças aos Union (\axref{union})
\& Powerset (\axref{powerset}):
$$
\centerAlignProof
\AxiomC{$A$}
\RightLabel{ZF6}
\UnaryInfC{$\Union A$}
\RightLabel{ZF5}
\UnaryInfC{$\pset\Union A$}
\RightLabel{ZF4}
\UnaryInfC{$C$}
\DisplayProof{}
$$
onde na aplicação de ZF4 usamos a fórmula
$
\exists u \exists v
\paren{ u \neq v \land \forall w(w \in x \liff w = u \lor w = v ) }
$.

\endexercise
%%}}}

\endsection
%%}}}

%%{{{ Classes vs. Sets (I) 
\section Classes vs{.}~Conjuntos (I).

%%{{{ What is a class? 
\note O que é uma classe?.
Sem dúvida, a notação $\set{ x \st \text{\lthole} }$
que temos usado até agora é natural e útil.
Ela denota a colecção de todos os objetos $x$ que satisfazem
a condição (ou ``passam o filtro'') que escrevemos no \lthole.
Dada uma condição definitiva $P$ então consideramos a colecção
$$
\set {x \st P(x)}
$$
de todos os objetos que a satisfazem.
Provavelmente vocé já percebeu que eu evitei usar a palavra
\emph{conjunto}, pois reservamos essa palavra para
apenas os conjuntos-objetos do nosso universo.
%%}}}

%%{{{ df: class 
\definition Classe.
\tdefined{classe}%
\tdefined{classe}[própria]%
\iisee{própria}[classe]{classe própria}%
Dada uma condição definitiva $P(\dhole)$ definimos a \dterm{classe}
$$
\class x {P(x)}
$$
como um sinónimo da própria condição $P$!
Chamamos a classe $P$ \dterm{própria} sse não existe \emph{conjunto}
$S$ que satisfaz a propriedade:
$$
x \in S \iff P(x).
$$
%%}}}

%%{{{ Abuso notacional 
\note Abuso notacional.
É muito comum abusar o símbolo $\in$, escrevendo
$$
x \in C
$$
mesmo quando $C$ é uma classe própria!
Nesse caso consideramos o ``$x \in C$'' apenas como uma
abreviação do $C(x)$.
Em outras palavras, esse $\in$ não é um símbolo de relação da nossa FOL de
teoria de conjuntos, mas sim uma abreviação em nossa \emph{metalinguagem},
no mesmo jeito que $\iff$ também não é, mas o $\liff$ é.
Para enfatisar essa diferença vamos usar um símbolo diferente:
%%}}}

%%{{{ df: inclass 
\definition.
\label{inclass}%
\sdefined{\holed x\inclass\holed C}{o objeto \holed x está na classe \holed C}%
Usamos o símbolo $\inclass$ na metalinguagem como ``pertence''
quando possivelmente o lado direito é uma classe própria.%
\footnote{Seguimos aqui o exemplo dos símbolos de equivalência na metalinguágem
($\Longleftrightarrow$) e na linguagem-objeto de lógica ($\leftrightarrow$).}
Com essa notação:
$$
x\inclass P
\defiff
\knuthcases{
P(x)    & se $P$ é uma classe própria\cr
x\in P  & se $P$ é um conjunto.
}
$$
%%}}}

%%{{{ eg: sets and proper classes 
\example.
Dados conjuntos $a$ e $b$,
das classes
$$
\munderbrace{\class x {x = a}} {\set a}
\qqquad
\munderbrace{\class x {x \in a \land x\in b}} {a\inter b}
\qqquad
\munderbrace{\class x {x \neq x}} {\emptyset}
\qqquad
\setst x {x = x}
$$
apenas a última é própria.
As outras são conjuntos mesmo!
\endexample
%%}}}

\endsection
%%}}}

%%{{{ Foundations of mathematics 
\section Fundações de matemática.

%%{{{ A low-level language for mathematics 
\note Uma linguagem ``low-level'' para matemática.
Fazendo uma analogia entre matemática e programação,
digamos que a teoria de conjuntos pode servir como uma certa low-level linguagem
em qual podemos ``compilar'' (traduzir, representar, \dots)~todos os high-level
conceitos que nos interessam em matemática!
As ``definições'' (assim entre aspas) que temos usado até agora para os vários
tipos e conceitos que encontramos não foram formais.
Nossa tarefa então aqui é tirar essas aspas.
Dar uma definição formal dum conceito significa o definir dentro da
teoria de conjuntos.
No final das contas, tudo vai ser representado dentro da FOL da teoria de conjuntos,
pegando como noções primitívas \emph{apenas} os ``$\Set$'' de ``ser conjunto''
e o ``$\in$'' de ``pertencer''.
Já temos uma primeira biblioteca construida até agora,
um certo ``açúcar\ii{açúcar sintáctico}\ sintáctico''.
E cada vez que conseguimos compilar algo dentro da teoria de conjuntos,
ganhamos não apenas o proprio algo para seus usos e suas aplicações,
mas também algo mais para usar para nossas próximas compilações.
%%}}}

%%{{{ Our inventory so far 
\note Nosso inventório até agora.
Vamos resumir todos os operadores e predicados que temos já definido
dentro da teoria de conjuntos, usando apenas os axiomas que encontramos
até agora.
Temos:
$$
\gather
\emptyset \;;\\
\set \dhole \;;\quad
\set{\dhole,\dhole} \;;\quad
\set{\dhole,\dotsc,\dhole} \;;\quad
\set{x \in \dhole \st \phi(x)} \;;\\
\dhole \inter \dhole \;;\quad
\dhole \union \dhole \;;\quad
\dhole \setminus \dhole \;;\quad
\dhole\symdiff \dhole \;;\quad
\Union \dhole \;;\quad
\pset \dhole
\endgather
$$
{e com o proviso de $a$ conjunto com $a\neq\emptyset$ também o}
$$
\dsize\Inter a\,.
$$
%%}}}

%%{{{ From specification to implementation 
\note De especificação para implementação.
Quando queremos representar algum \emph{tipo} de coisa
dentro do nosso mundo de conjuntos,
precisamos esclarecer qual é a \emph{especificação} desse tipo.
Quais propriedades desejamos dos objetos desse tipo?
O que precisamos para construir um objeto desse tipo?
Quando identificamos dois objetos desse tipo e os consideramos iguais?
Como podemos usar os objetos desse tipo?
Qual é a ``interface'' deles?
Talvez ajuda pensar que nossa tarefa é similhante de um ``vendedor
de implementações matemáticas''.
Nossos ``clientes'' são os próprios matemáticos que desejam usar certos
tipos de objetos matemáticos, e nos seus pedidos eles estão esclarecendo
quais são as propriedades que eles precisam.
Nosso trabalho então será:
\emph{implementar} essa especificação, ou seja,
\emph{representar fielmente} esses tipos e conceitos como conjuntos.
Para conseguir isto:
(1) \emph{definimos} os conceitos e objetos como conjuntos;
(2) \emph{provamos} que nossa implementação realmente atende as especificações.
Muitas vezes vamos até oferecer uma \emph{garantia de unicidade}
para mostrar para nosso cliente-matemático que ele não precisa
procurar outras implementações alternativas da nossa concorrência,
pois \emph{essencialmente} nem tem.
%%}}}

\blah.
Nosso próximo trabalho será representar as tuplas,
e por isso vamos analizar em muito detalhe essa especificação.
Depois relaxamos um pouco deixando uns detalhes tediosos como
``óbvios''.

\endsection
%%}}}

%%{{{ Constructing the tuples 
\section Construindo as tuplas.

%%{{{ Specification 
\note Especificação.
\label{tup_specification}%
Vamos começar com o trabalho de implementar tuplas de tamanho 2,
ou seja \emph{pares ordenados}.
Precisamos então definir \emph{um} operador
$\tup{\dhole,\dhole}$ que atende as especificações.
Primeiramente:
$$
\tup{x, y} = \tup{x', y'} \implies x = x' \mland y=y'.
\taglabel{TUP1}{spec_tup1}
$$
Mas precisamos mais que isso.
Dados conjuntos $A$ e $B$ queremos que
$$
\text{a \emph{classe}}
\quad
A\times B = \class z {
(\exists x \in A)
(\exists y \in B)
z = \tup {x,y}
}
\quad
\text{é um \emph{conjunto}}.
\taglabel{TUP2}{spec_tup2}
$$
Lembrando a idéia de tupla como black\ii{black box}[de tupla]\ box,
o interface que desejamos consiste em duas operações,
as \emph{projeções}
$\first$ e $\second$ tais que
$$
\first \tup{x,y} = x
\qqqquad
\text e
\qqqquad
\second \tup{x,y} = y.
$$
Queremos definir também um predicado $\Pair(\dhole)$
para afirmar que um certo objeto representa um par ordenado.
Isso é facil:
$$
\Pair(z) \defiff
\exists x
\exists y
\paren{
z = \tup {x,y}
}
$$
Finalmente, precisamos e confirmamos:
$$
\Pair(z) \iff z = \tup{ \first(z), \second(z) }.
$$
Anotamos também que assim que definir \emph{funções} dentro
da teoria de conjuntos, vamos mostrar a existência das funções
$$
\xalignat2
\pi_0 &\eqtype A\times B \to A &  \pi_1&\eqtype A\times B \to B\\
\pi_0\tup{x,y} &= x            &  \pi_1\tup{x,y} &= y
\endxalignat
$$
para todos os conjuntos $A$ e $B$.
%%}}}

%%{{{ x: op1_converse_direction_by_logic 
\exercise.
\label{op1_converse_direction_by_logic}%
No \ref{spec_tup1} botamos ``$\Longrightarrow$'' ao inves de ``$\Longleftrightarrow$''.
Por quê?  O que acontece com a ``$\Longleftarrow$''?

\hint
Como a direção ``$\Longleftarrow$'' poderia ser inválida?

\solution
A direção ``$\Longleftarrow$'' é garantida pela nossa lógica:
a propriedade da igualdade $=$,
que nos permite substituir iguais por iguais em qualquer expressão.

\endexercise
%%}}}

%%{{{ x: first_attempt_pair 
\exercise.
\label{first_attempt_pair}%
Prove que a operação
$$
\tup{x,y} \defeq \set{x,y}
$$
satisfaz uma das \ref{spec_tup1}~\&~\ref{spec_tup2}
mas não a outra, então essa
\emph{não} é uma implementação de par ordenado

\endexercise
%%}}}

\blah.
Nossa primeira tentativa não deu certo.
Mesmo assim é realmente possível implementar pares ordenados
como conjuntos!  Como?
\spoiler.

%%{{{ df: kuratowski_pair 
\definition par de Kuratowski.
\label{kuratowski_pair}%
\Kuratowski[par]%
Sejam $x,y$ objetos.
Definimos
$$
\tup{x,y} \defeq \kurpair x y.
$$
%%}}}

%%{{{ prop: kurpair_satisfies_tup1 
\proposition.
\label{kurpair_satisfies_tup1}%
O operador $\tup{\dhole, \dhole}$ de Kuratowski satisfaz a~\ref{spec_tup1}.
\sketch.
Suponha $\tup{x,y} = \tup{x',y'}$.
Logo
$$
\kurpair x y = \kurpair {x'} {y'}.
$$
Precisamos deduzir que $x=x'$ e $y=y'$ mas ainda não é claro.
O que temos é apenas igualdade desses dois conjuntos, que não
garanta o que queremos imediatamente.
Não podemos concluir nem que
$$
\alignat3
\set {x} &= \set {x'}
&\qquad&\mland\qquad&
\set {x,y} &= \set {x',y'},\\
\intertext{%
pois pela definição de igualdade de conjuntos (\axref{extensionality})
sabemos apenas que cada membro do conjunto no lado esquerdo
é algum membro do conjunto no lado direito e vice-versa.
Então, talvez
}
\set {x} &= \set { x',y'}
&\qquad&\mland\qquad &
\set {x,y} &= \set {x'}.
\endalignat
$$
Precisamos então separar em casos:
$x = y$ ou não.
Em cada caso argumentamos usando o (\axref{extensionality}) e a
cardinalidade dos conjuntos para progressar até chegar nos desejados
$x=x'$ e $y=y'$.
\qes
%%}}}

%%{{{ prop: kurpair_satisfies_tup2 
\proposition.
\label{kurpair_satisfies_tup2}%
O operador $\tup{\dhole, \dhole}$ de Kuratowski satisfaz a~\ref{spec_tup2}.
\proof.
Sejam $A,B$ conjuntos.
Precisamos mosrar que a classe
$$
A\times B = \class z {
(\exists x \in A)
(\exists y \in B)
z = \tup {x,y}
}
$$
é um conjunto.
Como já temos escrita a classe nessa forma, basta achar um conjunto $W$
que contem todos os pares que queremos, e aplicar esse mesmo filtro
para ficar apenas com eles mesmo.
Como parece o aleatorio $\tup{a,b} \in A\times B$?
$$
a\in A
\mland
b\in B
\implies
\tup{a,b} = \kurpair a b \in \mathord{?}
$$
Vamos ver:
$$
\alignat2
&\text{Suponha $a\in A$ e $b\in B$.}\\
&\text{\ttherefore $a,b \in A\union B$}                                 \qqby{def.~$\union$}\\
&\text{\ttherefore $\set {a}, \set{a,b} \subseteq A \union B$}          \qqby{def.~$\subseteq$}\\
&\text{\ttherefore $\set {a}, \set{a,b} \in \pset\paren{A \union B}$}   \qqby{def.~$\pset$}\\
&\text{\ttherefore $\kurpair a b \subseteq \pset\paren{A \union B}$}    \qqby{def.~$\subseteq$}\\
&\text{\ttherefore $\kurpair a b \in \pset\pset\paren{A \union B}$}     \qqby{def.~$\pset$}\\
&\text{\ttherefore $\tup {a, b} \in \pset\pset\paren{A \union B}$}     \qqby{def.~$\tup{a,b}$}
\endalignat
$$
Tome então $W\asseq \pset\pset\paren{A \union B}$
e defina
$$
A \times B \defeq \setst { z \in \pset\pset\paren{A\union B} } {
(\exists x \in A)
(\exists y \in B)
z = \tup {x,y}
}
$$
que é conjunto graças ao Separation~(\axref{separation}).
\qed
%%}}}

%%{{{ Being agnostic 
\note Sendo agnósticos.
\tdefined{agnóstico}%
Acabamos de encontrar \emph{um} operador de par ordenado,
do Kuratowski\Kuratowski{}.
Ele não é o único possível, mas para continuar com nossa teoria,
precisamos apenas mostrar que existe um.
Vamos usar o símbolo $\tup{\dhole,\dhole}$ sem esclarecer se realmente é
a implementação de Kuratowski que usamos, ou alguma outra implementação.
Tomando cuidado, sobre esse operador nos permitimos usar
\emph{apenas as propriedades da sua especificação e nada mais}:
as \ref{spec_tup1}~\&~\ref{spec_tup2} do \refn{tup_specification}.
Falamos então que estamos sendo $\tup{\,}$-\dterm{agnósticos}.
Por exemplo, não podemos afirmar que $\set {x} \in \tup {x,y}$.
Sim, isso é valido com a implementação de Kuratowski, mas é uma
\emph{coincidência} e não uma \emph{conseqüência} da especificação
de par ordenado.
Talvez outra implementação não tem essa propriedade,
como tu vai descobrir agora provando que o $\tup{\dhole,\dhole}$
do~Wiener\Wiener{} também é uma implementação de par ordenado.
%%}}}

%%{{{ x: wiener_pair 
\exercise par de Wiener.
\label{wiener_pair}%
\Wiener[par]%
A operação $\tup{\dhole,\dhole}$ definida pela
$$
\tup{x,y} \defeq \bigset{ \set{ \emptyset, \set{x} }, \set{ \set{ y } } }
$$
é uma implementação de par ordenado, ou seja,
satisfaz as~\ref{spec_tup1}~\&~\ref{spec_tup2}.%
\footnote{Wiener definiu essa operação uns anos \emph{antes} da definição de Kuratowski.}

\endexercise
%%}}}

%%{{{ x: another_pair 
\exercise.
\label{another_pair}%
Considere $0$ e $1$ dois objetos distintos (algo que nossos axiomas garantam).
Por exemplo, tome $0\defeq\emptyset$ e $1\defeq\set\emptyset$.
Prove que a operação
$$
\tup{x,y} \defeq \bigset{ \set{0, x}, \set{1, y} }
$$
também é uma implementação de par ordenado.

\hint
Calcule os $\tup {0,0}$, $\tup {0,1}$, $\tup {1,0}$, $\tup {1,1}$.

\hint
Separe em casos: $x=y$ ou não.

\endexercise
%%}}}

%%{{{ x: bad_pair 
\exercise.
\label{bad_pair}%
Prove que não podemos usar a operação
$$
\tup{x,y} \defeq \bigset{ x, \set{x, y} }
$$
como uma implementação de par ordenado.

\endexercise
%%}}}

\endsection
%%}}}

%%{{{ Constructing the disjoint union 
\section Construindo a união disjunta.

\endsection
%%}}}

%%{{{ Constructing the relations 
\section Construindo as relações.

\TODO Fechos: top-down revisitado.

\endsection
%%}}}

%%{{{ Constructing the functions 
\section Construindo as funções.

%%{{{ x: eqc_formally_defined 
\exercise.
\label{eqc_formally_defined}%
Defina o predicado $\eqc$.  Sejam $a,b$ conjuntos.
$$
a \eqc b \defiff (a \bijto b) \neq \emptyset.
$$

\endexercise
%%}}}

%%{{{ x: parto_formally_defined 
\exercise.
\label{parto_formally_defined}%
Defina o termo ``função parcial'' na teoria de conjuntos
e mostre que, se $a,b$ são conjuntos, o
$$
(a \parto b) \defiff \class f { f : a \parto b }
$$
também é.

\endexercise
%%}}}

%%{{{ df: parfun_compatibility 
\definition Compatibilidade.
\label{parfun_compatibility}%
\tdefined{função}[parcial!conflito]%
\tdefined{função}[parcial!compatibilidade]%
Sejam $A,B$ conjuntos e $\scr F$ uma família de funções parciais de $A$ para $B$:
$\scr F \subseteq (A\parto B)$.
Chamamos a $\scr F$ \dterm{compatível} sse $\Union {\scr F} \in (A\parto B)$.
Digamos que $\scr F$ tem \dterm{conflito} no $a\in A$ sse
existem $y,y'\in B$ com $y\neq y'$ e $\tup{a,y}, tup{a,y'}\in \scr F$;
equivalentemente
$$
\card{ \set{ \tup{a,y} \st y\in B } } > 1.
$$
%%}}}

%%{{{ df: approximation 
\definition Aproximação.
\label{parfun_approx}%
\tdefined{função}[parcial!aproximação]%
Seja $F : A \to B$.
Chamamos qualquer $f\subseteq F$ uma aproximação (parcial) da $F$.
Ela é \dterm{própria} se $f\subsetneq F$.
%%}}}

%%{{{ x: fun_approxes_is_set 
\exercise.
\label{fun_approxes_is_set}%
Seja $F : A \to B$.
Prove que a classe
$$
\class f { \text{$f$ é uma aproximação da $F$} }
$$
é um conjunto.

\solution
Usando o Separation~(\axref{separation}) escrevemos
$$
\set {f \in (A\parto B)} {f \subseteq F}.
$$

\endexercise
%%}}}

%%{{{ x: fun_with_approxes 
\def\Approxes{{\scr F}}
\def\FinApproxes{\Approxes_{\textrm{f}}}
\exercise.
\label{fun_with_approxes}%
Seja $F : A \to B$.
(i) Prove que:
$$
F = \Union \Approxes
$$
onde $\scr F$ é o conjunto de todas as aproximações da $F$.
(ii) É verdade também que
$$
F = \Union \FinApproxes
$$
onde $\FinApproxes$ é o conjunto de todas as aproximações finitas da $F$?

\solution
{%
\def\Approxes{{\scr F}}
\def\FinApproxes{\Approxes_{\textrm{f}}}
(i)
Tome $\tup{x,y} \in F$.
Para concluir que $\tup{x,y} \in \Union \Approxes$
precisamos achar uma aproximação $f\in\Approxes$ tal que $\tup{x,y} \in f$.
Aqui uma: a aproximação $\set{\tup{x,y}} \subseteq F$.
Conversamente, tome $\tup{x,y}$ in $\Union\Approxes$.
Pela definição de $\Union$ então temos que $\tup{x,y}\in f$ para alguma aproximação
$f\in\Approxes$.  Pela definição de aproximação agora, $f\subseteq F$, ou seja:
$\tup{x,y}\in f \subseteq F$.
\endgraf
(ii) Sim.  A direção $\Union \FinApproxes \subseteq F$ é trivial graças ao (i),
e a direção oposta também provamos no (i) pois a aproximação $\set{\tup{x,y}}$
que escolhemos nessa direção é realmente finita.
}

\endexercise
%%}}}

\endsection
%%}}}

%%{{{ Classes vs. Sets (II) 
\section Classes vs{.}~Conjuntos (II).

\note Relações e funções ``grandes demais''.
\tdefined{predicado}%
\tdefined{class-relação}%

%%{{{ df: functionlike 
\definition.
\label{functionlike}%
Uma fórmula $\Phi(x,y)$ é \dterm{function-like}, sse:
$$
\forall x
\unique y
\Phi(x,y)
\qquad
\text{ou seja,}
\qquad
\forall x
\exists y
\bigparen{
\Phi(x,y)
\land
\forall y'
\paren{\Phi(x,y') \limplies y = y'}}.
$$
Nesse caso, também usamos os termo \dterm{função-classe}, \dterm{operador},
e a notação comum
$$
\Phi(x) = y
\qqquad\text{como sinónimo de}\qqquad
\Phi(x,y).
$$
Seguindo essa linha denotamos por $\Phi(x)$ o único objeto $y$ tal que
$\Phi(x,y)$.
Assim o $\Phi(x)$ denota um \emph{objeto}, mas o $\Phi(x,y)$ uma \emph{afirmação}.
%%}}}

\endsection
%%}}}

%%{{{ Cardinal arithmetic 
\section Aritmética cardinal.

\endsection
%%}}}

%%{{{ The axiom of infinity 
\section O axioma da infinidade.

%%{{{ No infinite sets but Infinite(-) predicate 
\note.
Com todos os nossos axiomas até agora, mesmo tendo conseguido
representar tanta matemática fielmente dentro da teoria de conjuntos,
ainda \emph{não é garantida} a existência de nenhum conjunto infinito.
Mesmo assim, a noção de ``ser infinito'' pode sim ser expressada
em nossa dicionário, num jeito genial graças ao Dedekind\Dedekind{},
que deu a primeira definição de infinito que não presupõe
a definição dos números naturais.  Como?
\spoiler.
%%}}}

%%{{{ df: Dedekind-infinite 
\definition Dedekind-infinito.
\tdefined{Dedekind-infinito}
\sdefined {\Infinite(\holed a)} {o conjunto $a$ é Dedekind-infinito}
Seja $A$ conjunto.  Chamamos o $A$ \Dedekind[infinito]\dterm{Dedekind-infinito} sse
ele pode ser ``injetado'' para um subconjunto proprio dele, ou seja,
sse existem $X\subsetneq A$ e $f : A \bijto X$.
Definimos então o predicado
$$
\Infinite(a) \defiff \exists x
\paren{
x\subsetneq a \land (a \eqc x)
}.
$$
%%}}}

%%{{{ Set successor 
\note Conjunto-sucessor.
\ii{conjunto-sucessor}
%%}}}

%%{{{ df: setsucc 
\definition Zermelo, von Neumann.
\tdefined {conjunto-sucessor}%
\sdefined {\setsucc {\holed x}} {o conjunto-sucessór de $x$}%
Definimos
\vonNeumann[conjunto-sucessor]%
\Zermelo[conjunto-sucessor]%
o \dterm{conjunto-sucessor} dum conjunto $x$ ser o
$$
\align
\setsucc x &\defeq \set x           \tag{Zermelo}\\
\setsucc x &\defeq x \union \set x. \tag{von Neumann}
\endalign
$$
Como não existe ambigüidade, omitimos parenteses escrevendo por exemplo
$\setsucc{\setsucc{\setsucc x}}$ ao inves de
$\setsucc{(\setsucc{(\setsucc x)})}$.
%%}}}

%%{{{ ax: Infinity 
\axiom Infinity.
\tdefined{axioma}[Infinity]%
\label{infinity}%
Existe um conjunto que tem o $\emptyset$ como membro e é fechado pela
operação $(x \mapsto \setsucc x)$.
$$
\exists s
\bigparen{
\emptyset \in s
\land
\forall x
\paren{
x \in s
\limplies
\setsucc x \in s
}
}
\axtaglabel{ZF7}{infinity}
$$
%%}}}

%%{{{ x: infinity_axiom_guarantees_natlike_object 
\exercise.
Verdade ou falso?
Com o axioma Infinity~(\axref{infinity}) é garantida a existéncia \emph{do} conjunto
$$
\set {
\emptyset,
\setsucc\emptyset,
\setsucc{\setsucc{\emptyset}},
\setsucc{\setsucc{\setsucc{\emptyset}}},
\dotsc
}.
$$
(Escrevemos ``do'' ao inves de ``dum'' pois o axioma
Extensionality~(\axref{extensionality}) garanta que se existe, existe único.)

\hint
\emph{Não é} uma conseqüência imediata do~Infinity~(\axref{infinity}).
Por que não?

\endexercise
%%}}}

%%{{{ df: wrong_definition_of_I 
\definition.
\label{wrong_definition_of_I}%
Seja $I$ o conjunto cuja existência é garantida pelo axioma Infinity~(\axref{infinity}).
Ou seja, o conjunto que satisfaz a condição:
$$
\emptyset \in I \land \forall x \paren{x \in I \limplies \setsucc x \in I}.
$$
\mistake
%%}}}

%%{{{ x: def_of_I_used_definite_articile 
\exercise.
\label{def_of_I_used_definite_articile}%
Qual o problema com a \ref{wrong_definition_of_I}?

\hint
O artigo.

\solution
Para definir $I$ como \emph{o} conjunto que satisfaz tal propriedade
precisamos: \emph{existência \& unicidade}.
Existéncia é o que~\axref{infinity} garanta;
mas não temos---e nem podemos provar---unicidade.
Então precisamos definir o $I$ como \emph{um} conjunto
que satisfaz aquela condição.

\endexercise
%%}}}

%%{{{ Effects and side-effects 
\note Efeitos e efeitos colaterais.
O axioma Infinity~(\axref{infinity}) é o segundo dos nossos axiomas
que garanta diretamente a existência dum certo objeto;
o primeiro foi o Emptyset~(\axref{emptyset}).%
\footnote{Pois todos os outros começam com quantificadores universais.}
Assim que aceitamos o Emptyset, nos definimos o símbolo $\emptyset$
para ser \emph{o} conjunto vazio.
Para poder fazer isso precisamos \emph{provar} a unicidade
do conjunto vazio~(\ref{uniqueness_of_emptyset}).
Mas a condição que aparece no~\axref{infinity} não é
suficientemente forte para ganhar unicidade pelo~\axref{extensionality}!
Possivelmente (e realmente, como nos vamos ver) nosso mundo tem muitos
conjuntos com essa propriedade!
%%}}}

%%{{{ x: def_of_I_used_definite_articile 
\exercise.
\label{infinitely_many_infinite_sets}%
Mostre que já é garantida uma infinidade de conjuntos infinitos.

\hint
Olha para os subconjuntos de $I$.

\solution
Tome
$$
\align
I_0 &\asseq I\\
I_1 &\asseq I \setminus \set{ \emptyset }\\
I_2 &\asseq I \setminus \set{ \emptyset, \setsucc\emptyset }\\
I_3 &\asseq I \setminus \set{ \emptyset, \setsucc\emptyset, \setsucc\emptyset }\\
    &\eqvdots
\endalign
$$

\endexercise
%%}}}

%%{{{ How does this infinite set look like? 
\note Como parece esse conjunto infinito?.
Bem; sabemos que $I$ é infinito e tal, mas quais são os elementos dele?
É tentador pensar que $I$ é o conjunto
$$
I_* \pseudodefeq \set { \emptyset, \setsucc\emptyset, \setsucc{\setsucc\emptyset}, \setsucc{\setsucc{\setsucc{\emptyset}}},\dotsc}.
$$
No final das contas, vendo o~\axref{infinity},
\emph{o que mais poderia estar no $I$?}
Nada.  Certo?
Não.  Bem o oposto!
\emph{Absolutamente tudo} pode pertencer nesse $I$, pois a única informação
que temos sobre ele não tira nenhum objeto como possível membro dele!
Realmente, os únicos elementos \emph{garantidos} no $I$ são aqueles que escrevemos
em cima como membros do $I_*$, mas o $I$ pode ter mais: pode ter ``lixo'', como esse:
$$
I_{\spadesuit,\heartsuit} \pseudodefeq \set {
\emptyset,
\spadesuit,
\heartsuit,
\setsucc\emptyset,
\setsucc{\setsucc\emptyset},
\setsucc{\setsucc{\setsucc{\emptyset}}},
\dotsc
}.
$$
Aqui os $\spadesuit$ e $\heartsuit$ denotam dois objetos do nosso universo,
talvez nem são conjuntos, talvez denotam os próprios símbolos
``$\spadesuit$'' e ``$\heartsuit$'', talvez somos nos, eu e tu, etc.
Para a gente, é o lixo.%
\footnote{Sem ofensa.}
\endgraf
Na verdade, esse último conjunto não pode ser o nosso $I$, pois ele não
satisfaz a condição do~(\axref{infinity}) pois
$$
\spadesuit \in I
\qquad\text{mas}\qquad
\setsucc\spadesuit \notin I.
$$
Podemos então entender melhor nosso $I$.
Ele é um superconjunto do $I_*$---isto é garantido pelo~(\axref{infinity}) mesmo.
O que mais ele tem?  Não sabemos dizer, mas sabemos que \emph{se tem} outros objetos,
ele obrigatoriamente tem uma infinidade de conjuntos para cada um deles:
$$
\set {
\emptyset,
\spadesuit,
\heartsuit,
\setsucc{\emptyset},
\setsucc{\spadesuit},
\setsucc{\heartsuit},
\setsucc{\setsucc\emptyset},
\setsucc{\setsucc{\spadesuit}},
\setsucc{\setsucc{\heartsuit}},
\setsucc{\setsucc{\setsucc{\emptyset}}},
\setsucc{\setsucc{\setsucc{\spadesuit}}},
\setsucc{\setsucc{\setsucc{\heartsuit}}},
\dotsc
}.
$$
Vamos revisar esse $I$ então.
{\def\noise{\mathord{\;\vdots\;}}
Sabemos que ele parece assim:
$$
I = \set {
\emptyset,
\noise,
\setsucc{\emptyset},
\noise,
\setsucc{\setsucc{\emptyset}},
\noise,
\setsucc{\setsucc{\setsucc{\emptyset}}},
\noise,
\setsucc{\setsucc{\setsucc{\setsucc{\emptyset}}}},
\noise,
\dotsc
}
$$
onde os $\noise$ representam o lixo,
e nosso próximo trabalho será achar um jeito para nos livrar desse lixo!
}
%%}}}

\endsection
%%}}}

%%{{{ Constructing the natural numbers 
\section Construindo os números naturais.

%%{{{ Specification 
\note Especificação.
Primeiramente precisamos esclarecer o que precisamos implementar.
Qual é o ``pedido'' do cliente que queremos atender?
Quais são as leis (suficientes e necessárias) que os números naturais devem
respeitar?
\spoiler.
%%}}}

%%{{{ Peano system 
\definition Sistema Peano.
\tdefined{sistema Peano}%
\tdefined{princípio}[da indução, Peano]%
\label{Peano_system}%
Um \dterm{sistema Peano}\Peano[sistema]~é um conjunto estruturado
$\cal N = \sset \Nats {\Zero,\Succ}$ que satisfaz as leis:
$$
\alignat2
&\text{Zero é um número natural:}&\qquad&\Zero \in \Nats                                 \tag{P1}\\
&\text{O sucessor é uma operação unária nos naturais:}&&\Succ \eqtype \Nats \to \Nats  \tag{P2}\\
&\text{Naturais diferentes tem sucessóres diferentes:}&&\Succ \eqtype \Nats \injto \Nats \tag{P3}\\
&\text{Zero não é o sucessor de nenhum natural:}&&\Zero \notin \img \Succ \Nats          \tag{P4}\\
&\text{Os naturais satisfazem o princípio da indução:}                                   \tag{P5}
\endalignat
$$
\dterm{Princípio da indução}\/:
\emph{para todo $X\subseteq \Nats$,}
$$
\bigparen{
\Zero\in X
\land
\forall n
\paren{
n \in X \limplies \Succ n \in X
}
} \limplies X = \Nats.
$$
Observe que graças às (P3) e (P4) temos
$$
\gather
\Succ n = \Succ m \implies n = m \\
\Succ n \neq \Zero
\endgather
$$
para todos os $n,m\in \Nats$.
%%}}}

%%{{{ Defining a Peano system 
\note Definindo um sistema Peano.
Então o que precisamos implementar é um conjunto estruturado
$\cal N = \sset \Nats {\Zero, \Succ}$.
Isso é fácil: nosso $\cal N$ vai ser uma tripla $\tup {\Nats, \Zero, \Succ}$,
onde seus membros $\Nats$, $\Zero$, e $\Succ$ são tais objetos que as leis
(P1)--(P5) são satisfeitas.
Sabemos que o $\Nats$ precisa ser infinito, então temos que o procurar
entre os conjuntos infinitos da nossa teoria.
Uma primeira idéia seria botar $\Nats \defeq I$, mas essa não parece
uma idéia boa---será difícil ``vender'' uma implementação com lixo!
O que realmente queremos botar como $\Nats$ é o $I_*$
(que não consiguimos ainda defini-lo).
Mas vamos supor que o $I_*$ realmente é um conjunto;
ele vai representar os números naturais,
mas quais serão nossos $\Zero$ e $\Succ$?
Pela especificação dos naturais, $\Zero$ tem que ser um dos membros do $I_*$,
e bem naturalmente escolhemos o $\emptyset$ como o zero.
E o $\Succ$?  Obviamente queremos botar $\Succ = \lam x {\setsucc x}$,
mas para realmente definir o $\Succ$ como função, lembramos que em nosso
dicionário ``função'' é um certo tipo de conjunto de pares.
Botamos então
$$
\align
\Succ &\pseudodefeq \class {\tup{n,m}} { m = \setsucc n }
\intertext{e agora só basta achar um conjunto que tem todos esses pares
como membros.  Fácil:}
\Succ &\defeq \setst {\tup{n,m}\in \Nats\times \Nats} { m = \setsucc n }.
\endalign
$$
Então falta só definir esse $I_*$.
%%}}}

%%{{{ Getting rid of noise 
\note Jogando fora o lixo.
Queremos definir o $I_*$ como conjunto; construí-lo pelos axiomas.
Uma primeira tentativa seria começar com o proprio $I$,
e usar o Separation~(\axref{separation}) para filtrar
seus elementos, separando os quais queremos do lixo,
assim botando
$$
I_* = \setst {x\in I} {\phi(x)}.
$$
para algum certo filtro $\phi(\dhole)$.
Qual fórmula vamos usar?
\spoiler.
%%}}}

%%{{{ A top-down approach 
\note Uma abordagem top-down.
Não podemos descrever um filtro em nossa linguagem de lógica
(FOL de teoria de conjuntos)!  (Lembra-se, uma fórmula não pode
ter tamanho infinito.)
Precisamos então alguma outra idéia para nos livrar dos elementos ``extra'' do $I$.
\emph{Vamos definir o conjunto $I_*$ com a abordagem top-down!}
Sabemos que nosso $I_*$ desejado é um subconjunto de $I$.
Então vamos começar com a colecção de todos os subconjuntos de $I$
que satisfazem a condição do Infinity:
$$
\scr I
\defeq
\setst {i \in \pset I} {\emptyset \in i \land \forall x\paren{x\in i \limplies \setsucc x \in i}
}.
$$
Queremos agora selecionar o ``menor'' elemento dessa família $\scr I$.
Menor no sentido de ``aquele que está contido em todos''.
Sim, nosso $I_*$ é o $\subseteq$-menor elemento do $\scr I$!
Para defini-lo basta tomar a intersecção da família $\scr I$,
que podemos pois
$\scr I \neq \emptyset$~(\ref{why_is_scrI_nonempty}):
$$
I_* \defeq \Inter {\scr I}.
$$
%%}}}

%%{{{ x: why_is_scrI_nonempty 
\exercise.
\label{why_is_scrI_nonempty}%
Por que $\scr I \neq \emptyset$?

\solution
Pois $I\in\scr I$.

\endexercise
%%}}}

%%{{{ thm: existence_of_nats 
\theorem Existência dos naturais.
\label{existence_of_nats}%
Existe pelo menos um sistema Peano $\cal N = \sset \Nats {\Zero, \Succ}$.
\sketch.
Definimos:
$$
\align
\cal N &\defeq \tup{\Nats, \Zero, \Succ},\\
\text{onde:}\qquad
\Nats &\defeq \Inter \setst {i \in \pset I} {\emptyset \in i \land \forall x \paren{x\in i \limplies \setsucc x \in i}}\\
\Zero &\defeq \emptyset\\
\Succ &\defeq \setst {\tup{m,n}\in \Nats\times\Nats} { n = \setsucc m }.
\endalign
$$
Temos já justificado que cada objeto que aparece nessa definição
é conjunto pelos axiomas.  Basta só verificar que as (P1)--(P5)
são satisfeitas.
\qes
\proof.
A única coisa que deixamos para completar a prova foi
verificar os (P1)--(P5), que é feito nos
exercícios \refn{zero_is_a_nat}--\refn{nat_has_induction}.
\qed
%%}}}

%%{{{ x: P1 zero_is_a_nat 
\exercise P1.
\label{zero_is_a_nat}%
Prove que $\Zero\in\Nats$.

\endexercise
%%}}}

%%{{{ x: P2 succ_is_a_function 
\exercise P2.
\label{succ_is_a_function}%
Prove que $\Succ$ é uma função.

\endexercise
%%}}}

%%{{{ x: P3 succ_is_injective 
\exercise P3.
\label{succ_is_injective}%
Prove que $\Succ : \Nats \injto \Nats$.

\endexercise
%%}}}

%%{{{ x: P4 zero_is_not_a_succ 
\exercise P4.
\label{zero_is_not_a_succ}%
Prove que $\Zero\notin \img \Succ \Nats$.

\endexercise
%%}}}

%%{{{ x: P5 nat_has_induction 
\exercise P5.
\label{nat_has_induction}%
Seja $X\subseteq \Nats$ tal que:
\beginil
\item{(1)} $\Zero \in X$;
\item{(2)} para todo $k\in \Nats$, se $k\in X$ então $\Succ k \in X$.
\endil
\noindent
Prove que $X=\Nats$.

\endexercise
%%}}}

%%{{{ thm: uniqueness_of_nats 
\theorem Unicidade dos naturais.
\ii{unicidade!dos naturais}%
\label{uniqueness_of_nats}%
Se  $\cal N_1 = \sset {\Nats_1} {\Zero_1, \Succ_1}$
e   $\cal N_2 = \sset {\Nats_2} {\Zero_2, \Succ_2}$
são sistemas Peano, então são isomorfos: $\cal N_1 \iso \cal N_2$.
\wrongproof.
Precisamos definir um homomorfismo
$\phi : \cal N_1 \iso \cal N_2$.
Definimos a função $\phi : \Nats_1 \to \Nats_2$ usando recursão:
$$
\align
\phi(\Zero_1) &= \Zero_2\\
\phi(\Succ_1n) &= \Succ_2\phi(n).
\endalign
$$
Pela sua definição, a $\phi$ é um homomorfismo.
Basta só verificar que a $\phi$ é bijetora,
algo que tu vai fazer agora nos exercícios
\refn{homomorphism_of_nats_is_mono}~\&~\refn{homomorphism_of_nats_is_epi}.
\mistaqed
%%}}}

%%{{{ x: homomorphism_of_nats_is_mono 
\exercise.
\label{homomorphism_of_nats_is_mono}
Prove que a $\phi : \Nats_1 \to \Nats_2$ definida
no~\ref{uniqueness_of_nats} é um monomorfismo.

\endexercise
%%}}}

%%{{{ x: homomorphism_of_nats_is_epi 
\exercise.
\label{homomorphism_of_nats_is_epi}
Prove que a $\phi : \Nats_1 \to \Nats_2$ definida
no~\ref{uniqueness_of_nats} é um epimorfismo.

\endexercise
%%}}}

%%{{{ x: we_do_not_have_recursion_yet 
\exercise.
\label{we_do_not_have_recursion_yet}%
Qual o erro na prova do~\ref{uniqueness_of_nats}?

\hint
O que é uma função em nosso dicionário,
e quem garanta 

\solution
Não temos provado que dando equações recursivas como na
prova desse teorema podemos realmente definir uma função.
Fazemos isso no~\ref{recursion_theorem}, assim
realmente botando o $\qedsymbol$ no~\ref{uniqueness_of_nats}.

\endexercise
%%}}}

\endsection
%%}}}

%%{{{ Constructing more numbers 
\section Construindo mais números.

\TODO Os números inteiros.

\TODO Os números racionais.

\TODO Os números reais.

\endsection
%%}}}

%%{{{ Recursion theorems 
\section Teoremas de recursão.

%%{{{ thm: recursion_theorem 
\theorem Recursão.
\ii{teorema}[de recursão]%
\label{recursion_theorem}%
Sejam $\cal N = \sset \Nats {\Zero, \Succ}$ um sistema Peano,
$A$ conjunto,
$a \in A$,
e $h: A \to A$.
Então existe única função $F : \Nats \to A$ que satisfaz as equações:
$$
\align
F(\Zero)    &= a \tag{1}\\
F(\Succ n)  &= h(F(n)), \quad\text{para todo $n\in\Nats$}.\tag{2}
\endalign
$$
\sketch.
Nosso plano é
\beginil
\item{(i)} construir o objeto $F$ como conjunto;
\item{(ii)} mostrar que $F : \Nats \to A$;
\item{(iii)} mostrar que $F$ satisfaz as (1)--(2);
\item{(iv)} unicidade.
\endil
\endgraf
(i)
Vamos construir o $F$ \emph{bottom-up}, juntando umas das suas aproximações finitas:
funções parciais $f : \Nats\parto E$ onde a idéia é que elas ``concordam'' com a $F$
desejada onde elas estão definidas.
Nem vamos considerar todas elas: para nossa conveniência queremos apenas aquelas
cujo domínio é algum $\finord n$.
Por exemplo, as primeiras aproximações seriam as seguintes:
$$
\align
f_0 &= \emptyset\\
f_1 &= \bigset{ \tup{0,a} }\\
f_2 &= \bigset{ \tup{0,a}, \tup{1,h(a)} }\\
f_3 &= \bigset{ \tup{0,a}, \tup{1,h(a)}, \tup{2,h(h(a))} },\\
    &\eqvdots
\endalign
$$
Definimos o conjunto $\scr A$ de todas as aproximações aceitáveis:
$$
\scr A \defeq \setst {f \in (\Nats\parto E)} {\text{$f$ é aproximação aceitável}}
$$
onde falta descrever com uma fórmula nossa idéia de ``ser aproximação aceitável''
(\ref{acceptable_approximation}).
(Das aproximações em cima a $f_0$ não é aceitável.)
Agora podemos já definir o $F$:
$$
F \defeq \Union {\scr A}.
$$
\endgraf
(ii)
Precisamos mostrar a compatibilidade da $\scr F$ e a totalidade da $F$,
ou seja: que não existem conflitos\ii{função!parcial!conflito}{}
(em outras palavras: que a família de funções parciais
$\scr A$ é compatível\ii{função!parcial!compatibilidade});
e que $\dom F = \Nats$.
Esses são os exercícios~\ref{compatibility_of_scrF}~\&~\ref{totality_of_F}
respectivamente.
\endgraf
(iii)
Precisamos verificar a corretude da $F$, que ela atende sua especificação.
Essa parte deve seguir da definição de ``aproximação aceitável''.
Confirmamos isso no~\ref{correctness_of_F}.
\endgraf
(iv)
Para a unicidade da $F$, precisamos mostrar que se $G : \Nats \to A$ tal que
satisfaz as (1)--(2), então $F = G$.  Isso é o~\ref{uniqueness_of_F}, e com
ele terminamos nossa prova.
\qes
\proof.
A prova completa segue do seu esboço junto com os exercícios:
\refn{acceptable_approximation},
\refn{compatibility_of_scrF},
\refn{totality_of_F},
\refn{correctness_of_F}, e
\refn{uniqueness_of_F}.
\qed
%%}}}

%%{{{ x: acceptable_approximation 
\exercise.
\label{acceptable_approximation}
No contexto do~\ref{recursion_theorem} defina formalmente a afirmação ``$f$ é uma aproximação aceitável''.

\endexercise
%%}}}

%%{{{ x: compatibility_of_scrF 
\exercise Compatibilidade.
\label{compatibility_of_scrF}%
No contexto do~\ref{recursion_theorem} mostre que $\scr F$ é compatível.

\endexercise
%%}}}

%%{{{ x: totality_of_F 
\exercise Totalidade da $F$.
\label{totality_of_F}%
No contexto do~\ref{recursion_theorem} mostre que $\dom F = \Nats$.

\endexercise
%%}}}

%%{{{ x: correctness_of_F 
\exercise Corretude da $F$.
\label{correctness_of_F}%
No contexto do~\ref{recursion_theorem} mostre que $F$ atende sua especificação,
ou seja:
$$
\align
F(\Zero)    &= a \tag{1}\\
F(\Succ n)  &= h(F(n)), \quad\text{para todo $n\in\Nats$}.\tag{2}
\endalign
$$

\endexercise
%%}}}

%%{{{ x: uniqueness_of_F 
\exercise Unicidade da $F$.
\label{uniqueness_of_F}%
No contexto do~\ref{recursion_theorem} prove a unicidade da $F$, ou seja:
se $G : \Nats \to A$ tal que
$$
\align
G(\Zero)    &= a \tag{1}\\
G(\Succ n)  &= h(G(n)), \quad\text{para todo $n\in\Nats$}.\tag{2}
\endalign
$$
então $F = G$.
Em outras palavras, as duas equações (1)--(2)
\emph{determinam} a função no $\Nats$.

\endexercise
%%}}}

\endsection
%%}}}

%%{{{ Consequences of recursion 
\section Conseqüências de recursão.

\note Operações.
Na~\refn{Nats_formally} definimos recursivamente operações nos naturais.
Graças o~\ref{recursion_theorem}, ganhamos todas essas operações em qualquer
sistema Peano.

\endsection
%%}}}

%%{{{ Problem intermission 
\problems Intervalo de problemas.

%%{{{ common context for problems 
Para qualquer sistema de naturais $\cal N = \sset \Nats {\Zero, \Succ}$,
definimos as operações de adição e de multiplicação
$$
\xxalignat5
&\text{(a1)}  & n + \Zero   &= n          &&& n \ntimes \Zero   &= \Zero                &&\text{(m1)}\\
&\text{(a2)}  & n + \Succ m &= \Succ(n+m) &&& n \ntimes \Succ m &= \Succ (n \ntimes m)  &&\text{(m2)}
\endxxalignat
$$
e a relação de ordem no $\Nats$
$$
    n \leq m \defiff (\exists k)[n + k = m].
$$
Sejam dois sistemas de naturais
$\cal N_1 = \sset {\Nats_1} {\Zero_1, \Succ_1}$ e 
$\cal N_2 = \sset {\Nats_2} {\Zero_2, \Succ_2}$, e
suas operações de adição $+_1$ e $+_2$, e suas relações de ordem $\leq_1$ e $\leq_2$.
Seja $\phi:\Nats_1\bijto\Nats_2$ o isomorfismo definido pelas
$$
\align
    \phi(\Zero_1)   &= \Zero_2           \\
    \phi(\Succ_1 n) &= \Succ_2(\phi(n)).
\endalign
$$
%%}}}

%%{{{ prob: any_peano_morphism_respects_the_order 
\problem.
Prove que $\phi$ respeita a adição e a multiplicação, ou seja:
$$
\align
\phi(n +_1 m)       &= \phi(n) +_2 \phi(m)\\
\phi(n \ntimes_1 m) &= \phi(n) \ntimes_2 \phi(m).
\endalign
$$
Mostre que $\phi$ \emph{respeita a ordem} também, ou seja:
$$
n\leq_1 m \iff \phi(n)\leq_2 \phi(m).
$$

\hint
Prove as duas direções da ``$\Longleftrightarrow$'' separadamente.

\endproblem
%%}}}

\endproblems
%%}}}

%%{{{ More axioms 
\section Mais axiomas.

%%{{{ Credits 
\note Créditos.
\label{fol_filter_by_fraenkel_and_skolem}
Todos os axiomas que temos visto até agora são essencialmente os axiomas
de Zermelo\Zermelo{}, e falta apenas um dos seus axiomas originais
(o axioma da escolha que encontramos na~\refn{Axioms_of_choice}).
Sobre o axioma Separation~(\axref{separation}), Zermelo usou o termo
``propriedade definitiva'', que temos usado também sobre o ``filtro'',
mas foram \Fraenkel{}Fraenkel~e~\Skolem{}Skolem que consideraram definir isso
como uma fórmula da linguagem da FOL com $=$ e $\in$.
%%}}}

%%{{{ A letter from Fraenkel to Zermelo 
\note Uma carta de Fraenkel para Zermelo.
\Fraenkel{}Fraenkel percebeu (e comunicou no 1921 para Zermelo) que
com os seus axiomas não é possível provar a existência duns certos
conjuntos interessantes, como por exemplo o
$$
\bigset{
\Nats, \pset\Nats, \pset\pset\Nats, \pset\pset\pset\Nats, \dotsc
}.
$$
Sim, podemos construir cada um dos seus elementos, mas não a
colecção deles como conjunto!
%%}}}

%%{{{ ax: replacement 
\axiom Replacement (schema).
\tdefined{axioma}[Replacement (schema)]%
\label{replacement}%
{\rm Para cada função-classe $\Phi(\dhole)$, o seguinte:}
Para todo conjunto $a$, a classe
$
\classimg \Phi a \defeq \class {\Phi(x)} {x \in a}
$
é um conjunto.
$$
\forall a
\exists b
\forall y
\bigparen{
    y \in b
    \liff
    \lexists {x \in a} {\Phi(x) = y}
}
\axtaglabel{ZF8}{replacement}
$$
%%}}}

%%{{{ x: powersingleton_without_powerset 
\exercise.
\label{powersingleton_without_powerset}%
Resolve o~\ref{powersingleton} sem usar o~Powerset~(\axref{powerset}).

\hint
Qual operador $\Phi(\dhole)$ tu podes definir para o aplicar no $a$?

\solution
Definimos o operador $\Phi(\dhole)$ assim:
$$
\Phi(x) \defeq \set x.
$$
Facilmente, pelo Replacement~(\axref{replacement}) aplicado no $a$ com
esse $\Phi$ temos que
$\classimg \Phi a$
é um conjunto: o conjunto que procuramos!

\endexercise
%%}}}

%%{{{ A mortal game 
\note Um jogo mortal.
Teu oponente escolha um conjunto (e ele não participa mais no jogo):
esse é o ``conjunto da mesa''.
Em cada rodada do jogo tu tem que escolher um dos membros do conjunto da mesa,
e ele se vira o novo conjunto da mesa.
O objetivo é simples: \emph{continuar jogando pra sempre}.
(Imagine que se o jogo acabar, tu morre---e que tu queres viver---ou algo desse tipo.)
Então uma partida onde o oponente escolheu o conjunto
$
\set{ \emptyset, \fsset{\emptyset}, \fsset{\emptyset, \fsset{\emptyset}}}
$
seria a seguinte
(sublinhamos as escolhas do jogador onde possível):
$$
\matrix
\format
\c\quad & \c\quad & \c \\
\text{Rodada} & \text{Conjunto} & \text{Movimento}\\
1 & \bigset{ \emptyset, \ \fsset{\emptyset}, \ \underline{\fsset{\emptyset, \fsset{\emptyset}}}, \ \fsset{ \emptyset, \fsset{\emptyset}, \fsset{\emptyset, \fsset{\emptyset}} } } & \fsset{\emptyset, \fsset{\emptyset}}\\
2 & \bigset{\emptyset, \underline{\fsset{\emptyset}}} & \fsset{\emptyset} \\
3 & \bigset{\underline{\emptyset}} & \emptyset\\
4 & \emptyset & \boohoo
\endmatrix
$$
Talvez esse jogador não foi o mais esperto, mas facilmente confirmamos
que qualquer possível estratégia dele é condenada com morte certa depois
dum finito número de rodadas.
\emph{E se o proprio jogador começa escolhendo qual é o conjunto da mesa inicial?}
Qual conjunto tu escolheria?
Pode \emph{imaginar} algum conjunto que seria uma boa opção que porderia garantir
vitória?
\spoiler.
%%}}}

%%{{{ x: how_to_win_the_wf_game 
\exercise.
\label{how_to_win_the_wf_game}%
Considere os conjuntos da mesa seguintes:
\beginil
\item{(1)} o conjunto $x$, onde
$$
x = \bigset{ \emptyset, \ \Nats, \ \set{ \emptyset, \set{\set{\emptyset}}}, \ x };
$$
\item{(2)} o conjunto $a$, onde
$$
\xalignat3
a &= \bigset{ \emptyset, \ b } &
b &= \bigset{ \set{\emptyset}, \ c } &
c &= \bigset{ \set{\emptyset}, \ \set{ \set{a}, \set{\set{\set{\emptyset}}}}};
\endxalignat
$$
\item{(3)} o conjunto $\Omega$, onde
$$
\Omega = \bigset{ \Omega }.
$$
\endil
\noindent
Como tu jogaria nesses jogos?

\endexercise
%%}}}

%%{{{ Can we win? 
\note Podemos ganhar?.
Talvez.  Nossos axiomas não garantam a existência de nenhum conjunto
que nos permitaria ganhar; no outro lado, nem garantam a ausência de
conjuntos como os $x,a,b,c$ do~\ref{how_to_win_the_wf_game}.
O axioma seguinte resolve essa questão, afirmando que qualquer partida
desse jogo seria realmente mortal.
%%}}}

%%{{{ ax: foundation 
\axiom Foundation.
\tdefined{axioma}[Foundation]%
\label{foundation}%
Todo conjunto não vazio tem membro disjunto com ele mesmo.
$$
\paren{\forall a\neq\emptyset}
\lexists {z \in a} {z\inter a = \emptyset}
\axtaglabel{ZF9}{foundation}
$$
%%}}}

\blah.
Vamos agora pesquisar umas conseqüências desse axioma, também conhecido como
\dterm{Regularity}\iisee{axioma}[Regularity]{Foundation}.

%%{{{ x: x_notin_x 
\exercise.
\label{x_notin_x}%
Prove diretamente que para todo conjunto $x$, $x\notin x$.
Quais axiomas tu precisou?

\hint
Seja $x$ conjunto.
Não podemos aplicar o~\refn{foundation} diretamente no $x$,
pois talvez $x = \emptyset$.
Uma abordagem seria separar casos.
Ao inves disso, aplique o~\refn{foundation} no conjunto $\set x$.

\endexercise
%%}}}

%%{{{ cor: univ_not_in_univ 
\corollary.
\label{unit_not_in_univ}
$\Univ\notin\Univ$, ou seja, $\Univ$ não é um conjunto.
%%}}}

%%{{{ x: x_y_cannot_belong_to_each_other 
\exercise.
\label{x_y_cannot_belong_to_each_other}%
Prove que é impossível para dois conjuntos $x,y$ ter
$x \in y$ e também $y \in x$.

\endexercise
%%}}}

%%{{{ x: no_infinite_in_descending_chain_of_sets 
\exercise.
\label{no_infinite_in_descending_chain_of_sets}%
Prove que não existe seqüência infinita de conjuntos
$$
x_0 \ni x_1 \ni x_2 \ni x_3 \ni \dotsb
$$
e mostre que assim ganhamos os
exercícios~\refn{x_notin_x}~\&~\refn{x_y_cannot_belong_to_each_other}
como corolários.

\endexercise
%%}}}

%%{{{ x: bad_pair_becomes_good_pair 
\exercise.
\label{bad_pair_becomes_good_pair}%
Prove que com o axioma Foundation podemos sim usar a operação
do~\ref{bad_pair}
$$
\tup{x,y} \defeq \bigset{ x, \set{x, y} }
$$
como uma implementação de par ordenado.

\endexercise
%%}}}

\endsection
%%}}}

%%{{{ Axioms of choice 
\section Axiomas de escolha.
\label{Axioms_of_choice}%

\blah.
Finalmente encontramos aqui o último axioma de Zermelo,
o mais infame, seu \emph{axioma de escolha}.
Mas, primeiramente, umas definições.

%%{{{ df: choice_set 
\definition Conjunto-escolha.
\label{choice_set}%
\tdefined{conjunto-escolha}%
Seja $\scr A$ uma familia de conjuntos.
Chamamos o $E$ um \dterm{conjunto-escolha} da $\scr A$ sse
(1) $E \subseteq \union \scr A$, e
(2) para todo $A \in \scr A$, a intersecção $E \inter A$ é um singleton.
%%}}}

%%{{{ eg: choice_sets 
\example.
\label{choice_sets}%
Aqui umas famílias de conjuntos e um exemplo de conjunto-escolha para cada uma:
$$
\xalignat2
\scr A_1 &= \set{ [0,2], [1,4], [3,5] }
&&{\set{ 0, \,5/2, \,5 }}\\
\scr A_2 &= \set{ \set{a,b}, \set{b,c}, \set{d} }
&&{\set{a,c,d}}\\
\scr A_3 &= \set{ \set{a,b}, \set{b,c}, \set{c,a} }
&&\text{não tem conjunto-escolha}\\
\scr A_4 &= \set{ \emptyset, \set{\emptyset}, \set{\set{\emptyset}}, \set{\set{\set{\emptyset}}}, \set{\set{\set{\set{\emptyset}}}}, \dots }
&&\text{não tem conjunto-escolha}\\
\scr A_5 &= \set{            \set{\emptyset}, \set{\set{\emptyset}}, \set{\set{\set{\emptyset}}}, \set{\set{\set{\set{\emptyset}}}}, \dots }
&&\scr A_4\\
\scr A_6 &= \set{ \finord 1, \finord 2, \finord 3, \dotsc }
&&\text{não tem conjunto-escolha}
\endxalignat
$$
\endexample
%%}}}

%%{{{ df: choice_function 
\definition Função-escolha.
\label{choice_function}%
\tdefined{função-escolha}[de conjunto]%
\tdefined{função-escolha}[de família]%
Seja $A$ conjunto.
Chamamos a $\varepsilon : \pset A \setminus \set{\emptyset} \to A$
uma \dterm{função-escolha do conjunto} $A$ sse $\varepsilon(X) \in X$ para todo $X\in\dom\varepsilon$.
\endgraf
Seja $\scr A$ família de conjuntos.
Chamamos a $\varepsilon : \scr A \to \union \scr A$ uma
\dterm{função-escolha da família de conjuntos} $\scr A$ sse
$\varepsilon(A) \in A$ para todo $A\in \scr A$.
%%}}}

%%{{{ ax: choice_ac 
\axiom Choice (AC).
\tdefined{axioma}[Choice (AC)]%
\label{choice_ac}%
Seja $\scr A$ família de conjuntos não vazios.
Então
$$
\gathered
\text{
existe
$\varepsilon : \scr A \to \Union \scr A$,
tal que
}\\
\text{
para todo $A\in\scr A$,
$\varepsilon(A) \in A$.
}
\endgathered
\axtaglabel{AC}{ac}
$$
%%}}}

%%{{{ ax: choice_ac_disjoint 
\axiom Choice (forma disjunta).
\label{choice_ac_disjoint}%
Seja $\scr D$ família disjunta de conjuntos não vazios.
Então
$$
\gathered
\text{
existe
$\varepsilon : \scr D \to \Union \scr D$,
tal que
}\\
\text{
para todo $D\in\scr D$,
$\varepsilon(D) \in D$.
}
\endgathered
\axtaglabel{ACdis}{acdisj}
$$
%%}}}

%%{{{ ax: choice_ac_pset 
\axiom Choice (forma powerset).
\label{choice_ac_powerset}%
Seja $M$ conjunto não vazio.
Então
$$
\gathered
\text{
existe
$\varepsilon : \pset M\setminus\set{\emptyset} \to M$,
tal que
}\\
\text{
para todo
$\emptyset\neq A\subseteq M$,
$\varepsilon(A) \in A$.
}
\endgathered
\axtaglabel{AC$\pset$}{acpset}
$$
%%}}}

%%{{{ x: first_ac_equivalences 
\exercise.
\label{first_ac_equivalences}
Todas as ``formas'' do axioma de escolha (AC) que vimos até agora
são logicamente equivalentes.
Prove isso seguindo o ``round-robin'' seguinte:
$$
\axref{ac}
\implies \axref{acdisj}
\implies \axref{acpset}
\implies \axref{ac}.
$$

\endexercise
%%}}}

\blah.
Depois vamos encontrar mais teoremas (assumindo o \axref{ac}),
que na verdade são equivalentes com ele.

\endsection
%%}}}

%%{{{ Desired and controversial consequences 
\section Conseqüências desejáveis e controversiais.

%%{{{ thm: banach_tarski 
\theorem Banach--Tarski.
\label{banach_tarski}%
\Banach{}\Tarski{}%
\ii{teorema}[Banach--Tarski]%
Podemos decompor a bola sólida unária
$$
B = \set{\tup{x,y,z} \in \reals^3 \st x^2 + y^2 + z^2 \leq 1}
$$
em $5$ subconjuntos
$S_1, S_2, S_3, S_4, S_5 \subseteq S$,
rodar e transladar eles, criando duas cópias sólidas de $B$.
%%}}}

%%{{{ thm: wellordering_thm 
\theorem Bem-ordenação (Zermelo).
\label{wellordering_thm}%
\ii{teorema}[da bem-ordenação]
Todo conjunto $A$ pode ser bem ordenado.
%%}}}

%%{{{ thm: cardinal_comparability_thm 
\theorem Comparabilidade de cardinais.
\label{cardinal_comparability_thm}%
Para todos conjuntos $A,B$, temos $A\eqc B$ ou $B \eqc A$.
%%}}}

\endsection
%%}}}

%%{{{ Weaker choices 
\section Escolhas mais fracas.

\endsection
%%}}}

%%{{{ Other set theories 
\section Outras teorias de conjuntos.

\endsection
%%}}}

%%{{{ Other foundations 
\section Outras fundações.

\endsection
%%}}}

%%{{{ Problems 
\problems.

%%{{{ someset_problem 
\problem Someset is the new Emptyset.
\label{someset_problem}%
Considere o axioma seguinte:
\endgraf
\noindent
{\bf Someset.}
{\proclaimstyle Existe algo.}
$$
\exists s \paren{ s = s }
\axtaglabel{ZF2*}{someset}
$$
Mostre que no sistema axiomático
\axref{extensionality}%
+\axref{someset}%
+\axref{pairset}%
+\axref{separation}%
+\axref{powerset}%
+\axref{unionset},
existe o $\emptyset$.%
\footnote{Dependendo do uso e do contexto, podemos considerar como parte da lógica
que o universo não é vazio, ou seja, existe algo.  Nesse caso nem precisamos
o Someset~(\axref{someset}), pois seria implícito.}

\endproblem
%%}}}

%%{{{ triset_problem 
\problem Triset is the new Pairset.
\label{triset_problem}%
Considere o axioma seguinte:
\endgraf\noindent
{\bf Triset.}
{\proclaimstyle
Dados tres objetos distintos existe conjunto com exatamente
esses membros.
}
$$
\forall a
\forall b
\forall c
\lparen{
\bigparen{
a\neq b \land b \neq c \land c \neq a
}
\limplies
\exists s
\forall x
\bigparen{x\in s \liff x = a \lor x = b \lor x = c}
}
\axtaglabel{ZF3*}{triset}
$$
%\beginil
\item{(0)}
No sistema
\axref{extensionality}%
+\axref{emptyset}%
+\axref{pairset}%
+\axref{separation}%
+\axref{powerset}%
+\axref{unionset}
construa conjunto de cardinalidade~$3$.
\item{(1)}
Prove que no mesmo sistema
podemos substituir o axioma Pairset~(\axref{pairset}) pelo axioma
Triset~(\axref{triset}) ``sem perder nada''.
Em outras palavras, prove que no sistema
\axref{extensionality}%
+\axref{emptyset}%
+\axref{triset}%
+\axref{separation}%
+\axref{powerset}%
+\axref{unionset}
\emph{para todos os $a,b$ existe o conjunto $\set{a,b}$}.
\item{(2)}
Podemos provar a mesma coisa no sistema
\axref{extensionality}%
+\axref{emptyset}%
+\axref{triset}%
+\axref{separation}%
+\axref{unionset}%
?
%\endil

\hint
Para o (1), veja o (2); para o (2), veja o (1)!

\endproblem
%%}}}

%%{{{ consax_problem 
\problem.
\label{consax_problem}%
Considere o axioma seguinte:
$$
\forall h
\forall t
\exists s
\forall x
\bigparen{
x \in s
\liff
x = h
\lor
x \in t
}.
\axtaglabel{CONS}{consax}
$$

\endproblem
%%}}}

%%{{{ one_class_set_the_other_proper_problem 
\problem.
\label{one_class_set_the_other_proper_problem}%
Sejam $a,b$ conjuntos.
Mostre pelos axiomas que:
\item{(i)}  a classe $\set{ \set{x,\set y} \st x \in a \land y \in b}$ é conjunto;
\item{(ii)} a classe $\set{ \set{x,y} \st \text{$x,y$ conjuntos com $x\neq y$}}$ é propria.

\hint
Absurdo.

\endproblem
%%}}}

%%{{{ replacement_replaces_separation 
\problem Replacement is the new Separation---or is it?.
\label{replacement_replaces_separation}%
Podemos tirar o Separation scheme~(\axref{separation})
dos nossos axiomas ``sem perder nada'', se temos o
Replacement scheme~(\axref{replacement}) no lugar dele?

\hint
Tente achar uma class-function $\Phi$ tal que aplicada
nos elementos dum conjunto $A$, vai ``identificar'' todos
aqueles que \emph{não} tem a propriedade $\phi(\dhole)$,
mas mesmo assim sendo injetora quando restrita naqueles
que satisfazem a.

\hint
Alem de tudo isso, os ``originais'' $a\in A$ tem que ser
recuperáveis pelas suas imagens atraves da $\Phi$.

\solution
Seja $A$ conjunto e $\phi(x)$ fórmula.
Definimos a class-function
$$
\Phi(x) =
\knuthcases{
    \set {x},   &se $\phi(x)$\cr
    \emptyset,  &se não.
}
$$
Agora aplicamos o Replacement com essa class-function no conjunto $A$,
ganhando assim como conjunto o $\classimg\Phi A$, cujos elementos são exatamente os
\emph{singletons} $\set{a}$ de todos os $a\in A$ que satisfazem a $\phi(a)$,
e o $\emptyset$.
Usando o ZF6 chegamos no $\Union \classimg\Phi A$ que realmente é o desejado
conjunto
$\set { a \in A \st \phi(a) }$.

\endproblem
%%}}}

%%{{{ replacement_replaces_pairset 
\problem Replacement is the new Pairset---or is it?.
\label{replacement_replaces_pairset}%
Podemos tirar o Pairset~(\axref{pairset}) dos
nossos axiomas ``sem perder nada'', se temos o
Replacement scheme~(\axref{replacement}) no lugar dele?

\hint
Podemos sim.
Dados \emph{objetos} $a,b$, mostre que existe o conjunto
$\set{a,b}$ que consiste em exatamente esses objetos.

\hint
Tente achar uma class-function $\Phi$ tal que aplicada
nos elementos dum conjunto suficientemente grande,
vai ter como imagem o desejado $\set{a,b}$.

\solution
Sejam $a,b$ objetos.
Considere a class-function
$$
\Phi (x) \asseq 
\knuthcases{
    a, &se $x=\emptyset$\cr
    b, &se $x\neq\emptyset$.
}
$$
Agora precisamos apenas construir um conjunto $S$ tal que:
$\emptyset \in S$, e $|S| \geq 2$.
Pelo Emptyset, temos o $\emptyset$.
Pelo Powerset aplicado no $\emptyset$ ganhamos o $\set{\emptyset}$, e aplicando mais uma vez o Powerset chegamos no $\set{\emptyset, \set{\emptyset}}$.
Usando o Replacement com a $\Phi(x)$ nesse conjunto, construimos o desejado $\set{a,b}$.
\endgraf
Em forma de arvore:
$$
\centerAlignProof
\AxiomC{}
\RightLabel{Empty}
\UnaryInfC{$\emptyset$}
\RightLabel{Power}
\UnaryInfC{$\set{\emptyset}$}
\RightLabel{Power}
\UnaryInfC{$\set{\emptyset, \set{\emptyset}}$}
\RightLabel{Repl; $\Phi$}
\UnaryInfC{$\set{a, b}$}
\DisplayProof{}
$$

\endproblem
%%}}}

%%{{{ two_sets_and_one_proper_class_problem 
\problem.
\label{two_sets_and_one_proper_class_problem}%
Sejam $a,b$ conjuntos.
Mostre pelos axiomas que as classes
$$
\align
    C &= \set{
          \set{x, \set{x,y}}
          \st
          x \in a
          \land
          y \in b
    }\\
    D &= \set{
          \set{x,y}
          \st
          \paren{
              x \in a
              \lor
              x \in \Union a
          }
          \land
          \paren{y \in b \lor y \subseteq b}
    }\\
\intertext{são conjuntos, mas não a classe}
    Z &= \set{
        {\Inter\Inter z}
            \st z \neq \emptyset \land \Inter z \neq \emptyset
    }.
\endalign
$$

\endproblem
%%}}}

%%{{{ three_classes_are_sets_problem 
\problem.
\label{three_classes_are_sets_problem}%
Seja $a$ conjunto.
\emph{Sem usar} o Separation~(\axref{separation}),
mostre pelo resto dos axiomas que as classes seguintes são conjuntos:
$$
\xalignat3
    E &= \setlst {\set{x, \Union x,\pset x}} {x\in a} ;&
    F &= \setlst x {x\subsetneq a}                    ;&
    G &= \setst  x {x\neq\emptyset \land x = \Inter x}.
\endxalignat
$$

\endproblem
%%}}}

%%{{{ choice_rel_problem 
\problem.
\label{choice_rel_problem}%
Mostre que o Choice~\axref{ac} é equivalente com o seguinte axioma:
\endgraf
\noindent
{\bf Choice (Rel).}
{\proclaimstyle
Se uma relação $R \in \relspace A B$ tem a propriedade de totalidade,
então existe função $f : A\to B$ com $x \mathrel{R} f(x)$ para todo $x\in A$.
}
$$
\bigparen{
R \in \relspace A B
\land
\dom R = A
}
\limplies
\bigparen{
\paren{\exists f : A \to B }
\lforall {x\in A} {x \mathrel R f(x)}
}
\axtaglabel{ACrel}{acrel}
$$

\endproblem
%%}}}

%%{{{ prob: multiset_formally_defined 
\problem Multisets.
\label{multiset_formally_defined}%
Alguem te deu a seguinte especificação de multiset
e tu queres implementá-la dentro da teoria de conjuntos.
(Veja também \ref{Multisets} no \ref{Sets}.)
\endgraf
\noindent
{\bf ``Definição''.}
Um \dterm{multiset} (ou \dterm{bag}) $M$ é como um conjunto
onde um elemento pode pertencer no $M$ mais que
uma vez (mas não uma infinidade de vezes).
Ou seja, a ordem não importa (como nos conjuntos),
mas a ``multiplicidade'' importa sim.
\endgraf
Queremos tres operações em multisets, exemplificadas assim:
$$
\align
    \bag{ x, y, y, z, z, z, w } \bagunion
    \bag{ x, y, z, z, u, v, v } &=
    \bag{ x, y, y, z, z, z, u, v, v, w }\\
    \bag{ x, y, y, z, z, z, w } \baginter
    \bag{ x, y, z, z, u, v, v } &=
    \bag{ x, y, z, z }\\
    \bag{ x, y, y, z, z, z, w } \bagplus
    \bag{ x, y, z, z, u, v, v } &=
    \bag{ x, x, y, y, y, z, z, z, z, z, u, v, v, w }
\endalign
$$
Também queremos um predicado de ``pertencer'' $\inbag$
e uma relação de ``submultiset'' $\subbag$ tais que:
$$
\xalignat2
x&{}\inbag \bag{ x, y, y, z, z, z, w }           &\bag{ x, y, z, z }&{}\subbag    \bag{ x, x, y, y, z, z }            \\
z&{}\inbag \bag{ x, y, y, z, z, z, w }           &\bag{ x, y, z, z }&{}\notsubbag \bag{ x, x, y, y, z }               \\
u&{}\notinbag \bag{ x, y, y, z, z, z, w }        &\bag{ x, y, z, z }&{}\subbag    \bag{ x, y, z, z }                  \\
x&{}\notinbag \emptybag \quad\text{para todo $x$}&M                 &{}\subbag    M \quad\text{para todo multiset $M$}\\
 &                                               &\emptybag         &{}\subbag    M \quad\text{para todo multiset $M$}. 
\endxalignat
$$
(MS1) Para os multisets $A$ e $B$ temos $A = B$ sse eles tem os mesmos membro
com as mesmas multiplicidades.
Por exemplo,
$$
\bag{x, y, z, z, y} = \bag{x, y, y, z, z} \neq \bag{x,y,z}.
$$
(MS2) Para cada conjunto $A$, a classe
$$
\setst M {\text{$M$ é multiset e $\forall x(x \inbag M \limplies x \in A)$}}
$$
de todos os multisets formados por membros de $A$ é um conjunto.
\item{(i)}
Defina formalmente (em teoria de conjuntos) o termo ``multiset'' e mostre
(como exemplos) como são representados os multisets seguintes:
$$
\emptybag
\qqqquad
\bag{0, 1, 2, 2, 1}
\qqqquad
\bag{1, 2, 2, 3, 3, 3, 4, 4, 4, 4, \dotsc }.
$$
\item{(ii)}
Defina as operações de multisets ($\bagunion, \baginter, \bagplus$)
e os predicados ($\inbag$, $\subbag$).
\item{(iii)}
Prove pelos axiomas ZF que tua definição satisfaz as (MS1)--(MS2).

\hint
Para representar a multiplicidade, use uma função com codomínio o $\nats_{>0}$.

\solution
(i)
Um \dterm{multiset} é uma tupla $\cal M = \tup{ M ; f }$
onde $M$ é um conjunto e $f : M \to \nats_{>0}$.
$$
\align
\emptybag                       &= \tup{ \emptyset ; \emptyset }\\
\bag{0, 1, 2, 2, 1}             &= \tup{ \set{0,1,2} ; f }\\
\bag{1, 2, 2, 3, 3, 3, \dotsc}  &= \tup{ \nats_{>0} ; \idof {\nats_{>0}} }\\
\endalign
$$
onde $f : \set{0,1,2} \to \nats_{>0}$ é a função definida pela
$$
f(n) = \knuthcases{
    1,  &se $n = 0$\cr
    2,  &se $n = 1$\cr
    2,  &se $n = 2$.
}
$$
\endgraf
(ii)
$$
\align
\tup{A;\alpha} \bagunion \tup{B;\beta}&\defeq\tup{A\union B \;;\; \lambda x. \max\set{\alpha(x),\beta(x)}}\\
\tup{A;\alpha} \baginter \tup{B;\beta}&\defeq\tup{A\inter B \;;\; \lambda x. \max\set{\alpha(x),\beta(x)}}\\
\tup{A;\alpha} \bagplus  \tup{B;\beta}&\defeq\tup{A\union B \;;\; \lambda x. (\alpha(x) + \beta(x))}\\
x \inbag \tup{A;\alpha} &\defiff x \in A\\
\tup{A;\alpha} \subbag \tup{B;\beta} &\defiff A\subseteq B \mland (\forall x \in A)[ \alpha(x) \leq \beta(x)]
\endalign
$$
\endgraf
(iii)
A (MS1) é obviamente satisfeita graças à nossa definição de múltiset como tupla
de conjunto e função: ganhamos assim a (MS1) pelas definições de $=$ nos três
tipos involvidos: conjunto; tupla; função.
Vamos verificar a (MS2).
Seja $A$ conjunto.
O arbitrário multiset $\cal M$ com membros de $A$ tem a forma
$\cal M = \tup{X , f}$ para algum $X\subseteq A$ e $f : X \to\nats_{>0}$.
Então $\cal M \in \pset A \times (A \parto \nats_{>0})$ e construimos o conjunto
de todos os multisets com membros de $A$ usando o ZF4:
$$
\namedop{Multisets}(A) \defeq
    \set{
        \cal M\in \pset A \times (A \parto \nats_{>0})
        \st
        \text{$\cal M$ é um multiset}.
    }
$$
Para mostrar que o $\pset A \times (A \parto \nats\setminus\set{0})$ é um conjunto,
precisamos os operadores $\pset$, $\times$, $\parto$, $\setminus$, e o próprio $\nats$, que já temos construido pelos ZF1--ZF7.

\endproblem
%%}}}

%%{{{ df: terminating_game 
\definition Jogo terminante.
\label{terminating_game}%
\tdefined{jogo}[terminante]%
Consideramos jogos entre 2 jogadores.
Chamamos um jogo \dterm{terminante} sse não tem partidas infinitas.
Ou seja, seguindo suas regras cada partida termina depois um finíto número de turnos.
%%}}}

%%{{{ df: hypergame 
\definition Hypergame (Zwicker).
\label{hypergame}%
\tdefined{hypergame}%
\Zwicker[hypergame]{}%
Considere o jogo seguinte $\cal H$, chamado \dterm{hypergame}:
O $\cal H$ começa com o {\scshape Player~I} que escolha um jogo terminante $G$.
O {\scshape Player~II} começa jogar o jogo $G$ contra o {\scshape Player~I}.
Quem ganha nesse jogo $G$ é o vencedor do jogo $\cal H$.
%%}}}

%%{{{ eg: hypergame_plays 
\example.
\label{hypergame_plays}%
Por exemplo, sendo um bom jogador de ``jogo da velha'' e um pessimo jogador
de xadrez, se eu for o {\scshape Player~I} num hypergame, meu primeiro
movimento seria escolher o jogo terminante ``jogo da velha'' para jogar.
Meu oponente, se for o {\scshape Player~I} duma partida de hypergame,
seu primeiro movimento seria escolher o jogo terminante ``xadrez''.
Depois desse movimento eu viro o {\scshape Player~I} no xadrez.
Quem vai ganhar nesse xadrez, vai ser o vencidor dessa partida de hypergame.
\endexample
%%}}}

%%{{{ hypergame_paradox 
\note O paradoxo de hypergame.
\label{hypergame_paradox}%
{\def\P{\text{\tt P}}
\def\O{\text{\tt O}}
Zwicker\Zwicker[hypergame]{} percebeu o seguinte paradoxo, se perguntando
se o próprio Hypergame é um jogo terminante ou não.
Claramente tem que ser, pois a primeira regra do jogo obriga o {\scshape Player~I}
escolher um jogo terminante.  Logo, depois de $n\in\nats$ turnos, esse jogo
termina, e junto com ele termina a partida do hypergame (em $n+1$ turnos).
Então hypergame é terminante.
Logo, numa partida de hypergame, o {\scshape Player~I} pode escolher o próprio
hypergame.  Assim começamos uma sub-partida de hypergame, onde o
{\scshape Player~II} toma o papel de {\scshape Player~I}.
Se ele escolher, por exemplo, ``jogo de velha'', a partida parece assim:
$$
\def\drawTTTboard{%
\draw (-1, 3)     -- (-1,-3);
\draw ( 1, 3)     -- ( 1,-3);
\draw (-3, 1)     -- ( 3, 1);
\draw (-3,-1)     -- ( 3,-1);
}
\align
\P:\quad& \hbox{Escholho ``Hypergame''}\\
\O:\quad& \hbox{Escholho ``Jogo de velha''}\\
\P:\quad&
\gathered
\tikzpicture[scale=0.15]
\drawTTTboard
\draw (-0.7,-0.7) -- (0.7, 0.7);
\draw (-0.7, 0.7) -- (0.7,-0.7);
\endtikzpicture
\endgathered\\
\phantom{\P}\vdots\quad&\vdots\\
\O:\quad&
\gathered
\tikzpicture[scale=0.15]
\drawTTTboard
\draw (-0.7,-0.7) -- ( 0.7, 0.7);
\draw (-0.7, 0.7) -- ( 0.7,-0.7);
\draw (-2.7,-0.7) -- (-1.3, 0.7);
\draw (-2.7, 0.7) -- (-1.3,-0.7);
\draw (2,2)  circle (0.7);
\draw (0,-2) circle (0.7);
\endtikzpicture
\endgathered\\
\P:\quad&
\gathered
\tikzpicture[scale=0.15]
\drawTTTboard
% X-moves
\draw (-0.7,-0.7) -- ( 0.7, 0.7);
\draw (-0.7, 0.7) -- ( 0.7,-0.7);
\draw (-2.7,-0.7) -- (-1.3, 0.7);
\draw (-2.7, 0.7) -- (-1.3,-0.7);
\draw ( 1.3,-0.7) -- ( 2.7, 0.7);
\draw ( 1.3, 0.7) -- ( 2.7,-0.7);
% O-moves
\draw (2,2)  circle (0.7);
\draw (0,-2) circle (0.7);
% draw winning line
\draw (-2.7, 0.0) -- ( 2.7, 0.0);
\endtikzpicture
\endgathered\\
\endalign
$$
Onde denotamos os dois jogadores com $\P$ e $\O$ (de ``Player'' e ``Opponent'').
Mas, agora a partida seguinte é possível:
$$
\align
\P:\quad& \hbox{Escholho ``Hypergame''}\\
\O:\quad& \hbox{Escholho ``Hypergame''}\\
\P:\quad& \hbox{Escholho ``Hypergame''}\\
\O:\quad& \hbox{Escholho ``Hypergame''}\\
\phantom{\P}\vdots\quad&\vdots
\endalign
$$
e achamos uma partida infinita do hypergame!
Logo o hypergame não é terminante.
}
%%}}}

%%{{{ prob: from_hypergame_to_cantors_theorem 
\problem.
\label{from_hypergame_to_cantors_theorem}%
Seja conjunto $A$ e suponha que existe injecção $\phi : A \injto \pset A$.
Para todo $x\in A$, denota com $S_x$ o $\phi(x)$, ou seja, $S_x$
é o subconjunto de $A$ associado com o $x$.
Seja $a\in A$.  Chame um \dterm{caminho} de $a$ qualquer seqüência
finita ou infinita $\set{a_i}_i$ de elementos de $A$ que satisfaz:
$$
\align
a_0     &= a\\
a_{n+1} &\in S_{a_n}.
\endalign
$$
Finalmente, chame um $a\in A$ \dterm{terminante} se todos os caminhos
de $a$ são finítos.  Use o paradoxo do Hypergame para provar que
a $\phi$ não pode ser sobrejetora, achando assim uma nova prova
do teorema de Cantor~\refn{cantor_theorem}.

\hint
Seja $T \subseteq A$ o conjunto de todos os terminantes elementos de $A$.
Queremos provar que $T\notin \img \phi A$, ou seja, que para todo $x\in A$,
$S_x \neq T$, provando assim que $\phi$ não é bijetora.

\hint
Suponha para chegar num absurdo que $T=S_a$ para algum $a\in A$.

\hint
$T = \emptyset$?

\endproblem
%%}}}

\endproblems
%%}}}

%%{{{ Further reading 
\further.

\cite{fromfregetogodel},
\cite{halmosnaive},
\cite{ynmnst},
\cite{kunenfoundations}.
\cite{kunen2011},
\cite{cohensetch},
\cite{kunen1980},
\cite{jechset}.

\endfurther
%%}}}

}
\endchapter
%%}}}

%%{{{ chapter: Computability and complexity 
\chapter Computabilidade e complexidade.
\label{Computability_and_complexity}%

%%{{{ A surprisingly difficult question 
\section Uma questão surpresamente difícil.

\endsection
%%}}}

%%{{{ Models of computation 
\section Modelos de computação.

\endsection
%%}}}

%%{{{ Decision problems 
\section Problemas de decisão.

\endsection
%%}}}

%%{{{ The halting problem 
\section O problema da parada.

\endsection
%%}}}

%%{{{ Asymptotic notation 
\section A notação assintótica.

\endsection
%%}}}

%%{{{ Analysis of algorithms 
\section Análise de algorítmos.

\endsection
%%}}}

%%{{{ Computational complexity 
\section Complexidade computacional.

\endsection
%%}}}

%%{{{ Problems 
\problems.

\endproblems
%%}}}

%%{{{ Further reading 
\further.

\cite{cutlandcomputability},
\cite{kleeneIM},
\cite{rogersrecursive}.

\cite{kozenac}, \cite{kozentc}.

\cite{dpv},
\cite{jonescomputability}.

\endfurther
%%}}}

\endchapter
%%}}}

%%{{{ chapter: Theory of recursive functions 
\chapter Teoria de funções recursivas.
\label{Recursive_functions}%

%%{{{ Problems 
\problems.

\endproblems
%%}}}

%%{{{ Further reading 
\further.

\cite{kleeneIM},
\cite{shoenfieldrecursion},
\cite{ynmrandc},
\cite{rogersrecursive}.

\endfurther
%%}}}

\endchapter
%%}}}

%%{{{ chapter: Formal languages 
\chapter Linguagens formais.

%%{{{ Problems 
\problems.

\endproblems
%%}}}

%%{{{ Further reading 
\further.

\endfurther
%%}}}

\endchapter
%%}}}

%%{{{ chapter: Ordered sets 
\chapter Conjuntos ordenados.
\label{Posets}%

%%{{{ Concept, notation, properties 
\section Conceito, notação, propriedades.

\note Conceito.

%%{{{ df: poset 
\definition poset.
\label{poset}%
\tdefined{poset}%
Chamamos o conjunto estruturado $\cal P = \sset P \leq$
um \dterm{poset} (ou conjunto parcialmente ordenado),
sse $\leq$ é uma relação de ordem parcial:
$$
\gather
x \leq x\\
x \leq y \mland y \leq z \implies x \leq z\\
x \leq y \mland y \leq x \implies x = y
\endgather
$$
%%}}}

\blah.
Nas definições seguintes nosso contexto é um poset $\sset P \leq$.

%%{{{ df: incomparable 
\definition.
\label{incomparable}%
\tdefined{incomparável}%
Se $x \nleq y$ e $y \nleq x$ chamamos $x$ e $y$ \dterm{incomparáveis}.
Denotamos assim:
$$
x \incomp y \defiff x \nleq y \mland y \nleq x.
$$
%%}}}

%%{{{ df: chain_antichain 
\definition.
\label{chain_antichain}%
\tdefined{cadeia}%
\tdefined{anticadeia}%
Seja $X\subseteq P$.
Chamamos o $X$ uma \dterm{cadeia} sse todos os seus elementos são comparáveis.
No caso oposto, onde todos são comparáveis apenas com eles mesmo,
chamamos o $X$ \dterm{anticadeia}.  Simbolicamente:
$$
\align
\text{cadeia:}\quad&       x, y \in X  \implies  x \leq y \mlor y\leq x;\\
\text{anticadeia:}\quad&   x, y \in X  \mland x \leq y \implies x = y.
\endalign
$$
%%}}}

%%{{{ df: covby 
\definition.
\label{covby}%
\tdefined{cobertura}%
\sdefined {\holed x \covby \holed y} {o $x$ é coberto por o $y$}%
Se $x\leq y$ e não existe nenhum $z$ estritamente entre os $x$ e $y$
falamos que $y$ \dterm{cobre} o $x$.  Simbolicamente:
$$
x\covby y \defiff x \leq y \mland \lnot\exists z (x < z < y).
$$
%%}}}

%%{{{ Diagramas Hasse 
\note Diagramas Hasse.
\Hasse{}
%%}}}

%%{{{ df: min_max 
\definition.
\label{min_max}%
\tdefined{mínimo}%
\tdefined{máximo}%
\sdefined{\bot} {o bottom (o mínimo elemento dum poset)}%
\sdefined{\top} {o top (o máximo elemento dum poset)}%
\emph{Dualmente} chamamos o $x$ \dterm{máximo} de $P$ sse $y\leq x$ para todo $y\in P$.
Denotamos o mínimo de $P$, se existe, por $\min P$ ou $\bot$ (\dterm{bottom}),
e seu máximo, se existe, por $\max P$ ou $\top$ (\dterm{top}).\mistake
%%}}}

%%{{{ x: uniqueness_of_min_max 
\exercise.
\label{uniqueness_of_min_max}%
Qual o erro na definição em cima?
Ache-o e corrija-o.

\endexercise
%%}}}

%%{{{ df: minimal_maximal 
\definition.
\label{minimal_maximal}
\tdefined{minimal}%
\tdefined{maximal}%
Chamamos o $x$ um elemento \dterm{minimal} de $P$
sse nenhum elemento está embaixo dele, e, \emph{dualmente}
chamamos o $x$ \dterm{maximal} de $P$ sse nenhum elemento
está em cima dele.  Simbolicamente:
$$
\align
\text{$x$ minimal}&\defiff  \lforall {y \in P} {y \leq x \implies x = y};\\
\text{$x$ maximal}&\defiff  \lforall {y \in P} {x \leq y \implies x = y}.
\endalign
$$
%%}}}

%%{{{ df: downset_upset 
\definition.
\label{downset_upset}%
\tdefined{downset}%
\tdefined{upset}%
Chamamos o $A\subseteq P$ um \dterm{down-set} de $A$
sse o $A$ é ``fechado para baixo'', e \emph{dualmente},
o chamamos \dterm{up-set} sse ele é ``fechado para cima''.
Formalmente,
$$
\align
\text{$A$ down-set} &\defiff {a \in A \mland x \leq a \implies x\in A};\\
\text{$A$ up-set}   &\defiff {a \in A \mland a \leq x \implies x\in A}.
\endalign
$$
Dado um poset $P$, usamos $\downs P$ para denotar o conjunto de todos os seus
down-sets.  Simbolicamente,
$$
\downs P \defeq \set{ D \subseteq P \st \text{$D$ é um down-set de $P$} }.
$$
%%}}}

%%{{{ Notational abuse 
\note Abusos notacionais.
Extendemos o ``tipo'' do predicado $\dhole\leq\dhole$ de elementos de $P$
para elementos e/ou subconjuntos de $P$, definindo:
$$
\align
a \leq Y &\defiff a \leq y, \quad\text{para todo $y\in Y$};\\
X \leq b &\defiff x \leq b, \quad\text{para todo $x\in X$};\\
X \leq Y &\defiff x \leq y, \quad\text{para todo $x\in X$ e $y\in Y$}.
\endalign
$$
%%}}}

\endsection
%%}}}

%%{{{ Operations and constructions 
\section Operações e construções.

%TODO (Construções de conjutos ordenados)
%TODO (Dualidade)
%TODO (Ordens canônicas, ordens lexicográficas)

\endsection
%%}}}

%%{{{ Mappings 
\section Mapeamentos.

%%{{{ x: downsets_of_poset_iso_downsets_of_dual 
\exercise.
\label{downsets_of_poset_iso_downsets_of_dual}%
Defina um $\varphi : \downsets P \iso \downsets { \dual P }$.

\solution
$\varphi(X) = P\setminus X$

\endexercise
%%}}}

%%{{{ x: downsets_of_disjunion_iso_product_of_downsets 
\exercise.
\label{downsets_of_disjunion_iso_product_of_downsets}%
Defina um
$\psi : \downsets { P_1 \disjunion P_2 } \iso \downsets {P_1} \times \downsets {P_2}$.

\solution
$\psi(D) = \tup{ D \inter P_1, D \inter P_2 }$

\endexercise
%%}}}

\endsection
%%}}}

%%{{{ Problems 
\problems.

%%{{{ prob: posets_of_divisors 
\problem.
\label{posets_of_divisors}%
Para $n\in\nats$, definimos o poset
${\cal D}_n \defeq \sset {D_n} {\divides}$
onde $D_n \defeq \set{ d \in \nats \st d \divides n }$.
\item{(i)}
Desenha o diagrama Hasse de ${\cal D}_{30}$.
\item{(ii)}
Ache conjunto $A$ tal que ${\cal D}_{30} \iso \sset {\pset A} {\subseteq}$
e defina um isomorfismo $\phi : D_{30} \to \pset A$.
\item{(iii)}
Existe conjunto $B$ tal que ${\cal D}_0 \iso \sset {\pset B} {\subseteq}$?
Se sim, ache o $B$ e defina um isomorfismo
$\phi : D_0 \to {\pset B}$;
se não, prove que é impossível.

\solution%%{{{
\noindent (i)
Primeiramente calculamos: $D_{30} = \set{1, 2, 3, 5, 6, 10, 15, 30}$.
$$
\tikzpicture
\node (max) at (0,4)  {$30$};
\node (a)   at (-2,2) {$6$};
\node (b)   at (0,2)  {$10$};
\node (c)   at (2,2)  {$15$};
\node (d)   at (-2,0) {$2$};
\node (e)   at (0,0)  {$3$};
\node (f)   at (2,0)  {$5$};
\node (min) at (0,-2) {$1$};
\draw (min) -- (d) -- (a) -- (max) -- (b) -- (f)
(e) -- (min) -- (f) -- (c) -- (max)
(d) -- (b);
\draw[preaction={draw=white, -,line width=6pt}] (a) -- (e) -- (c);
\endtikzpicture
$$
\endgraf
\noindent (ii)
Tome o $A = \set{2,3,5}$ e defina a função $\phi : \pset A \to D_{30}$
pelas equações:
$$
\align
\phi(30) &= A\\
\phi(15) &= \set{3,5}\\
\phi(10) &= \set{2,5}\\
\phi(6)  &= \set{2,3}\\
\phi(2)  &= \set{2}\\
\phi(3)  &= \set{3}\\
\phi(5)  &= \set{5}\\
\phi(1)  &= \emptyset
\endalign
$$
Seu diagrama Hasse parece assim:
$$
\tikzpicture
\node (max) at (0,4)  {$\set{2,3,5}$};
\node (a)   at (-2,2) {$\set{2,3}$};
\node (b)   at (0,2)  {$\set{2,5}$};
\node (c)   at (2,2)  {$\set{3,5}$};
\node (d)   at (-2,0) {$\set 2$};
\node (e)   at (0,0)  {$\set 3$};
\node (f)   at (2,0)  {$\set 5$};
\node (min) at (0,-2) {$\emptyset$};
\draw (min) -- (d) -- (a) -- (max) -- (b) -- (f)
(e) -- (min) -- (f) -- (c) -- (max)
(d) -- (b);
\draw[preaction={draw=white, -,line width=6pt}] (a) -- (e) -- (c);
\endtikzpicture
$$
\noindent
Obs: qualquer conjunto $A$ com $|A|=3$ serve!
Uma avantagem desse é que podemos bem elegantemente definir a bijeção
inversa, mandando cada subconjunto de $\set{2,3,5}$ para seu produtório!
\endgraf
\noindent (iii)
Não existe, pois $D_0 = \nats$ (contável)
e logo não pode ser equinúmero com o powerset de nenhum conjunto $B$.
\endgraf
\noindent (iv)
Verdade, a função $\phi : D_0 \to \set{D_n \st n\in \nats}$ definida pela
$$
\phi(n) = D_n
$$
é um isomorfismo, pois:
$$
n \divides m \iff D_n \subseteq D_m.
$$
%%}}}

\endproblem
%%}}}

\endproblems
%%}}}

%%{{{ Further reading 
\further.

\cite{DaveyPriestley},
\cite{goldreisets}.

\endfurther
%%}}}

\endchapter
%%}}}

%%{{{ chapter: Lattices 
\chapter Reticulados.

%%{{{ Lattices as posets 
\section Reticulados como posets.

%%{{{ df: lattice_as_poset 
\definition.
\label{lattice_as_poset}%
\tdefined{lattice}[como poset]%
\iisee{lattice}{reticulado}
Um poset $\cal L = \sset L {\leq}$
é um \dterm{lattice} (ou \dterm{reticulado}) sse 
para todo $x,y \in L$, os $\sup\set{x,y}$ e $\inf\set{x,y}$
existem.
%%}}}

\endsection
%%}}}

%%{{{ Lattices as algebraic structures 
\section Reticulados como estruturas algébricas.

%%{{{ df: lattice_as_algebra 
\definition.
\label{lattice_as_algebra}%
\tdefined{lattice}[como algebra]%
Seja $\cal L = \sset L {\join, \meet}$ conjunto estruturado
onde $\join,\meet$ são operadores binários no $L$.
Chamamos o $L$ um \dterm{lattice} (ou \dterm{reticulado}) sse
as operações $\join,\meet$ são associativas, comutativas,
e idempotentes, e satisfazem as leis seguintes \emph{de absorção}:
$$
\align
a \join (b \meet a) &= a\\
a \meet (b \join a) &= a
\endalign
$$
%%}}}

%%{{{ x: abs_and_com_imply_idem 
\exercise.
\label{abs_and_com_imply_idem}%
Num lattice $L$, prove que podemos inferir as leis da idempotência pelas outras.

\hint
$a\join(a\meet(a\join a))$.

\solution
Seja $a \in L$.
Calculamos
$$
\alignat2
a \join ( (a\join a) \meet a )
&= a \join a                        \qqby{$\meet$-abs.}\\
\intertext{e também}
a \join ( (a\join a) \meet a)
&=  ( (a\join a) \meet a) \join a   \qqby{$\join$-com.}\\
&=  ( a \meet (a\join a) ) \join a  \qqby{$\meet$-com.}\\
&=  a                               \qqby{$\join$-abs.}
\endalignat
$$
Logo $a\join a = a$.

\endexercise
%%}}}

%%{{{ x: two_equiv_ways_to_define_an_order_on_an_algebraic_lattice 
\exercise.
\label{two_equiv_ways_to_define_an_order_on_an_algebraic_lattice}%
Seja $\cal L = \sset L {\join,\meet}$ um láttice pela~\ref{lattice_as_algebra}.
Prove que:
$$
a\join b = b \iff a\meet b = a.
$$

\solution
\lrdir:
Suponha $b = a \join b$.
Calculamos
$$
\alignat2
a \meet b
&= a \meet (a \join b) \qqby{hipótese}\\
&= (a \join b) \meet a \qqby{$\meet$-com.}\\
&= a.                  \qqby{$\meet$-abs.}
\endalignat
$$
\endgraf\noindent
\rldir:
Similar.

\endexercise
%%}}}

%%{{{ df: semilattice 
\definition.
\label{semilattice}%
\tdefined{semilattice}%
\iisee{semireticulado}{semilattice}%
Seja $\cal L = \sset L {\diamond}$ conjunto estruturado
onde $\diamond$ é um operador binário no $L$.
Chamamos o $L$ um \dterm{semilattice} (ou \dterm{semireticulado})
sse a operação $\diamond$ é associativa, comutativa, e idempotente.
%%}}}

%%{{{ x: from_semilattice_to_lattice 
\exercise.
\label{from_semilattice_to_lattice}%
\tdefined{lattice}%
Defina o conceito ``lattice'' usando o conceito ``semilattice''.

\solution
O conjunto estruturado $\cal L = \sset L {\join,\meet}$ é um
\dterm{lattice} sse os conjuntos estruturados
$\sset L {\join}$ e $\sset L {\meet}$ são semilattices,
e as leis de absorção são satisfeitas.

\endexercise
%%}}}

\endsection
%%}}}

%%{{{ Complete lattices 
\section Reticulados completos.

\endsection
%%}}}

%%{{{ Irreducible elements 
\section Elementos irredutíveis.

\endsection
%%}}}

%%{{{ The Knaster--Tarski fixpoint theorem 
\section O teorema fixpoint de Knaster--Tarski.

\endsection
%%}}}

%%{{{ Boolean algebras 
\section Álgebras booleanas.

\endsection
%%}}}

%%{{{ Problems 
\problems.

%%{{{ df: cofinito_em_nats 
\definition.
\label{cofinito_em_nats}%
\tdefined{cofinito}%
Chamamos um $A\subseteq \nats$ \dterm{cofinito} sse seu complemento $\nats\setminus A$ é finito.
%%}}}

%%{{{ prob: family_of_cofinite_lattice 
\problem.
\label{family_of_cofinite_lattice}%
\tdefined{reticulado}[de conjuntos]%
Mostre que as famílias
$$
\align
\scr L_1 &\asseq \set{ A \subseteq \nats \st \text{$A$ é cofinito} }\\
\scr L_2 &\asseq \set{ A \subseteq \nats \st \text{$A$ é finito ou cofinito} }
\endalign
$$
são \dterm{reticulados de conjuntos}, ou seja,
reticulados com relação de ordem $\subseteq$.

\solution
\proofstyle{Sobre o $\scr L_1$.}
Sejam $A, B \in \scr L_1$.
Pela definição então $\nats\setminus A$ e $\nats\setminus B$ são finitos.
Vou mostrar que $A\union B$ e $A\inter B$ são cofinitos, e logo pertencem no $\scr L_1$ também.
Calculamos 
$$
\align
\nats\setminus(A\union B) &=\paren{\nats\setminus A} \inter \paren{\nats\setminus B}\\
\nats\setminus(A\inter B) &=\paren{\nats\setminus A} \union \paren{\nats\setminus B},
\endalign
$$
logo os dois conjuntos no lado esquerdo são finitos como intersecção e união de
finitos respectivamente.
\endgraf
\proofstyle{Sobre o $\scr L_2$.}
Sejam $A, B \in \scr L_2$.
Separamos em casos:
\endgraf
\casestyle{Caso ambos finitos:}
Facilmente $A\union B$ e $A \inter B$ são finitos também,
como intersecção e união de conjuntos finitos.
Logo ambos pertencem no $\scr L_2$.
\endgraf
\casestyle{Caso ambos cofinitos:}
Provamos isso na demonstração sobre o $\scr L_1$.
\endgraf
\casestyle{Caso contrário:}
Temos um dos $A,B$ finito e o outro cofinito.
Sem perda de generalidade, suponha que $A$ finito, $B$ cofinito.
O $A\inter B$ é trivialmente finito como intersecção de finito com qualquer conjunto.
Logo $A\inter B \in \scr L_2$.
O $A\union B$ é cofinito, pois
$$
\nats\setminus(A \union B) = \paren{\nats\setminus A} \inter \paren{\nats\setminus B}
$$
que é finito para o mesmo motivo (intersecção com o $\nats\setminus B$ que é finito).
Logo $A\union B \in \scr L_2$.

\endproblem
%%}}}

%%{{{ prob: family_of_cofinite_not_complete_lattice 
\problem.
\label{family_of_cofinite_not_complete_lattice}%
Mostre que nenhum dos $\scr L_1,\scr L_2$ do~\ref{family_of_cofinite_lattice}
é completo.

\hint
Seja $A_n \asseq \nats\setminus\set{0,2,\dotsc, 2n - 2}$ o $\nats$ sem os
primeiros $n$ números pares.
Mostre que:
\emph{se $B \subseteq A_n$ para todo $n\in \nats$, então $B$ não é cofinito}.

\solution
Vamos provar primeiramente a afirmação seguinte:
\emph{se $B \subseteq A_n$ para todo $n\in \nats$, então $B$ não é cofinito}.
\endgraf
\proofstyle{Prova da afirmação:}
Suponha que para todo $n\in\nats$, $B \subseteq A_n$.
Logo
$$
\align
    B &\subseteq \Inter_{i=0}^\infty A_i\\
\intertext{e complementando os dois lado,}
\nats\setminus B
    &\supseteq \nats\setminus\Inter_{i=0}^\infty A_i\\
    &= \Union_{i=0}^\infty(\nats\setminus A_i)\\
    &= \Union_{i=0}^\infty(\set{0,2,\dotsc, 2n-2})\\
    &= 2\nats.
\endalign
$$
Como $2\nats$ é infinito, o $B$ não é cofinito.
\endgraf
Agora voltamos a provar que nenhum dos $\scr L_1, \scr L_2$ é completo.
Considere o conjunto
$$
\cal S \asseq \set { A_0, A_1, A_2, \dotsc }
$$
Observe que
$\cal S \subseteq \scr L_1, \scr L_2$
pois todos os seus elementos são claramente cofinitos.
Mesmo assim, o $\Inter \cal S$ não é cofinito
$\Inter \cal S$ é o conjunto de todos os ímpares)
e logo não pertence em nenhum dos $\scr L_1, L_2$.

\endproblem
%%}}}

%%{{{ prob: banach_decomposition_theorem 
\problem Teorema de decomposição de Banach.
\label{banach_decomposition_theorem}%

\endproblem
%%}}}

%%{{{ prob: schroder_bernstein_lattice_proof 
\problem Teorema de Schröder--Bernstein.
\label{schroder_bernstein_lattice_proof}%

\endproblem
%%}}}

\endproblems
%%}}}

%%{{{ Further reading 
\further.

\cite{DaveyPriestley},
\cite{gratzerlatticefirst},
\cite{gratzerlatticefoundation}.

\cite{halmosboolean}.

\endfurther
%%}}}

\endchapter
%%}}}

%%{{{ chapter: Well-orderings and transfinite induction 
\chapter Bem-ordens e indução transfinita.

%%{{{ intro 
\chapintro
O termo \emph{wellorder} (ou \emph{well-order}) de inglês tem sido traduzido
como \emph{boa ordem} em português.  Aqui eu uso o termo \emph{bem-ordem}.
Além de ficar mais perto no termo internacional, a palavra ``bem'' tem
um significado \emph{bem} mais usado em português do que em inglês,
onde não é muito \emph{well} known, exemplificado aqui:%
\footnote{e com certeza \emph{well} beyond nossos interesses aqui}
\endgraf\smallskip
Uma ordem é uma relação legal.\endgraf
Uma \emph{bem}-ordem é uma relação \emph{bem}~legal---vamos descobrir isso neste capítulo.
%%}}}

%%{{{ df: woset 
\definition Bem-ordem.
\label{woset}%
\tdefined{woset}%
\iisee{conjunto}[bem-ordenado]{woset}%
Seja $\sset A <$ um conjunto totalmente ordenado.
Sua ordem $<$ é uma \dterm{bem-ordem}, sse
\emph{cada subconjunto $S\subseteq A$ possui um elemento mínimo}.
Nesse caso chamamos o $A$ \dterm{bem-ordenado} ou \dterm{woset}
(de \dterm{well-ordered set}).
%%}}}

%%{{{ Transfinite induction 
\section Indução transfinita.

\endsection
%%}}}

%%{{{ Transfinite recursion 
\section Recursão transfinita.

\endsection
%%}}}

%%{{{ Problems 
\problems.

\endproblems
%%}}}

%%{{{ Further reading 
\further.

\cite{goldreisets},
\cite{ynmnst}.

\endfurther
%%}}}

\endchapter
%%}}}

%%{{{ chapter: Ordinal arithmetic 
\chapter Aritmética ordinal.

%%{{{ x: omegaomega_plus_one_is_well_ordered 
\exercise.
\label{omegaomega_plus_one_is_well_ordered}%
O $\omega^2+1$ é bem ordenado.

\hint
Lembre-se que usamos a ordem (anti)lexicográfica nos produtos.
Tome $A$ tal que $\emptyset\neq A \subseteq \omega^2 + 1$ e ache se u mínimo.
Separe casos dependendo se $A=\set{\top}$ ou não,
onde $\top$ o máximo elemento do $\omega^2+1$.

\solution
Seja $A\subseteq \omega^2 + 1$ com $A\neq\emptyset$.
Temos a seguinte ordem no $\omega^2 + 1$:
$$
\munderbrace{
 \munderbrace{\tup{0,0} < \tup{1,0} < \tup{2,0} < \dotsb}{\dsize\omega} 
 <
 \munderbrace{\tup{0,1} < \tup{1,1} < \tup{2,1} < \dotsb}{\dsize\omega} 
 <
 \dotsb
}{\dsize\omega^2}
<
\set\top.
$$
\noindent\casestyle{Caso $A = \set {\top}$}:
$\min A = \top$.
\endgraf
\noindent\casestyle{Caso $A \neq \set {\top}$}:
Como $A\neq \emptyset$, concluimos que $A\inter \omega^2 \neq \emptyset$.
Sejam:
$$
\align
y_0 &\asseq \min\set{ y\in\nats \st \lexists {x\in\nats}{\tup{x,y}\in A}}\\
x_0 &\asseq \min\set{ x\in\nats \st \tup{x,y_0}\in A}
\endalign
$$
onde os dois mínima existem graças ao PBO dos naturais.
Facilmente, $\min A = \tup{x_0,y_0}$.

\endexercise
%%}}}

%%{{{ x: solving_for_ordinals 
\exercise.
\label{solving_for_ordinals}%
O que podes concluir sobre os ordinais $\alpha$ e $\beta$ se\dots:
$$
\xxalignat3
\text{(i)}~&\omega + \alpha = \omega & \text{(iii)}~& \omega \cdot \alpha = \omega & \text{(v)}~& \alpha +     \beta = \omega\\
\text{(ii)}~&\alpha + \omega = \omega & \text{(iv)}~& \alpha \cdot \omega = \omega & \text{(vi)}~& \alpha \cdot \beta = \omega
\endxxalignat
$$

\solution
\item{(i)}   $\alpha = 0$
\item{(ii)}  $\alpha$ é finito
\item{(iii)} $\alpha=1$
\item{(iv)}  $\alpha$ finíto \& $\alpha\neq 0$
\item{(v)}   $\text{ou}\,\left\{\aligned&\text{$\alpha$ finíto      \& $\beta=\omega$}\\&\text{$\alpha=\omega$ \& $\beta=0$}\endaligned\right.$
\item{(vi)}  $\text{ou}\,\left\{\aligned&\text{$1\leq\alpha<\omega$ \& $\beta=\omega$}\\&\text{$\alpha=\omega$ \& $\beta=1$}\endaligned\right.$

\endexercise
%%}}}

%%{{{ Problems 
\problems.

\endproblems
%%}}}

%%{{{ Further reading 
\further.

\cite{goldreisets},
\cite{ynmnst},
\cite{kunen2011}.

\endfurther
%%}}}

\endchapter
%%}}}

%%{{{ chapter: Denotational semantics 
\chapter Semântica denotacional.

%TODO: Check http://pages.cs.wisc.edu/~horwitz/CS704-NOTES/
%TODO: use Stoy and Tennent

%%{{{ A language of binary numerals 
\section Uma linguagem de numerais binários.

\endsection
%%}}}

%%{{{ A little programming language 
\section Uma pequena linguagem de programação.

\endsection
%%}}}

%%{{{ Domain theory 
\section Teoria de domínios.

\endsection
%%}}}

%%{{{ Problems 
\problems.

\endproblems
%%}}}

%%{{{ Further reading 
\further.

\cite{stoysemantics},
\cite{tennent1976},
\cite{tennentsemantics}.

\cite{streicherdomainsfp}.

\cite{lpbook}.

\cite{winskelsemantics}.

\endfurther
%%}}}

\endchapter
%%}}}

%%{{{ chapter: Universal algebra 
\chapter Álgebra universal.
\label{Universal_algebra}%

%%{{{ Problems 
\problems.

\endproblems
%%}}}

%%{{{ Further reading 
\further.

\endfurther
%%}}}

\endchapter
%%}}}

%%{{{ chapter: Intuitionistic logic 
\chapter Lógica intuicionista.
\label{Intuitionistic_logic}%

%%{{{ Problems 
\problems.

\endproblems
%%}}}

%%{{{ Further reading 
\further.

\cite{heytingintuitionism}.
\cite{dummettintuitionism}.
\cite{proofsandtypes}.
\cite{lecturesch}.
\cite{bishopfca}~\&~\cite{bishopbridgesconstructive}.
\cite{girardblindspot}.

\endfurther
%%}}}

\endchapter
%%}}}

%%{{{ chapter: Proof theory 
\chapter Teoria de provas.
\label{Proof_theory}%

%%{{{ Systems à la Hilbert 
\section Sistemas à la Hilbert.

\endsection
%%}}}

%%{{{ Dedução natural 
\section Natural deduction.

\endsection
%%}}}

%%{{{ Sequent calculus 
\section Sequent calculus.

\endsection
%%}}}

%%{{{ Problems 
\problems.

\endproblems
%%}}}

%%{{{ Further reading 
\further.

\cite{vonplatoelements},
\cite{bimboproof},
\cite{negrivonplatospt},
\cite{takeuti},
\cite{prawitz},
\cite{proofsandtypes},
\cite{olthdem},
\cite{kleeneIM}.
\cite{lecturesch}.
\cite{girardblindspot}.

\endfurther
%%}}}

\endchapter
%%}}}

%%{{{ chapter: Type theory 
\chapter Teoria de tipos.
\label{Type_theory}%

%%{{{ Problems 
\problems.

\endproblems
%%}}}

%%{{{ Further reading 
\further.

\cite{nederpeltgeuvers},
\cite{piercetapl}.
\cite{hott}.

\endfurther
%%}}}

\endchapter
%%}}}

%%{{{ chapter: Linear logic 
\chapter Lógica linear.
\label{Linear_logic}%

%%{{{ Problems 
\problems.

\endproblems
%%}}}

%%{{{ Further reading 
\further.

\cite{girardllss}.
\cite{girardblindspot}.

\endfurther
%%}}}

\endchapter
%%}}}

%%{{{ chapter: Category theory 
\chapter Teoria de categorias.
\label{Category_theory}%

%%{{{ Problems 
\problems.

\endproblems
%%}}}

%%{{{ Further reading 
\further.

\cite{babylawvere},
\cite{arbibmanesarrows},
\cite{papalawvere}.

\cite{goldblatttopoi}.

\cite{awodeycats},
\cite{barrwellscatscs},
\cite{joyofcats}.

\cite{riehlcats},
\cite{maclanecats}.

\endfurther
%%}}}

\endchapter
%%}}}

