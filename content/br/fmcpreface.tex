%{{{ [vim] 
% vim:foldmarker=%{{{,%}}}
% vim:foldmethod=marker
% vim:foldcolumn=3
%}}}
%% preface.tex
%% author: Thanos Tsouanas <thanos@tsouanas.org>

\chapteroid \prefacename.

%%{{{ How to read this text 
\sectionoid Como ler esse texto.

%%{{{ Fight it! 
\blah Fight it.
\tdefined{Halmos}[túmulo]%
Nas palavras do grande Paul \Halmos{}Halmos:
\spoken
\wq{Don't just read it; fight it!
Ask your own questions,
look for your own examples,
discover your own proofs.
Is the hypothesis necessary?
Is the converse true?
What happens in the classical special case?
What about the degenerate cases?
Where does the proof use the hypothesis?}
\endspoken
Cada demonstração de teorema é marcada com um \symq{$\qedsymbol$},
conhecido como \dterm{Halmos\Halmos[tombstone]~(tombstone)};%
\footnote{A idéia é que conseguimos matar nosso alvo com nossa demonstração,
e logo mostramos (com orgulho) o seu túmulo!}
os resolvidos exemplos no texto terminam com \symq{$\qexsymbol$}.
Para te motivar tentar demonstrar os teoremas antes de estudar suas demonstrações,
muitas vezes eu apresento apenas um esboço da demonstração,
cujo fim eu marco com um \symq{$\qessymbol$}.
Nesses casos, as demonstrações completas aparecem no apendice.
Eu marco com \symq{$\activitysymbol$} os itens do texto que
precisam ser resolvidos.
\endgraf
\emph{Cuidado! Nunca leia matemática passivamente!}
Umas das demonstrações e definições têm falhas e/ou erros
(nesses casos o \symq{$\qedsymbol$} vira-se \symq{$\mistakesymbol$}).
Os problemas e exercísios pedem identificar os erros.
%%}}}

%%{{{ Spoiler alerts 
\blah Spoiler alerts.
Em certos pontos aparecem ``spoiler alerts''.
Isso acontece quando eu tenho acabado de perguntar algo, ou preciso
tomar uma decisão, etc., e seria bom para o leitor pensar sozinho
como ele continuaria desse ponto, antes de virar a página
e ler o resto.
%%}}}

%%{{{ Hints and solutions 
\blah Dicas e resoluções.
Para muitos dos exercícios e problemas eu tenho dicas para te ajudar
chegar numa resolução.  Tente resolver sem procurar dicas.
Se não conseguir, procure uma primeira dica no apêndice ``Dicas \#1'',
e volte a tentar.
Se ainda parece difícil, procure segunda dica no ``Dicas \#2'', etc., etc.
Finalmente, quando não tem mais dicas para te ajudar,
no apêndice tem resoluções completas.
Quando tem dicas, no fim do enunciado aparecem numerinhos em parenteses
e cada um deles é um link que te leva para a dica correspondente.
Não tenho link para a resolução: a idéia é dificultar a vida de quem
quer desistir facilmente e procurar logo a resolução.
%%}}}

\endsectionoid
%%}}}

%%{{{ About the reader 
\sectionoid Sobre o leitor.

\blah Prerequisitos.
Nada demais: o texto é bastante ``self-contained''.
Supostamente o leitor sente confortável com as propriedades básicas
de aritmética, manipulações algébricas, convenções e notação relevante.
Idealmente experiência com programação ajudaria entender muitos exemplos
e metaforas que uso para explicar uns conceitos matemáticos, mas quem
não programou nunca na vida ainda consegue acompanhar.
Se esse é o teu caso, sugiro começar a aprender ambas as artes de
\emph{demonstrar} e de \emph{programar} em paralelo, até chegar
na conclusão que não se trata de duas artes mesmo, mas da mesma,
só com roupas diferentes.

\endsectionoid
%%}}}

\endchapteroid
