%{{{ [vim] 
% vim:foldmarker=%{{{,%}}}
% vim:foldmethod=marker
% vim:foldcolumn=3
%}}}
%% preface.tex
%% author: Thanos Tsouanas <thanos@tsouanas.org>

\chapteroid \prefacename.

\sectionoid Para o leitor.

Cada prova de teorema é marcada com um ``\thinspace\qedsymbol\thinspace'',
conhecido como ``Halmos\Halmos[tombstone]~(tombstone)'';
os resolvidos exemplos no texto terminam com ``\thinspace\qexsymbol\thinspace''.
Para te motivar tentar provar os teoremas antes de estudar suas provas,
muitas vezes eu apresento apenas um rascunho de prova,
cujo fim eu marco com um ``\thinspace\qessymbol\thinspace''.
Nesses casos, as provas completas aparecem no apendice.
Eu marco com ``\thinspace\activitysymbol\thinspace'' os itens do texto que
precisam ser resolvidos.

Para muitos dos exercícios e problemas eu tenho dicas para te ajudar
chegar numa resolução.  Tente resolver sem procurar dicas.
Se não conseguir, procure uma primeira dica no apêndice ``Dicas \#1'',
e volte a tentar.
Se ainda parece difícil, procure segunda dica no ``Dicas \#2'', etc., etc.
Finalmente, quando não tem mais dicas para te ajudar,
no apêndice tem resoluções completas.

\emph{Cuidado! Nunca leia matemática passivamente!}
Umas das provas tem falhas e/ou erros
(nesses casos o ``\thinspace\qedsymbol\thinspace''
vira-se ``\thinspace\mistakesymbol\thinspace'').
Os problemas e exercísios pedem identificar os erros,
e caso que os teoremas são realmente prováveis suas provas
inteiras e corretas também aparecem no apêndice.

Em certos pontos aparecem ``spoiler alerts''.
Isso acontece quando eu tenho acabado de perguntar algo, ou preciso
tomar uma decisão, etc., e seria bom para o leitor pensar sozinho
como ele continuaria desse ponto, antes de virar a página
e ler o resto.

\endsectionoid

\endchapteroid
