%{{{ [vim] 
% vim:foldmarker=%{{{,%}}}
% vim:foldmethod=marker
% vim:foldcolumn=8
%}}}
%% fmc.tex
%% author: Thanos Tsouanas <thanos@tsouanas.org>

\input fmcmac

%%{{{ Frontmatter 

%%{{{ Front cover 
{
\titlepagetrue
\nodecorspage
{\hrule width \hsize height 1pt depth 1pt }
\vskip 4pt
\line{\noindent\twelvett \ifprint Printed edition\else Electronic edition\fi\hfil \ifdraft WORK IN PROGRESS: DRAFT\fi}
\line{\twelvett \url{http://www.tsouanas.org/fmc/}\hfil
\twelvett
\safereadfile{lasttmp}
}
\vskip 2ex
\vfil\vfil
{
{\noindent\centerline{\twentyfourbf Matemática Fundacional}}
\vskip 2ex
{\hrule width \hsize height 1pt\relax}
\vskip 2ex
{\noindent\centerline{\twentytt para Computação}}
}
\vskip 2ex
\vfil
\leftline{\largest\larger Thanos {\scshape Tsouanas}}
\vskip 2pt
{\it
\leftline{Universidade Federal do Rio Grande do Norte}
}
\vskip 1ex
\vfil
\centerline{\bf\today/}
\centerline{\oldstyle{\the\time}~minute\ifnum\time=1\else s\fi~past midnight,}
\centerline{Rio Grande do Norte, Brasil}
\vfil\vfil
{\twelvessdc
\centerline{Instituto Metrópole Digital}
\centerline{Universidade Federal do Rio Grande do Norte}
}
\vskip 4pt
{\hrule width \hsize height 1pt depth 1pt }
\eject
}
%%}}}

%%{{{ meta info 
{
\titlepagetrue
\leftline{\twelverm Please report any errors by email:
{\twelvett {thanos$\@$tsouanas.org}}}
\vfil
\centerline{Escrito em português \strikeout{do Brasil} do Θάνος.}
\vfil
\vfil
\vfil
\centerline{Typeset in \TeX\ with the \ThaTeX~package;}
\centerline{built with the \XeTeX~engine.}
\eject
}
%%}}}

\ignore{

%%{{{ dedication 
{
\hbox{}\vfil
\centerline{{\fourteenit To someone.}}
\vfil\vfil\vfil\vfil
\eject
}
%%}}}

}

%%{{{ short toc 
{
\oddpage\titlepagetrue
\baselineskip=18pt
%\def\tocchapterentry#1#2#3{{\line{\llap{{#2}.~}#1\dotfill#3}}}
\def\tocchapterentry#1#2#3{{\line{\llap{{#2}.~}#1\dotfill\hlprintpage{#3}}}}
\def\tocappendixentry#1#2#3{{\line{\llap{{\Alphabetnumeral#2}.~}#1\dotfill\hlprintpage{#3}}}}
\def\tocchapteroidentry#1#2{{\line{#1\dotfill\hlprintpage{#2}}}}
\def\tocsectionentry#1#2#3{}
\def\tocsectionoidentry#1#2{}
{\larger\bf
\parindent=0pt
\baselineskip=16pt
\centerline{\fourteenrm\toupper{\shorttocname}}
\vskip\chapteroidtitleskip
\readtocfile
}
\vfil
\eject
}
%%}}}

%%{{{ long toc 
{
\oddpage
\titlepagetrue
\nodecorspage
\def\tocchapterentry#1#2#3{{\bf\vskip 2ex\line{\llap{{#2}.~}#1\enspace\hrulefill\enspace\hlprintpage{#3}}}\vskip 1ex}
\def\tocappendixentry#1#2#3{{\bf\vskip 2ex\line{\llap{{\Alphabetnumeral#2}.~}#1\enspace\hrulefill\enspace\hlprintpage{#3}}}\vskip 1ex}
\def\tocchapteroidentry#1#2{{\bf\vskip 2ex\line{#1\enspace\hrulefill\enspace\hlprintpage{#2}}}\vskip 1ex}
\def\tocsectionentry#1#2#3{{\line{{\qquad} \llap{\S#2.~}{#1}{\dotfill}{\hlprintpage{#3}}\qquad\qquad}}}
\def\tocsectionoidentry#1#2{{\line{{\qquad} {#1}{\dotfill}{\hlprintpage{#2}}\qquad\qquad}}}
{
\parindent=0pt
\centerline{\fourteenrm\toupper{\tocname}}
\vskip\chaptertitleskip
\readtocfile
}
\vfil
\eject
}
%%}}}

\ignore{

%%{{{ list of algorithms 
\chapteroid \loaname.

\endchapteroid
%%}}}

%%{{{ list of code 
\chapteroid \locname.

\endchapteroid
%%}}}

}

%%{{{ Preface & acknowledgements 
\chapteroid \prefacename.

\sectionoid Para o leitor.

Cada prova de teorema é marcada com um ``\thinspace\qedsymbol\thinspace'',
conhecido como ``Halmos\Halmos[tombstone]~(tombstone)'';
os resolvidos exemplos no texto terminam com ``\thinspace\qexsymbol\thinspace''.
Para te motivar tentar provar os teoremas antes de estudar suas provas,
muitas vezes eu apresento apenas um rascunho de prova,
cujo fim eu marco com um ``\thinspace\qessymbol\thinspace''.
Nesses casos, as provas completas aparecem no apendice.
Eu marco com ``\thinspace\activitysymbol\thinspace'' os itens do texto que
precisam ser resolvidos.

Para muitos dos exercícios e problemas eu tenho dicas para te ajudar chegar
numa resolução.  Tente resolver sem procurar dicas.
Se não conseguir, procure uma primeira dica no apêndice ``Dicas \#1'',
e volte a tentar.
Se ainda parece difícil, procure segunda dica no ``Dicas \#2'',
etc., etc.
Finalmente, no apêndice tem resoluções completas.

\emph{Cuidado! Nunca leia matemática passivamente!}
Umas das provas tem falhas e/ou erros
(nesses casos o ``\thinspace\qedsymbol\thinspace''
vira-se ``\thinspace\mistakesymbol\thinspace'').
Os problemas e exercísios pedem identificar os erros,
e caso que os teoremas são realmente prováveis suas provas
inteiras e corretas também aparecem no apêndice.

\endsectionoid

\endchapteroid
%%}}}

%%}}}

\input fmcmain

%%{{{ Appendix 

%%{{{ APPENDICES page 
{
\partpage
\hbox{}\vfil
\twentyfourbf
\centerline{\appendicesname}
\titlepagetrue
\vfil
\vfil
\eject
}
%%}}}

%%{{{ Proofs 
\appendix \proofsname.

{%
\immediate\closeout\fullproofsfile
\innernewcount\curchapterno

\def\fullproofentry#1#2#3{%
\ifnum#1>\curchapterno
\curchapterno=#1
\centerline{\fourteenbf\chaptername~#1}\vskip\fullproofskip
\fi
\noindent{\bf #2.}
#3
\goodbreak\vskip\fullproofskip
}

\safereadfile{proofs}
}

\endappendix
%%}}}

%%{{{ Hints 
\appendix \hintsname.

{%
\immediate\closeout\hintsfile
\input repeat
\repeat \for{i} \from{1} \to{\maxhintno} \do{
\expandafter\hintsection \number\i.
\hldestizepage%

\def\hintentry#1#2#3#4#5#6{%
\ifnum\number\i=#5
\noindent{\llap{\bf \refn{#4}.~}}%
{\scshape\hintname~\oldstyle{#5}:}
\definexref{#4_hint_#5}{#5}{hint}%
#6
%
\goodbreak
\vskip\hintskip
\fi
\goodbreak
}

\safereadfile{hints}

\endhintsection
}% end repeat
}

\endappendix
%%}}}

%%{{{ Solutions 
\appendix \solutionsname.

{%
\immediate\closeout\solutionsfile
\innernewcount\curchapterno

\def\solutionentry#1#2#3#4{%
\ifnum#2>\curchapterno
\curchapterno=#2
\centerline{\fourteenbf\chaptername~#2}\vskip\solutionskip
\fi
\noindent\llap{\bf #3.\enspace}%
#4
\vskip\solutionskip
}

\safereadfile{solutions}
}

\endappendix
%%}}}

%%{{{ Program code 
\appendix Código de programas.

\safereadfile{programs}

\endappendix
%%}}}

%%}}}

%%{{{ Backmatter 

%%{{{ Referências 
\chapteroid \referencesname.

\bibliographystyle{alpha}
\bibliography{fmcrefs}

\endchapteroid
%%}}}

%%{{{ Symbol definitions 
\chapteroid Definições de símbolos.

{%
\immediate\closeout\symbolsfile
\multiply\baselineskip by 2

\def\symbolentry#1#2#3#4#5#6#7{%
\line{\llap{#6}\quad\hbox{#7}\quad\dotfill{}\enspace{{\ref{#5},}\rlap{\enspace{\bf #4}}}}%
}

%\def\symbolentry#1#2#3#4#5#6#7{%
%{#6} & \hbox{#7}\dotfill & {\ref{#5}} & \rlap{\bf #4}\cr
%}

%\halign to \hsize{
%\hfil# & #\hfil  & #\hfil     & \hfil#\cr
\safereadfile{symbols}
%}
}

\endchapteroid
%%}}}

%%{{{ Indice de nómes 
\chapteroid \nameindexname.

\readindexfile n

\endchapteroid
%%}}}

%%{{{ Indice de termos 
\chapteroid \indexname.

\readindexfile i

\endchapteroid
%%}}}

%%{{{ Back cover

%%}}}

%%}}}

%%{{{ \Report 
\Report
\WriteReport{exercises}{\the\exercisetotal}
\WriteReport{problems}{\the\problemtotal}
\WriteReport{chapters}{\the\chaptertotal}
\WriteReport{sections}{\the\sectionno}
\WriteReport{pages}{\the\pageno}
\EndReport
%%}}}

\bye

